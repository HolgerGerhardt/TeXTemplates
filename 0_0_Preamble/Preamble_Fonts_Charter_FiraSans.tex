% !TeX TXS-program:compile = txs:///pdflatex/
% !TeX TS-program = pdflatex
% !TeX TXS-program:bibliography = txs:///biber
% !BIB program = biber




%%%%%%%%%%%%%%%%%%%%%%%%%%%%%%%%%%%%
%%  FONT SETTINGS AND TYPOGRAPHY  %%
%%%%%%%%%%%%%%%%%%%%%%%%%%%%%%%%%%%%


% Use Fira Sans as the sans-serif font.
% Load first because it otherwise doesn't work properly.
% ==>
\usepackage[lining, scale=0.88]{FiraMono}  % To match the x-height of the Charter serif font
\usepackage[book, semibold, lining, tabular, scale=0.88]{FiraSans}[2018/05/23]
\makeatletter
\@ifpackagelater{FiraSans}{2018/05/23}
	{}
	{\PackageError{FiraSans}
	     {Outdated 'FiraSans' package}
	     {Upgrade to FiraSans 2018/05/23 or newer!\MessageBreak
	      Otherwise the sans-serif font may not work properly, or the compilation might fail.\MessageBreak
	      This is a fatal error. I'm aborting now.}%
	   \endinput}
\makeatother
%% In case you wish to use the ``light'' options of FiraSans: -->
%\usepackage{afterpackage}
%\AfterPackage{siunitx}{%
%  \providecommand*{\lseries}{\fontseries{l}\selectfont}
%}
%% See https://tex.stackexchange.com/questions/164791/combination-of-siunitx-and-light-weight-font-fails-to-compile.
%% <--
% <==

%% This template also works with the ``IBM Plex Sans'' and ``IBM Plex Mono'' fonts,
%% https://ctan.org/pkg/plex ==>
%\usepackage[semibold, scale=0.89]{plex-mono}  % To match the x-height of the Charter serif font
%\usepackage[text, semibold, scale=0.89]{plex-sans}[2018/12/31]
%	% Earlier versions of the ``plex-sans'' package do not include support for LGR Greek.
%% <==

%% This template also works with the ``Alegreya'' and ``Alegreya Sans'' fonts,
%% https://ctan.org/pkg/alegreya ==>
%\usepackage[semibold, scale=0.89]{plex-mono}  % To match the x-height of the Charter serif font
%\usepackage[lining, tabular, scale=1.02]{AlegreyaSans}[2018/08/02]
%	% Earlier versions of the ``Alegreya'' package do not include support for LGR Greek.
%% <==

%% This template also works with the ``Google Noto'' fonts,
%% https://ctan.org/pkg/noto ==>
%\usepackage[scale=0.85]{noto-mono}
%\usepackage[lining, tabular, scale=0.85]{noto-sans}[2019/01/11]
%	% Earlier versions of the ``noto'' package do not include support for LGR Greek.
%% <==

% Save the font family and font series definitions for later use: ==>
\newcommand{\savesffamily}{\sfdefault}
\makeatletter
\newcommand{\savesfmdseries}{\mdseries@sf}
\newcommand{\savesfbfseries}{\bfseries@sf}
\makeatother
% <==

% Use Bitstream Charter as the math font
% ==>
\usepackage[charter, greeklowercase=italicized, greekuppercase=italicized, sfscaled=false, ttscaled=false]{mathdesign}
%% Use Alegreya as the text font -->
%\usepackage[lining, tabular, scale=1.02]{Alegreya}[2018/08/02]
%\ifxetex \else \DisableLigatures[T]{family = {rm*}} \fi
%% <--
%% Use Noto Serif as the text font -->
%\usepackage[lining, tabular, scale=0.85]{noto-serif}[2019/01/11]
%% <--
% Save the font family and font series definitions for later use:
\let \savermfamily   \rmdefault
\let \savermmdseries \mddefault
\let \savermbfseries \bfdefault
% <==

% Use Bitstream Charter as the text font
% ==>
\usepackage[scaled=.96, lining, sups]{XCharter}  % option ``sups'' not working properly with subimport
% Save the font family and font series definitions for later use:
\let \savermdefaultfortext \rmdefault
\let \savemddefaultfortext \mddefault
\let \savebfdefaultfortext \bfdefault
\renewcommand{\textellipsis}{\mbox{.{\kern.09em}.{\kern.09em}.}}
\makeatletter
\newcommand{\@makefnmarkorig}{%
	\hbox{\sufigures\hspace*{.04em}\@thefnmark\hspace*{.04em}}%
}  % Copied from XCharter.sty
\makeatother
% <==

\ifxetex
	% Do nothing.
\else
	\DisableLigatures[f]{family = {rm*, tt*}}
	% Disable the f* ligatures for Charter because the font does not provide sufficient support.
	% Disable the f* ligatures for the ``typewriter'' font because they make no sense for a monospaced font.
	\SetExtraKerning[unit=space]
		{encoding=*, family=*, series=*, size={*, normalsize, footnotesize}, font = */*/*/*/*}
		{\textemdash = {325, 325},
		           / = {100, 100},
		           : = { 50,   0},
		           ; = { 50,   0}}
\fi

\usepackage{xfrac}	% Provides \sfrac

\usepackage[normalem]{ulem}	% Do NOT change \emph to \uline!

% Since math mode uses a different font encoding, issuing \euro/\texteuro in math mode
% produces an incorrect sign. We fix this. ==>
\AtBeginDocument{%
	\renewcommand*{\euro}[1]{%
		\relax\ifmmode\text{\texteuro}#1\else\texteuro #1\fi%
	}%
}
% <==