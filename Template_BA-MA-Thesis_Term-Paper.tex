% !TEX program = pdflatex
% !BIB program = biber
% !TEX spellcheck = en-us




%%%%%%%%%%%%%%%%%%%%%%%%%%
%%  AUTHOR INFORMATION  %%
%%%%%%%%%%%%%%%%%%%%%%%%%%


\newcommand{\studentfullname}{\"{A}nn\'{i} Sn\^{e}\v{z}{\`a}na H\r{a}land-\c{C}a{\l}han\H{o}\u{g}l\"{u}}
\newcommand{\studentemail}{a.s.haaland-calhanoglu@example.org}  % Only relevant for term papers, ignored for bachelor’s and master’s theses
\newcommand{\studentID}{1234567}
\newcommand{\manuscripttitle}{Template for Term\nolinebreak\ Papers, Bachelor\texorpdfstring{'}{’}s\nolinebreak\ Theses, and Master\texorpdfstring{'}{’}s\nolinebreak\ Theses}
\newcommand{\manuscriptdate}{2025-10-01}  % In ISO format: yyyy-mm-dd
\newcommand{\manuscripttype}{MA}  % BA for bachelor's thesis, MA for master's thesis, otherwise term paper
\newcommand{\manuscriptlanguage}{EN}  % DE for German, otherwise (US) English
\newcommand{\seminartitle}{%
	Theoretical and Empirical Microeconomics and Macroeconomics with Implications for Social Policy All Around the World%
}  % Only relevant for term papers, ignored for bachelor’s and master’s theses
\newcommand{\advisorname}{Prof.~Dr. Vae-Ree Smart}
\newcommand{\advisorgender}{x}  % f for female, m for male, otherwise nonbinary




%%%%%%%%%%%%%%%%%%%%%%
%%  DOCUMENT CLASS  %%
%%%%%%%%%%%%%%%%%%%%%%


\documentclass[12pt, a4paper, oneside]{article}
	% In line with the official guidelines of the Department of Economics at the University of Bonn:
	% https://www.econ.uni-bonn.de/examinations/de/informationen/bachelor/bachelorarbeit/dokumente/ba-merkblatt-2016-05-23.pdf
	% https://www.econ.uni-bonn.de/examinations/en/information/master-economics/master-thesis/documents/ma-master-thesis-style-guide-2014-06-10.pdf
%\documentclass[11pt, a4paper, oneside]{article}
	% Alternative with 11pt font size
	% Margins are adjusted below such that approximately the same amount of text fits per page




%%%%%%%%%%%%%%%%%%%%%%%%%%%%
%%  FUNDAMENTAL PACKAGES  %%
%%%%%%%%%%%%%%%%%%%%%%%%%%%%


\usepackage[utf8]{inputenc}
\usepackage[LGR, T1]{fontenc}
\usepackage{textgreek}  % Use Greek letters in text mode
\AtBeginDocument{\DeclareTextGreekSymbol{nu}{n}[\char23]}  % \textnu is overwritten by some package

\usepackage{ifthen}
\ifthenelse{%
	\equal{\manuscriptlanguage}{DE}%
}{% German
	\usepackage[USenglish, ngerman]{babel}
}{% English
	\usepackage[ngerman, english]{babel}
}

\usepackage{xpatch}
\usepackage{etoolbox}
\usepackage{calc}  % Enables addition, subtraction, etc. of lengths

\usepackage{graphicx}
\usepackage[svgnames, x11names]{xcolor}
\definecolor{UBonnBlue}{RGB}{0, 78, 159}  % #004e9f  % was 7/82/154 before 2022
\definecolor{UBonnBlueTinted}{RGB}{0, 73, 143}  % darker header on uni-bonn.de pages  % or (6, 73, 138), as the darker footer on uni-bonn.de pages
\definecolor{UBonnBlueDark}{RGB}{5, 61, 115}  % https://confluence.team.uni-bonn.de/spaces/UNISERVICEPORT/pages/187724137/Design+Gestalten+Corporate+Identity#Design%2CGestalten%26CorporateIdentity-PowerpointVorlagen(PPT%2Cbarrierefrei)
\definecolor{UBonnGray}{RGB}{144, 144, 133}
\definecolor{UBonnYellow}{RGB}{252, 186, 0}
\colorlet{UBonnGrey}{UBonnGray}

\usepackage{amsmath, mathtools}
\usepackage{amsthm, thmtools}
\renewcommand{\qedsymbol}{\ensuremath{\blacksquare}}

\usepackage[ngerman, UKenglish, USenglish, cleanlook]{isodate}

\usepackage[bottom, multiple, splitrule, stable]{footmisc}
\let \footnoteruleOrig \footnoterule
\renewcommand{\footnoterule}{\footnoteruleOrig\vskip 1.5ex}
\renewcommand{\multfootsep}{,\,}
\makeatletter
\patchcmd{\@footnotetext}{#1}{\strut\ignorespaces #1}{}{}
\makeatother

\PassOptionsToPackage{hyphens}{url}
\usepackage{hyperref}
\hypersetup{
	allcolors = UBonnBlueDark,  % alternatively, use UBonnBlue (brighter) or UBonnBlueDark (darker)
	colorlinks = true,
	pdftitle  = {\manuscripttitle},
	pdfauthor = {\studentfullname},
	bookmarksnumbered = true,
}

\newcommand*{\Appendixautorefname}{Appendix}
	% See https://tex.stackexchange.com/questions/207744/no-autoref-name-for-appendix

\usepackage{titlecaps}




%%%%%%%%%%%%%%%%%%%
%%  PAGE LAYOUT  %%
%%%%%%%%%%%%%%%%%%%


\flushbottom

% Adjust font sizes and page margins based on class option 11pt/12pt
% such that same amount of text fits per page for 11pt and 12pt
\newlength{\baselinedist}
\newlength{\smalllinespacing}
\newlength{\footnotelinespacing}
\newlength{\fsLarge}
\newlength{\fslarge}
\newlength{\fsnormal}
\newlength{\fssmall}
\newlength{\fsfootnote}
\let \LargeOrig \Large
\let \largeOrig \large
\let \normalsizeOrig \normalsize
\let \smallOrig \small
\let \footnotesizeOrig \footnotesize
\makeatletter
\ifdim \f@size pt < 11.5pt
	\usepackage[
		bottom = 2.8cm,
		left = 2.5cm,
		right = 4cm,
		top = 2.8cm,
		headheight = 15pt,  % Numerous warning messages otherwise
		headsep = 22.5pt,
		marginpar = 3.25cm,
		footnotesep = 19.5pt plus 39pt,
	]{geometry}
	\setlength{\baselinedist}{19.5pt}
	\setlength{\skip\footins}{19.5pt plus 58.5pt}
	\setlength{\smalllinespacing}{15pt}
	\setlength{\footnotelinespacing}{13.5pt}
	\setlength{\fsLarge}{14pt}
	\setlength{\fslarge}{12pt}
	\setlength{\fsnormal}{11pt}
	\setlength{\fssmall}{10pt}
	\setlength{\fsfootnote}{9pt}
	% Adapted from size11.clo:
	\def\normalsize{%
		\normalsizeOrig%
		\fontsize{\fsnormal}{\baselinedist}\selectfont%
		\abovedisplayskip 13.5pt plus 6.75pt minus 3.375pt%
		\belowdisplayskip \abovedisplayskip%
		\abovedisplayshortskip 6.75pt plus 3.375pt minus 1.6875pt%
		\belowdisplayshortskip 6.75pt plus 3.375pt minus 1.6875pt%
	}
\else
	\usepackage[
		bottom = 2cm,
		left = 3cm,
		right = 2cm,
		top = 2cm,
		headheight = 16.5pt,  % Numerous warning messages otherwise
		headsep = 16.5pt,
		marginpar = 1.5cm,
		footnotesep = 20.7pt plus 41.4pt,
	]{geometry}
	\setlength{\baselinedist}{20.7pt}
	\setlength{\skip\footins}{20.7pt plus 62.1pt}
	\setlength{\smalllinespacing}{16.5pt}
	\setlength{\footnotelinespacing}{15pt}
	\setlength{\fsLarge}{16.5pt}
	\setlength{\fslarge}{14pt}
	\setlength{\fsnormal}{12pt}
	\setlength{\fssmall}{11pt}
	\setlength{\fsfootnote}{10pt}
	% Adapted from size12.clo:
	\def\normalsize{%
		\normalsizeOrig%
		\fontsize{\fsnormal}{\baselinedist}\selectfont%
		\abovedisplayskip 15pt plus 7.5pt minus 3.75pt%
		\belowdisplayskip \abovedisplayskip%
		\abovedisplayshortskip 7.5pt plus 3.75pt minus 1.775pt%
		\belowdisplayshortskip 7.5pt plus 3.75pt minus 1.775pt%
	}
\fi
\makeatother
\normalsize

\renewcommand{\Large}{\LargeOrig\fontsize{\fsLarge}{\baselinedist}\selectfont}
\renewcommand{\large}{\largeOrig\fontsize{\fslarge}{\baselinedist}\selectfont}
\renewcommand{\small}{\smallOrig\fontsize{\fssmall}{\smalllinespacing}\selectfont}
\renewcommand{\footnotesize}{\footnotesizeOrig\fontsize{\fsfootnote}{\footnotelinespacing}\selectfont}

\setlength{\textheight}{35\baselinedist}  % Permit exactly 35 lines of text per page

\usepackage{ragged2e}  % For ragged right/left text with hyphenation enabled
\usepackage[all]{nowidow}  % Prevent orphans and widows

\renewenvironment{quote}{%
	\par\list{}{%
		\leftmargin = \baselineskip%
		\rightmargin = \leftmargin%
		\baselineskip = \smalllinespacing% 15pt%
	}
	\item\relax%
}{%
	\endlist%
}
\renewenvironment{quotation}{%
	\list{}{%
		\leftmargin = \baselineskip%
		\rightmargin = \leftmargin%
		\listparindent = \smalllinespacing%
		%\itemindent = \listparindent%
		\parsep = 0pt%
		\baselineskip = \smalllinespacing% 15pt
	}%
	\item\relax%
}{%
	\endlist%
}

\AtBeginDocument{%
	\setlength{\baselineskip}{\baselinedist}%
	\setlength{\topskip}{\baselineskip}%
	\setlength{\parindent}{\baselineskip}%
	\setlength{\parskip}{0pt}%
	\frenchspacing% Reduce amount of white space after a period
	\let \nonfrenchspacing=\frenchspacing% because babel's \selectlanguage{...} resets \frenchspacing
	\sloppy% Prevent overfull hboxes (at the expense of more uneven whitespace)
}

\usepackage{fancyhdr}
\pagestyle{fancy}
\fancyhf{}
\renewcommand{\headrulewidth}{0pt}
\fancyhead[L]{\sffamily\small\strut\studentfullname: \textit{\manuscripttitle}}
\fancyhead[R]{\sffamily\small\strut\thepage}




%%%%%%%%%%%%%%%%%%
%%  COVER PAGE  %%
%%%%%%%%%%%%%%%%%%


\makeatletter
\newcommand*{\maketitleEN}{%
	\thispagestyle{empty}
	\pdfbookmark[1]{\manuscripttitle}{coverpage}
	\begin{center}
		\Large
		\null\vfill
		{\fontsize{27.5pt}{32pt}\bfseries\manuscripttitle\par}
		\vfill\vfill
		{\huge\ifthenelse{\equal{\manuscripttype}{BA}}{Bachelor's Thesis}{%
			\ifthenelse{\equal{\manuscripttype}{MA}}{Master's Thesis}{%
				Term Paper%
			}%
		}\par}
		\ifthenelse{\equal{\manuscripttype}{BA} \OR \equal{\manuscripttype}{MA}}{%
			\vspace{1.75\baselineskip}
			Presented to the \\
			Department of Economics at the \\
			University of Bonn \par
			\vspace{1.25\baselineskip}
		}{%
			\vspace{1.25\baselineskip}
			for the Seminar \par
			\vspace{1.25\baselineskip}
		}
		\ifthenelse{\equal{\manuscripttype}{BA} \OR \equal{\manuscripttype}{MA}}{%
			In Partial Fulfillment of the Requirements for the Degree of \\
			\ifthenelse{\equal{\manuscripttype}{BA}}{Bachelor of Science (B.Sc.)}{Master of Science (M.Sc.)}
		}{
			``\seminartitle''
		}
		\par
		\vfill\vfill%\vspace{2\baselineskip}
		\ifthenelse{\equal{\manuscripttype}{BA} \OR \equal{\manuscripttype}{MA}}{Supervisor}{Instructor}: \\[0.5\baselineskip]
		\advisorname
		\vfill\vfill%\vfill
		\ifthenelse{\equal{\manuscripttype}{BA} \OR \equal{\manuscripttype}{MA}}{%
			Submitted in \printdayoff\printdate{\manuscriptdate}%
		}{%
			Submitted on \printdate{\manuscriptdate}%
		}
		by \\[1.5\baselineskip]
		{\LARGE \@author} \\[1.5\baselineskip]
		\ifthenelse{\equal{\manuscripttype}{BA} \OR \equal{\manuscripttype}{MA}}{}{%
			E-Mail: \href{mailto:\studentemail}{\studentemail}\\%
		}
		Matriculation Number: \studentID
		\vfill
	\end{center}
}
\newcommand*{\maketitleDE}{%
	\thispagestyle{empty}
	\pdfbookmark[1]{\manuscripttitle}{Titelseite}
	\begin{center}
		\Large
		\null\vfill
		{\fontsize{27.5pt}{32pt}\bfseries\manuscripttitle\par}
		\vfill\vfill
		{\huge\ifthenelse{\equal{\manuscripttype}{BA}}{Bachelorarbeit}{%
			\ifthenelse{\equal{\manuscripttype}{MA}}{Masterarbeit}{%
				Hausarbeit%
			}%
		}\par}
		\ifthenelse{\equal{\manuscripttype}{BA} \OR \equal{\manuscripttype}{MA}}{%
			\vspace{1.75\baselineskip}
			zur Erlangung des \\
			Grades \ifthenelse{\equal{\manuscripttype}{BA}}{Bachelor of Science (B.\,Sc.)}{Master of Science (M.\,Sc.)} \\
			im Studiengang Volkswirtschaftslehre an der \\
			Rheinischen Friedrich-Wilhelms-Universität Bonn%
		}{%
			\vspace{1.25\baselineskip}
			zum Seminar \par
			\vspace{1.25\baselineskip}
			\quotedblbase\seminartitle\textquotedblleft
		}
		\par
		\vfill\vfill%\vspace{2\baselineskip}
		Themensteller\ifthenelse{\equal{\advisorgender}{f}}{in}{\ifthenelse{\equal{\advisorgender}{m}}{}{*in}}: \\[0.5\baselineskip]
		\advisorname
		\vfill\vfill%\vfill
		\ifthenelse{\equal{\manuscripttype}{BA} \OR \equal{\manuscripttype}{MA}}{%
			Vorgelegt im \datengerman\printdayoff\printdate{\manuscriptdate}%
		}{
			Vorgelegt am \datengerman\printdate{\manuscriptdate}%
		}
		von \\[1.5\baselineskip]
		{\LARGE \@author} \\[1.5\baselineskip]
		\ifthenelse{\equal{\manuscripttype}{BA} \OR \equal{\manuscripttype}{MA}}{}{%
			E-Mail: \href{mailto:\studentemail}{\studentemail}\\%
		}
		Matrikelnummer: \studentID
		\vfill
	\end{center}
}
\makeatother

\ifthenelse{%
	\equal{\manuscriptlanguage}{DE}%
}{% German
	\AtBeginDocument{\selectlanguage{ngerman}}
	\renewcommand*{\maketitle}{\maketitleDE}
}{% English
	\AtBeginDocument{\selectlanguage{USenglish}}
	\renewcommand*{\maketitle}{\maketitleEN}
}%




%%%%%%%%%%%%%%%%%%%%%%%%%%%%%%%%%%%%%%%%%%%
%%  FORMATTING OF THE TABLE OF CONTENTS  %%
%%%%%%%%%%%%%%%%%%%%%%%%%%%%%%%%%%%%%%%%%%%


\let \tableofcontentsOrig \tableofcontents

% If you would like to increase the horizontal spacing in the table of contents
%\makeatletter
%\patchcmd{\l@section}{1.5em}{2em}{}{}  % \l@section is defined in article.cls
%%\newcommand*\l@subsection{\@dottedtocline{2}{1.5em}{2.3em}}  % definition in article.cls
%\renewcommand*{\l@subsection}{\@dottedtocline{2}{2em}{3em}}
%\makeatother

%\makeatletter
%\pretocmd{\l@section}{\vspace{-1ex}}% Reduce vertical white space before section entry
%\makeatother

\makeatletter
\renewcommand{\tableofcontents}{%
	\begingroup  % Local changes only
	\setlength{\baselineskip}{\smalllinespacing}%
	\microtypesetup{protrusion = false}% Disable optical marginal alignment for TOC so that page numbers align properly
	\pdfbookmark[1]{\contentsname}{toc}% Add bookmark for TOC to PDF file
	\setcounter{tocdepth}{2}%
	\renewcommand{\@dotsep}{200}% Remove dotted leader between subsection headings and page numbers in TOC
	\renewcommand{\@pnumwidth}{4em}% Increase width of page numbers so that also ``A-12'' fits
	\tableofcontentsOrig%
	\endgroup%
}
\makeatother




%%%%%%%%%%%%%%%%%%%%%
%%  FONT SETTINGS  %%
%%%%%%%%%%%%%%%%%%%%%


\usepackage{relsize}  % Allow for fine-grained scaling of font sizes
\renewcommand{\RSpercentTolerance}{0}  % Permit precise scaling

% Times (https://en.wikipedia.org/wiki/Times_New_Roman)
\usepackage[scale = 0.995, shrink = 0.025em, stretch = .15em]{newtxtext}  % Line breaks as similar as possible to Word
\usepackage[scale = 0.995, slantedGreek]{newtxmath}

%% Palatino (https://en.wikipedia.org/wiki/Palatino)
%\usepackage[scale = 0.899, shrink = 0.025em, stretch = .15em]{newpxtext}  % Line breaks as similar as possible to Times
%\usepackage[scale = 0.899, slantedGreek]{newpxmath}

%% Utopia (https://en.wikipedia.org/wiki/Utopia_(typeface))
%\usepackage[scale = 0.9485, shrink = 0.025em, stretch = .15em]{erewhon}  % Line breaks as similar as possible to Times
%\usepackage[erewhon, scale = 0.9485, slantedGreek]{newtxmath}

%% Charter (https://en.wikipedia.org/wiki/Bitstream_Charter)
%\usepackage[scale = 0.9245]{XCharter}  % Line breaks as similar as possible to Times
%\usepackage[xcharter, scale = 0.9245, slantedGreek]{newtxmath}

%% STIX Two (https://en.wikipedia.org/wiki/STIX_Fonts_project#STIX_2.0.0)
%\usepackage[scale = 0.957]{stickstootext}  % Line breaks as similar as possible to Times
%\usepackage[stix2, scale = 0.957, slantedGreek]{newtxmath}

%% Libertinus Serif (https://en.wikipedia.org/wiki/Libertinus)
%\usepackage[serif, ScaleRM = 0.9875]{libertinus}  % Line breaks as similar as possible to Times
%\usepackage{libertinust1math}

%% New Century Schoolbook (https://en.wikipedia.org/wiki/Century_type_family)
%\usepackage[scaled = 0.875]{scholax}  % Line breaks as similar as possible to Times
%\usepackage[scaled = 0.940625, ncf, slantedGreek]{newtxmath}

% Sans-serif and monospaced font
\usepackage[book, lining, scale = 0.85, semibold, tabular]{FiraSans}
\usepackage[lining, scale = 0.85]{FiraMono}  % No genuine italics, but slanted replacement

\usepackage[protrusion = compatibility, expansion = false]{microtype}

\usepackage{bm}  % Improved bold math font support
% Emulate "text math" fonts from the unicode-math package
\newcommand{\mathup}[1]{\mathrm{#1}}
\newcommand{\mathbfit}[1]{\bm{#1}}
\newcommand{\mathbfup}[1]{\mathbf{#1}}
\newcommand{\dd}{\mathrm{d}}
\newcommand*{\divslash}{%
	\mathbin{%
		\nonscript\mskip-2mu / \nonscript\mskip-2mu%
	}%
}  % Emulate the \divslash glyph defined by the unicode-math package

\usepackage[explicit]{titlesec}
\setcounter{secnumdepth}{4}
\titleformat{\section}{\normalfont\Large\bfseries}{\strut\thesection}{1em}{#1\strut}
\titlespacing*{\section}{0pt}{2\baselinedist plus 4ex}{1\baselinedist plus 2ex}
\titleformat{\subsection}{\normalfont\large\bfseries}{\strut\thesubsection}{1em}{#1\strut}
\titlespacing*{\subsection}{0pt}{1\baselinedist plus 2ex}{1\baselinedist plus 2ex}
\titleformat{\subsubsection}{\normalfont\normalsize\bfseries}{\strut\thesubsubsection}{1em}{#1\strut}
\titlespacing*{\subsubsection}{0pt}{0.5\baselinedist plus 1ex}{0.5\baselinedist plus 1ex}
\titleformat{\paragraph}{\normalfont\normalsize\upshape}{\strut\theparagraph}{1em}{#1\strut}
\titlespacing*{\paragraph}{0pt}{0.5\baselinedist plus 1ex}{0pt}
\titleformat{name = \subparagraph, numberless}[runin]{}{}{0pt}{\itshape}[]
\titlespacing{\subparagraph}{\parindent}{0pt}{2\wordsep}[]

\usepackage{soul}
\renewcommand{\caps}[1]{{\textscale{0.97}{\textls[50]{\MakeUppercase{#1}}}}}
	% Redefine \caps so that it makes more sense and also works inside \ul




%%%%%%%%%%%%%%%%%%%%%%%%%%%
%%  FORMATTING OF LISTS  %%
%%%%%%%%%%%%%%%%%%%%%%%%%%%


\usepackage[inline]{enumitem}
% General settings:
\setlist{
	leftmargin = \parindent, listparindent = \parindent, itemsep = 0pt, parsep = 0pt, partopsep = 0pt
}
\setlist[1]{topsep = 0.5\baselinedist plus 1ex}
\setlist[{2, 3, 4}]{topsep = 0pt}
% Type-specific settings
\setlist[enumerate]{leftmargin = \parindent, labelsep = *}
	% Fine as long as the list does not include more than 9 items.
\setlist[enumerate, 1]{label = (\arabic*)}
\setlist[enumerate, 2]{label = \alph*., align = right}
	% Taken from the Chicago Manual of Style (16th ed., Section 6.126)
\setlist[enumerate, 3]{label = \roman*., align = right, widest* = 3, labelsep = 0.3\parindent}
\setlist[itemize]{labelsep = 0.5\parindent}
\setlist[itemize, 2]{label = --}




%%%%%%%%%%%%%%%%%%%%%%%%%%%%%%%%%%%%%%%%
%%  FORMATTING OF FIGURES AND TABLES  %%
%%%%%%%%%%%%%%%%%%%%%%%%%%%%%%%%%%%%%%%%


\usepackage{tabularray}
	% Modern package that comprises the functionality of (probably) all earlier table packages that you can think of: threeparttable, tabularx/tabulary/tabu, longtable, xltabular, multirow, makecell, booktabs, colortbl, ...
\UseTblrLibrary{booktabs, siunitx}
\addtolength{\heavyrulewidth}{0.1pt}%
\addtolength{\lightrulewidth}{-0.05pt}%
\setlength{\cmidrulewidth}{\lightrulewidth}
\addtolength{\aboverulesep}{1pt}%
\addtolength{\belowrulesep}{1pt}%
\addtolength{\belowbottomsep}{\smallskipamount}  % Add vertical space below closing rule of tables
\SetTblrInner[booktabs]{
	hborder{1} = {abovespace = \abovetopsep, belowspace = \belowrulesep},
	hline{1} = {\heavyrulewidth, solid},  % Draw heavy rule at the top of each table
	hborder{2} = {abovespace = \aboverulesep, belowspace = \belowrulesep},
	hline{2} = {\lightrulewidth, solid},    % By default, draw regular rule after column headings
	hborder{Z} = {abovespace = \aboverulesep, belowspace = \belowbottomsep},
	hline{Z} = {\heavyrulewidth, solid},  % Draw heavy rule at the bottom of each table
	column{1} = {leftsep = 0pt},
	column{Z} = {rightsep = 0pt},
	row{1} = {guard},
	width = \textwidth,
}
%\AtEndEnvironment{booktabs}{\vspace{\belowbottomsep}}
\SetTblrStyle{caption}{halign = c, hang = 0pt, indent = 0pt}
\DeclareTblrTemplate{caption-sep}{default}{.~\strut}
\SetTblrStyle{caption-tag}{font = \small\bfseries}
\SetTblrStyle{caption-sep}{font = \small\bfseries}
\SetTblrStyle{caption-text}{font = \small}
% Prints a colon only if \InsertTblrRemarkTag is not empty
\newcommand{\RemarkTagMaybeColon}{%
	\ifdefempty{\InsertTblrRemarkTag}{\hspace{\smalllinespacing}\ignorespaces}{:~\ignorespaces}%
}
\SetTblrStyle{remark}{halign = j, hang = 0pt, indent = 0pt}  % indent sets \parindent
\DeclareTblrTemplate{remark-sep}{default}{\RemarkTagMaybeColon\strut}  % Set to empty to suppress the default colon
\SetTblrStyle{remark-tag}{font = \footnotesize\itshape}
\SetTblrStyle{remark-sep}{font = \footnotesize\itshape}
\SetTblrStyle{remark-text}{font = \footnotesize}

%\usepackage{siunitx}  % Already loaded by tabularray
% Allows, among others, for alignment of decimal numbers in tables at the decimal point.
\sisetup{
	%detect-all,
	%round-integer-to-decimal = true,
	%group-digits             = true,
	%group-minimum-digits     = 5,
	%group-separator          = {\kern 1pt},
	number-unit-product      = \textup{~},
	table-align-text-pre     = false,
	table-align-text-post    = false,
	input-signs              = + -,
	%input-symbols            = {*} {**} {***} \sigstar,
	input-open-uncertainty = ,
	input-close-uncertainty = ,
	retain-explicit-plus,
}

\usepackage[
	font = small,
	justification = centering,
	labelfont = {bf, small},
	labelsep = period,
	margin = 0pt,
	%textfont = it,
]{caption}  % Set figure captions and table headings in font size \small
\captionsetup[figure]{position = below}  % Figure captions go below the figure.
\captionsetup[table]{position = above}  % Table titles go above the table.
\makeatletter
\g@addto@macro\@floatboxreset{\vskip 0.45\baselineskip}  % Move floats down a bit, since we increase \topskip
\g@addto@macro\@floatboxreset{\centering}  % Automatically center all figures and tables
\g@addto@macro\@floatboxreset{\small}  % and make the font size small
\makeatother

% Copied from AEA.cls
\newenvironment{tablenotes}[1][Note]
	{\par\justifying\medskip\begingroup\footnotesize\noindent\strut\textit{#1:} \ignorespaces}
	{\par\endgroup}
\newenvironment{figurenotes}[1][Note]
	{\par\justifying\medskip\begingroup\footnotesize\noindent\strut\textit{#1:} \ignorespaces}
	{\par\endgroup}

% Let even relatively big floats (long tables, spacious figures) be placed on text pages ==>
\renewcommand{\textfraction}{0.05}
\renewcommand{\topfraction}{0.95}
\renewcommand{\bottomfraction}{0.95}
% <== See http://tex.stackexchange.com/questions/39017/how-to-influence-the-position-of-float-environments-like-figure-and-table-in-lat/39020#39020
\setlength{\textfloatsep}{1.5\baselineskip plus 3\baselineskip minus 0.0pt}
% originally, \textfloatsep: 20.0pt plus 2.0pt minus 4.0pt




%%%%%%%%%%%%%%%%%%%%%%%%%%%%%%
%%  FORMATTING OF THEOREMS  %%
%%%%%%%%%%%%%%%%%%%%%%%%%%%%%%


\newtheoremstyle{Plain}% name
	{1\baselinedist plus 2ex}  % Space above: Use \topsep to make the space identical to the one around lists
	{1\baselinedist plus 2ex} % Space below
	{\itshape}  % Body font
	{}  % Indent amount (empty = no indent, \parindent = paragraph indent)
	{\bfseries} % Theorem head font
	{.}  % Punctuation after theorem head
	{\fontdimen2\font plus \fontdimen3\font}  % Space after theorem head: normal interword space
	{\thmname{#1}\thmnumber{\:#2}\thmnote{\bfseries\upshape\ (#3)}}
		% Theorem head spec (changed such that also the ``theorem note'' is printed in boldface)

%\theoremstyle{plain}
\theoremstyle{Plain}
\newtheorem{theorem}{Theorem}
\newtheorem{lemma}[theorem]{Lemma}
\ifthenelse{
	\equal{\manuscriptlanguage}{DE}
}{% German
	\newtheorem{conjecture}[theorem]{Vermutung}
	\newtheorem{corollary}[theorem]{Korollar}
	\newtheorem{proposition}[theorem]{Satz}
}{% English
	\newtheorem{conjecture}[theorem]{Conjecture}
	\newtheorem{corollary}[theorem]{Corollary}
	\newtheorem{proposition}[theorem]{Proposition}
}

\newtheoremstyle{Definition}% name
	{1\baselinedist plus 2ex}  % Space above: Use \topsep to make the space identical to the one around lists
	{1\baselinedist plus 2ex}  % Space below
	{\upshape}  % Body font
	{}  % Indent amount (empty = no indent, \parindent = paragraph indent)
	{\bfseries} % Theorem head font
	{.}  % Punctuation after theorem head
	{\fontdimen2\font plus \fontdimen3\font}  % Space after theorem head: normal interword space
	{\thmname{#1}\thmnumber{\:#2}\thmnote{\bfseries\upshape\ (#3)}}
		% Theorem head spec (changed such that also the ``theorem note'' is printed in boldface)

%\theoremstyle{definition}
\theoremstyle{Definition}
\newtheorem{axiom}{Axiom}
\newtheorem{definition}{Definition}
\newtheorem{problem}{Problem}
\ifthenelse{
	\equal{\manuscriptlanguage}{DE}
}{% German
	\newtheorem{algorithm}{Algorithmus}
	\newtheorem{assertion}{Zusicherung}  % {Aussage}?  % {Bestätigung}?
	\newtheorem{assumption}{Annahme}
	\newtheorem{condition}{Bedingung}
	\newtheorem{criterion}{Kriterium}
	\newtheorem{example}{Beispiel}
	\newtheorem{exercise}{\"Ubung}
	\newtheorem{hypothesis}{Hypothese}
	\newtheorem{question}{Frage}
	\newtheorem{result}{Resultat}
}{% English
	\newtheorem{algorithm}{Algorithm}
	\newtheorem{assertion}{Assertion}
	\newtheorem{assumption}{Assumption}
	\newtheorem{condition}{Condition}
	\newtheorem{criterion}{Criterion}
	\newtheorem{example}{Example}
	\newtheorem{exercise}{Exercise}
	\newtheorem{hypothesis}{Hypothesis}
	\newtheorem{question}{Question}
	\newtheorem{result}{Result}
}

\newtheoremstyle{Remark}% name
	{0.5\baselinedist plus 1ex}  % Space above: Use \topsep to make the space identical to the one around lists
	{0.5\baselinedist plus 1ex}  % Space below
	{\upshape}  % Body font
	{}  % Indent amount (empty = no indent, \parindent = paragraph indent)
	{\itshape} % Theorem head font
	{.}  % Punctuation after theorem head
	{0.5em}  % Space after theorem head: " " = normal interword space; \newline = linebreak
	{\thmname{#1}\thmnumber{\:#2}\thmnote{\itshape\ (#3)}}
		% Theorem head spec (changed such that also the ``theorem note'' is printed in boldface)

%\theoremstyle{remark}
\theoremstyle{Remark}
\ifthenelse{
	\equal{\manuscriptlanguage}{DE}
}{% German
	\newtheorem{acknowledgment}{Kenntnisnahme}
	\newtheorem{case}{Fall}
	\newtheorem{claim}{Behauptung}
	\newtheorem{conclusion}{Schlussfolgerung}
	\newtheorem{notation}{Notation}
	\newtheorem{note}{Hinweis}
	\newtheorem{property}{Eigenschaft}
	\newtheorem{remark}{Anmerkung}
	\newtheorem{summary}{Zusammenfassung}
}{% English
	\newtheorem{acknowledgment}{Acknowledgment}
	\newtheorem{case}{Case}
	\newtheorem{claim}{Claim}
	\newtheorem{conclusion}{Conclusion}
	\newtheorem{notation}{Notation}
	\newtheorem{note}{Note}
	\newtheorem{property}{Property}
	\newtheorem{remark}{Remark}
	\newtheorem{summary}{Summary}
}




%%%%%%%%%%%%%%%%%%%%%%%%%%%%%%%%%%%%%%
%%  FORMATTING OF THE BIBLIOGRAPHY  %%
%%%%%%%%%%%%%%%%%%%%%%%%%%%%%%%%%%%%%%


\usepackage[
	authordate,
	backend = biber,
	backref, backrefstyle = three,
	compresspages = false,
	dashed = false,
	doi = only, isbn = false,
	natbib = true,
	noibid,  % See https://tex.stackexchange.com/questions/129487/bug-in-the-parencite-command-of-the-biblatex-chicago-package
	sortcites = true, sorting = ynt,  % sort the in-text citations by year of publication
	useprefix = true,
]{biblatex-chicago}

\usepackage[babel, german = quotes]{csquotes}  % Needed for correct German quotes via BibLaTeX's \mkbibquote{...}

\let \citeOrig \cite
\let \cite \textcite
\let \citealp \citeOrig

\urlstyle{same}  % Since biblatex-chicago sets \urlstyle{rm}
% Prevent weird line-breaking of URLs, see
% https://tex.stackexchange.com/questions/134191/line-breaks-of-long-urls-in-biblatex-bibliography:
\setcounter{biburllcpenalty}{10000}
\setcounter{biburlucpenalty}{10000}
\setcounter{biburlnumpenalty}{10000}
\AtBeginDocument{\biburlbigskip = 0mu\relax}  % To prevent excessive whitespace in URLs and DOIs

\renewcommand{\bibfont}{\small}
\setlength{\bibitemsep}{0pt}  % No vertical space between bibliography entries

\DeclareFieldFormat[report]{title}{\mkbibquote{#1}}  % Use quotation marks (instead of italic font) for titles of reports, as AER and JPE do
\DeclareFieldFormat[report]{citetitle}{\mkbibquote{#1}}

% Adjust formatting of back-references, Econometrica style
\DefineBibliographyStrings{english}{%
	backrefpage  = {},	% originally ``cit. on p.''
	backrefpages = {}	% originally ``cit. on pp.''
}
\DefineBibliographyStrings{ngerman}{%
	backrefpage  = {},	% originally ``Siehe Seite''
	backrefpages = {}	% originally ``Siehe Seiten''
}
\renewcommand*{\finentrypunct}{}
\renewbibmacro*{pageref}{%
	\addperiod
	\iflistundef{pageref}
	{}
	{\printtext[brackets]{%
			\ifnumgreater{\value{pageref}}{1}
				{\bibstring{backrefpages}}
				{\bibstring{backrefpage}}%
			\printlist[pageref][-\value{listtotal}]{pageref}%
		}%
	}%
}

\bibliography{biblatex-examples-ms.bib}  % From the "biblatex-ms" package




%%%%%%%%%%%%%%%%%%%%%%%%%%%%%%%%%%
%%  FORMATTING OF THE APPENDIX  %%
%%%%%%%%%%%%%%%%%%%%%%%%%%%%%%%%%%


%\usepackage[title, titletoc]{appendix}

\usepackage{chngcntr}  % to innclude section numbers/letters in the figure/table/equation counters

\AtBeginEnvironment{appendix}{%
	\clearpage%
	\small%
	\counterwithin{equation}{section}%
	\counterwithin{figure}{section}%
	\counterwithin{table}{section}%
	\counterwithin{equation}{section}%
	\renewcommand{\thepage}{A-\arabic{page}}%
	\setcounter{page}{1}%
	%\pdfbookmark[0]{\appendixname}{Appendix}%
	\titleformat{\section}{\normalfont\Large\bfseries}{\strut\appendixname~\thesection}{1em}{#1\strut}%
}




%%%%%%%%%%%%%%%%%%%%%%%%%%%%%%%%%%%
%%  SETTINGS FOR CODE LISTINGS   %%
%%%%%%%%%%%%%%%%%%%%%%%%%%%%%%%%%%%


\usepackage{listings}

\colorlet{backgroundcolor}{UBonnGray!5}
\definecolor{stringcolor}{RGB}{191, 0, 0}  % dark red
\definecolor{commentcolor}{RGB}{53, 128, 0}  % green
\definecolor{keywordcolor}{RGB}{0, 0, 153}  % dark blue
\definecolor{digitcolor}{RGB}{0, 0, 204}  % brighter blue
\definecolor{builtincolor}{RGB}{191, 0, 191}  % pink
\definecolor{symbolcolor}{RGB}{102, 102, 102}  % gray

\lstdefinestyle{RStudio-Xcode}{
	backgroundcolor = \color{backgroundcolor},
	basicstyle = \ttfamily\tiny,
	commentstyle = \color{commentcolor},
	extendedchars = true,  % Do not convert UTF-8 characters, since we ue \usepackage[utf8]{inputenc} anyway
	frame = single, % Adds frame around the code
	keywords = [1]{break, else, F, FALSE, for, function, if, in, Inf, library, NA, NA_character_, NA_complex_, NA_integer_, NA_real_, NaN, next, NULL, repeat, require, return, switch, T, TRUE, tryCatch, while},
	keywordstyle = [1]\color{builtincolor},
	literate =
		{“}{{\textquotedblleft}}{1}
		{”}{{\textquotedblright}}{1}
		{€}{{\texteuro}}{1}
		{β}{{\textbeta}}{1}
		{δ}{{\textdelta}}{1}
		{ρ}{{\textrho}}{1}
		{σ}{{\textsigma}}{1}
		{Ä}{{\"A}}1
		{Ö}{{\"O}}1
		{Ü}{{\"U}}1
		{ü}{{\"u}}1
		{ä}{{\"a}}1
		{ö}{{\"o}}1
		{ß}{{\ss}}1
	,  % https://stackoverflow.com/questions/1116266/listings-in-latex-with-utf-8-or-at-least-german-umlauts
	morecomment = [l]{\#},
	morestring = [b]{"},
	morestring = [b]{'},
	numbers = left,  % Position of line numbers
	numbersep = 1.333em,  % Distance between line numbers and code
	numberstyle = \tiny\sffamily\color{UBonnGray},  % Style of the line numbers
	rulecolor = \color{backgroundcolor},
	stringstyle = \color{stringcolor},     % String literal style
}
\lstset{style = RStudio-Xcode}




%%%%%%%%%%%%%%%%%%%%%
%%  MISCELLANEOUS  %%
%%%%%%%%%%%%%%%%%%%%%


% Here, you can add further packages that you would like to use and custom commands

\usepackage[most]{tcolorbox}  % Loads tikz
\tcbset{
	after = {\par\vskip 0.25\baselinedist plus 10ex\noindent\unskip\ignorespacesafterend},
	after upper = \strut,
	arc = 0pt,
	before = {\par\vskip 0.75\baselinedist plus 10ex\noindent\ignorespaces},
	before upper = {%
		\fontsize{\fsnormal}{\smalllinespacing}\selectfont%
		\setlength{\parindent}{\smalllinespacing}%
		\noindent\strut%
	},
	bottom = 1.5ex, left = 2ex, right = 2ex, top = 1.5ex,
	breakable,
	boxrule = 0pt,
	colback = UBonnBlue!5,  % Set background color to 5% blue
	colframe = UBonnBlue!5,
}

% Illustrate the baseline grid:
\usetikzlibrary{calc}
\newcommand{\displaybaselinegrid}{%
	\begin{tikzpicture}[remember picture, overlay, x = 1mm, y = \baselineskip]
		\foreach \i in {-5,...,-1,0,1,2,...,35}{
			\draw[magenta!50] ($(current page.north west) + (0mm, -1in-\voffset-\topmargin-\headheight-\headsep-\topskip) + (0, -\i-2)$) -- ($(current page.north east) + (0mm, -1in-\voffset-\topmargin-\headheight-\headsep-\topskip) + (0, -\i-2)$);
		}
		\draw[DarkOrchid2] ($(current page.north west) + (0mm, -1in-\topmargin-\headheight-\headsep)$) -- ($(current page.north east) + (0, -1in-\topmargin-\headheight-\headsep)$);
		\draw[DarkOrchid2] ($(current page.north west) + (0mm, -1in-\topmargin-\headheight-\headsep-\textheight)$) -- ($(current page.north east) + (0mm, -1in-\topmargin-\headheight-\headsep-\textheight)$);
	\end{tikzpicture}%
}

\usepackage{printlen}
\uselengthunit{PT}
\renewcommand{\unitspace}{~}

\newcommand{\beware}[1]{%
	\marginpar{%
		\RaggedRight\scriptsize\sffamily\color{Red3}\bfseries\sloppy%
		#1%
	}%
}




%%%%%%%%%%%%
%%  BODY  %%
%%%%%%%%%%%%


\begin{document}


\pagenumbering{roman}


\title{\manuscripttitle}
\author{\studentfullname}
\date{\printdate{\manuscriptdate}}

\maketitle


\clearpage


\tableofcontents


\clearpage


\pagenumbering{arabic}


\section{Structure of the Thesis\slash Term Paper}% Note: \slash typesets a slash after which a line break may happen
\label{sec:structure}

\subsection{Parts of the Manuscript}

\newcommand{\PDFcompiler}{%
	\ifthenelse{\boolean{PDFTeX}}{PDFLaTeX: \\[\medskipamount] \mdseries\texttt{\pdftexbanner}}{}%
	\ifthenelse{\boolean{LuaTeX}}{LuaLaTeX: \\[\medskipamount] \mdseries\texttt{\luatexbanner}}{}%
	\ifthenelse{\boolean{XeTeX}}{XeLaTeX, \\[\medskipamount] \mdseries version \texttt{\number\eTeXversion\eTeXrevision-\number\XeTeXversion\XeTeXrevision}}{}%
	\texttt{ (format version: \fmtversion)}
}
\beware{%
	For compiling this document without any changes, a~\emph{complete (and up-to-date)} TeX distribution is required. \\
	Options include
	\href{https://tug.org/mactex/mactex-download.html}{MacTeX},
	\href{https://miktex.org/download}{MiKTeX},
	and \href{https://tug.org/texlive/acquire-netinstall.html}{TeX Live}. \\[\medskipamount]
	\mdseries For MiKTeX, please choose the ``Net Installer'' to install the \emph{complete} MiKTeX distribution, rather than the ``Basic Installer.'' \\[\medskipamount]
	\bfseries Compilation via \url{https://overleaf-students.astro.uni-bonn.de/} is also supported.
	\\[\medskipamount]
	\mdseries This document was compiled on \today\ with \PDFcompiler.%
}%
A~thesis generally consists of the following parts:
\begin{enumerate}
	\item a~title page,
	\item a~table of contents,
	\item optionally, a~list of abbreviations, tables, and/or figures (these are generally unnecessary for theses and more relevant for books),
	\item the main text (body),
	\item a~bibliography,
	\item optionally, an appendix, and
	\item a~written declaration of authorship.
\end{enumerate}
It may make sense to place the bibliography after potential appendices. Please ask your supervisor for their preference. The parts before the main text are called ``front matter.''

\subsection{Structure of the Main Text}

For the main text of theses, two basic outlines are very common:
\begin{itemize}
	\item Three-part structure: This structure divides an academic paper (including theses) into the three parts
		\begin{enumerate}
			\item introduction,
			\item main text, and
			\item discussion/concluding remarks.
		\end{enumerate}
	\item Four-part structure/``IMRaD''\footnote{For detailed information, see \url{https://en.wikipedia.org/wiki/IMRAD} and \url{https://skillsearthsciences.sites.uu.nl/writing/structure/imrad/}.}: This structure divides an academic paper (including theses) into the four parts
		\begin{enumerate}
			\item introduction,
			\item methods,
			\item results, and
			\item discussion (conclusion).
		\end{enumerate}
\end{itemize}

The IMRaD structure is very common in the natural sciences and life sciences. It is also generally suited for manuscripts in economics. Economists, however, usually do not call the methods section ``Methods'' but give it a~different name or even split it up into a~sequence of sections, such as ``Theoretical Model''\slash ``Theoretical Predictions,'' followed by ``Empirical Strategy'' and ``Data''\slash ``Dataset'' (or the other way round) or by ``Experimental Design'' in the case that the paper features an experiment. Similarly, economists frequently split the results section into a~sequence of sections, such as ``Treatment Effect,'' followed by ``Analysis of Heterogeneity,'' ``Evidence on Mechanisms'' and\slash or ``Robustness Checks.'' And some theoretical papers have the structure ``Model,'' ``Results,'' ``Applications.'' None, however, do away with the ``Introduction'' and ``Discussion''\slash ``Conclusion'' section. It is also not uncommon---but by no means the standard---in economics to have a~dedicated ``Related Literature'' section after the intro\-duction.

Other structures are possible, of course. Please consult with your advisor to find a~good structure for your manuscript.

\subsection{Structuring Introduction and Discussion: Funnel and Reverse Funnel}

\begingroup

\SetTblrInner[tblr]{
	colspec = {X[2.25] X[l]},
	column{1} = {leftsep = 0pt},
	column{Z} = {rightsep = 0pt},
	column{2} = {font = \sffamily\footnotesize\setlength{\baselineskip}{2.75ex}\RaggedRight},
	hborder{1-Z} = {abovespace = 0pt, belowspace = 0pt},
	hline{1, Z} = {0pt},
	vborder{2} = {rightspace+ = 0.75em},
	width = \linewidth,
}

A~general structure that can be found in many research papers is the so-called X-shaped or ``hourglass'' structure.\footnote{See \url{https://skillsearthsciences.sites.uu.nl/writing/structure/\#attachment_2387}, Fig.\,1, and \url{https://pmc.ncbi.nlm.nih.gov/articles/PMC5405644/pdf/JHRS-10-3.pdf}, Figure~1.} This structure is brought about when the so-called ``funnel'' technique is used to write the introduction,\footnote{See  \url{https://read.aupress.ca/read/read-think-write/section/9e945e9b-62cf-4dbb-b8cd-823549ede292\#fig1401}, Figure~14.1, and \url{https://kib.ki.se/en/write-cite/academic-writing/structure-academic-texts}} and when a~``reverse funnel'' or ``upside-down funnel'' is used to write the discussion\slash concluding section:\footnote{See \url{https://read.aupress.ca/read/read-think-write/section/9e945e9b-62cf-4dbb-b8cd-823549ede292\#fig1402}, Figure~14.2, and \url{https://kib.ki.se/en/write-cite/academic-writing/writing-conclusion}.}
\begin{enumerate}
	\item The paper starts with a~\emph{broad} perspective at the beginning of the introduction, which becomes \emph{narrower} with every sentence, leading to the paper's narrow research question.
	\item The \emph{narrowest} part is when the exact research question is posed, the research methodology is described in detail, and the results are derived\slash presented.
	\item When the results are discussed and related to the literature in the concluding section, the perspective becomes \emph{broad} again.
\end{enumerate}

Let us take a~look at the ``Introduction'' section from \citet{sigfridsson} to see the funneling approach at work:%
\footnote{See \url{https://lucris.lub.lu.se/ws/portalfiles/portal/135489685/25_charges.pdf}.}
\begin{tcolorbox}
	\begin{tblr}{}
		Molecular simulation methods have become an important technique in many areas of chemistry through the recent advent of effective and wide-spread software for molecular mechanics, molecular dynamics, and Monte Carlo simulations, and procedures for the estimation of free energy differences. In all such methods, a~proper description of the electrostatic interactions within the simulated system is of key importance, \textellipsis\ However, in most implementations, \textellipsis, the simplest possible solution is used, \textellipsis\ Thus, most classical simulation methods need a~point-charge parameterisation of the molecules of interest.
		&
		{%
			\textit{Paragraph~1} \\
			\textmd{Broad introduction with general description of the research field (``many areas of
			chemistry''), research topic (``molecular simulation methods''), and recent development (``through the recent advent of \textellipsis''). Followed by a~more detailed description of crucial aspects in this context (``In all such methods, a~proper description \textellipsis\ is of key importance'') and of shortcomings of existing methods.}
		}
		\\
		\hspace{\smalllinespacing}Unfortunately, atomic charges are not observables, i.e. they cannot unambiguously be determined by experiments or quantum chemical calculations. Therefore, a~large number of methods have been suggested for the estimation of point charges [1]. Several groups have tried to derive the charges directly from experimental quantities, \textellipsis\ Yet, relevant data is usually missing or too scarce to allow a determination of all charges in interesting molecules. Instead, most techniques derive atomic charges from quantum chemical calculations.
		&
		{%
			\textit{Paragraph~2} \\
			Narrowing down the topic: Description of the state of the art and of issues and limitations observed in the literature.%
		}
		\\
		{%
			\hspace{\smalllinespacing}The simplest way to determine quantum chemical charges is the Mulliken population
			analysis. \textellipsis\ Although, Mulliken charges are known to strongly depend on the basis set [6] and to reproduce electrostatic moments poorly [1], they are still widely used due to their simplicity. Several other \textellipsis\ methods have been devised to cure the problems of the Mulliken charges [1], \textellipsis
			\\
			\null\hspace{\smalllinespacing}\textellipsis
		}
		&
		{%
			\textit{Paragraph 3} \\
			Further description of the state of the art: strengths and weaknesses of existing approaches. \\
			\strut \\[2ex]
			\textit{Paragraphs 4--6} \\
			Further description of unresolved issues.
		}
		\\
		\hspace{\smalllinespacing}In this paper, we make a critical analysis of the four most popular potential-based point-charge methods, \textellipsis\ It is shown that these methods may give widely different results and possible explanations for this are discussed. Two alternative methods for deriving atomic charges are suggested, which avoid the arbitrariness in the selection of potential points, and their performance is judged using a~number of different quality criteria.
		&
		{%
			\textit{Paragraph~7} \\
			Narrow description of the specific contribution of this paper: Identification of undesirable arbitrary variation in simulation results of existing methods, which inspired development of an ``alternative methods'' that avoids this issue.
		}
	\end{tblr}
\end{tcolorbox}

Let us now inspect the ``Concluding Remarks'' subsection from \citet{sigfridsson} to see the ``reverse funneling'' approach at work:
\begin{tcolorbox}
	\begin{tblr}{}
		Potential-based charges strongly depend on the way the potential points are selected. We have presented and tested a~new method that avoids such a~dependence, \caps{CHELMO}. It fits the charges directly to the electrostatic moments and it performs as well, or better, as the best electrostatic potential method judging from the calculated electrostatic potential and moments. \textellipsis\ The major disadvantage of the method is that it cannot be used for large molecules, since there is a~limited number of moments. No more than 25~independent charges in a~molecule can be determined if all moments up to hexadecapole moments are used. In practice it should not be used for molecules with more than about 20 atoms since otherwise the higher moments may be poorly reproduced \textellipsis\ This could be remedied if moments higher than hexadecapole moments are calculated but such moments are not included in the output of normal quantum chemical packages.
		&
		{%
			\textit{Paragraph~1} \\
			Narrow recap of the first contribution of the paper: brief description of the issue and how it was solved by the new method. Followed by description of the limitations of the new method. Outlook how these limitations could be overcome.
		}
		\\
		\hspace{\smalllinespacing}In order to get a~method that can be used for larger system, we constructed the \caps{CHELP-BOW} method, which estimates the charges from a Boltzmann-weighted fit to the electrostatic potential. It has the advantage over the other electrostatic potential methods that \textellipsis\ The present implementation of the \caps{CHELP-BOW} method is very simplified, but in a future publication we will refine and thoroughly test the method. However, the results are conclusive enough to show that the method is very promising and has the desired behaviour. \textellipsis\ Clearly, this is the method we recommend for general use.
		&
		{%
			\textit{Paragraph~2} \\
			Narrow recap of the central contribution of the paper and how it relates to the literature: how existing methods were further improved upon by the second method proposed in this paper. Followed by outlook on future advances and recommendation for researchers in the field.
		}
		\\
		\hspace{\smalllinespacing}An important advantage of the methods developed in this paper is that they are general, i.e. they can be used with any quantum chemical program and they can use experimental data (\textellipsis\!) as well. The only thing that has to be changed is the input section of the programs. Thus any quantum chemical basis sets can be used, and any level of theory that gives a~wave function may be employed. Furthermore, the methods can easily be adapted to \textellipsis
		&
		{%
			\textit{Paragraph~3} \\
			Broader implications due to the generality of the new method.
		}
		\\
		\hspace{\smalllinespacing}Finally, a comment on the traditional electrostatic potential methods. \textellipsis\ Thus, we cannot see any justification to use \caps{CHELP} charges except for backward comparisons.
		&
		{%
			\textit{Paragraph~4} \\
			Other broader implications.
		}
	\end{tblr}
\end{tcolorbox}

\endgroup

\subsection{Some Questions That May Help You Guide Writing Your Thesis}

You may find the following list of questions helpful as a~guide to making sure that your manu\-script covers all important aspects. The questions are primarily intended for doctoral and advanced researchers, but they may also be helpful to students writing term papers and theses.

Your introduction might answer the following questions (not necessarily in this order, but this order will give rise to a~``funnel'' structure):
\begin{enumerate}[labelindent = \parindent, leftmargin = 2\parindent]
	\item What do we already know?
	\item What do we not know yet (the ``knowledge gap'')?
	\item Why is it interesting\slash relevant to close the gap?
	\item \emph{What is the research question? [R]}
	\item Why is it interesting to answer the research question?
	\item How do we contribute to closing the knowledge gap: How do we answer the research question? (Which method do we use?)
	\item Why is the chosen method (more) suitable (than alternative methods) for answering the research question?
\end{enumerate}

And your conclusion/discussion might answer the following questions (again, not necessarily in this order, but this order will give rise to a~``reverse funnel'' structure):
\begin{enumerate}[resume, labelindent = \parindent, leftmargin = 2\parindent]
	\item What was the knowledge gap before the current study? [Similar to the introduction.]
	\item \emph{How did we contribute to closing that gap, and what did we find? [A]} (Which method\slash dataset did we use, and what are the results?---Recap and take-home message.)
	\item How do our results relate to (confirm\slash contradict\slash complement\slash extend) the results from previous and other contemporaneous studies?
	\item What part of the gap is still open? (Phrased a~bit more negatively: What are the limitations of our approach?)
	\item Next steps\slash avenues for future research: How could we go about closing the remaining knowledge gap (by removing the limitations of our current approach)?\footnote{John Cochrane might do away with question~12. See his ``Writing Tips for Ph.D. Students,'' \url{https://www.fma.org/assets/docs/membercontent/writing_cochrane.pdf}.}
\end{enumerate}

In her book \textit{The Little Book of Research Writing},\footnote{For a quick summary of the book, see \url{https://mauve-porcupine-8992.squarespace.com/s/Chaubey_Research_Writing-bxdd.pdf}. See also \url{https://www.econscribe.org/about} and \url{https://arxiv.org/pdf/2012.07787}.} Varanya Chaubey proposes a~simpler list of only three items---the ``RAP method'':
\begin{itemize}
	\item The ``R'' stands for ``research question.''
	\item The ``A'' stands for ``answer'' (to the research question, of course).
	\item The ``P'' stands for ``positioning statement.'' (How does our manuscript relate to the literature: Does it corroborate or extend or contradict others' findings?)
\end{itemize}

Relating the ``RAP'' approach to the the list of questions above, question~4 is the ``R,'' the answer to question~9 is the ``A,'' and the remaining questions spell out the ``P.''

\subsection{Structuring Paragraphs in Academic Manuscripts}

\begingroup

\setul{0.3ex}{0.5ex}

\SetTblrInner[tblr]{
	colspec = {X[2.15] X[l]},
	column{1} = {leftsep = 0pt},
	column{Z} = {rightsep = 0pt},
	column{2} = {font = \sffamily\bfseries\footnotesize\setlength{\baselineskip}{2.75ex}},
	hborder{1-Z} = {abovespace = 0pt, belowspace = 0pt},
	hline{1, Z} = {0pt},
	vborder{2} = {rightspace+ = 0.75em},
	width = \linewidth,
}

Obviously, there is no single ``right'' way of constructing paragraphs. However, a~few basic rules can help you write paragraphs that are \emph{easy to comprehend} by your readers. A~common approach is to compose sentences such that they reflect the following functional sequence:\footnote{See \url{https://libguides.newcastle.edu.au/writing-paragraphs/structure}, \url{https://www.une.edu.au/library/students/academic-writing/write-paragraphs/paragraphs/Academic-paragraphs_v2.pdf}, \url{https://learningessentials.auckland.ac.nz/writing-effectively/paragraph/}}
\begin{enumerate}
	\item \emph{topic sentence (\caps{TS})}---which, if necessary, connects to the previous paragraph;
	\item \emph{supporting sentences (\caps{SS})}---which elaborate on the topic sentence; and
	\item \emph{concluding sentence or connecting sentence (\caps{CS})}---which refers back to the topic sentence and\slash or provides a~link to the subsequent paragraph.
\end{enumerate}

There are various ways in which the supporting sentences can be filled with content. The University of Sheffield suggests\footnote{See \url{https://sheffield.ac.uk/study-skills/writing/academic/paragraph}.} that the supporting sentences could consist of
\begin{enumerate}
	\item ``explanation or definitions (optional),''
	\item ``evidence and examples,'' and
	\item ``comment'' (an interpretation and evaluation of the evidence by you).
\end{enumerate}
The University of Hull and Newcastle University (\caps{UK}) consider the mnemonic ``\caps{PEEL}'' helpful, although they define its meaning in slightly different ways: ``Point, Evidence, Explanation, Link''\footnote{See \url{https://libguides.hull.ac.uk/writing/paras}.} versus ``Point, Evidence, Evaluation, Link''\footnote{See \url{https://www.ncl.ac.uk/academic-skills-kit/writing/academic-writing/paragraphing/}}. One may view ``Explanation''---``why the point is important and how it helps with your overall argument''---as narrower than ``Evaluation''---which is a~general critical reflection on the informativeness of the evidence and can include an explanation of its importance. ``\caps{PEEL}'' relates to \caps{TS}\slash \caps{SS}\slash \caps{CS} as follows:
\begin{enumerate}
	\item ``P,'' the Point, corresponds to the topic sentence.
	\item ``E,'' Evidence, is the first part of the supporting sentences.
	\item ``E,'' Evaluation/Explanation, forms the second part of the supporting sentences.
	\item ``L,'' Link, refers to the connecting sentence that links to the next paragraph.
\end{enumerate}

The Wilfrid Laurier University (Canada) suggests a~different mnemonic: ``\caps{MEAL}.''\footnote{See \url{https://students.wlu.ca/academics/support-and-advising/student-success/assets/resources/writing/how-to-structure-an-academic-paragraph.html}.}
\begin{enumerate}
	\item ``M: Main Point Sentence'' corresponds to the topic sentence.
	\item ``E: Evidence'' is the first part of the supporting sentences.
	\item ``A: Analysis/Synthesis'' forms the second part of the supporting sentences.
	\item ``L: Linking-Back Sentence'' refers to the concluding sentence that summarizes the paragraph and links back to the topic sentence or even the topic of the entire paper.
\end{enumerate}

The first three components are largely identical according to these approaches. A~difference exists in the meaning of ``L'': While ``\caps{PEEL}'' stresses the link to the \emph{next} step in the chain of thought, ``\caps{MEAL}'' stresses underscoring the message of the \emph{current} paragraph. That is, the latter advocates \emph{closure}: summarizing the paragraph and linking \emph{back} to the topic sentence.

Yet slightly different is the approach put forth by Academic Writing \caps{UK}:\footnote{See \url{https://academic-englishuk.com/paragraphing/}.}
\begin{enumerate}
	\item Topic sentence---key topic in this paragraph.
	\item Development---the main idea\slash topic discussed in more detail.
	\item Example---support\slash evidence\slash data\slash statistics that show your development is valid\slash credible.
	\item Summary---overall main point summarized\slash evaluated.
\end{enumerate}

It is perfectly fine to mix the different approaches. For some paragraphs, a~closer explanation of the topic sentence (``development'') may be necessary, while it is unnecessary for others. And in some paragraphs, you may want underscore the central message by reiterating it in the final sentence (concluding sentence), while for other paragraphs, you will use the final sentence to set the stage for the next step in your chain of reasoning (connecting sentence).

One thing that \emph{all} these suggestions have in common is, however, that they start with a~\emph{topic sentence} and end with a~\emph{concluding\slash connecting sentence}. Hence, no matter how you fill the supporting sentences, it is probably best to start a~paragraph with a~topic sentence and end with a~concluding\slash connecting sentence.

Let us illustrate the ``\caps{MEAL}''\slash``\caps{PEEL}'' structure, with an optional explanation\slash def\-i\-ni\-tion included, through some example sentences from \citet{sigfridsson}:
\begin{tcolorbox}
	\microtypesetup{protrusion = false}%
	\begin{tblr}{}
		\textellipsis\ {\setulcolor{DarkOrchid3!67}\ul{Thus, most classical simulation methods need a~point-charge parameterisation of the molecules of interest.}}
		&
		\textcolor{DarkOrchid3}{L: Concluding sentence}
		\\
		\hspace{\smalllinespacing}{\setulcolor{Purple4!67}\ul{Unfortunately, atomic charges are not observables,}}
		{\setulcolor{RoyalBlue4!67}\ul{i.e. they cannot unambiguously be determined by experiments or quantum chemical calculations.}}
		{\setulcolor{DeepPink4!67}\ul{Therefore, a~large number of methods have been suggested for the estimation of point charges [1].}}
		{\setulcolor{Teal!67}\ul{Several groups have tried to derive the charges directly from experimental quantities, e.g. dipole moments, electrostatic potentials, or free energy differences [2--4]. Yet, relevant data is usually missing or too scarce to allow a determination of all charges in interesting molecules.}}
		{\setulcolor{DarkOrchid3!67}\ul{Instead, most techniques derive atomic charges from quantum chemical calculations.}}%
		&
		{%
			\textcolor{Purple4}{M/P: Topic sentence \\ \textmd{(reference to previous paragraph)}} \\ \strut \\
			\textcolor{RoyalBlue4}{Optional explanation/definition} \\ \strut \\
			Supporting sentences: \\
			\textcolor{DeepPink4}{E: Evidence} \\
			\textcolor{Teal}{A/E: Analysis/Evaluation} \\ \strut \\
			\textcolor{DarkOrchid3}{L: Connecting sentence \\ \textmd{(link to next paragraph)}}%
		}
		\\
		\hspace{\smalllinespacing}{\setulcolor{Purple4!67}\ul{The simplest way to determine quantum chemical charges is the Mulliken population analysis.}} \textellipsis
		&
		\textcolor{Purple4}{M/P: Topic sentence \\ \textmd{(reference to previous paragraph)}}
	\end{tblr}
\end{tcolorbox}

The rule to finish a~paragraph with a~concluding sentence entails, in particular, that a~theoretical or empirical finding should never be the last thing that you mention in a~paragraph. An ``evidence'' sentence should always be followed by an interpretation in which you tell the reader what we learn from that finding. Here is an example from \citet{sigfridsson}:
\begin{tcolorbox}
	\microtypesetup{protrusion = false}%
	\begin{tblr}{}
		\textellipsis\ {\setulcolor{DeepPink4!67}\ul{With these radii, the charges almost coincide with those obtained by the {\caps{CHELPG}} method (within ${0.02\,e}$).}}
		{\setulcolor{DarkOrchid3!67}\ul{Thus, we can conclude that the main difference between the Merz--Kollman and the {\caps{CHELPG}} method is the 1.4~times larger effective radii of the former method, whereas the sampling schemes are almost equivalent.}}
		&
		{%
			\textcolor{DeepPink4}{Evidence: \\ Supporting sentence with \\ empirical finding} \\ \strut \\
			\textcolor{DarkOrchid3}{Concluding sentence with \\ interpretation of that finding}%
		}
	\end{tblr}
\end{tcolorbox}

To reiterate, there is no single ``right'' way of constructing paragraphs. However, keeping the \caps{TS}/\caps{SS}/\caps{CS} structure in mind as a~guideline is bound to help you produce a~text that enables readers to follow your chain of reasoning.

\endgroup

\subsection{Additional Online Resources on Effective Writing in Economics}

Additional useful recommendations and tips can be found in Plamen Nikolov's ``Writing Tips for Crafting Effective Economics Research Papers -- 2023-2024 Edition'' (\url{https://docs.iza.org/dp16276.pdf}) and in John~H. Cochrane's ``Writing Tips for Ph.D. Students'' (\url{https://www.fma.org/assets/docs/membercontent/writing_cochrane.pdf}).


\section{Layout and Design of the Thesis}
\label{sec:layout}

\subsection{Introductory Remark}
The following recommendations are based on the official guidelines%
\footnote{See \url{https://www.econ.uni-bonn.de/examinations/de/informationen/bachelor/bachelorarbeit/dokumente/ba-merkblatt-2016-05-23.pdf} and \url{https://www.econ.uni-bonn.de/examinations/en/information/master-economics/master-thesis/documents/ma-master-thesis-style-guide-2014-06-10.pdf}.}
of the examination office of the Department of Economics at the University of Bonn.

\subsection{Typeface and Font Sizes}
This template uses the Times typeface by default for the body text. Alternatively, you can choose a~typeface of equal size. Equal size here means that the same amount of text (or less) fits on a~page with the layout---the page size, margins, and line spacing---defined below.

It is probably a~good choice to use a~serif font. This is because papers and theses in economics often feature mathematical formulas and equations. With serif fonts, characters can usually be disambiguated relatively well, and often better than with sans-serif fonts (compare: Al/AI, {\relscale{0.90}\fontfamily{phv}\selectfont Al/AI}). Here are some examples of glyphs that are easy to confuse in particular fonts:
\begin{tcolorbox}
	Digit one, lowercase l, uppercase I, vertical bar: \\
	\small%
	{\fontfamily{ntxtlf}\selectfont Times: 1\,l\,I\,|} \quad
	{\relscale{0.88}\fontfamily{plxSerif-TLF}\selectfont IBM Plex Serif: 1\,l\,I\,|} \quad
	{\relscale{0.85}\fontfamily{pag}\selectfont Avant Garde: 1\,l\,I\,|} \quad
	{\relscale{0.90}\fontfamily{phv}\selectfont Helvetica: 1\,l\,I\,|} \quad {\relscale{0.87}\fontfamily{plxSans-TLF}\selectfont IBM Plex Sans: 1\,l\,I\,|}
	\par \smallskip
	\normalsize%
	Lowercase o, digit zero, uppercase O: \\
	\small%
	{\fontfamily{ntxtlf}\selectfont Times: o\,0\,O}\quad
	{\relscale{0.89}\fontfamily{Cabin-TLF}\selectfont Cabin: o\,0\,O}\quad
	{\relscale{0.92}\fontfamily{AlegreyaSans-OsF}\selectfont Alegreya Sans: o\,0\,O}\quad
	{\relscale{0.82}\fontfamily{plxMono-TLF}\selectfont IBM Plex Mono: o\,0\,O}\quad
	{\relscale{0.82}\fontfamily{CascadiaCode-TLF}\fontseries{sl}\selectfont Cascadia Code: o\,0\,O}
	\par \smallskip
	\normalsize%
	Italic Latin a, p, u, v, y, Y vs. italic Greek alpha, rho, upsilon, nu, gamma, Upsilon: \\
	\small%
	Times: $a \, \alpha$, $p \, \rho$, $u \, \upsilon$, $v \, \nu$, $y \, \gamma$, $Y \, \mathit{\Upsilon}$
	\quad
	{\relscale{0.975}\fontfamily{FiraSans-TLF}\selectfont Fira Sans: {\itshape a\,\textalpha}, {\itshape p\,\textrho}, {\itshape u\,\textupsilon}, {\itshape v\,\textnu}, {\itshape y\,\textgamma}, {\itshape Y\,\textUpsilon}}
\end{tcolorbox}

When using Times or Times New Roman as the typeface, the following font sizes apply:
\begin{itemize}
	\item The font size is \printlength{\fsnormal} for body text.
	\item The font sizes, font weights, and font shapes for headings are as follows:
		\begin{itemize}
			\item Level-1 headings (\verb|\section|): \printlength{\fsLarge}, boldface.
			\item Level-2 headings (\verb|\subsection|): \printlength{\fslarge}, boldface.
			\item Level-3 headings (\verb|\subsubsection|): \printlength{\fsnormal} (like body text), italic.
			\item Level-4 headings (\verb|\paragraph|): \printlength{\fsnormal}, upright (like body text). (Level-4 headings are usually unnecessary in documents that have the scope of theses.)
		\end{itemize}
	\item The font size is \printlength{\fssmall} for tables (not including the table of contents), table titles, figure captions, list of references, and appendix.
	\item The font size is \printlength{\fsfootnote} for footnotes, table notes, and figure notes.
\end{itemize}

If you use a different typeface, adjust the font sizes such that the same amount of text fits on a page as if you had used Times New Roman with the font sizes mentioned above.

\subsection{Page Size and Page Margins}

The examination office's guidelines require that bachelor's and master's theses be printed on A4~paper (29.7~cm width, 21.0~cm height).

\makeatletter
\ifdim \f@size pt < 11.5pt
	The following margins are required when using a~font size of \printlength{\fsnormal} to give about the same amount of text per page as it is produced when following the guidelines to the letter:
	\begin{itemize}
		\item The sum of the left and right margin has to equal 6.5~cm (default: 2.5~cm left margin, 4.0~cm right margin; the latter is relatively wide in order to provide space for annotations by reviewers). Consequently, the text width amounts to 14.5~cm.
		\item The sum of the top and bottom margin has to equal 5.6 cm (default: 2.8~cm top margin, 2.8~cm bottom margin). Consequently, the text height amounts to 24.1~cm.
	\end{itemize}
\else
	The guidelines specify the following margins when using a~font size of \printlength{\fsnormal}:
	\begin{itemize}
		\item The sum of the left and right margin has to equal 5~cm (default: 3~cm left margin, 2~cm right margin). Consequently, the text width amounts to 16~cm.
		\item The sum of the top and bottom margin has to equal 4 cm (default: 2~cm top margin, 2~cm bottom margin). Consequently, the text height amounts to 25.7~cm.
	\end{itemize}
\fi
\makeatother

\subsection{Line Spacing}

\begin{itemize}
	\item Line spacing is \printlength{\baselinedist} for the body text. The reason for the generous---but not particularly aesthetic---line spacing is that many reviewers like to place annotations in-between the lines of text. Apart from that, manuscripts in economics often include mathematical expressions with fractions, subscripts, superscripts, sums, integrals, etc. which all require generous line spacing.
	\item This results in 35~lines of text per page
		\makeatletter%
		\ifdim \f@size pt < 11.5pt%
			(${35 \times 19.5\textup{~pt}} = 682.5\textup{~pt} = 24.08\textup{~cm}$)%
		\else%
			(${35 \times 20.7\textup{~pt}} = 724.5\textup{~pt} = 25.56\textup{~cm}$)%
		\fi%
		\makeatother%
		, as ``one-and-a-half line spacing'' would produce in Microsoft Word.
	\item There is no additional vertical space between paragraphs. Paragraphs that do not follow a~heading should have a~first-line indent (that corresponds to the line height of \printlength{\baselinedist}).
	\item The line spacing for headings is also \printlength{\baselinedist}, in combination with vertical space above and below as follows:
		\newlength{\baselinedistDouble}
		\setlength{\baselinedistDouble}{2\baselinedist}
		\newlength{\baselinedistHalf}
		\setlength{\baselinedistHalf}{0.5\baselinedist}
		\begin{itemize}
			\item Level-1 headings (\verb|\section|) are spaced \printlength{\baselinedistDouble} above and \printlength{\baselinedist} below.
			\item Level-2 headings (\verb|\subsection|) are spaced \printlength{\baselinedist} above and \printlength{\baselinedist} below.
			\item Level-3 headings (\verb|\subsubsection|) are spaced \makeatletter\SI[round-mode = places, round-precision = 2]{\strip@pt\baselinedistHalf}{pt}\makeatother\ above and \makeatletter\SI[round-mode = places, round-precision = 2]{\strip@pt\baselinedistHalf}{pt}\makeatother\ below.
			\item Level-4 headings (\verb|\paragraph|) are spaced \makeatletter\SI[round-mode = places, round-precision = 2]{\strip@pt\baselinedistHalf}{pt}\makeatother\ above and 0~pt below. (Level-4 headings are usually unnecessary in documents that have the scope of theses.)
		\end{itemize}
	\item The line spacing is \printlength{\smalllinespacing} for tables (including table of contents), table titles, figure captions, list of references, and appendix.
	\item The line spacing is \printlength{\footnotelinespacing} for footnotes, table notes, and figure notes.
\end{itemize}

\subsection{Page Numbering}

Page numbering starts with the first page of the main text. The main text is numbered using Arabic numerals. (After the main text, page numbering can be continued with Arabic numerals. For appendices, a~prefix like ``A'' can be added to the page numbers.)

For the front matter, page numbers can be omitted, or Roman numerals can be used. In this case, the (Roman) numbering starts with the cover page. (However, no page number is printed on the cover page itself, as it is evident that it is the first page.)


\section{Citing Other Authors and Sources}
\label{sec:citations}

\subsection{In-Text Citations}

\subsubsection{General Rules}

Both literal citations from other texts as well as the paraphrasing of other authors' ideas must be identified as such. The cited author(s) is (are) indicated right before or after the citation, including the year of the manuscript that you are citing/paraphrasing.

\subsubsection{Paraphrasing Other Authors' Thoughts}

Here is an example of a~paraphrasing in-text citation:
\begin{tcolorbox}
	\cite{sarfraz} suggest an algorithm to automatically capture the outlines of fonts.
\end{tcolorbox}

The author's name/authors' names can be embedded in the sentence, as in the example above, or surrounded by parentheses:
\begin{tcolorbox}
	Many researchers in computer science have worked on the problem of converting the outlines of objects to mathematically describable curves, such as B\'{e}zier curves (see, e.g., \citealp{sarfraz}).
\end{tcolorbox}
Paraphrasing citations of this type usually start with ``see.''

\subsubsection{Literal Citations}

Literal citations must be indicated via quotation marks. Use double quotation marks for this purpose. For quotes within direct quotes, use single quotation marks. For literal citations, the page number must also be provided, if possible. Here is an example of a~literal citation:
\begin{tcolorbox}
	\cite[795]{sarfraz} state that ``another major difference lies in the curve model for the description of the design curve. The outline capturing technique, instead of traditional B\'{e}zier cubics, is based upon a~generalized Hermite cubic.'' They conclude (p.\,796): ``Accordingly, the design curve will be $GC^1$ continuous.''
\end{tcolorbox}

Omissions must be indicated by ellipses (``\textellipsis\!''), as in the following example:
\begin{tcolorbox}
	\cite[795]{sarfraz} state: ``Another major difference lies in the curve model for the description of the design curve. The outline capturing technique, \textellipsis, is based upon a~generalized Hermite cubic.''
\end{tcolorbox}

\pagebreak
Changes must be placed in square brackets, as in the following example:
\begin{tcolorbox}
	\cite[377]{sigfridsson} find that the ``\caps{CHELMO} [Charges from Electrostatic Moments] method gives the best multipole moments for small and medium-sized polar systems.''
\end{tcolorbox}

Longer quotations---as a~rule of thumb, anything that spans more than two lines of text---should be formatted as block quotes:
\begin{tcolorbox}[bottom = -0.75\baselinedist, breakable = false]
	\cite[377]{sigfridsson} summarize their findings as follows:
	\begin{quote}\vspace{-\smallskipamount}%
		The \caps{CHELMO} method gives the best multipole moments for small and medium-sized polar systems, whereas the \caps{CHELP-BOW} charges reproduce best the total interaction energy in actual simulations. Among the standard methods, the Merz--Kollman charges give the best moments and potentials, but they show an appreciable dependence on the orientation of the molecule.
	\end{quote}
\end{tcolorbox}
Even longer quotations---which span more than one paragraph---are typeset as follows: \begin{tcolorbox}[bottom = -0.75\baselinedist, breakable = false]
	\cite[801]{sarfraz} describe the results of their approach as follows:
	\begin{quotation}\vspace{-\smallskipamount}%
		The scheme and the algorithm has been implemented and tested for various shapes. Visually quite elegant results have been observed. \par
		The algorithm has been tested for four fonts in Figs. 4(a), 5(a), 6(a) and (7a). The fonts in Figs. 4(a) and 5(a) are Arabic, whereas the images in Figs. 6(a) and 7(a) are Greek and Kanji characters respectively. \par
		Fig.~4(b) is the outline of Fig.~4(a) using modified Avrahami and Pratt algorithm~[24]. Fig. 4(c) displays the initial characteristic points in Fig. 4(b) using the method of Davis~[23]. Figs. 4(d) and (e) demonstrate the initial characteristic points as well as the intermediate characteristic points which have been obtained after minimizing the errors between the original outline and the computed outline.
	\end{quotation}
\end{tcolorbox}

\subsection{List of References}

All works referenced in the manuscript---and only those---must be included in the list of references. It is either placed right after the main text, before the appendix, or at the very end, following all appendices. The publications should be arranged by author (or editor) in alphabetical order. Some examples of entries in the reference list can be found at the end of this guide. If you refer to Internet sources, the complete web address and the date on which you accessed it should be mentioned in the reference list.

I~recommend generating citations following the guidelines of the \textit{Chicago Manual of Style}. The reasons for this recommendation are as follows:
\begin{itemize}
	\item The citation styles of several economics journals (such as the \href{https://www.aeaweb.org/journals/aer}{\textit{American Economic Review}} and the \href{https://www.journals.uchicago.edu/journal/jpe}{\textit{Journal of Political Economy}}) are based on the \textit{\caps{CMOS}} citation style.
	\item The \textit{\caps{CMOS}} citation style is very well documented online.\footnote{\url{https://www.chicagomanualofstyle.org/tools_citationguide/citation-guide-2.html}.}
	\item The \textit{\caps{CMOS}} citation style is implemented in both Zotero and BibLaTeX.\footnote{See \url{https://www.zotero.org/styles?q=chicago} and \url{https://ctan.org/pkg/biblatex-chicago}, respectively.}
\end{itemize}

The \textit{\caps{CMOS}} style posits that titles of books and journals be italicized, while titles of journal articles and newspaper articles be placed in double quotation marks.

\subsection{Additional Recommendations Regarding Citations and Formatting}

\subsubsection{Citations Embedded in a~Sentence}

The settings in this document output citations as they would be generated by the popular \texttt{natbib} package. That is, \verb|\cite|, \verb|\citet|, and \verb|\textcite| yield the same result:
\begin{tcolorbox}
	\cite{sarfraz}, \citet{sarfraz}, \textcite{sarfraz}
\end{tcolorbox}
Citing the same author multiple times:
\begin{tcolorbox}
	\cite{aristotle:anima, aristotle:physics, aristotle:poetics} \textellipsis
\end{tcolorbox}

\pagebreak
If a~citation includes a~``von'' part and occurs at the beginning of a~sentence, it has to be capitalized, which is done via the \verb|\Citet| command:
\begin{tcolorbox}
	\Citet{vangennep} \textellipsis, \citet{vangennep} \textellipsis
\end{tcolorbox}

Citing a~work with more than three authors via \verb|\cite|, \verb|\citep|, \verb|\citet|, etc. abbreviates the author list to first author plus ``et~al.'' by default:
\begin{tcolorbox}
	\setlength{\parindent}{0pt}%
	The book \citetitle{gerhardt} \citep{gerhardt} has a~single author. \par
	The report \citetitle{chiu} \citep{chiu} has two authors. \par
	The \citetitle{companion} \citep{companion} has three authors. \par
	The article \citetitle{murray} \citep{murray} has many authors and was published in \mkbibemph{\citefield{murray}{journaltitle}}.
\end{tcolorbox}
The full list of author names is printed when using the starred commands \verb|\citep*|, \verb|\citet*|, etc:
\begin{tcolorbox}
	The article \citetitle{murray} \citep*{murray} has many authors and was published in \mkbibemph{\citefield{murray}{journaltitle}}.
\end{tcolorbox}

\subsubsection{Citations Included in Parentheses}

A~citation in parentheses:
\begin{tcolorbox}
	This process has been described in the literature in detail \citep[see][and the references therein]{gerhardt}.
\end{tcolorbox}
An alternative way to include additional text in the parentheses is the following:
\begin{tcolorbox}
	Its properties are covered in publications and reports from various disciplines (for instance, \citealp{chiu, markey, padhye, sarfraz} are of relevance).
\end{tcolorbox}

\subsubsection{Citing Only Parts of Works}
To cite particular pieces of information of a~work, several commands are available, such as \verb|\citeauthor|, \verb|\citeyear|, \verb|\citetitle|, etc.:
\begin{tcolorbox}
	The book \citetitle{knuth:ct:a}, written by Donald~E. \citeauthor{knuth:ct:a}, was published in \citeyear{knuth:ct:a}. It is available online via  \url{https://ctan.org/pkg/texbook}.%
	\vspace{-1pt}
\end{tcolorbox}


\section{Recommendations for Typesetting Mathematical Expressions}
\label{sec:math}

\subsection{Equations and Equation Numbering}
Short equations can be typeset in-line:
\begin{tcolorbox}
	Let $\pi(x) = p(x)\,x - c(x)$ denote the profit of a~monopolist, where $c(x)$ is the cost of producing quantity~$x$.
\end{tcolorbox}

Equations that are taller than a~normal line of text and important equations should be typeset as display equations:
\begin{tcolorbox}[bottom = -2.75ex, top = -2.75ex]
	\begin{equation}
		(x+a)^n= \sum_{k=0}^n \binom{n}{k} \, x^k a^{n-k}
	\end{equation}%
\end{tcolorbox}

All display equations should be numbered because this makes it easier to refer to them:
\begin{tcolorbox}[top = -2.75ex]
	\begin{equation}
		\label{eq:series}
		(1+x)^n = 1 + \frac{n\,x}{1!} + \frac{n\,(n-1)\,x^2}{2!} + \cdots
	\end{equation} \par
	Equation~\eqref{eq:series} posits that \textellipsis
\end{tcolorbox}
Some authors prefer to number only those equations that they actually reference in the body text. This practice is debatable, however, since the very purpose of the numbering is to make an equation easy to find. When encountering an unnumbered equation, one does not know whether the numbered equation that one is looking for can be found above or below the unnumbered equation. Hence, please number \emph{all} equations. (In the same vein, we also numbers \emph{all} pages, because selectively numbering only particular pages would make little sense.)

\subsection{Italic Shape for Variables}
All variables should be consistently italicized. This makes it easier to identify them and to com\-pre\-hend the text. In LaTeX, this is achieved by using math mode (e.g., \verb|$x$| or \verb|\(x\)| to produce~$x$):
\begin{tcolorbox}
	Let $p$ denote the price, and $c$ marginal cost.
\end{tcolorbox}

\subsection{Upright Shape for Mathematical Constants and Functions with Established Meaning}
In line with the \caps{ISO} norm 80000-2:2019(E), functions, numerical constants, and operators with an established, ``well-defined'' meaning should \emph{not} be italicized but set upright to disambiguate them from variables. ``Well-defined'' here means that the meaning does not change across contexts. This applies to, for instance, the exponential function ($\exp$) and the logarithmic function ($\log$, $\ln$), to Euler's constant ($\mathrm{e}$) and the circular constant ($\uppi$), as well as to the symbols for the differential ($\mathrm{d}$) and the expected value ($\operatorfont{E}$). The \caps{ISO}~80000-2:2019(E) standard%
\footnote{See \url{https://cdn.standards.iteh.ai/samples/64973/329519100abd447ea0d49747258d1094/ISO-80000-2-2019.pdf},~p.\,1. See also the descriptions and discussions in \url{https://tug.org/tugboat/tb18-1/tb54becc.pdf} and in \url{https://nhigham.com/2016/01/28/typesetting-mathematics-according-to-the-iso-standard/}.}%
\textsuperscript{,\kern0.04em}%
\footnote{Note that the decimal separator used by the \caps{ISO} is a~comma (as it is standard, e.g., in German texts) rather than a~period (as it is standard in, e.g., English texts). The \caps{ISO}~80000-1 standard permits both.}
states:

\begin{quotation}
	\sloppy%
	\textbf{4\quad Variables, functions and operators}\par\vskip 0.25\baselineskip
	\noindent It is customary to use different sorts of letters for different sorts of entities, e.g. $x$, $y$, $\dots$ for numbers or elements of some given set, $f$, $g$ for functions, etc. This makes formulas more readable and helps in setting up an appropriate context. \par
	Variables such as $x$, $y$, etc., and running numbers, such as $i$ in $\sum_i x_i$ are printed in italic type. Parameters, such as $a$, $b$, etc., which may be considered as constant in a~particular context, are printed in italic type. The same applies to functions in general, e.g. $f$, $g$. \par
	An explicitly defined function not depending on the context is, however, printed in upright type, e.g. $\sin$, $\exp$, $\ln$, $\upGamma$. Mathematical constants, the values of which never change, are printed in upright type, e.g. ${\mathrm{e} = 2{,}718\,281\,828\dots}$; ${\uppi = 3{,}141\,592\dots}$; ${\mathrm{i}^2 = -1}$. Well-defined operators are also printed in upright type, e.g. $\mathbf{div}$, $\updelta$ in $\updelta x$ and each~$\dd$ in ${\dd f \divslash \dd x}$. Some transforms use special capital letters (\textellipsis\!). \par
	Numbers expressed in the form of digits are always printed in upright type, e.g. $351\,204$; $1{,}32$; $7/8$. \par
	Binary operators, for example $+$, $-$, $/$, shall be preceded and followed by thin spaces. This rule does not apply in case of unary operators, as in $-17{,}3$. \par
	The argument of a~function is written in parentheses after the symbol for the function, without a~space between the symbol for the function and the first parenthesis, e.g. $f(x)$, ${\cos(\omega\,t + \phi)}$. If the symbol for the function consists of two or more letters and the argument contains no operation symbol, such as $+$, $-$, $\times$, or $/$, the parentheses around the argument may be omitted. In these cases, there shall be a~thin space between the symbol for the function and the argument, e.g. $\operatorname{int} 2{,}4$; $\sin n{\kern 0.1em}\uppi$; $\operatorname{arcosh} 2A$; $\operatorname{Ei} x$. \par
	If there is any risk of confusion, parentheses should always be inserted. For example, write $\cos(x) + y$; do not write $\cos x + y$, which could be mistaken for $\cos(x + y)$. \par
	A comma, semicolon or other appropriate symbol can be used as a~separator between numbers or expressions. The comma is generally preferred, except when numbers with a~decimal comma are used. \par
	If an expression or equation must be split into two or more lines, the following method shall be used:
	\begin{itemize}[leftmargin = \baselineskip, labelsep = 0.25\baselineskip]
		\item[---] Place the line breaks immediately before one of the symbols $=$, $+$, $-$, $\pm$, or $\mp$, or, if necessary, immediately before one of the symbols $\times$, $\cdot$, or $/$.
	\end{itemize}
	The symbol shall not be given twice around the line break; two minus signs could for example give rise to sign errors. If possible, the line break should not be inside of an expression in parentheses.
\end{quotation}

Here are some examples:
\begin{tcolorbox}[extras last = {bottom = -4ex}, extras unbroken = {bottom = -4ex}]
	\setlength{\parindent}{0pt}%
	Profit $\pi$ (variable) vs.\ radial constant $\uppi$ (numerical constant). \par
	Effort $e$ (variable) vs.\ Euler's number $\mathrm{e}$ (numerical constant). \par
	Expected value $\operatorname{E}[X]$, variance $\operatorname{Var}[X]$, covariance $\mathrm{Cov}[X, Y]$, probability $\Pr[{X \le x}]$. \par
	Exponential function and logarithm: $\exp x$, $\log y$, $\lg y$, $\ln y$. \par
	Sine and cosine function: $\sin \theta$, $\cos \theta$.
	\begin{equation}
		f(x) = a_0 + \sum_{n=1}^\infty \left(a_n \cos \frac{n \uppi x}{L} + b_n \sin \frac{n \uppi x}{L} \right).
	\end{equation} \par
	Variable $d$ vs.\ differential $\dd$, difference operator $\upDelta$, limit $\lim$.
	\begin{equation}
		\mathrm{E}[X] \coloneq \int_{-\infty}^{\infty} x\,f(x)\,\dd x; \quad
		f'(x) \coloneq \frac{\dd f(x)}{\dd x} \coloneq \lim_{\upDelta x \to 0} \frac{f(x + \upDelta x) - f(x)}{\upDelta x}.
	\end{equation}
\end{tcolorbox}

\pagebreak

Following the \caps{ISO} recommendations, vectors and matrices should be typeset using boldface. Also for vectors and matrices, italics should be used when they denote variables, and upright font should be used for vector-\slash matrix-valued operators and functions with a~``well-defined'' meaning:
\begin{tcolorbox}[extras last = {bottom = -3.5ex}, extras unbroken = {bottom = -3.5ex}]
	A multivariate random variable $\bm{X}$ has the expected value $\mathbf{E}[\boldsymbol{X}]$ (a~vector of scalar expected values). The variance--covariance matrix $\bm{\Sigma}$ is a~symmetric ${(n \times n)}$ matrix with the following entries:
	\begin{equation}
		\bm{\Sigma} = \mathbf{Cov}[\bm{X}] =
		\left[\;\begin{matrix}
			\mathrm{Var}[X_1] & \mathrm{Cov}[X_1, X_2] & \cdots & \mathrm{Cov}[X_1, X_n] \\
			\mathrm{Cov}[X_2, X_1] & \mathrm{Var}[X_2] & \cdots & \mathrm{Cov}[X_2, X_n]
			\\
			\vdots & & \ddots & \vdots \\
			\mathrm{Cov}[X_n, X_1] & \mathrm{Cov}[X_n, X_2] & \cdots & \mathrm{Var}[X_n]
		\end{matrix}\;\right].
	\end{equation}
\end{tcolorbox}

Descriptive subscripts and superscripts should be printed upright so that they can be disambiguated from exponential operations etc.:
\begin{tcolorbox}
	In the equation ${\hat{\bm{\beta}} = (\bm{X}^\mathrm{T}\bm{X})^{-1}\bm{X}^\mathrm{T}\bm{y}}$, the (upright) uppercase letter~$\mathrm{T}$ denotes the matrix transpose. \par
	By contrast, in the expression ${\bm{X} = (\bm{x}_t)_{t = 1}^T}$, the (italic) uppercase letter~$T$ denotes the total number of observations, with individual observations indexed by lowercase~$t$.
\end{tcolorbox}


\section{Portability}

\subsection{Compatibility with Other Typesetting Software}
\label{sec:portability:compatibility}

The settings in this template regarding the font family (by default, Times), font sizes, and line spacing were chosen such that a~virtually identical layout can be achieved with other typesetting software such as Microsoft Word and Apple Pages and the open-source alternatives Libre\-Office Writer (\url{https://www.libreoffice.org}) and Typst (\url{https://typst.app}). This means taking into account, for instance, that Word restricts font sizes to multiples of 0.5~pt.

\subsection{Clean Code/Semantic Coding: Separating Content and Formatting}

All written texts are structured into words and sentences. If they are long enough, texts are also structured into paragraphs. Academic texts, in addition, feature sections, subsections, and often subsubsections, as well as figures, tables, lists of references, and potentially appendices. These structural elements are associated with headings\slash titles and captions.

The structure of a~document is conveyed to the reader through \emph{formatting}. Effective formatting of academic documents, thus, requires that all elements of a~particular type---say, section headings---be identifiable as such. Identifiability is achieved through styling that is \emph{consistent within type} and \emph{distinct across types}. For example, all section headings must be formatted identically, and at the same time, they must look different from subsection headings.

As a~consequence, one should refrain from using manual, discretionary formatting whenever possible. In line with this principle, the source code of this template refers to the document's structure as much as possible. This is also called ``semantic coding'': using code that describes the \emph{meaning} of the content and not \emph{how} those elements should look.
\begin{itemize}
	\item Examples for \emph{semantic} commands are \verb|\author|, \verb|\title|, \verb|\section|,  \verb|\emph|, \verb|\url|, \verb|\cite|, \verb|\begin{figure}| \textellipsis\ \verb|\end{figure}|, \verb|\caption|, \verb|\begin{quote}| \textellipsis\ \verb|\end{quote}|.
	\item Examples for LaTeX commands that are \emph{not semantic}---and should be avoided---are \verb|\textbf|, \verb|\textit|, \verb|\newpage|, \verb|\pagebreak|, \verb|\linebreak|, \verb|\\|, \verb|\noindent|, \verb|\smallskip|, \verb|\centering|, \verb|\noindent|, \verb|\hspace|, \verb|\vspace|, \verb|\cellcolor|, \verb|\multirow|.
\end{itemize}

The reason why one should stick to using semantic commands is that they keep the code \emph{portable}: Semantic code can be \emph{reused} across document without any changes (or at least without major adjustments). This is not then case with nonsemantic code, that is, code that includes manual formatting instructions. Instructions like \verb|\bigskip|, \verb|\hspace|, \verb|\\| (even worse, \verb|\\ \\|), etc. may result in decent formatting in a~particular document but are bound to lead to undesirable formatting in another document. Moreover, manual formatting is error-prone, as it easily leads to inconsistent styling.

This entails two things: First, the formatting should be determined, as much as possible, in the \emph{preamble} of the LaTeX document. This comes at the cost, of course, of a~rather long preamble. Second, instead of defining custom commands, one should \emph{redefine} standard LaTeX commands or use (widely adopted) packages that are available on \caps{CTAN} (\url{https://ctan.org}).

This template follows the above principles, resulting in clean, portable code: The source code of the body of this template (apart from the boxes with examples) consists almost completely of semantic, basic LaTeX, with a~few commands from popular packages on top. And while the preamble of this document is rather long---to achieve the formatting described in \autoref{sec:portability:compatibility}---the packages that are absolutely necessary for successful compilation are few:
\begin{itemize}
	\item \verb|amsmath|,
	\item \verb|amsthm|,
	\item \verb|babel| with options \verb|USenglish| and \verb|ngerman|,
	\item \verb|biblatex-chicago| with options \verb|authordate|, \verb|backend = biber|, and \verb|natbib|;
	\item \verb|bm|,
	\item \verb|enumitem|,
	\item \verb|fontenc| with options \verb|LGR| and \verb|T1|,
	\item \verb|hyperref|,
	\item \verb|isodate|,
	\item \verb|listings|,
	\item \verb|mathtools|,
	\item \verb|microtype|,
	\item \verb|newtxmath| and \verb|newtxtext|,
	\item \verb|printlen|,
	\item \verb|ragged2e|,
	\item \verb|relsize|,
	\item \verb|soul|,
	\item \verb|tabularray| with \verb|\UseTblrLibrary{booktabs, siunitx}|,
	\item \verb|tcolorbox| with option \verb|most|,
	\item \verb|textgreek|,
	\item \verb|titlecaps|,
	\item \verb|xcolor| with options \verb|svgnames| and \verb|x11names|.
\end{itemize}

\subsection{Using BibLaTeX (Rather Than BibTeX)}

This template uses the modern BibLaTeX framework (the \textit{biblatex-chicago} package, \url{https://ctan.org/pkg/biblatex-chicago}, to be precise) instead of the vintage BibTeX framework to manage the references included in the document. The reason for this choice is that BibLaTeX is much more flexible than BibTeX and also handles multi-language references much better.

Moreover, in line with the principle describe above, BibLaTeX permits much better separation of a~document's content (in this case, the \textit{.bib} file) and its formatting (of the citations and the bibliography) than BibTeX. In particular, suppressing output of information that is present in a~\textit{.bib} file is cumbersome with BibTeX: It amounts to editing the ``bibliography style'' \textit{.bst} file, and \textit{.bst} files have a~syntax that is completely different from LaTeX and difficult to learn. It is usually easier to just remove the information from the \textit{.bib} file. This, however, impacts the portability of the \textit{.bib} file: In other circumstances one might want that very information to be present. BibLaTeX simplifies skipping information that is present in the \textit{.bib} file.

An example: Researchers frequently present their research in the form of posters. Space on posters is usually quite limited. Hence, one might not want authors' first names to be abbreviated on a~poster. In the associated paper, however, the names should be printed in full to minimize ambiguity. With BibTeX, this requires time-consuming adjustments of the \textit{.bst} file. With BibLaTeX, it is much easier: In the \textit{.tex} file for the poster, one can include the command
\begin{quote}
	\verb|\ExecuteBibliographyOptions{giveninits = true}|
\end{quote}

Similarly, in a~paper, the list of references should include \caps{URL}s or \caps{DOI}s (digital object identifiers) whenever they exist, to make it easy for readers to locate a~cited work. On a~poster, by contrast, one might want to suppress \caps{DOI}s and \caps{URL}s. With BibTeX, this would require time-consuming adjustments of the \textit{.bst} file. With BibLaTeX, one can simply do, for instance,
\begin{quote}
\begin{verbatim}
\AtEveryBibitem{%
  \ifentrytype{article}{\clearfield{doi}\clearfield{url}}{}%
}
\end{verbatim}
\end{quote}

Using BibLaTeX requires using the program \verb|biber| instead of \verb|bibtex| for compiling the bibliography. When using the Overleaf online service, using \verb|biber| instead of \verb|bibtex| is as simple as it gets: Overleaf automatically chooses \verb|biber| to compile the bibliography when it encounters BibLaTeX's \verb|backend = biber| option.\footnote{This applies to the commercial version (\url{https://www.overleaf.com}) as well as to free versions \url{https://overleaf-students.astro.uni-bonn.de/} and \url{https://uni-bonn.sciebo.de/apps/overleaf_nextcloud/launcher/launch}.} And when using the TeXstudio editor (\url{https://texstudio.org}), an easy way to make this happen is to include the ``magic comment''
\begin{quote}
	\verb|% !BIB program = biber|
\end{quote}
at the beginning of the \textit{.tex} file.

\subsection{Using the \mbox{\textit{tabularray}} Package (Rather Than \mbox{\textit{tabularx}}\slash \mbox{\textit{tabulary}}\slash \mbox{\textit{tabu}}, \mbox{\textit{booktabs}}, \mbox{\textit{longtable}}, \mbox{\textit{xltabular}}, \mbox{\textit{multirow}}, \mbox{\textit{makecell}}, \mbox{\textit{colortbl}}, \mbox{\textit{threeparttable}}, \textellipsis\!)}

The principle of separating the content from its formatting as much as possible can also be applied to tables. The \mbox{\textit{tabularray}} package (\url{https://ctan.org/pkg/tabularray}) allows you to do just that. \mbox{\textit{tabularray}} has been around since May 2021, so it is relatively new---but it has already matured into a~package that is extremely useful. \mbox{\textit{tabularray}} replaces (or loads) all the other table-related LaTeX packages that you may have used so far---like \mbox{\textit{tabularx}}\slash\mbox{\textit{tabulary}}\slash\mbox{\textit{tabu}}, \mbox{\textit{longtable}}, \mbox{\textit{xltabular}}, \mbox{\textit{multirow}}, \mbox{\textit{makecell}}, \mbox{\textit{booktabs}}, \mbox{\textit{colortbl}}, \mbox{\textit{threeparttable}}, \textellipsis

Separating the contents of a~table from its formatting means that no formatting instructions whatsoever need to be placed in the table cells---neither for drawing rules between table rows nor for highlighting entries nor for increasing spacing nor for changing alignment, and so on. Instead, the placement of rules at the top and bottom of a~table or between particular rows, the specification of column widths, the specification of font sizes and cell alignment, desired highlighting via boldface or a~background color, etc., can all be coded as arguments to the table environment. This is beneficial in at least three aspects:
\begin{itemize}
	\item The source code of the table's contents is very clean. This particularly applies to animated Beamer presentations: no need to put \verb|\pause|\slash \verb|\only|\slash \verb|\cellcolor|\slash\textellipsis in table cells.
	\item  It makes table contents \emph{portable}: A~table's contents can be reused easily across multiple documents that are formatted in different ways.
	\item  It also drastically simplifies the inclusion of table contents that are produced by an external program such as Python, R, or Stata.
\end{itemize}

Further additional helpful features of \mbox{\textit{tabularray}} are:
\begin{itemize}
	\item \mbox{\textit{tabularray}} makes it easy to bottom-align or top-align particular table rows or individual cells entries, while keeping a~different alignment for the other rows\slash cells.
	\item Table entries that consist of multiple lines of text become easy to produce: Just enclose them in curly braces, like in this example: \verb|... & {Line 1 \\ Line 2} & ... \\|. No~need to use \verb|\multirow| or \verb|\makecell| any more.
	\item You can set document-wide formatting defaults to achieve consistent formatting of all tables. For instance, you can set defaults such that all tables feature the same top and bottom rules. Or, you could make all column heads be typeset in boldface document-wide.
\end{itemize}

When using \verb|\UseTblrLibrary{siunitx}|, \mbox{\textit{tabularray}} loads the \mbox{\textit{siunitx}} package. This is another useful package which permits separating content from formatting. In particular, the \verb|S|~column type is able to round numbers to a~specified precision. This means that one can include table content produced by an external program with as many decimal digits as that program produces and have LaTeX take care of the rounding.

\section{Tests, Including Example Figures}
\label{sec:tests}

\begin{figure}
	\includegraphics[width = 0.45\textwidth]{example-grid-100x100bp.png}% From the "mwe" package
	\caption{An example \caps{PNG} image. Figure captions go \emph{below} the figures and close with a~period.}
	\label{fig:png}
	\begin{figurenotes}
		Figure captions should be short; by default they should not span more than a~single line. Any additional information that is necessary for understanding a~figure should go in a~figure note, as illustrated here.
	\end{figurenotes}
\end{figure}

\subsection{Tests of List Formatting}
\label{sec:tests:lists}

\newcommand{\testlist}{%
	\item A~journey into the unknown.
	\item The art of tea brewing.
	\item Exploring the depths of the ocean.
	\item The mysteries of ancient civilizations.
	\item Innovations in renewable energy.
	\item The beauty of a~starry night.
}

If the order of the list entries does not matter---that is, the list is a~collection of items without a~hierarchy---use an unordered list, that is, a~list with bullet points:
\begin{itemize}
	\item Nested bulleted list (please avoid more than two list levels):
		\begin{itemize}
			\testlist
		\end{itemize}
	\testlist
\end{itemize}

\renewcommand{\testlist}{%
	\item We first provide an overview of the foundational concepts in LaTeX, including its purpose as a~typesetting system that is particularly well-suited for producing complex documents such as academic papers and theses. LaTeX allows for precise control over document structure and formatting, making it a~preferred choice for many professionals in academia and research.
	\item We delve into\footnote{This text was \caps{AI}-generated \textellipsis} the various environments available in LaTeX, such as the \texttt{itemize} and \texttt{enumerate} environments. Each environment serves a~different purpose, with \texttt{itemize} creating bullet points and \texttt{enumerate} generating numbered lists.
	\item We discuss the importance of packages in LaTeX, which enhance its functionality. Packages like \texttt{amsmath} for advanced mathematical typesetting and \texttt{graphicx} for including images are essential for expanding the capabilities of basic LaTeX. Users should familiarize themselves with how to include and utilize these packages to maximize their LaTeX experience.
	\item Lastly, we explore the common pitfalls and troubleshooting tips for LaTeX users. Issues such as compilation errors, misformatted text, and missing packages can often arise. Knowing how to read error messages and where to find help, such as online forums and documentation, can significantly improve the user experience and help resolve issues efficiently.
}

\displaybaselinegrid%
If the order of the list entries matters---such as when describing a~sequence of events, a~hierarchy, or items sorted by priority---use an ordered list, that is, a~numbered list:%
\begin{enumerate}
	\item Nested numbered list (please avoid more than two list levels!):
		\begin{enumerate}
			\testlist
		\end{enumerate}
	\testlist
\end{enumerate}

\begin{description}
	\item[Apple.] A~sweet and crunchy fruit that comes in various colors such as red, green, and yellow. Apples are rich in dietary fiber and vitamin~C, making them a~healthy snack. They are also used in a~variety of recipes, including pies, salads, and juices.\footnote{This description was \caps{AI}-generated.}
	\item[Banana.] A~long, yellow fruit that is soft and sweet when ripe. Bananas are known for being an excellent source of potassium, which helps maintain proper muscle and nerve function. They are often eaten as a~quick snack or used in smoothies, baked goods, and desserts.\footnote{This description was \caps{AI}-generated, too.}
	\item[Cherry.] A~small, round fruit that is typically red or dark purple in color and contains a~hard pit in the center. Cherries are known for their sweet-tart flavor and are commonly consumed fresh, or used in jams, pies, and juices. They are also rich in antioxidants and anti-inflammatory compounds.\footnote{This description was \caps{AI}-generated, too.}
\end{description}

\pagebreak

\begingroup
\setlength{\parindent}{0pt}
Line 01---This is the first line on the page.\displaybaselinegrid \par
Line 02. Checking the alignment of body text with the baseline grid. \par
Line 03 \par
Line 04 \par
Line 05 \par
Line 06 \par
Line 07 \par
Line 08 \par
Line 09 \par
Line 10 \par
Line 11 \par
Line 12 \par
Line 13 \par
Line 14 \par
Line 15 \par
Line 16 \par
Line 17 \par
Line 18 \par
Line 19 \par
Line 20 \par
Line 21 \par
Line 22 \par
Line 23 \par
Line 24 \par
Line 25 \par
Line 26 \par
Line 27 \par
Line 28 \par
Line 29 \par
Line 30 \par
Line 31 \par
Line 32 \par
Line 33 \par
Line 34 \par
Line 35---This should be the last line on the page. \par
This line should be on the next page. \par
Body text. \par
Body text. \par
\section*{Checking alignment with the baseline grid}
\subsection*{Checking alignment with the baseline grid}
\subsubsection*{Checking alignment with the baseline grid}
Body text.\displaybaselinegrid \par
Body text. \par
Body text. \par
Body text. \par
Body text. \par
Body text. \par
Body text. \par
Body text. \par
Body text. \par
Body text. \par
Body text. \par
\subsubsection*{Checking alignment with the baseline grid}
Body text. \par
Body text. \par
Body text. \par
Body text. \par
Body text. \par
Body text. \par
Body text. \par
Body text. \par
Body text. \par
Body text. \par
Final line on the page. \par

\pagebreak

First line on a~new page. \par
Body text. \par
Body text. \par
Body text. \par
Body text. \par
Body text. \par
Body text. \par
Body text. \par
Body text. \par
Body text. \par
Body text. \par
Body text. \par
Body text. \par
\section*{Checking the alignment of headings and body text with the baseline grid: a~\texttt{\textbackslash section} heading}
Body text.\displaybaselinegrid \par
Body text. \par
Body text. \par
Body text. \par
Body text. \par
Body text. \par
Body text. \par
Body text. \par
Body text. \par
Body text. \par
Body text. \par
Body text. \par
Body text. \par
Body text. \par
Body text. \par
Body text. \par
Final line on the page. \par

\pagebreak

First line on a~new page. \par
Body text. \par
Body text. \par
Body text. \par
Body text. \par
Body text. \par
Body text. \par
Body text. \par
Body text. \par
Body text. \par
Body text. \par
Body text. \par
Body text. \par
Body text. \par
\subsection*{Checking the alignment of headings and body text with the baseline grid: a~\texttt{\textbackslash subsection} heading}
Body text.\displaybaselinegrid \par
Body text. \par
Body text. \par
Body text. \par
Body text. \par
Body text. \par
Body text. \par
Body text. \par
Body text. \par
Body text. \par
Body text. \par
Body text. \par
Body text. \par
Body text. \par
Body text. \par
Body text. \par
Final line on the page. \par

\pagebreak

First line on a~new page. \par
Body text. \par
Body text. \par
Body text. \par
Body text. \par
Body text. \par
Body text. \par
Body text. \par
Body text. \par
Body text. \par
Body text. \par
Body text. \par
Body text. \par
Body text. \par
Body text. \par
\subsubsection*{Checking the alignment of headings and body text with the baseline grid: a~\texttt{\textbackslash subsubsection} heading}
Body text.\displaybaselinegrid \par
Body text. \par
Body text. \par
Body text. \par
Body text. \par
Body text. \par
Body text. \par
Body text. \par
Body text. \par
Body text. \par
Body text. \par
Body text. \par
Body text. \par
Body text. \par
Body text. \par
Body text. \par
Final line on the page. \par
\endgroup

\pagebreak

\subsection{Test of Theorems, Lemmas, Hypotheses, Results, Proofs}

\begin{conjecture}[Poincaré conjecture]
	\label{conjecture:poincare}
	Every three-dimensional topological manifold which is closed, connected, and has trivial fundamental group is homeomorphic to the three-dimensional sphere.
\end{conjecture}

\begin{corollary}
	\label{corollary:budde}
	Let the performance measurement system~$\bm{P}$ be minimal and balanced. If not all measures of~$\bm{P}$ are verifiable, the first-best effort cannot be induced by an explicit contract.
\end{corollary}

\begin{lemma}
	\label{lemma:budde-kraekel-1}
	To implement a~given action $a$ at minimal cost, higher-powered
	incentives are necessary under a~more risky performance measure.
\end{lemma}

\begin{proof}%[Proof of \autoref{lemma:budde-kraekel-1}]
	Obvious from (IC$'$).
\end{proof}

\begin{lemma}
	\label{lemma:budde-goex-1}
	The optimal capacity choice of firm~$i$ is given by
	\begin{equation*}
		k^* =
		\begin{cases*}
			\strut \; 1 & for\;\; $\underline{c} \leq c_i < \hat{c}$, \\
			\strut \; 0 & for\;\; $\hat{c} \leq c_i \leq \overline{c}$,
		\end{cases*}
		\quad \text{where} \quad \hat{c} = H^{-1}(1 \divslash \theta).
	\end{equation*}
\end{lemma}

\begin{proof}
	See \autoref{app:proofs}.
\end{proof}

\begin{proposition}
	\label{proposition:budde-1}
	Let $\bm{P}$ be balanced with respect to the principal's objective $V$. Then the following statements hold:
	\begin{enumerate}[label = (\roman*), align = right, leftmargin = \baselinedist, labelsep = 0.3\baselinedist]
		\item The first-best solution can be achieved by a~linear contract based on $\bm{P}$.
		\item If $\bm{P}$ is also minimal, all measures have nonzero value in the optimal contract.
	\end{enumerate}
\end{proposition}

\begin{proposition}
	\label{proposition:budde-kraekel-3}
	If the principal strictly prefers the implemented action in a~model with a~finite action space, there is a~positive risk--incentive relationship.
\end{proposition}

\begin{proposition}
	\label{proposition:budde-kraekel-4}
	If the agent is risk neutral, there is a~positive risk--incentive relationship as long as the probability of a~high payment exceeds \textonehalf.
\end{proposition}

\begin{theorem}[Lindeberg--Lévy Central Limit Theorem]
	%\newcommand{\mathup}[1]{\mathrm{#1}}
	\label{theorem:CLT}
	Let $X_1, X_2, \ldots, X_n$ be i.i.d. random variables with expected value ${\operatorname{E}[X_i] = \mu < \infty}$ and variance ${0 < \operatorname{Var}[X_i] = \sigma^2 < \infty}$. Then, the random variable\hfill\null
	\begin{equation*}
		Z_{n}
		= \frac{\overline{X}-\mu}{\sigma \divslash \sqrt{n}}
		= \frac{X_1 + X_2 + \dots + X_{n} - n\,\mu}{\sqrt{n} \, \sigma}
	\end{equation*}
	converges in distribution to the standard normal random variable as n goes to infinity, that is
	\begin{equation*}
		\lim_{n \rightarrow \infty} \Pr[Z_{n} \leq x] = \upPhi(x), \quad \text{for all } x \in \mathbb{R},
	\end{equation*}
	where $\upPhi(x)$ is the standard normal \caps{CDF}.
\end{theorem}

\begin{assumption}[\caps{MLRP}]
	\label{assumption:mlrp}
	The signals $y_i$ fulfill the monotone likelihood ratio property:
	$p_i(y_i \,|\, a_i) \divslash p_i(y_i \,|\, a_i)$ is increasing in $y_i$.
\end{assumption}

\begin{assumption}[\caps{CDFC}]
	\label{assumption:cdfc}
	The signals $y_i$ fulfill the convexity of the distribution function condition:
	$\partial^2 F_i(y_i \,|\, a_i) \divslash \partial a_i^2 \ge 0$.
\end{assumption}

\begin{definition}
	\label{definition:budde-kraekel}
	We define a~performance signal~$\hat{s}$ to be \textit{more risky} (less informative with respect to the agent's action) than a~signal~$s$, if $\hat{a}$~is a~garbling of~$s$, i.e., if there exists a~number ${b \in (0, 1/2]}$ such that
	\begin{equation*}
		\hat{p}(a) = (1 - b) \, p(a) + b \, (1 - p(a)) = b + (1 - 2\,b)\,p(a).
	\end{equation*}
\end{definition}

\begin{example}
	\label{example:budde-kraekel-2}
	Let the agent's action space be $\{a_1, a_2\}$ with ${v(a_1) = 0}$ and
	${v(a_2) = 4}$, the agent's utility be given by ${U(w, a) = \sqrt{w} - c(a)}$ with ${c(a_1) = 0}$ and ${c(a_2) = 1}$, and success probabilities be ${p(a_1) = 0.25}$ and ${p(a_2) = 0.75}$. Then, high effort~$a_2$ is implemented by wages ${w_L = 0}$ and ${w_H = 4}$. The expected compensation cost is ${\operatorname{E}[w] = 0.25 \times 0 + 0.75 \times 4 = 3}$, and the principal's net profit is ${4 - 3 = 1}$. The principal strictly prefers $a_2$ to $a_1$, since implementing $a_1$ yields a~net profit of~$0$. \par
	Now consider a~garbling of the form proposed in our definition of riskiness. Success probabilities become ${\hat{p}(a_1) = 0.25 + 0.5\,b}$ and ${\hat{p}(a_2) = 0.75 - 0.5\,b}$. High effort is implemented by wages ${w_L = 0}$ and ${w_H = 1 \divslash (0.5 - b)^2}$. As ${1 \divslash (0.5 - b)^2 > 4}$, ${\forall b \in (0, 1/2)}$, incentives become higher powered. The principal's net profit under $a_2$ is ${4 - (0.75 - 0.5\,b) \divslash (0.5 - b)^2}$, which is larger than his zero profit under $a_1$ for ${b < (7 - \sqrt{33}) \divslash 16}$ or ${b > (7 + \sqrt{33}) \divslash 16}$. For these levels of $b$, the principal still implements $a_H$ after the garbling, and there is a~positive risk--incentive relationship.
\end{example}

\begin{hypothesis}
	\label{hypothesis:dertwinkel-kalt-1}
	Subjects allocate more money to payoffs that are concentrated on a~single date than to equal-sized payoffs that are dispersed over multiple earlier dates, ${d^{\mathrm{CB}} > 0}$ (in contrast to standard discounting).
\end{hypothesis}

\begin{hypothesis}
	\label{hypothesis:dertwinkel-kalt-2}
	The effect described in \autoref{hypothesis:dertwinkel-kalt-1} is the more pronounced, the more dispersed a~payoff is, that is, ${d^{\mathrm{CB}}_8 > d^{\mathrm{CB}}_4 > d^{\mathrm{CB}}_2 > 0}$.
\end{hypothesis}

\begin{result}[Test of \autoref{hypothesis:dertwinkel-kalt-1}]
	\label{result:dertwinkel-kalt-1}
	On average, subjects allocate more money to payoffs that are concentrated rather than dispersed, that is, our measure of concentration bias, $\hat{d}$, is significantly larger than zero.
\end{result}

\begin{result}[Test of \autoref{hypothesis:dertwinkel-kalt-2}]
	\label{result:dertwinkel-kalt-2}
	Our measure of concentration bias is the greater, the more dispersed payments in the \textsc{Conc-Disp} and \textsc{Disp-Conc} condition are, that is, ${\hat{d}_8 > \hat{d}_4 > \hat{d}_2 > 0}$.
\end{result}

\begin{remark}[Formatting of theorem-like segments]
	\label{remark:formatting}
	The formatting of theorems and theorem-like segments of academic papers varies widely between journals and publishers. There is no agreed-upon standard. Some journals indent these segments, others do not; some print lemmas, propositions, and theorems in italics, others use upright letters. Therefore, the categorization and formatting of the different theorem-like environments in this template follows the formatting suggested by the American Mathematical Society in \url{https://mirrors.ctan.org/macros/latex/required/amscls/doc/amsthdoc.pdf}: \par
	\begin{quotation}
		These default settings are provided: \par\vskip7.5pt
		\begin{itemize}[leftmargin = \baselinedist, labelsep = 0.45\baselinedist]
			\item \texttt{plain}: italic text, extra space above and below;
			\item \texttt{definition}: upright text, extra space above and below;
			\item \texttt{remark}: upright text, no extra space above or below.
		\end{itemize} \par\vskip7.5pt
		\textellipsis \par
		The following list summarizes the types of structures which are normally
		associated with each theorem style. \par\vskip7.5pt
		\begin{tblr}{
			colspec = {l X},
			hline{1} = {0pt},
			hline{Z} = {0pt},
			width = \linewidth-\parindent,
		}
			\texttt{plain} &
			Theorem, Lemma, Corollary, Proposition, Conjecture, Criterion, Assertion
			\\
			\texttt{definition} &
			Definition, Condition, Problem, Example, Exercise, Algorithm, Question, Axiom, Property, Assumption, Hypothesis
			\\
			\texttt{remark} &
			Remark, Note, Notation, Claim, Summary, Acknowledgment, Case, Conclusion
		\end{tblr}
	\end{quotation}
\end{remark}

\subsection{Tests of Cross-Referencing}

\hyperref[sec:math]{\titlecap{\sectionautorefname}~\ref{sec:math}}, \autoref{sec:tests}, \autoref{sec:tests:lists}, \autoref{sec:tests:lorem-ipsum:beginning}, \autoref{sec:tests:lorem-ipsum:main:duden}, \autoref{app:math-torture}, \autoref{app:math-test:alphabets}.

\autoref{fig:png}, \autoref{fig:pdf}, \autoref{tab:siunitx-example}, \autoref{tab:reg-table}.

\autoref{conjecture:poincare}, \autoref{corollary:budde}, \autoref{proposition:budde-1}, \autoref{lemma:budde-goex-1}, \autoref{lemma:budde-kraekel-1}, \autoref{proposition:budde-kraekel-3}, \autoref{proposition:budde-kraekel-4}, \autoref{theorem:CLT}.

\autoref{hypothesis:dertwinkel-kalt-1}, \autoref{hypothesis:dertwinkel-kalt-2}, \autoref{result:dertwinkel-kalt-1} \autoref{result:dertwinkel-kalt-2}.

\autoref{assumption:mlrp}, \autoref{assumption:cdfc}, \autoref{definition:budde-kraekel}, \autoref{example:budde-kraekel-2}, \autoref{remark:formatting}.

\begin{figure}[t]
	\includegraphics[trim=0mm 0mm 17.65mm 17.65mm, clip, width = 0.55\textwidth]{example-grid-100x100bp.pdf}% From the "mwe" package
	\caption{An example cropped \caps{PDF} image.}
	\label{fig:pdf}
	\begin{figurenotes}
		Figures as well as tables should be placed at the top of a~page. If the top is already occupied, they can also be placed at the bottom of a~page. This corresponds to LaTeX's default placement order. That is, figures and tables should \emph{not} be placed in the midst of body text.
	\end{figurenotes}
\end{figure}


\section{Text Samples, Including Example Tables}

\subsection{Overview}

The whole Latin alphabet in a~short sentence: Amazingly few discotheques provide jukeboxes. All letters between other letters and some punctuation: Incredibly, he makes a major life-change! For example: ``I'll require that the system have two sizes.'' All digits and f-ligatures: Fifty-five fjord truffles offer sufficient flavor, although ${537 + 489 = 1026}$. All French and German special characters: Le c\oe ur d\'e\c cu mais l'\^ame plut\^ot na\"\i ve, Lou\"ys r\^eva de crapa\"uter en cano\"e au del\`a des \^\i les, pr\`es du m\"alstr\"om o\`u br\^ulent les nov\ae. Die Faltung einer Gau\ss-Kurve mit einer Lorentz-Kurve:
\begin{equation}
\int_{-\infty}^\infty \frac{y\,\mathrm{e}^{-t^2}}{(x-t)^2+y^2}\,\mathrm{d} t =
\uppi \, \Re[w(x + i\,y)], \quad w(z) = \mathrm{e}^{-z^2} \operatorname{erfc}(-i\, z), \quad x \in \mathbb{R}, y>0.
\end{equation}
The cross product $\bm{\alpha} \times \bm{\beta}$ of two vectors $\bm{\alpha}$ and $\bm{\beta}$ is defined by
\begin{equation}
	\bm{\alpha} \times \bm{\beta} \coloneq |\bm{\alpha}| \, |\bm{\beta}| \sin (\theta) \, \hat{\bm{n}}.
\end{equation}

The remaining text samples in this section were generated with the help of \url{https://www.blindtextgenerator.de}.

\begin{table}
	\caption{My first table (table titles go \emph{above} the tables and do \emph{not} have a~closing period)}
	\label{tab:siunitx-example}
	\begin{booktabs}{
		colspec = {
			r
			X[c, si={table-format = 2.3}]
			X[c, si={table-format = +1.3}]
			X[c, si={table-format = 2.3}]
			X[c, si={table-format = 1.3}]
			X[c, si={table-format = 1.3}]
			X[c, si={table-format = +3.3}]
			Q[c, si={table-format = +3.3}]
		},
		column{1} = {rightsep+ = 15pt},
		column{Z} = {leftsep+ = 3pt},
		hline{6, 10} = {\lightrulewidth, solid},
		hborder{6, 10} = {abovespace = \aboverulesep, belowspace = \belowrulesep},
		row{1} = {guard},
	}
		{$m$} & {$\Re\{\underline{\mathfrak{X}}(m)\}$} & {$-\Im\{\underline{\mathfrak{X}}(m)\}$} & {$\mathfrak{X}(m)$} & {$\frac{\mathfrak{X}(m)}{23}$} & {$A_m$} & {$ \varphi(m) \divslash {^{\circ}}$} & {$\varphi_m \divslash {^{\circ}}$} \\
		 1 & 16.128 & +8.872 & 16.128 & 1.402 & 1.373 & -146.6 & -137.6 \\
		 2 & 3.442  & -2.509 & 3.442  & 0.299 & 0.343 & 133.2  & 152.4  \\
		 3 & 1.826  & -0.363 & 1.826  & 0.159 & 0.119 & 168.5  & -161.1 \\
		 4 & 0.993  & -0.429 & 0.993  & 0.086 & 0.08  & 25.6   & 90     \\
		 5 & 1.29   & +0.099 & 1.29   & 0.112 & 0.097 & -175.6 & -114.7 \\
		 6 & 0.483  & -0.183 & 0.483  & 0.042 & 0.063 & 22.3   & 122.5  \\
		 7 & 0.766  & -0.475 & 0.766  & 0.067 & 0.039 & 141.6  & -122   \\
		 8 & 0.624  & +0.365 & 0.624  & 0.054 & 0.04  & -35.7  & 90     \\
		 9 & 0.641  & -0.466 & 0.641  & 0.056 & 0.045 & 133.3  & -106.3 \\
		10 & 0.45   & +0.421 & 0.45   & 0.039 & 0.034 & -69.4  & 110.9  \\
		11 & 0.598  & -0.597 & 0.598  & 0.052 & 0.025 & 92.3   & -109.3 \\
	\end{booktabs}
	\begin{tablenotes}[Source]
		Add the source of your data if you are using data that someone else collected. This table was adapted from \url{https://tex.stackexchange.com/a/112382/156280}.
	\end{tablenotes}
	\begin{tablenotes}[Notes]
		Table titles should be short; by default they should not span more than a~single line. Any additional information that is necessary for understanding a~table should go in a~table note, as illustrated here. A~useful tool for generating LaTeX tables is \url{https://www.tablesgenerator.com}. The tables in this template use the environments provided by the \mbox{\textit{tabularray}} package (\url{https://ctan.org/pkg/tabularray}).
	\end{tablenotes}
\end{table}

\subsection{Kafka in \texttt{\textbackslash normalsize}}

\newcommand{\Kafka}{%
	Jemand musste Josef K. verleumdet haben, denn ohne dass er etwas Böses getan hätte, \mbox{wurde} er eines Morgens verhaftet. ,,Wie ein Hund!{}`` sagte er, es war, als sollte die Scham ihn über\-leben. Als Gregor \mbox{Samsa} eines Morgens aus unruhigen Träumen erwachte, fand er sich in seinem Bett zu einem ungeheueren Ungeziefer verwandelt. Und es war ihnen wie eine Bestätigung ihrer neuen Träume und guten Absichten, als am Ziele ihrer Fahrt die Tochter als erste sich erhob und ihren jungen Körper dehnte. ,,Es ist~ein eigentümlicher Apparat``, sagte der Offizier zu dem Forschungsreisenden und überblickte mit einem gewissermaßen bewundernden Blick den ihm doch wohlbekannten Apparat. Sie hätten noch~ins Boot springen können, aber der Reisende hob ein schweres, geknotetes Tau vom Boden, drohte ihnen damit und hielt sie dadurch von dem Sprunge ab. In den letzten Jahrzehnten ist das Interesse an Hungerkünstlern sehr zurück\-ge\-gangen. Aber sie überwanden sich, umdrängten den Käfig und \textellipsis \par%
}

\begin{otherlanguage}{ngerman}\normalsize\noindent%
	\Kafka
\end{otherlanguage}

\subsection{Kafka in \texttt{\textbackslash small}}

\begin{otherlanguage}{ngerman}\vskip-\smallskipamount\noindent\small%
	\Kafka
\end{otherlanguage}

\subsection{Kafka in \texttt{\textbackslash footnotesize}}

\begin{otherlanguage}{ngerman}\vskip-\smallskipamount\noindent\footnotesize%
	\Kafka
\end{otherlanguage}

\subsection{The Story of \textit{Lorem Ipsum}}

\subsubsection{The Beginning}
\label{sec:tests:lorem-ipsum:beginning}
Far far away, behind the word mountains, far from the countries Vokalia and Consonantia, there live the blind texts. Separated they live in Bookmarksgrove right at the coast of the Semantics, a~large language ocean. A~small river named Duden flows by their place and supplies it with the necessary regelialia. It is a~paradisematic country, in which roasted parts of sentences fly into your mouth. Even the all-powerful Pointing has no control about the blind texts it is an almost unorthographic life One day however a~small line of blind text by the name of Lorem Ipsum decided to leave for the far World of Grammar.

\begin{table}
	\caption{Example of a~regression table---alignment at the decimal point via the \mbox{\textit{siunitx}} package}
	\label{tab:reg-table}
	\newcolumntype{U}{
		S[table-format = +1.3, round-mode = places, round-precision = 3, table-space-text-pre = {**}, table-space-text-post = {-**}, round-pad = false, table-column-width = 0.1175\textwidth]
	}
	\begin{booktabs}{
		cell{1}{2} = {c = 4}{halign = c},  % Merge cells 2-4 of row 1 and center the content
		colspec = {
			X[l]
			*{5}{S[table-format = +1.3, round-mode = places, round-precision = 3, table-space-text-pre = {**}, table-space-text-post = {-**}, round-pad = false, table-column-width = 0.1175\textwidth]}
		},
		colsep = 2pt,
		column{2} = {leftsep = 0pt},
		column{Z} = {rightsep = 6pt},
		hborder{2, 4, 16, 18} = {abovespace = \aboverulesep, belowspace = \belowrulesep},
		hline{2} = {0pt},  % Remove default rule after column headings
		hline{2} = {2-Y}{\lightrulewidth, solid, endpos = true, leftpos = -1, rightpos = -1},
		hline{4, 16, 18} = {\lightrulewidth, solid},
		row{1-3} = {guard},
	}
		& Choice List & Choice List & Choice List & Choice List & Combined \\
		& A & B & C & D \\
		&	(1) & (2) & (3) & (4) & (5) \\
		Treatment
			&	-0.390	&	-0.228	&	-0.729*	&	-0.449*	&	-0.453**	\\
			&	(+0.352)	&	(-0.205)	&	[+0.377]	&	[-0.245]	&	{\{}+0.204{\}}	\\
		Female
			&	0.948***	&	0.0607	&	0.188	&	0.305	&	0.385*	\\
			&	(0.354)	&	(0.233)	&	(0.372)	&	(0.226)	&	(0.222)	\\
		$\text{Female} \times \text{Treatment}$
			&	0.169	&	0.251	&	0.892*	&	0.454	&	0.439	\\
			&	(0.514)	&	(0.325)	&	(0.533)	&	(0.341)	&	(0.307)	\\
		Final high school grade
			&	-0.101	&	0.0132	&	0.0759	&	0.117	&	0.0394	\\
			&	(0.198)	&	(0.144)	&	(0.224)	&	(0.146)	&	(0.133)	\\
		Trait self-control
			&	-0.0156	&	0.00214	&	-0.0160	&	-0.000122	&	-0.00678	\\
			&	(0.0162)	&	(0.00994)	&	(0.0147)	&	(0.0104)	&	(0.00935)	\\
		Constant
			&	2.357***	&	1.512***	&	-0.322	&	2.158***	&	1.437***	\\
			&	(0.239)	&	(0.144)	&	(0.265)	&	(0.161)	&	(0.152)	\\
		Observations
			&	{303}	&	{289}	&	{295}	&	{304}	&	{1191}	\\
		$R^2$
			&	0.057	&	0.008	&	0.039	&	0.043	&	0.024	\\
		$\text{Treatment} \times (1 + \textup{Female})$
			&	-0.221	&	0.0228	&	0.163	&	0.00435	&	-0.0138	\\
		\makebox[0pt][l]{$p_F[\text{Treatment} \times (1 + \textup{Female}) = 0]$}
			&	0.327	&	0.00767	&	0.192	&	0.000314	&	0.00343	\\
	\end{booktabs}
	\begin{tablenotes}[Notes]%
		Dependent variable: $m_\sim$. Robust standard errors (cluster-corrected for column~5) in parentheses. Missing observations (${N < 308}$) due to exclusion of trials in which subjects behaved irrationally (i.e., chose a~dominated option). The regressors Final high school grade and Trait self-control are mean-centered. *\,${p < 0.1}$, **\,${p < 0.05}$, ***\,${p < 0.01}$.
	\end{tablenotes}
\end{table}

The Big Oxmox \citep{wilde} advised her not to do so, because there were thousands of bad Commas, wild Question Marks and devious Semikoli, but the Little Blind Text didn't listen. She packed her seven versalia, put her initial into the belt and made herself on the way. When she reached the first hills of the Italic Mountains, she had a~last view back on the skyline of her hometown Bookmarksgrove, the headline of Alphabet Village and the subline of her own road, the Line Lane. Pityful a~rethoric question ran over her cheek, then she continued her way. On her way she met a~copy. The copy warned the Little Blind Text, that where it came from it would have been rewritten a~thousand times and everything that was left from its origin would be the word ``and'' and the Little Blind Text should turn around and return to its own, safe country.

\begin{table}[tb!]
	\caption{Initial capital structures of large projects}
	\label{tab:init-cap-structure}
	\begin{booktabs}{
		colspec = {X *{12}{r}},
		colsep = 3.2pt,
		column{1} = {leftsep = 0pt},
		column{Z} = {leftsep = 11pt, rightsep = 0pt},
		hborder{2, 9, 11} = {abovespace = \aboverulesep, belowspace = \belowrulesep},
		hline{2, 9, 11} = {\lightrulewidth, solid},
	}
		Debt-to-Assets Ratio & 2002 & 2003 & 2004 & 2005 & 2006 & 2007 & 2008 & 2009 & 2010 & 2011 & 2012 & Total \\
		<\,50\%              & --\% & 10\% &  5\% &  7\% & 11\% &  2\% & --\% & --\% & --\% & --\% &  3\% &  2\% \\
		50\%--59.9\%         & 20\% &  5\% & 16\% &  7\% & 11\% &  8\% & 13\% & 11\% &  9\% & 15\% &  8\% & 11\% \\
		60\%--69.9\%         & --\% & 10\% &  5\% & 13\% & 11\% & 10\% & 16\% & 16\% & 17\% & 11\% & 15\% & 13\% \\
		70\%--79.9\%         & --\% & 10\% & 37\% & 13\% & 15\% &  8\% & 24\% & 29\% & 16\% & 14\% & 35\% & 22\% \\
		80\%--89.9\%         & 40\% & 10\% &  5\% & 17\% &  7\% & 24\% & 18\% & 18\% & 21\% &  6\% &  8\% & 15\% \\
		>\,90\%              & 40\% & 55\% & 32\% & 43\% & 46\% & 49\% & 29\% & 26\% & 38\% & 24\% & 35\% & 37\% \\
		Mean                 & 85\% & 80\% & 77\% & 80\% & 78\% & 85\% & 79\% & 79\% & 82\% & 77\% & 80\% & 80\% \\
		Median               & 85\% & 94\% & 76\% & 81\% & 80\% & 87\% & 79\% & 77\% & 82\% & 75\% & 72\% & 80\% \\
		No. of Projects      &    5 &   20 &   19 &   30 &   46 &   51 &   68 &   38 &   58 &   54 &   26 &  415 \\
	\end{booktabs}
	\begin{tablenotes}[Notes]
		Adapted from \url{https://tex.stackexchange.com/a/373932/156280}. Large projects are defined as having a~capitalization of at least \$1bn.
	\end{tablenotes}
\end{table}

\subsubsection{The Main Text}

\paragraph{The Duden River}
\label{sec:tests:lorem-ipsum:main:duden}
But nothing the copy said could convince her and so it didn't take long until a few insidious Copy Writers ambushed her, made her drunk with Longe and Parole and dragged her into their agency, where they abused her for their projects again and again. And if she hasn't been rewritten, then they are still using her. Far far away, behind the word mountains, far from the countries Vokalia and Consonantia, there live the blind texts.%
\footnote{The quick, brown fox jumps over a lazy dog \citep{knuth:ct:b, knuth:ct:c}. DJs flock by when \caps{MTV} ax quiz prog. Junk \caps{MTV} quiz graced by fox whelps. Bawds jog, flick quartz, vex nymphs. Waltz, bad nymph, for quick jigs vex! Fox nymphs grab quick-jived waltz. Brick quiz whangs jumpy veldt fox. Bright vixens jump; dozy fowl quack. Quick wafting zephyrs vex bold Jim. Quick zephyrs blow, vexing daft Jim. Sex-charged fop blew my junk \caps{TV} quiz. How quickly daft jumping zebras vex. Two driven jocks help fax my big quiz. Quick, Baz, get my woven flax jodhpurs! ``Now fax quiz Jack!'' my brave ghost pled. Five quacking zephyrs jolt my wax bed. ``But I~must explain to you how all this mistaken idea of denouncing pleasure and praising pain was born and I~will give you a~complete account of the system, and expound the actual teachings of the great explorer of the truth, the master-builder of human happiness. No one rejects, dislikes, or avoids pleasure itself, because it is pleasure, but because those who do not know how to pursue pleasure rationally encounter consequences that are extremely painful. Nor again is there anyone who loves or pursues or desires to obtain pain of itself, because it is pain, but because occasionally circumstances occur in which toil and pain can procure him some great pleasure. To take a~trivial example, which of us ever undertakes laborious physical exercise, except to obtain some advantage from it? %But who has any right to find fault with a~man who chooses to enjoy a~pleasure that has no annoying consequences, or one who avoids a~pain that produces no resultant pleasure? On the other hand, we denounce with righteous indignation and dislike men who are so beguiled and demoralized by the charms of pleasure of the moment, so blinded by desire, that they cannot foresee. the pain and trouble that are bound to ensue; and equal blame belongs to those who fail in their duty through weakness of will, which is the same as saying through shrinking from toil and pain. These cases are perfectly simple and easy to distinguish.
\citep[210.]{cicero}}

Separated they live in Bookmarksgrove right at the coast of the Semantics, a~large language ocean. A~small river named Duden flows by their place and supplies it with the necessary regelialia. It is a~paradisematic country, in which roasted parts of sentences fly into your mouth. Even the all-powerful Pointing has no control about the blind texts it is an almost unorthographic life One day however a small line of blind text by the name of Lorem Ipsum decided to leave for the far World of Grammar. The Big Oxmox advised her not to do so, because there were thousands of bad Commas, wild Question Marks and devious Semikoli, but the Little Blind Text didn't listen. She packed her seven versalia, put her initial into the belt and made herself on the way. When she reached the first hills of the Italic Mountains, she had a last view back on the skyline of her hometown Bookmarksgrove, the headline of Alphabet Village and the subline of her own road, the Line Lane. Pityful a rethoric question ran over her cheek, then she continued her way.%

\begin{table}
	\caption{Project funding by source}
	\label{tab:project-funding}
	\begin{booktabs}{
		colspec = {
			X *{8}{Q[r, wd = 0.0675\textwidth]} Q[r]
		},
		colsep = 2.5pt,
		column{1} = {leftsep = 0pt},
		column{Z} = {leftsep = 10pt, rightsep = 0pt},
		hborder{2, 5-8, 11, 12} = {abovespace = \aboverulesep, belowspace = \belowrulesep},
		hline{2, 8} = {\lightrulewidth, solid},
		hline{5-7, 11, 12} = {2-Z}{\lightrulewidth, solid, endpos = true, leftpos = -1},
	}
		& 1995 & 1996 & 1997 & 1998 & 1999 & 2000 & 2001 & 2002 & Total \\
		{Bank Loans} & 23.33 & 42.83 & 67.43 & 56.65 & 72.39 & 110.89 & 108.48 & 62.20 & 557.88 \\
		{Bonds} & 3.79 & 4.79 & 7.70 & 9.79 & 19.79 & 20.81 & 25.00 & 13.80 & 109.26 \\
		{Development Agencies} & 17.59 & 18.96 & 22.05 & 20.97 & 16.62 & 17.69 & 18.75 & 18.75 & 162.63 \\
		{Total Debt} & 44.71 & 66.58 & 96.98 & 87.41 & 108.80 & 149.39 & 152.23 & 94.75 & 829.77 \\
		{Equity} & 19.16 & 28.54 & 41.56 & 37.46 & 46.70 & 64.02 & 65.24 & 40.61 & 355.68 \\
		{Total} & 63.88 & 95.12 & 138.54 & 124.87 & 155.68 & 213.40 & 217.47 & 135.36 & 1185.63 \\
		{Bank Loans} & 37\% & 45\% & 49\% & 45\% & 46\% & 52\% & 50\% & 46\% & 47\% \\
		{Bonds} & 6\% & 5\% & 5\% & 8\% & 13\% & 10\% & 11\% & 10\% & 9\% \\
		{Development Agencies} & 28\% & 20\% & 16\% & 17\% & 11\% & 8\% & 9\% & 14\% & 14\% \\
		{Total Debt} & 70\% & 70\% & 70\% & 70\% & 70\% & 70\% & 70\% & 70\% & 70\% \\
		{Equity} & 30\% & 30\% & 30\% & 30\% & 30\% & 30\% & 30\% & 30\% & 30\% \\
	\end{booktabs}
	\begin{tablenotes}[Notes]
		Absolute amounts in US\$ billions. Adapted from \url{https://tex.stackexchange.com/a/373932/156280}.
	\end{tablenotes}
\end{table}

On her way she met a~copy. The copy warned the Little Blind Text, that where it came from it would have been rewritten a~thousand times and everything that was left from its origin would be the word ``and'' and the Little Blind Text should turn around and return to its own, safe country. But nothing the copy said could convince her and so it didn't take long until a~few insidious Copy Writers ambushed her, made her drunk with Longe and Parole and dragged her into their agency, where they abused her for their projects again and again. And if she hasn't been rewritten, then they are still using her.

Far far away, behind the word mountains, far from the countries Vokalia and Consonantia, there live the blind texts. Separated they live in Bookmarksgrove right at the coast of the Semantics, a~large language ocean. A~small river named Duden flows by their place and supplies it with the necessary regelialia. It is a~paradisematic country, in which roasted parts of sentences fly into your mouth. Even the all-powerful Pointing has no control about the blind texts it is an almost unorthographic life One day however a~small line of blind text by the name of Lorem Ipsum decided to leave for the far World of Grammar. The Big Oxmox advised her not to do so, because there were thousands of bad Commas, wild Question Marks and devious Semikoli, but the Little Blind Text didn't listen.

\begin{table}
	\caption{Global project bank facility lead arrangers}
	\begin{booktabs}{
		colspec = {r X rrrr},
		columns = {colsep+ = 0.6em},
		column{1} = {leftsep = 0pt},
		column{Z} = {rightsep = 0pt},
		hborder{2, 13} = {abovespace = \aboverulesep, belowspace = \belowrulesep},
		hline{2} = {\lightrulewidth, solid},
		hline{13} = {2-Z}{\lightrulewidth, solid, endpos = true, leftpos = -1},
		row{1} = {valign = b},
		width = \textwidth,
	}
		Rank & Lead Arranger & {Number \\ of Deals} & Amount & {Market \\ Share} & {Equator \\ Principles \\ Adoption} \\
		 1 & State Bank of India & 52 & \$21,631.6 & 10.1\% & N/A \\
		 2 & Mitsubishi UFJ Financial & 88 & \$9,486.1 & 4.4\% & Dec.~2005 \\
		 3 & Sumitomo Mitsui & 71 & \$8,188.1 & 3.8\% & Jan.~2006 \\
		 4 & Credit Agrocole & 60 & \$6,506.4 & 3.1\% & Jun.~2005 \\
		 5 & Mizuho Financial & 55 & \$5,797.5 & 2.7\% & Oct.~2003 \\
		 6 & Soci\'{e}t\'{e} Generale & 55 & \$5,760.5 & 2.7\% & Sep.~2007 \\
		 7 & BNP Paribas & 55 & \$5,390.8 & 2.5\% & Oct.~2008 \\
		 8 & Axis Bank & 18 & \$5,216.9 & 2.4\% & N/A \\
		 9 & IDBI Bank & 10 & \$5,162.3 & 2.4\% & N/A \\
		10 & ING & 49 & \$4,916.1 & 2.3\% & Jun.~2003 \\
		   & Others & 102 & \$135,430.4 & 63.6\% & \\
		   & Total Market & 615 & \$213,486.7 & 100.0\% & \\
	\end{booktabs}
	\begin{tablenotes}
		Adapted from \url{https://tex.stackexchange.com/a/373932/156280}.
	\end{tablenotes}
\end{table}

\paragraph{The Seven Versalia}
She packed her seven versalia, put her initial into the belt and made herself on the way. When she reached the first hills of the Italic Mountains, she had a~last view back on the skyline of her hometown Bookmarksgrove, the headline of Alphabet Village and the subline of her own road, the Line Lane. Pityful a~rethoric question ran over her cheek, then she continued her way. On her way she met a~copy \citep{shore}.

\subparagraph{Warning.} The copy warned the Little Blind Text, that where it came from it would have been rewritten a~thousand times and everything that was left from its origin would be the word ``and'' and the Little Blind Text should turn around and return to its own, safe country. But nothing the copy said could convince her and so it didn't take long until a~few insidious Copy Writers ambushed her, made her drunk with Longe and Parole and dragged her into their agency, where they abused her for their projects again and again. And if she hasn't been rewritten, then they are still using her.

\subparagraph{Vokalia and Consonantia.} Far far away, behind the word mountains, far from the countries Vokalia and Consonantia, there live the blind texts. Separated they live in Bookmarksgrove right at the coast of the Semantics, a~large language ocean. A~small river named Duden flows by their place and supplies it with the necessary regelialia. It is a~paradisematic country, in which roasted parts of sentences fly into your mouth.

Even the all-powerful Pointing has no control about the blind texts it is an almost unorthographic life One day however a~small line of blind text by the name of Lorem Ipsum decided to leave for the far World of Grammar. The Big Oxmox advised her not to do so, because there were thousands of bad Commas, wild Question Marks and devious Semikoli, but the Little Blind Text didn't listen. She packed her seven versalia, put her initial into the belt and made herself on the way.

\subsubsection{The End}
When she reached the first hills of the Italic Mountains, she had a~last view back on the skyline of her hometown Bookmarksgrove, the headline of Alphabet Village and the subline of her own road, the Line Lane. Pityful a~rethoric question ran over her cheek, then she continued her way. On her way she met a~copy. The copy warned the Little Blind Text, that where it came from it would have been rewritten a~thousand times and everything that was left from its origin would be the word ``and'' and the Little Blind Text should turn around and return to its own, safe country. But nothing the copy said could convince her and so it didn't take long until a~few insidious Copy Writers ambushed her, made her drunk with Longe and Parole and dragged her into their agency, where they abused her for their projects again and again. And if she hasn't been rewritten, then they are still using her. Far far away, behind the word mountains, far from the countries Vokalia and Consonantia, there live the blind texts. Separated they live in Bookmarksgrove right at the coast of the Semantics, a~large language ocean.%
\displaybaselinegrid

\begin{table}
	\caption{An additional table to check the horizontal spacing of columns}
	\begin{booktabs}{X}
		\begin{tabular*}{\textwidth}{@{\extracolsep{\fill}} *{8}{l} @{}}
			Test & Test 1 & Test 12 & Test 123 & Test 1234 & Test 12345 & Test 123456 & Test 1234567
		\end{tabular*}
	\end{booktabs}
	\begin{tablenotes}
		One can nest traditional environments (e.g., \verb|tabular|, \verb|tabular*|) in \mbox{\textit{tabularray}} environments. This way, the outer formatting can be kept consistent (e.g., automatically adding \verb|\toprule| and \verb|\bottomrule|), while at the same time, formatting via the traditional environments can be used (e.g., \verb|@{\extracolsep{\fill}}|).
	\end{tablenotes}
\end{table}

\begin{table}[b]
	\caption{A~table to illustrate rounding with the help of the \mbox{\textit{siunitx}} package}
	\begin{booktabs}{
		colspec = {X r c l S[table-format = 11.4, round-precision = 4, round-mode = places]},
		hborder{2, 4} = {abovespace = \aboverulesep, belowspace = \belowrulesep},
		hline{2} = {\lightrulewidth, solid},
		hline{4} = {2-Z}{\lightrulewidth, solid, endpos = true, leftpos = -1},
		width = \textwidth,
	}
		Expandable column (\texttt{X}) & right-aligned & center-aligned & left-aligned & {\texttt{S} (from \mbox{\textit{siunitx}})} \\
		Some text & 1.23 & 4.56 & \phantom{0}7.89 & 9012345678.90123456789 \\
		Some text & 1.23 & 4.56 & \phantom{0}7.89 & 9012345678.90123456789 \\
		Sum	& 2.46 & 9.12 & 15.78 & 18024691357.80246913578 \\
	\end{booktabs}
	\begin{tablenotes}[Notes]
		We place this table at the bottom of the page because the top is already occupied by another table. \par
		Check the source code of this table to see that the \texttt{S}~column type provided by the \mbox{\textit{siunitx}} package takes care of the rounding in the right-most column.
	\end{tablenotes}
\end{table}

A small river named Duden flows by their place and supplies it with the necessary regelialia. It is a~paradisematic country, in which roasted parts of sentences fly into your mouth. Even the all-powerful Pointing has no control about the blind texts it is an almost unorthographic life One day however a~small line of blind text by the name of Lorem Ipsum decided to leave for the far World of Grammar. The Big Oxmox advised her not to do so \citep[see, e.g.,][]{baez/article}, because there were thousands of bad Commas, wild Question Marks and devious Semikoli, but the Little Blind Text didn't listen. She packed her seven versalia, put her initial into the belt and made herself on the way.

When she reached the first hills of the Italic Mountains, she had a~last view back on the skyline of her hometown Bookmarksgrove, the headline of Alphabet Village and the subline of her own road, the Line Lane. Pityful a~rethoric question ran over her cheek, then she continued her way. On her way she met a~copy. The copy warned the Little Blind Text, that where it came from it would have been rewritten a~thousand times and everything that was left from its origin would be the word ``and'' and the Little Blind Text should turn around and return to its own, safe country.

\begin{table}
	\begin{booktabs}[
		caption = {%
			A~\mbox{\textit{tabularray}}-based table with natural column widths and a~total width of less than \texttt{\textbackslash textwidth}%
		},
		label = {fig:test},
		headsep = 10pt,  % See https://ftp.agdsn.de/pub/mirrors/latex/dante/macros/latex/contrib/caption/caption.pdf, Section 2.6
		long,
		postsep = 0pt,
		presep = 0pt,  % No additional space, since we are within a table environment
		remark{Notes} = {%
			\footnotesize\strut As this example shows, \mbox{\textit{tabularray}} can produce table titles and notes whose width is identical to that of the table's body, just as the \mbox{\textit{threeparttable}} package would produce them. So, no \mbox{\textit{threeparttable}} needed, \mbox{\textit{tabularray}} does it all.%
		},
		remark{} = {%
			You might agree, however, that tables which have the same width as the body text look better in most cases.
			\smallskip%
		},
		remark{Source} = {%
			You can combine several table notes, for instance, to indicate the source(s) of the data that you are presenting or analyzing.%
		},
	]{
		colspec = {*{8}{l}},
		column{1} = {leftsep = 0pt},
		column{Z} = {rightsep = 0pt},
		hborder{2} = {abovespace = \aboverulesep, belowspace = \belowrulesep},
		hline{2} = {\lightrulewidth, solid},
		row{2} = {valign = m},
	}
		Test & Test 1 & Test 12 & Test 123 & Test 1234 & Test 12345 & Test 123456 \\
		$f(x)$ & $\displaystyle\frac{\sqrt{3}}{n}$ & Test & \SetCell[c=3]{c} Content spanning multiple cells & & & {Row~1 \\ Row~2}
	\end{booktabs}
\end{table}


%%  BIBLIOGRAPHY  %%

\phantomsection
\begin{refcontext}[sorting=nyt]  % Sort BIBLIOGRAPHY by alphabet (while CITATIONS are sorted by year)
	\printbibliography[heading=bibintoc]
\end{refcontext}


%%  APPENDIX: TESTING MATH FONTS


\begin{appendix}

\section{Additional Proofs}
\label{app:proofs}

\begin{proof}[Proof of \autoref{lemma:budde-goex-1}]
	Since expected profit is linear in $k_i$, the firm chooses the maximum capacity level ${k = 1}$ when (3) is positive and the minimum capacity ${k_i = 0}$ when (3) is negative. Taking into account that in equilibrium the condition ${B^{-1}(b_i^*) = c_i}$ must hold for all~$i$, the critical value~$\hat{c}$ for the indifferent bidder is obtained by solving~(3) for~$c_i$.
\end{proof}

\clearpage

\raggedbottom

\section{Example of a~Long Table That Spans More Than One Page}

\newcommand{\pz}{\phantom{0}}
\def\tablecapstrut{\vrule height\baselinedist depth1ex width0pt}

\begingroup

\DeclareTblrTemplate{contfoot-text}{normal}{\strut Continued on next page.}
\SetTblrStyle{contfoot-text}{font = \itshape\footnotesize}
\DeclareTblrTemplate{contfoot}{leftaligned}{
	\noindent\raggedright%
	\UseTblrTemplate{contfoot-text}{normal} \par
}
\SetTblrTemplate{contfoot}{leftaligned}
\DeclareTblrTemplate{conthead-text}{normal}{\tablecapstrut(continued)}
\SetTblrTemplate{conthead-text}{normal}
\DeclareTblrTemplate{capcont}{default}{%
	\UseTblrTemplate{caption-tag}{default}
	\UseTblrTemplate{conthead-text}{default}%
}
\begin{booktabs}[
	caption = {Parameter values used in the time experiment},
	label = {tab:parameter_values_time},
	footsep = 2\belowbottomsep,
	headsep = 10pt,  % See https://ftp.agdsn.de/pub/mirrors/latex/dante/macros/latex/contrib/caption/caption.pdf, Section 2.6
	long,
	presep = \smallskipamount,  %\textfloatsep,
	postsep = \medskipamount,  %\textfloatsep,
	remark{Note} = {Taken from \url{https://www.econtribute.de/RePEc/ajk/ajkdps/ECONtribute_336_2024.pdf}.},
]{
	cells = {font = \footnotesize},
	colspec = {X *{7}{l}},
	column{2-6} = {rightsep+ = -1pt},
	column{7} = {rightsep+ = 12pt},
	hborder{1-Z} = {abovespace = 0pt, belowspace = 0pt},
	hborder{1} = {abovespace = \abovetopsep, belowspace = \belowrulesep},
	hborder{Z} = {abovespace = \aboverulesep, belowspace = \belowbottomsep},
	hborder{2, 36, 38, 74, 79} = {abovespace = \aboverulesep, belowspace = \belowrulesep},
	hline{2, 38, 79} = {\lightrulewidth, solid},
	rowhead = 1,
	%rows = {ht = 0.765\baselinedist},
	rows = {ht = 0.0216\textheight},
	stretch = 0,  % Reduce row height slightly
}
	{\strut Decision \\ \strut situation \\ \strut \#} &
	{Base payment \\ at earlier date \\ $b_t$ (\texteuro)} &
	{Base payment \\ at later date \\ $b_{t+k}$ (\texteuro)} &
	{Curvature \\ ~ \\ $z$} &
	{Front-end \\ delay \\ $t$ (weeks)} &
	{Delay \\ ~ \\ $k$ (weeks)} &
	{Budget \\ ~ \\ $m$ (\texteuro)} &
	{Price ratio \\ ~ \\ $p_{t+k} \divslash p_t = 1 \divslash R$} \\
	\pz1	&	1.50	&	1.50	&	0.0	&	1	&	\pz5	&	\pz20.00	&	1.42857	\\
	\pz2	&	1.50	&	1.50	&	0.0	&	1	&	\pz5	&	\pz17.50	&	1.25000	\\
	\pz3	&	1.50	&	1.50	&	0.0	&	1	&	\pz5	&	\pz15.56	&	1.11111	\\
	\pz4	&	1.50	&	1.50	&	0.0	&	1	&	\pz5	&	\pz14.70	&	1.05000	\\
	\pz5	&	1.50	&	1.50	&	0.0	&	1	&	\pz5	&	\pz14.00	&	1.00000	\\
	\pz6	&	1.50	&	1.50	&	0.0	&	1	&	\pz5	&	\pz14.00	&	0.95238	\\
	\pz7	&	1.50	&	1.50	&	0.0	&	1	&	\pz5	&	\pz14.00	&	0.90000	\\
	\pz8	&	1.50	&	1.50	&	0.0	&	1	&	\pz5	&	\pz14.00	&	0.80000	\\
	\pz9	&	1.50	&	1.50	&	0.0	&	1	&	\pz5	&	\pz14.00	&	0.70000	\\
	10	&	1.50	&	1.50	&	0.0	&	1	&	\pz5	&	\pz25.00	&	1.25000	\\
	11	&	1.50	&	1.50	&	0.0	&	1	&	\pz5	&	\pz21.00	&	1.05000	\\
	12	&	1.50	&	1.50	&	0.0	&	1	&	\pz5	&	\pz20.00	&	1.00000	\\
	13	&	1.50	&	1.50	&	0.0	&	1	&	\pz5	&	\pz20.00	&	0.95238	\\
	14	&	1.50	&	1.50	&	0.0	&	1	&	\pz5	&	\pz20.00	&	0.80000	\\
	15	&	1.50	&	1.50	&	0.0	&	1	&	10	&	\pz20.00	&	1.42857	\\
	16	&	1.50	&	1.50	&	0.0	&	1	&	10	&	\pz17.50	&	1.25000	\\
	17	&	1.50	&	1.50	&	0.0	&	1	&	10	&	\pz15.56	&	1.11111	\\
	18	&	1.50	&	1.50	&	0.0	&	1	&	10	&	\pz14.70	&	1.05000	\\
	19	&	1.50	&	1.50	&	0.0	&	1	&	10	&	\pz14.00	&	1.00000	\\
	20	&	1.50	&	1.50	&	0.0	&	1	&	10	&	\pz14.00	&	0.95238	\\
	21	&	1.50	&	1.50	&	0.0	&	1	&	10	&	\pz14.00	&	0.90000	\\
	22	&	1.50	&	1.50	&	0.0	&	1	&	10	&	\pz14.00	&	0.80000	\\
	23	&	1.50	&	1.50	&	0.0	&	1	&	10	&	\pz14.00	&	0.70000	\\
	24	&	1.50	&	1.50	&	0.0	&	1	&	10	&	\pz25.00	&	1.25000	\\
	25	&	1.50	&	1.50	&	0.0	&	1	&	10	&	\pz21.00	&	1.05000	\\
	26	&	1.50	&	1.50	&	0.0	&	1	&	10	&	\pz20.00	&	1.00000	\\
	27	&	1.50	&	1.50	&	0.0	&	1	&	10	&	\pz20.00	&	0.95238	\\
	28	&	1.50	&	1.50	&	0.0	&	1	&	10	&	\pz20.00	&	0.80000	\\
	29	&	1.50	&	1.50	&	0.0	&	0	&	\pz5	&	\pz14.70	&	1.05000	\\
	30	&	1.50	&	1.50	&	0.0	&	0	&	\pz5	&	\pz14.00	&	0.95238	\\
	31	&	1.50	&	1.50	&	0.0	&	0	&	\pz5	&	\pz21.00	&	1.05000	\\
	32	&	1.50	&	1.50	&	0.0	&	0	&	\pz5	&	\pz20.00	&	0.95238	\\
	33	&	1.50	&	1.50	&	0.0	&	0	&	10	&	\pz14.70	&	1.05000	\\
	34	&	1.50	&	1.50	&	0.0	&	0	&	10	&	\pz14.00	&	0.95238	\\
	35	&	1.50	&	1.50	&	0.0	&	0	&	10	&	\pz21.00	&	1.05000	\\
	36	&	1.50	&	1.50	&	0.0	&	0	&	10	&	\pz20.00	&	0.95238	\\
	37	&	1.50	&	1.50	&	0.4	&	1	&	\pz5	&	\pz20.00	&	1.42857	\\
	38	&	1.50	&	1.50	&	0.4	&	1	&	\pz5	&	\pz17.50	&	1.25000	\\
	39	&	1.50	&	1.50	&	0.4	&	1	&	\pz5	&	\pz15.56	&	1.11111	\\
	40	&	1.50	&	1.50	&	0.4	&	1	&	\pz5	&	\pz14.70	&	1.05000	\\
	41	&	1.50	&	1.50	&	0.4	&	1	&	\pz5	&	\pz14.00	&	1.00000	\\
	42	&	1.50	&	1.50	&	0.4	&	1	&	\pz5	&	\pz14.00	&	0.95238	\\
	43	&	1.50	&	1.50	&	0.4	&	1	&	\pz5	&	\pz14.00	&	0.90000	\\
	44	&	1.50	&	1.50	&	0.4	&	1	&	\pz5	&	\pz14.00	&	0.80000	\\
	45	&	1.50	&	1.50	&	0.4	&	1	&	\pz5	&	\pz14.00	&	0.70000	\\
	46	&	1.50	&	1.50	&	0.4	&	1	&	\pz5	&	\pz25.00	&	1.25000	\\
	47	&	1.50	&	1.50	&	0.4	&	1	&	\pz5	&	\pz21.00	&	1.05000	\\
	48	&	1.50	&	1.50	&	0.4	&	1	&	\pz5	&	\pz20.00	&	1.00000	\\
	49	&	1.50	&	1.50	&	0.4	&	1	&	\pz5	&	\pz20.00	&	0.95238	\\
	50	&	1.50	&	1.50	&	0.4	&	1	&	\pz5	&	\pz20.00	&	0.80000	\\
	51	&	1.50	&	1.50	&	0.4	&	1	&	10	&	\pz20.00	&	1.42857	\\
	52	&	1.50	&	1.50	&	0.4	&	1	&	10	&	\pz17.50	&	1.25000	\\
	53	&	1.50	&	1.50	&	0.4	&	1	&	10	&	\pz15.56	&	1.11111	\\
	54	&	1.50	&	1.50	&	0.4	&	1	&	10	&	\pz14.70	&	1.05000	\\
	55	&	1.50	&	1.50	&	0.4	&	1	&	10	&	\pz14.00	&	1.00000	\\
	56	&	1.50	&	1.50	&	0.4	&	1	&	10	&	\pz14.00	&	0.95238	\\
	57	&	1.50	&	1.50	&	0.4	&	1	&	10	&	\pz14.00	&	0.90000	\\
	58	&	1.50	&	1.50	&	0.4	&	1	&	10	&	\pz14.00	&	0.80000	\\
	59	&	1.50	&	1.50	&	0.4	&	1	&	10	&	\pz14.00	&	0.70000	\\
	60	&	1.50	&	1.50	&	0.4	&	1	&	10	&	\pz25.00	&	1.25000	\\
	61	&	1.50	&	1.50	&	0.4	&	1	&	10	&	\pz21.00	&	1.05000	\\
	62	&	1.50	&	1.50	&	0.4	&	1	&	10	&	\pz20.00	&	1.00000	\\
	63	&	1.50	&	1.50	&	0.4	&	1	&	10	&	\pz20.00	&	0.95238	\\
	64	&	1.50	&	1.50	&	0.4	&	1	&	10	&	\pz20.00	&	0.80000	\\
	65	&	1.50	&	1.50	&	0.4	&	0	&	\pz5	&	\pz14.70	&	1.05000	\\
	66	&	1.50	&	1.50	&	0.4	&	0	&	\pz5	&	\pz14.00	&	0.95238	\\
	67	&	1.50	&	1.50	&	0.4	&	0	&	\pz5	&	\pz21.00	&	1.05000	\\
	68	&	1.50	&	1.50	&	0.4	&	0	&	\pz5	&	\pz20.00	&	0.95238	\\
	69	&	1.50	&	1.50	&	0.4	&	0	&	10	&	\pz14.70	&	1.05000	\\
	70	&	1.50	&	1.50	&	0.4	&	0	&	10	&	\pz14.00	&	0.95238	\\
	71	&	1.50	&	1.50	&	0.4	&	0	&	10	&	\pz21.00	&	1.05000	\\
	72	&	1.50	&	1.50	&	0.4	&	0	&	10	&	\pz20.00	&	0.95238	\\
	73	&	1.50	&	1.50	&	0.0	&	1	&	10	&	150.00	&	1.25000	\\
	74	&	1.50	&	1.50	&	0.0	&	1	&	10	&	126.00	&	1.05000	\\
	75	&	1.50	&	1.50	&	0.0	&	1	&	10	&	120.00	&	1.00000	\\
	76	&	1.50	&	1.50	&	0.0	&	1	&	10	&	120.00	&	0.95238	\\
	77	&	1.50	&	1.50	&	0.0	&	1	&	10	&	120.00	&	0.80000	\\
	78	&	1.50	&	1.50	&	0.4	&	1	&	10	&	150.00	&	1.25000	\\
	79	&	1.50	&	1.50	&	0.4	&	1	&	10	&	126.00	&	1.05000	\\
	80	&	1.50	&	1.50	&	0.4	&	1	&	10	&	120.00	&	1.00000	\\
	81	&	1.50	&	1.50	&	0.4	&	1	&	10	&	120.00	&	0.95238	\\
	82	&	1.50	&	1.50	&	0.4	&	1	&	10	&	120.00	&	0.80000	\\
\end{booktabs}

\endgroup

\begin{table}[h]
	\caption{An additional table to check the \texttt{table} counter increment and the vertical spacing}
	\begin{booktabs}{X}
		\begin{tabular*}{\textwidth}{@{\extracolsep{\fill}} *{8}{l} @{}}
			Test & Test 1 & Test 12 & Test 123 & Test 1234 & Test 12345 & Test 123456 & Test 1234567
		\end{tabular*}
	\end{booktabs}
\end{table}


\section{An Example Code Listing}

Taken from \url{https://holgergerhardt.github.io/scbr/2024-10-13_estimation-illustration.R}:

\begin{lstlisting}
# This file illustrates the most important parts of the estimation of
# preference parameters performed by Holger Gerhardt & Rafael Suchy (2024),
# “Estimating Preference Parameters from Strictly Concave Budget Restrictions,”
# https://www.econtribute.de/RePEc/ajk/ajkdps/ECONtribute_336_2024.pdf.

# Version: 2024-10-13
# The results reported in this file were produced using R version 4.4.1
# (https://cloud.r-project.org/bin/macosx/big-sur-arm64/base/R-4.4.1-arm64.pkg)
# and the up-to-date versions of the packages mentioned below.




# PREAMBLE ------------------------------------------------------------------------------------


# Clean up environment
rm(list = ls())

options(max.print = 9999)  # Increase limit for omitting entries in output
options(scipen = 999)  # Increase limit of using scientific notation
round_prec <- 4  # Number of decimal places for reporting the parameter estimates

# Required packages
packages_required <- c(
  "AER",  # For "tobit() function"
  "cli",  # For colored error and warning messages
  "msm",  # For "deltamethod() function"
  "maxLik"  # For maximum likelihood estimation
)
# Install packages that are not installed yet
install.packages(setdiff(packages_required, rownames(installed.packages())))
# For updating already installed packages, use the following
# install.packages(packages_required)
# Load the required packages
for (name in packages_required) {
  library(name, character.only = TRUE)
}

# Alternatively, use groundhog:
# install.packages("groundhog")
# library("groundhog")
# groundhog.library(packages_required, "2024-10-10")




# BUDGET RESTRICTIONS -------------------------------------------------------------------------


# General functional form, based on Equation (9) from
# https://www.econtribute.de/RePEc/ajk/ajkdps/ECONtribute_336_2024.pdf:
# c_{t}^{1 + z} + (1 / R)^{1 + z} c_{t + k}^{1 + z} = m^{1 + z}.
# Thus, linear budget restrictions (LBRs) for z = 0, and
# strictly concave budget restrictions (SCBRs) for z > 0.
c_2 <- function(c_1, m, R, z, b_1, b_2) {
  R * (m^(1 + z) - c_1^(1 + z))^(1 / (1 + z))
}

# Experimental parameters
# (see Table E.1 in https://www.econtribute.de/RePEc/ajk/ajkdps/ECONtribute_336_2024.pdf)
# The presentation of the different budget restrictions in the experiment by Gerhardt & Suchy
# was randomized as described in the manuscript. We abstract from this randomization here.
budget_restritions <- matrix(c(
  c(01, 1.50, 1.50, 0.0, 1, 05, 020.00, 0.70000, 1.42857, 1, 1, 0.01, 02.5, 025),
  c(02, 1.50, 1.50, 0.0, 1, 05, 017.50, 0.80000, 1.25000, 1, 1, 0.01, 02.5, 025),
  c(03, 1.50, 1.50, 0.0, 1, 05, 015.56, 0.90000, 1.11111, 1, 1, 0.01, 02.5, 025),
  c(04, 1.50, 1.50, 0.0, 1, 05, 014.70, 0.95238, 1.05000, 1, 1, 0.01, 02.5, 025),
  c(05, 1.50, 1.50, 0.0, 1, 05, 014.00, 1.00000, 1.00000, 1, 1, 0.01, 02.5, 025),
  c(06, 1.50, 1.50, 0.0, 1, 05, 014.00, 1.05000, 0.95238, 1, 1, 0.01, 02.5, 025),
  c(07, 1.50, 1.50, 0.0, 1, 05, 014.00, 1.11111, 0.90000, 1, 1, 0.01, 02.5, 025),
  c(08, 1.50, 1.50, 0.0, 1, 05, 014.00, 1.25000, 0.80000, 1, 1, 0.01, 02.5, 025),
  c(09, 1.50, 1.50, 0.0, 1, 05, 014.00, 1.42857, 0.70000, 1, 1, 0.01, 02.5, 025),
  c(10, 1.50, 1.50, 0.0, 1, 05, 025.00, 0.80000, 1.25000, 1, 1, 0.01, 02.5, 025),
  c(11, 1.50, 1.50, 0.0, 1, 05, 021.00, 0.95238, 1.05000, 1, 1, 0.01, 02.5, 025),
  c(12, 1.50, 1.50, 0.0, 1, 05, 020.00, 1.00000, 1.00000, 1, 1, 0.01, 02.5, 025),
  c(13, 1.50, 1.50, 0.0, 1, 05, 020.00, 1.05000, 0.95238, 1, 1, 0.01, 02.5, 025),
  c(14, 1.50, 1.50, 0.0, 1, 05, 020.00, 1.25000, 0.80000, 1, 1, 0.01, 02.5, 025),
  c(15, 1.50, 1.50, 0.0, 1, 10, 020.00, 0.70000, 1.42857, 1, 1, 0.01, 02.5, 025),
  c(16, 1.50, 1.50, 0.0, 1, 10, 017.50, 0.80000, 1.25000, 1, 1, 0.01, 02.5, 025),
  c(17, 1.50, 1.50, 0.0, 1, 10, 015.56, 0.90000, 1.11111, 1, 1, 0.01, 02.5, 025),
  c(18, 1.50, 1.50, 0.0, 1, 10, 014.70, 0.95238, 1.05000, 1, 1, 0.01, 02.5, 025),
  c(19, 1.50, 1.50, 0.0, 1, 10, 014.00, 1.00000, 1.00000, 1, 1, 0.01, 02.5, 025),
  c(20, 1.50, 1.50, 0.0, 1, 10, 014.00, 1.05000, 0.95238, 1, 1, 0.01, 02.5, 025),
  c(21, 1.50, 1.50, 0.0, 1, 10, 014.00, 1.11111, 0.90000, 1, 1, 0.01, 02.5, 025),
  c(22, 1.50, 1.50, 0.0, 1, 10, 014.00, 1.25000, 0.80000, 1, 1, 0.01, 02.5, 025),
  c(23, 1.50, 1.50, 0.0, 1, 10, 014.00, 1.42857, 0.70000, 1, 1, 0.01, 02.5, 025),
  c(24, 1.50, 1.50, 0.0, 1, 10, 025.00, 0.80000, 1.25000, 1, 1, 0.01, 02.5, 025),
  c(25, 1.50, 1.50, 0.0, 1, 10, 021.00, 0.95238, 1.05000, 1, 1, 0.01, 02.5, 025),
  c(26, 1.50, 1.50, 0.0, 1, 10, 020.00, 1.00000, 1.00000, 1, 1, 0.01, 02.5, 025),
  c(27, 1.50, 1.50, 0.0, 1, 10, 020.00, 1.05000, 0.95238, 1, 1, 0.01, 02.5, 025),
  c(28, 1.50, 1.50, 0.0, 1, 10, 020.00, 1.25000, 0.80000, 1, 1, 0.01, 02.5, 025),
  c(29, 1.50, 1.50, 0.0, 0, 05, 014.70, 0.95238, 1.05000, 1, 1, 0.01, 02.5, 025),
  c(30, 1.50, 1.50, 0.0, 0, 05, 014.00, 1.05000, 0.95238, 1, 1, 0.01, 02.5, 025),
  c(31, 1.50, 1.50, 0.0, 0, 05, 021.00, 0.95238, 1.05000, 1, 1, 0.01, 02.5, 025),
  c(32, 1.50, 1.50, 0.0, 0, 05, 020.00, 1.05000, 0.95238, 1, 1, 0.01, 02.5, 025),
  c(33, 1.50, 1.50, 0.0, 0, 10, 014.70, 0.95238, 1.05000, 1, 1, 0.01, 02.5, 025),
  c(34, 1.50, 1.50, 0.0, 0, 10, 014.00, 1.05000, 0.95238, 1, 1, 0.01, 02.5, 025),
  c(35, 1.50, 1.50, 0.0, 0, 10, 021.00, 0.95238, 1.05000, 1, 1, 0.01, 02.5, 025),
  c(36, 1.50, 1.50, 0.0, 0, 10, 020.00, 1.05000, 0.95238, 1, 1, 0.01, 02.5, 025),
  c(37, 1.50, 1.50, 0.4, 1, 05, 020.00, 0.70000, 1.42857, 1, 2, 0.01, 02.5, 025),
  c(38, 1.50, 1.50, 0.4, 1, 05, 017.50, 0.80000, 1.25000, 1, 2, 0.01, 02.5, 025),
  c(39, 1.50, 1.50, 0.4, 1, 05, 015.56, 0.90000, 1.11111, 1, 2, 0.01, 02.5, 025),
  c(40, 1.50, 1.50, 0.4, 1, 05, 014.70, 0.95238, 1.05000, 1, 2, 0.01, 02.5, 025),
  c(41, 1.50, 1.50, 0.4, 1, 05, 014.00, 1.00000, 1.00000, 1, 2, 0.01, 02.5, 025),
  c(42, 1.50, 1.50, 0.4, 1, 05, 014.00, 1.05000, 0.95238, 1, 2, 0.01, 02.5, 025),
  c(43, 1.50, 1.50, 0.4, 1, 05, 014.00, 1.11111, 0.90000, 1, 2, 0.01, 02.5, 025),
  c(44, 1.50, 1.50, 0.4, 1, 05, 014.00, 1.25000, 0.80000, 1, 2, 0.01, 02.5, 025),
  c(45, 1.50, 1.50, 0.4, 1, 05, 014.00, 1.42857, 0.70000, 1, 2, 0.01, 02.5, 025),
  c(46, 1.50, 1.50, 0.4, 1, 05, 025.00, 0.80000, 1.25000, 1, 2, 0.01, 02.5, 025),
  c(47, 1.50, 1.50, 0.4, 1, 05, 021.00, 0.95238, 1.05000, 1, 2, 0.01, 02.5, 025),
  c(48, 1.50, 1.50, 0.4, 1, 05, 020.00, 1.00000, 1.00000, 1, 2, 0.01, 02.5, 025),
  c(49, 1.50, 1.50, 0.4, 1, 05, 020.00, 1.05000, 0.95238, 1, 2, 0.01, 02.5, 025),
  c(50, 1.50, 1.50, 0.4, 1, 05, 020.00, 1.25000, 0.80000, 1, 2, 0.01, 02.5, 025),
  c(51, 1.50, 1.50, 0.4, 1, 10, 020.00, 0.70000, 1.42857, 1, 2, 0.01, 02.5, 025),
  c(52, 1.50, 1.50, 0.4, 1, 10, 017.50, 0.80000, 1.25000, 1, 2, 0.01, 02.5, 025),
  c(53, 1.50, 1.50, 0.4, 1, 10, 015.56, 0.90000, 1.11111, 1, 2, 0.01, 02.5, 025),
  c(54, 1.50, 1.50, 0.4, 1, 10, 014.70, 0.95238, 1.05000, 1, 2, 0.01, 02.5, 025),
  c(55, 1.50, 1.50, 0.4, 1, 10, 014.00, 1.00000, 1.00000, 1, 2, 0.01, 02.5, 025),
  c(56, 1.50, 1.50, 0.4, 1, 10, 014.00, 1.05000, 0.95238, 1, 2, 0.01, 02.5, 025),
  c(57, 1.50, 1.50, 0.4, 1, 10, 014.00, 1.11111, 0.90000, 1, 2, 0.01, 02.5, 025),
  c(58, 1.50, 1.50, 0.4, 1, 10, 014.00, 1.25000, 0.80000, 1, 2, 0.01, 02.5, 025),
  c(59, 1.50, 1.50, 0.4, 1, 10, 014.00, 1.42857, 0.70000, 1, 2, 0.01, 02.5, 025),
  c(60, 1.50, 1.50, 0.4, 1, 10, 025.00, 0.80000, 1.25000, 1, 2, 0.01, 02.5, 025),
  c(61, 1.50, 1.50, 0.4, 1, 10, 021.00, 0.95238, 1.05000, 1, 2, 0.01, 02.5, 025),
  c(62, 1.50, 1.50, 0.4, 1, 10, 020.00, 1.00000, 1.00000, 1, 2, 0.01, 02.5, 025),
  c(63, 1.50, 1.50, 0.4, 1, 10, 020.00, 1.05000, 0.95238, 1, 2, 0.01, 02.5, 025),
  c(64, 1.50, 1.50, 0.4, 1, 10, 020.00, 1.25000, 0.80000, 1, 2, 0.01, 02.5, 025),
  c(65, 1.50, 1.50, 0.4, 0, 05, 014.70, 0.95238, 1.05000, 1, 2, 0.01, 02.5, 025),
  c(66, 1.50, 1.50, 0.4, 0, 05, 014.00, 1.05000, 0.95238, 1, 2, 0.01, 02.5, 025),
  c(67, 1.50, 1.50, 0.4, 0, 05, 021.00, 0.95238, 1.05000, 1, 2, 0.01, 02.5, 025),
  c(68, 1.50, 1.50, 0.4, 0, 05, 020.00, 1.05000, 0.95238, 1, 2, 0.01, 02.5, 025),
  c(69, 1.50, 1.50, 0.4, 0, 10, 014.70, 0.95238, 1.05000, 1, 2, 0.01, 02.5, 025),
  c(70, 1.50, 1.50, 0.4, 0, 10, 014.00, 1.05000, 0.95238, 1, 2, 0.01, 02.5, 025),
  c(71, 1.50, 1.50, 0.4, 0, 10, 021.00, 0.95238, 1.05000, 1, 2, 0.01, 02.5, 025),
  c(72, 1.50, 1.50, 0.4, 0, 10, 020.00, 1.05000, 0.95238, 1, 2, 0.01, 02.5, 025),
  c(73, 1.50, 1.50, 0.0, 1, 10, 150.00, 0.80000, 1.25000, 1, 3, 0.10, 10.0, 160),
  c(74, 1.50, 1.50, 0.0, 1, 10, 126.00, 0.95238, 1.05000, 1, 3, 0.10, 10.0, 160),
  c(75, 1.50, 1.50, 0.0, 1, 10, 120.00, 1.00000, 1.00000, 1, 3, 0.10, 10.0, 160),
  c(76, 1.50, 1.50, 0.0, 1, 10, 120.00, 1.05000, 0.95238, 1, 3, 0.10, 10.0, 160),
  c(77, 1.50, 1.50, 0.0, 1, 10, 120.00, 1.25000, 0.80000, 1, 3, 0.10, 10.0, 160),
  c(78, 1.50, 1.50, 0.4, 1, 10, 150.00, 0.80000, 1.25000, 1, 4, 0.10, 10.0, 160),
  c(79, 1.50, 1.50, 0.4, 1, 10, 126.00, 0.95238, 1.05000, 1, 4, 0.10, 10.0, 160),
  c(80, 1.50, 1.50, 0.4, 1, 10, 120.00, 1.00000, 1.00000, 1, 4, 0.10, 10.0, 160),
  c(81, 1.50, 1.50, 0.4, 1, 10, 120.00, 1.05000, 0.95238, 1, 4, 0.10, 10.0, 160),
  c(82, 1.50, 1.50, 0.4, 1, 10, 120.00, 1.25000, 0.80000, 1, 4, 0.10, 10.0, 160)
), nrow = 82, byrow = TRUE)
colnames(budget_restritions) <- c(
  "DecNum", "BasePayOne", "BasePayTwo", "Curvature", "FED", "Delay", "Budget",
  "PriceOneDivPriceTwo", "PriceTwoDivPriceOne", "ProbOne", "Block", "StepSize",
  "TickDist", "PlotMax"
)

# Generate data frame so that we can include all possible allocations that could be selected
# from the different budget restrictions
budget_restritions_df <- as.data.frame(budget_restritions)
# Add necessary columns, initialized with NA
budget_restritions_df$c_1_series <- list(NA)
budget_restritions_df$c_2_series <- list(NA)
budget_restritions_df$C_series <- list(NA)
budget_restritions_df$lb <- NA
budget_restritions_df$ub <- NA

# In the actual experiment, the sooner payment was displayed on the vertical axis, and the later
# payment was displayed on the horizontal axis. For convenience, we display c_1 on the horizontal
# and c_2 on the vertical axis here.

# Populate data frame
for (i in 1:dim(budget_restritions_df)[1]) {
  c_1_series <- seq(
    0, budget_restritions_df$Budget[i], budget_restritions_df$StepSize[i]
  )
  c_2_series <- round(c_2(
    c_1_series,
    budget_restritions_df$Budget[i],
    budget_restritions_df$PriceOneDivPriceTwo[i],
    budget_restritions_df$Curvature[i],
    budget_restritions_df$BasePayOne[i],
    budget_restritions_df$BasePayTwo[i]
  ), 2)  # Rounding to 2 decimal places because we displayed amounts with a precision of €0.01
  # Keep only c_1-c_2 pairs for c_2 >= baseline payment
  c_1_series <- c_1_series[c_2_series >= budget_restritions_df$BasePayTwo[i]]
  c_2_series <- c_2_series[1:length(c_1_series)]
  # Keep only c_1-c_2 pairs for c_1 >= baseline payment
  c_2_series <- c_2_series[c_1_series >= budget_restritions_df$BasePayOne[i]]
  c_1_series <- c_1_series[c_1_series >= budget_restritions_df$BasePayOne[i]]
  # Add horizontal/vertical segment at the position of the baseline payments
  c_1_series <- c(0, c_1_series, max(c_1_series))
  c_2_series <- c(max(c_2_series), c_2_series, 0)
  budget_restritions_df[[i, "c_1_series"]] <- as.list(c_1_series)
  budget_restritions_df[[i, "c_2_series"]] <- as.list(c_2_series)
  budget_restritions_df[[i, "C_series"]] <- as.list(c_1_series / c_2_series)
  budget_restritions_df[[i, "lb"]] <- budget_restritions_df[[i, "BasePayTwo"]] / max(c_2_series)
  budget_restritions_df[[i, "ub"]] <- max(c_1_series) / budget_restritions_df[[i, "BasePayTwo"]]
}
rm(c_1_series, c_2_series, i)




# PLOT BUDGET RESTRICTIONS --------------------------------------------------------------------


plotBRs <- function(df, curv) {
  df_aux <- df[df$Curvature == curv, ]
  plot_max <- max(df_aux[, "PlotMax"])
  plot(
    NaN, NaN,
    xlim = c(0, plot_max),
    ylim = c(0, plot_max),
    axes = FALSE,
    xlab = "", ylab = "",
    asp = 1,
    main = bquote("Budget restricions with curvature" ~ italic(z) ~ "=" ~ .(curv))
  )
  axis(1, seq(0, plot_max, 20), col = "gray25", col.axis = "gray25", pos = 0)
  text(x = plot_max / 2, y = -22.5, bquote(italic(c)[italic(t)]), xpd = TRUE)
  axis(2, seq(0, plot_max, 20), col = "gray25", col.axis = "gray25", pos = 0)
  text(x = -22.5, y = plot_max / 2, bquote(italic(c)[italic(t) + italic(k)]), xpd = TRUE, srt = 90)
  for (i in 1:dim(df_aux)[1]) {
    lines( unlist(df_aux[[i, "c_1_series"]]), unlist(df_aux[[i, "c_2_series"]]), col = "navy")
  }
}

for (curv in unique(budget_restritions_df$Curvature)) {
  plotBRs(budget_restritions_df, curv)
}
rm(curv)




# SIMULATE CHOICES ----------------------------------------------------------------------------


# For replicability of the results, use particular seed for the pseudorandom draws
set.seed(42)
# For new draws, remove the fixed seed via the following line
# set.seed(NULL)

# Optimal payment ratio:
# Equation (25) in https://www.econtribute.de/RePEc/ajk/ajkdps/ECONtribute_336_2024.pdf
C_star <- function(t, k, R, z, beta, delta, rho) {
  (1 / (beta^(t == 0) * delta^k * R^(1 + z)))^(1 / (rho + z))
}

# Add noise with Gaussian distribution:
# additive normally distributed noise on log(C_star) => multiplicative log-normal noise on C_star
C_star_noisy <- function(t, k, R, z, beta, delta, rho, sigma) {
  C_star(t, k, R, z, beta, delta, rho) * exp(rnorm(length(t), mean = 0, sd = sigma))
}

# Determine optimal (noisy) points on all BRs
opt_noisy_points_on_BR <- function(df_ind, beta, delta, rho, sigma) {
  ones <- rep(1, dim(df_ind)[1])
  beta_vec <- beta * ones
  delta_vec <- delta * ones
  rho_vec <- rho * ones
  sigma_vec <- sigma * ones
  # If rho is so small (negative) that the curvature of the utility function exceeds the curvature
  # of the budget restriction, replace rho by a value that is ever so slightly larger than
  # the curvature of the budget restriction. This way, the condition for an interior allocation
  # can still be applied, ideally leading to finite values (instead of NaNs), which can then be
  # converted to corner allocations.
  rho_vec[rho_vec <= -df_ind$Curvature] <-
    -df_ind$Curvature[rho_vec <= -df_ind$Curvature] + 0.000001
  C_star_noisy_list <- C_star_noisy(
    df_ind$FED,
    df_ind$Delay,
    df_ind$PriceOneDivPriceTwo,
    df_ind$Curvature,
    beta_vec, delta_vec, rho_vec, sigma_vec
  )
  # Initialize vectors with NA
  opt_noisy_C <- opt_noisy_C_idx <- opt_noisy_c_1 <- opt_noisy_c_2 <-
    rep(NA, length(C_star_noisy_list))
  # Populate vectors by iterating of the budget restrictions
  for (i in 1:length(C_star_noisy_list)) {
    # Find the point among the discrete points on the current budget restriction that is
    # closest to the (continuous) theoretical prediction
    dist <- abs(C_star_noisy_list[i] - unlist(df_ind[i, "C_series"]))
    # Use this allocation as the simulated choice
    opt_noisy_C_idx[i] <- which.min(dist)
    opt_noisy_C[i] <- unlist(df_ind[i, "C_series"])[opt_noisy_C_idx[i]]
    opt_noisy_c_1[i] <- unlist(df_ind[i, "c_1_series"])[opt_noisy_C_idx[i]]
    opt_noisy_c_2[i] <- unlist(df_ind[i, "c_2_series"])[opt_noisy_C_idx[i]]
    # Convert predicted allocation beyond the baseline payments to corner allocation
    max_C <- unlist(df_ind[i, "C_series"])[length(unlist(df_ind[i, "C_series"])) - 1]
    min_C <- unlist(df_ind[i, "C_series"])[2]
    if (opt_noisy_C[i] > max_C || is.nan(min(dist))) {
      opt_noisy_C[i] <- df_ind[[i, "ub"]]
      opt_noisy_c_1[i] <- max(unlist(df_ind[[i, "c_1_series"]]))
      opt_noisy_c_2[i] <- df_ind[[i, "BasePayTwo"]]
    }
    if (opt_noisy_C[i] < min_C) {
      opt_noisy_C[i] <- df_ind[[i, "lb"]]
      opt_noisy_c_1[i] <- df_ind[[i, "BasePayOne"]]
      opt_noisy_c_2[i] <- max(unlist(df_ind[[i, "c_2_series"]]))
    }
  }
  rm(dist)
  return(c(
    "C" = list(opt_noisy_C),
    "c_1" = list(opt_noisy_c_1),
    "c_2" = list(opt_noisy_c_2)
  ))
}

# Simulate participants with the following preference and noise parameters
# id, beta, delta, rho, sigma
params_sim <- matrix(c(
  1, 1.00, 1.00, 0.000, 0.00,  # Will be hard to estimate with z = 0 but easy with z = 0.4
  2, 1.00, 0.99, 0.000, 0.25,  # Will be hard to estimate with z = 0 but easy with z = 0.4
  3, 0.80, 0.99, 0.000, 0.25,  # Will be hard to estimate with z = 0 but easy with z = 0.4
  4, 0.80, 0.99, 0.100, 0.50,  # Should be estimable with both z = 0 and z = 0.4
  5, 0.85, 0.95, 0.150, 0.50,  # Should be estimable with both z = 0 and z = 0.4
  6, 1.00, 1.00, 1.000, 0.00,  # Should be estimable with both z = 0 and z = 0.4
  7, 1.00, 0.99, 100.0, 0.25,  # Should be estimable with both z = 0 and z = 0.4
  8, 1.00, 1.00, 200.0, 0.05,  # Should be estimable with both z = 0 and z = 0.4
  9, 1.00, 1.00, 200.0, 0.00   # May fail to converge (rho -> Inf), since c_1 = c_2 for all choices
), ncol = 5, byrow = TRUE)
colnames(params_sim) <- c("id", "beta", "delta", "rho", "sigma")

ids <- unique(params_sim[, "id"])

# Create new empty data frame
df <- budget_restritions_df[FALSE, ]

# Populate the data frame with the (noisy) choices of the simulated participants and
# plot the simulated choices
for (id in ids) {
  df_ind <- budget_restritions_df
  # Add column with IDs
  df_ind$id <- id
  # Store simulated choices
  points_sim <- opt_noisy_points_on_BR(
    df_ind,
    params_sim[params_sim[, "id"] == id, "beta"],
    params_sim[params_sim[, "id"] == id, "delta"],
    params_sim[params_sim[, "id"] == id, "rho"],
    params_sim[params_sim[, "id"] == id, "sigma"]
  )
  df_ind$payment_1 <- unlist(points_sim["c_1"])
  df_ind$payment_2 <- unlist(points_sim["c_2"])
  df_ind$payment_ratio <- unlist(points_sim["C"])
  # Append to the existing data frame
  df <- rbind(df, df_ind)
  # Generate a plot per level of curvature of the budget restrictions
  for (curv in unique(df_ind$Curvature)) {
    plotBRs(df_ind, curv)
    points(
      x = unlist(points_sim["c_1"])[df_ind$Curvature == curv],
      y = unlist(points_sim["c_2"])[df_ind$Curvature == curv],
      col = "navy",
      pch = 20,
    )
    mtext(bquote(
      "Simulated choices for ID" ~ .(id) * ":" ~
        italic(β) ~ "=" ~ .(params_sim[params_sim[, "id"] == id, "beta"]) * "," ~
        italic(δ) ~ "=" ~ .(params_sim[params_sim[, "id"] == id, "delta"]) * "," ~
        italic(ρ) ~ "=" ~ .(params_sim[params_sim[, "id"] == id, "rho"]) * "," ~
        italic(σ) ~ "=" ~ .(params_sim[params_sim[, "id"] == id, "sigma"])
    ), side = 3, col = "navy", line = 0.25)
    Sys.sleep(0.5)
  }
}
rm(curv, df_ind, id)
rm(points_sim)




# TOBIT ESTIMATION ----------------------------------------------------------------------------


# Generating the explanatory variables, see
# https://www.econtribute.de/RePEc/ajk/ajkdps/ECONtribute_336_2024.pdf, eq. (26) on p. 15:
df$cov_1 <-
  -as.integer(df$FED == 0)  # -I[t = 0], coefficient: gamma_beta
df$cov_2 <-
  -df$Delay  # -k, coefficient: gamma_delta
df$cov_3 <-
  -(1 + df$Curvature) * log(df$PriceOneDivPriceTwo)  # -(1 + z) ln(R), coefficient: gamma_rho

ids <- unique(df$id)
tobit_estimates_report <- list()

# Estimate separately for each curvature level of the budget restrictions
for (curv in unique(df$Curvature)) {
  # Initialize collection of estimates with NA
  tobit_estimates <- matrix(rep(NA, length(ids) * 5), ncol = 5)
  tobit_estimates <- cbind(ids, tobit_estimates)
  colnames(tobit_estimates) <- c("id", "beta", "delta", "rho", "sigma", "logL")
  # Estimate each individual separately
  for (id in ids) {
    print(paste0("Tobit estimation, Subject ID: ", id, "; BR curvature: ", curv))
    df_ind <- df[df$id == id & df$Curvature == curv, ]
    # Plot simulated choices
    plotBRs(df_ind, curv)
    points(
      df_ind$payment_1,
      df_ind$payment_2,
      col = "navy",
      pch = 20
    )
    mtext(bquote(
      "Simulated choices for ID" ~ .(id) * ":" ~
        italic(β) ~ "=" ~ .(params_sim[params_sim[, "id"] == id, "beta"]) * "," ~
        italic(δ) ~ "=" ~ .(params_sim[params_sim[, "id"] == id, "delta"]) * "," ~
        italic(ρ) ~ "=" ~ .(params_sim[params_sim[, "id"] == id, "rho"]) * "," ~
        italic(σ) ~ "=" ~ .(params_sim[params_sim[, "id"] == id, "sigma"])
    ), side = 3, col = "navy", line = 0.25)
    tryCatch(
      {
        withCallingHandlers(
          # Attempt Tobit regression
          {
            model <- tobit(
              log(payment_ratio) ~ cov_1 + cov_2 + cov_3 - 1,
              # "+" here because we included the minus sign above when creating the regressors
              left = log(df_ind$lb),
              right = log(df_ind$ub),
              data = df_ind
            )
          },
          # Show potential warning messages of the Tobit regression
          warning = function(w) {
            message(style_bold(col_blue("WARNING: ", conditionMessage(w))))
            invokeRestart("muffleWarning")
          }
        )
        vcov_matrix <- vcov(model)
        coeffs <- model$coefficients
        est_beta <-
          round(exp(as.numeric(coeffs[["cov_1"]]) / as.numeric(coeffs[["cov_3"]])), round_prec)
          # beta = exp(gamma_beta / gamma_rho), eq. (27) in Gerhardt & Suchy (2024)
        est_delta <-
          round(exp(as.numeric(coeffs[["cov_2"]]) / as.numeric(coeffs[["cov_3"]])), round_prec)
          # delta = exp(gamma_delta / gamma_rho), eq. (28)
        est_rho <-
          round(1 / as.numeric(coeffs[["cov_3"]]) - curv, round_prec)
          # rho = (1 / gamma_rho) - z, eq. (29)
        est_sigma <- round(model$scale, round_prec)
        tobit_estimates[tobit_estimates[, "id"] == id, "beta"] <- est_beta
        tobit_estimates[tobit_estimates[, "id"] == id, "delta"] <- est_delta
        tobit_estimates[tobit_estimates[, "id"] == id, "rho"] <- est_rho
        tobit_estimates[tobit_estimates[, "id"] == id, "sigma"] <- est_sigma
        tobit_estimates[tobit_estimates[, "id"] == id, "logL"] <- round(logLik(model), round_prec)
        # Use delta method to calculate standard errors of structural parameters
        est_beta_se <- deltamethod(list(~ exp(x1 / x3)), coeffs, vcov(model)[1:3, 1:3])
        est_delta_se <- deltamethod(list(~ exp(x2 / x3)), coeffs, vcov(model)[1:3, 1:3])
        est_rho_se <- deltamethod(list(~ 1 / x3 - curv), coeffs, vcov(model)[1:3, 1:3])
        # Add best-fitting allocations to plot (remove random component, i.e., sigma = 0)
        points(
          unlist(opt_noisy_points_on_BR(df_ind, est_beta, est_delta, est_rho, sigma = 0)["c_1"]),
          unlist(opt_noisy_points_on_BR(df_ind, est_beta, est_delta, est_rho, sigma = 0)["c_2"]),
          col = "#FFA50099",
          pch = 20
        )
        mtext(bquote(
          "Parameter estimates:" ~
            hat(italic(β)) ~ "=" ~ .(est_beta) * "," ~
            hat(italic(δ)) ~ "=" ~ .(est_delta) * "," ~
            hat(italic(ρ)) ~ "=" ~ .(est_rho) * "," ~
            hat(italic(σ)) ~ "=" ~ .(est_sigma)
        ), side = 3, col = "#FFA500", line = -1)
      },
      # If Tobit regression fails, issue error message
      error = function(e) {
        message(style_bold(bg_red(col_br_white("ERROR: ", conditionMessage(e)))))
      }
    )
    Sys.sleep(0.25)  # Short break to update plot
  }
  # Collect and display Tobit estimates
  tobit_estimates_report[[toString(curv)]] <- tobit_estimates
  print(cbind(params_sim, tobit_estimates_report[[toString(curv)]][, 2:6]))
}
# Collect estimates
all_estimates_report <- tobit_estimates_report
rm(curv, id, df_ind, tobit_estimates)
rm(coeffs, vcov_matrix)
rm(est_beta, est_beta_se, est_delta, est_delta_se, est_rho, est_rho_se, est_sigma)




# NONLINEAR MAXIMUM LIKELIHOOD ESTIMATION -----------------------------------------------------


# Log-likelihood contribution of a single observation according to eq. (30) in
# https://www.econtribute.de/RePEc/ajk/ajkdps/ECONtribute_336_2024.pdf
LL_contrib <- function(C_star_obs, t, k, lb, ub, R, z, beta, delta, rho, sigma) {
  C_star_pred <- C_star(t, k, R, z, beta, delta, rho)
  if (
    beta < 0 || delta < 0 || sigma < 0 || rho < -z
    # Parameters must not become negative, and
    # curvature of utility beyond curvature of BR cannot be identified
  ) {
    lnf <- -999999
  } else {
    # Interior solution
    lnf <- log(dnorm((log(C_star_obs) - log(C_star_pred)) / sigma) / sigma)
    # If observation is lower bound
    if (C_star_obs < lb + 0.00001) {
      lnf <- log(pnorm((log(lb) - log(C_star_pred)) / sigma))
    }
    # If observation is upper bound
    if (C_star_obs > ub - 0.00001) {
      lnf <- log(pnorm((log(C_star_pred) - log(ub)) / sigma))
    }
  }
  # Rule out NaNs and infinite values
  if (is.na(lnf) || is.nan(lnf) || lnf == -Inf) {
    lnf <- -999999
  }
  return(lnf)
}

# Log-likelihood contributions of all observations collected in vector
# (required by some optimization methods)
LL_contrib_vec <- function(C_star_obs, t, k, lb, ub, R, z, beta, delta, rho, sigma) {
  ln <- unlist(sapply(
    1:length(C_star_obs),
    function(i) {
      LL_contrib(
        C_star_obs[i],
        t[i], k[i], lb[i], ub[i], R[i], z[i],
        beta, delta, rho, sigma
      )
    }
  ))
  return(as.vector(ln))
}

ids <- unique(df$id)
mle_estimates_report <- list()

# Estimate separately for each curvature level of the budget restrictions
for (curv in unique(df$Curvature)) {
  # Initialize collection of estimates with NA
  mle_estimates <- matrix(rep(NA, length(ids) * 5), ncol = 5)
  mle_estimates <- cbind(ids, mle_estimates)
  colnames(mle_estimates) <- c("id", "beta", "delta", "rho", "sigma", "logL")
  # Estimate each individual separately
  for (id in ids) {
    print(paste0("NL-ML estimation, Subject ID: ", id, "; BR curvature: ", curv))
    df_ind <- df[df$id == id & df$Curvature == curv, ]
    # Objective function for maxLik must not contain any arguments except params
    LL_contrib_vec_filled <- function(params) {
      LL_contrib_vec(
        df_ind$payment_ratio,
        df_ind$FED, df_ind$Delay, df_ind$lb, df_ind$ub,
        df_ind$PriceOneDivPriceTwo, df_ind$Curvature,
        params[1], params[2], params[3], params[4]
      )
    }
    # Set the initial values for the numerical nonlinear estimation procedure
    # By default, take the outcome of the Tobit regression
    tobit_for_init <- tobit_estimates_report[[toString(curv)]]
    init_vals <- round(c(
      tobit_for_init[tobit_for_init[, "id"] == id, "beta"],
      tobit_for_init[tobit_for_init[, "id"] == id, "delta"],
      tobit_for_init[tobit_for_init[, "id"] == id, "rho"],
      tobit_for_init[tobit_for_init[, "id"] == id, "sigma"]
    ), 2)
    # If the Tobit regression yields nonsensical results, set different initical value
    if (init_vals["beta"] <= 0 | is.na(init_vals["beta"])) {
      init_vals["beta"] <- 0.95
    }
    if (init_vals["delta"] <= 0 | is.na(init_vals["delta"])) {
      init_vals["delta"] <- 0.95
    }
    if (is.na(init_vals["rho"])) {
      init_vals["rho"] <- 0.05
    } else if (init_vals["rho"] <= -curv) {
      init_vals["rho"] <- 1
    }
    if (init_vals["sigma"] <= 0 | is.na(init_vals["sigma"])) {
      init_vals["sigma"] <- 0.1
    }
    # As is frequently the case with numerical optimization procedures, individual-specific
    # initial values may be required, e.g., if Tobit did not converge. This is particularly likely
    # with linear budget restrictions and less so with strictly concave budget restrictions.
    if (id == 1 && curv == 0) {
      init_vals[c("beta", "delta", "rho", "sigma")] = c(1, 1, 0.01, 0.005)
    }
    if (id %in% c(2, 3) && curv == 0) {
      init_vals[c("beta", "delta", "rho", "sigma")] = c(0.995, 0.995, 0.01, 0.2)
    }
    # sum(LL_contrib_vec_filled(init_vals))  # Helpful for finding initial values
    # Plot simulated choices
    plotBRs(df_ind, curv)
    points(
      df_ind$payment_1,
      df_ind$payment_2,
      col = "navy",
      pch = 20
    )
    mtext(bquote(
      "Simulated choices for ID" ~ .(id) * ":" ~
        italic(β) ~ "=" ~ .(params_sim[params_sim[, "id"] == id, "beta"]) * "," ~
        italic(δ) ~ "=" ~ .(params_sim[params_sim[, "id"] == id, "delta"]) * "," ~
        italic(ρ) ~ "=" ~ .(params_sim[params_sim[, "id"] == id, "rho"]) * "," ~
        italic(σ) ~ "=" ~ .(params_sim[params_sim[, "id"] == id, "sigma"])
    ), side = 3, col = "navy", line = 0.25)
    tryCatch(
      {
        withCallingHandlers(
          # Attempt NL-MLE
          {
            mle_estim <- maxLik(
              logLik = LL_contrib_vec_filled,
              start = init_vals,
              # method = "BFGS",  # Broyden/Fletcher/Goldfarb/Shanno
              # method = "BFGSR",  # Broyden/Fletcher/Goldfarb/Shanno
              # method = "BHHH",  # Berndt/Hall/Hall/Hausman
              method = "NR", # Newton/Raphson
              # method = "SANN",  # Simulated Annealing
              control = list(
                gradtol = 10^-8, # Return code 1 (normal convergence)
                tol = 10^-8, # Return code 2 (normal convergence)
                steptol = -1, # Return code 3
                iterlim = 1000, # 10^6,  # Return code 4
                reltol = 10^-8 # Return code 8 (normal convergence)
              )
            )
            est_se <- stdEr(mle_estim, eigentol = 10^(-15))
          },
          # Show potential warning messages of the NL-MLE
          warning = function(w) {
            message(style_bold(col_blue("WARNING: ", conditionMessage(w))))
            invokeRestart("muffleWarning")
          }
        )
        # Add best-fitting allocations to plot (remove random component, i.e., sigma = 0)
        points(
          unlist(opt_noisy_points_on_BR(
            df_ind,
            mle_estim$estimate["beta"], mle_estim$estimate["delta"], mle_estim$estimate["rho"],
            sigma = 0
          )["c_1"]),
          unlist(opt_noisy_points_on_BR(
            df_ind,
            mle_estim$estimate["beta"], mle_estim$estimate["delta"], mle_estim$estimate["rho"],
            sigma = 0
          )["c_2"]),
          col = "#FF450088",
          pch = 20
        )
        mle_estim$estimate <- round(mle_estim$estimate, round_prec)
        mtext(bquote(
          "Parameter estimates:" ~
            hat(italic(β)) ~ "=" ~ .(mle_estim$estimate["beta"]) * "," ~
            hat(italic(δ)) ~ "=" ~ .(mle_estim$estimate["delta"]) * "," ~
            hat(italic(ρ)) ~ "=" ~ .(mle_estim$estimate["rho"]) * "," ~
            hat(italic(σ)) ~ "=" ~ .(mle_estim$estimate["sigma"])
        ), side = 3, col = "#FF4500", line = -1)
        mle_estimates[mle_estimates[, "id"] == id, "beta"] <- mle_estim$estimate["beta"]
        mle_estimates[mle_estimates[, "id"] == id, "delta"] <- mle_estim$estimate["delta"]
        mle_estimates[mle_estimates[, "id"] == id, "rho"] <- mle_estim$estimate["rho"]
        mle_estimates[mle_estimates[, "id"] == id, "sigma"] <- mle_estim$estimate["sigma"]
        mle_estimates[mle_estimates[, "id"] == id, "logL"] <- round(mle_estim$maximum, round_prec)
      },
      # If NL-MLE fails, issue error message
      error = function(e) {
        message(style_bold(bg_red(col_br_white("ERROR: ", conditionMessage(e)))))
      }
    )
    Sys.sleep(0.25)  # Short break to update plot
  }
  # Collect NL-MLE estimates
  mle_estimates_report[[toString(curv)]] <- mle_estimates
  cols = c("beta", "delta", "rho", "sigma", "logL")
  # Collect and display all estimates
  all_estimates_report[[toString(curv)]] <- cbind(
    params_sim,
    tobit_estimates_report[[toString(curv)]][, cols],
    mle_estimates_report[[toString(curv)]][, cols]
  )
  colnames(all_estimates_report[[toString(curv)]]) <-
    c(
      "id",
      "beta_sim", "delta_sim", "rho_sim", "sigma_sim",
      "beta_tobit", "delta_tobit", "rho_tobit", "sigma_tobit", "logL_tobit",
      "beta_mle", "delta_mle", "rho_mle", "sigma_mle", "logL_mle"
    )
  print(all_estimates_report[[toString(curv)]])
}
rm(curv, id, df_ind, mle_estimates, est_se, tobit_for_init)

all_estimates_report[["0"]][, 1:10]
# This should yield
#      id beta_sim delta_sim rho_sim sigma_sim beta_tobit delta_tobit rho_tobit sigma_tobit logL_tobit
# [1,]  1     1.00      1.00    0.00      0.00         NA          NA        NA      0.0006    32.1004
# [2,]  2     1.00      0.99    0.00      0.25     0.9557      0.9910    0.0000    155.4723 -9471.4181
# [3,]  3     0.80      0.99    0.00      0.25     0.9557      0.9910    0.0000    155.4723 -9471.4181
# [4,]  4     0.80      0.99    0.10      0.50     0.7998      0.9907    0.0961      0.3942   -15.9651
# [5,]  5     0.85      0.95    0.15      0.50     0.8554      0.9514    0.1556      0.4335   -14.5728
# [6,]  6     1.00      1.00    1.00      0.00     1.0000      1.0000    0.9999      0.0001   324.9009
# [7,]  7     1.00      0.99  100.00      0.25     0.6308      0.9941   -3.9738      0.1959     8.6606
# [8,]  8     1.00      1.00  200.00      0.05     0.1828      1.0957   57.7050      0.0496    64.9979
# [9,]  9     1.00      1.00  200.00      0.00     0.9814      1.0025  196.4053      0.0007   240.3492
# With linear budget restrictions:
# IDs 1, 2, and 3: Tobit does not converge for linear utility.
# Tobit converges to strongly convex utility instead of strongly concave utility for ID 7.
# ID 8: Strongly concave utility is hard to estimate in the presence of noise.

all_estimates_report[["0"]][, c(1:5, 11:15)]
# This should yield
#      id beta_sim delta_sim rho_sim sigma_sim beta_mle delta_mle  rho_mle sigma_mle logL_mle
# [1,]  1     1.00      1.00    0.00      0.00   1.0000    1.0000   0.0110    0.0003  34.8556
# [2,]  2     1.00      0.99    0.00      0.25   0.9905    0.9900   0.0001    0.1853   0.0000
# [3,]  3     0.80      0.99    0.00      0.25   0.9905    0.9900   0.0001    0.1853   0.0000
# [4,]  4     0.80      0.99    0.10      0.50   0.7998    0.9907   0.0961    0.3942 -15.9651
# [5,]  5     0.85      0.95    0.15      0.50   0.8554    0.9514   0.1556    0.4335 -14.5728
# [6,]  6     1.00      1.00    1.00      0.00   1.0000    1.0000   0.9999    0.0001 324.9009
# [7,]  7     1.00      0.99  100.00      0.25 555.2478    1.0991  57.9643    0.2008   7.6510
# [8,]  8     1.00      1.00  200.00      0.05   0.1745    1.0984  59.2789    0.0496  64.9978
# [9,]  9     1.00      1.00  200.00      0.00   0.9800    1.0000 196.4100    0.0049 179.5666
# With linear budget restrictions (LBRs):
# IDs 1, 2, and 3: Estimation becomes possible with NL-MLE by searching for suitable initial values,
# which, however, is cumbersome and difficult to automate.
# ID 7: Our NL-MLE routine converges to strongly concave utility, thereby fixing the weak point
# of Tobit. At the same time, with strongly concave utility all allocations are characterized by
# c_1 = c_2, with beta and delta hardly having any influence. Hence, identification of beta and
# delta becomes very hard.

all_estimates_report[["0.4"]][, 1:10]
# This should yield
#      id beta_sim delta_sim rho_sim sigma_sim beta_tobit delta_tobit rho_tobit sigma_tobit logL_tobit
# [1,]  1     1.00      1.00    0.00      0.00     1.0000      1.0000   -0.0002      0.0004   260.7333
# [2,]  2     1.00      0.99    0.00      0.25     0.9697      0.9903   -0.0093      0.2442    -0.3814
# [3,]  3     0.80      0.99    0.00      0.25     0.8085      0.9913    0.0063      0.2156     4.7404
# [4,]  4     0.80      0.99    0.10      0.50     0.8394      0.9889    0.0885      0.5054   -30.1979
# [5,]  5     0.85      0.95    0.15      0.50     0.9753      0.9509    0.0958      0.4442   -24.9066
# [6,]  6     1.00      1.00    1.00      0.00     0.9997      1.0000    0.9981      0.0004   258.3382
# [7,]  7     1.00      0.99  100.00      0.25     0.8767      0.9071   12.5038      0.2460    -0.6843
# [8,]  8     1.00      1.00  200.00      0.05     1.2020      1.0142   18.5775      0.0513    63.6310
# [9,]  9     1.00      1.00  200.00      0.00     0.9419      1.0080  196.3481      0.0006   248.9800
# With strictly concave budget restrictions (SCBRs):
# IDs 1, 2, and 3: Tobit converges for linear utility. Thus, SCBRs fix the weak point of LBRs.

all_estimates_report[["0.4"]][, c(1:5, 11:15)]
# This should yield
#      id beta_sim delta_sim rho_sim sigma_sim beta_mle delta_mle  rho_mle sigma_mle logL_mle
# [1,]  1     1.00      1.00    0.00      0.00   1.0000    1.0000  -0.0002    0.0004 260.7333
# [2,]  2     1.00      0.99    0.00      0.25   0.9697    0.9903  -0.0093    0.2442  -0.3814
# [3,]  3     0.80      0.99    0.00      0.25   0.8085    0.9913   0.0063    0.2156   4.7404
# [4,]  4     0.80      0.99    0.10      0.50   0.8394    0.9889   0.0885    0.5054 -30.1979
# [5,]  5     0.85      0.95    0.15      0.50   0.9753    0.9509   0.0958    0.4442 -24.9066
# [6,]  6     1.00      1.00    1.00      0.00   0.9997    1.0000   0.9981    0.0004 258.3382
# [7,]  7     1.00      0.99  100.00      0.25   0.8767    0.9071  12.5036    0.2460  -0.6843
# [8,]  8     1.00      1.00  200.00      0.05   1.2019    1.0142  18.5773    0.0513  63.6310
# [9,]  9     1.00      1.00  200.00      0.00   0.9400    1.0100 196.3500    0.0047 181.4055
# This illustrates that with the chosen optimization method ("NR"), our NL-MLE routine delivers
# (virtually) the same results as Tobit.
\end{lstlisting}

\clearpage

\begingroup

\normalsize

\section{Font Samples}

\subsection[Font Sample Times]{\fontfamily{ntxtlf}Font Sample Times}

\begin{otherlanguage}{ngerman}\noindent\normalsize%
	\fontfamily{ntxtlf}%
	\makeatletter%
	\ifdim \f@size pt < 11.5pt%
		\fontsize{11pt}{\baselinedist}%  % 10.945pt
	\else%
		\fontsize{12pt}{\baselinedist}%
	\fi%
	\makeatother%
	\selectfont%
	\Kafka
	\medskip
	\noindent\url{https://en.wikipedia.org/wiki/Times_New_Roman} \par
\end{otherlanguage}

\subsection[Font Sample Palatino]{\fontfamily{zpltlf}\selectfont Font Sample Palatino}

\begin{otherlanguage}{ngerman}\noindent%
	\fontfamily{zpltlf}%
	\makeatletter%
	\ifdim \f@size pt < 11.5pt%
		\fontsize{10pt}{\baselinedist}%
	\else%
		\fontsize{10.95pt}{\baselinedist}%
	\fi%
	\makeatother%
	\selectfont%
	\Kafka
	\medskip
	\noindent\url{https://en.wikipedia.org/wiki/Palatino} \par
\end{otherlanguage}

\clearpage

\section*{\strut}

\subsection[Font Sample Utopia]{\fontfamily{erewhon-TLF}\selectfont Font Sample Utopia}

\begin{otherlanguage}{ngerman}\noindent%
	\fontfamily{erewhon-TLF}%
	\makeatletter%
	\ifdim \f@size pt < 11.5pt%
		\fontsize{10.5pt}{\baselinedist}%
	\else%
		\fontsize{11.43pt}{\baselinedist}%
	\fi%
	\makeatother%
	\selectfont%
	\Kafka
	\medskip
	\noindent\url{https://en.wikipedia.org/wiki/Utopia_(typeface)} \par
\end{otherlanguage}

\subsection[Font Sample Charter]{\fontfamily{XCharter-TLF}\selectfont Font Sample Charter}

\begin{otherlanguage}{ngerman}\noindent%
	\fontfamily{XCharter-TLF}%
	\makeatletter%
	\ifdim \f@size pt < 11.5pt%
		\fontsize{10.26pt}{\baselinedist}%
	\else%
		\fontsize{11.1pt}{\baselinedist}%
	\fi%
	\makeatother%
	\selectfont%
	\Kafka
	\medskip
	\noindent\url{https://en.wikipedia.org/wiki/Bitstream_Charter} \par
\end{otherlanguage}

\clearpage

\section*{\strut}

\subsection[Font Sample {\textls[40]{STIX}} Two]{\fontfamily{SticksTooText-TLF}\selectfont Font Sample \textls[40]{STIX} Two}

\begin{otherlanguage}{ngerman}\noindent%
	\fontfamily{SticksTooText-TLF}%
	\makeatletter%
	\ifdim \f@size pt < 11.5pt%
		\fontsize{10.55pt}{\baselinedist}%
	\else%
		\fontsize{11.51pt}{\baselinedist}%
	\fi%
	\makeatother%
	\selectfont%
	\Kafka
	\medskip
	\noindent\url{https://en.wikipedia.org/wiki/STIX_Fonts_project#STIX_2.0.0} \par
\end{otherlanguage}

\subsection[Font Sample Libertinus Serif]{\fontfamily{LibertinusSerif-TLF}\selectfont Font Sample Libertinus Serif}

\begin{otherlanguage}{ngerman}\noindent%
	\fontfamily{LibertinusSerif-TLF}%
	\makeatletter%
	\ifdim \f@size pt < 11.5pt%
		\fontsize{10.92pt}{\baselinedist}%
	\else%
		\fontsize{11.85pt}{\baselinedist}%
	\fi%
	\makeatother%
	\selectfont%
	\Kafka
	\medskip
	\noindent\url{https://en.wikipedia.org/wiki/Libertinus} \par
\end{otherlanguage}

\clearpage

\section*{\strut}

\subsection[Font Sample New Century Schoolbook]{\fontfamily{TeXGyreScholaX-TLF}\selectfont{\relscale{0.925}Font Sample New Century Schoolbook}}

\begin{otherlanguage}{ngerman}\noindent%
	\fontfamily{TeXGyreScholaX-TLF}%
	\makeatletter%
	\ifdim \f@size pt < 11.5pt%
		\fontsize{9.6pt}{\baselinedist}%
	\else%
		\fontsize{10.57pt}{\baselinedist}%
	\fi%
	\makeatother%
	\selectfont%
	\Kafka
	\medskip
	\noindent\url{https://en.wikipedia.org/wiki/Century_type_family} \par
\end{otherlanguage}

\endgroup

\clearpage

\section{Math \texorpdfstring{``}{“}Torture Test\texorpdfstring{''}{”}}
\label{app:math-torture}

\makeatletter
\newcommand*{\checkgreekletters}{%
	\@for\@tempa:=%
	alpha,beta,gamma,delta,epsilon,varepsilon,zeta,eta,theta,vartheta,iota,kappa,lambda,mu,nu,xi,%
	omicron,pi,varpi,rho,varrho,sigma,varsigma,tau,upsilon,phi,varphi,chi,psi,omega,digamma,%
	Alpha,Beta,Gamma,Delta,Epsilon,Zeta,Eta,Theta,Iota,Kappa,Lambda,Mu,Nu,Xi,%
	Omicron,Pi,Rho,Sigma,Tau,Upsilon,Phi,Chi,Psi,Omega,Digamma%
	\do{$\csname\@tempa\endcsname,$ }%
}
\makeatother

\newcommand{\dit}{\mathit{d}}
\newcommand{\dup}{\mathup{d}}

\def\test#1{#1}

\def\testnums{%
  \test 0 \test 1 \test 2 \test 3 \test 4 \test 5 \test 6 \test 7
  \test 8 \test 9 }
\def\testupperi{%
  \test A \test B \test C \test D \test E \test F \test G \test H
  \test I \test J \test K \test L \test M }
\def\testupperii{%
  \test N \test O \test P \test Q \test R \test S \test T \test U
  \test V \test W \test X \test Y \test Z }
\def\testupper{%
  \testupperi\testupperii}

\def\testloweri{%
  \test a \test b \test c \test d \test e \test f \test g \test h
  \test i \test j \test k \test l \test m }
\def\testlowerii{%
  \test n \test o \test p \test q \test r \test s \test t \test u
  \test v \test w \test x \test y \test z }
\def\testlower{%
  \testloweri\testlowerii}

\def\testupgreeki{%
  \test A \test B \test\Gamma \test\Delta \test E \test Z \test H
  \test\Theta \test I \test K \test\Lambda \test M }
\def\testupgreekii{%
  \test N \test\Xi \test O \test\Pi \test P \test\Sigma \test T
  \test\Upsilon \test\Phi \test X \test\Psi \test\Omega }
\def\testupgreek{%
  \testupgreeki\testupgreekii}

\def\testlowgreeki{%
  \test\alpha \test\beta \test\gamma \test\delta \test\epsilon
  \test\zeta \test\eta \test\theta \test\iota \test\kappa \test\lambda
  \test\mu }
\def\testlowgreekii{%
  \test\nu \test\xi \test o \test\pi \test\rho \test\sigma \test\tau
  \test\upsilon \test\phi \test\chi \test\psi \test\omega }
\def\testlowgreekiii{%
  \test\varepsilon \test\vartheta \test\varpi \test\varrho
  \test\varsigma \test\varphi}
\def\testlowgreek{%
  \testlowgreeki\testlowgreekii\testlowgreekiii}

\newcommand{\showfamily}{}

Most of the following examples are taken from \textit{The \TeX book} \citep[][see \url{https://ctan.org/pkg/texbook}]{knuth:ct:a} and were adapted for \LaTeX\ from Karl Berry's torture test for plain \TeX\ math fonts.

\noindent $x + y - z$, \quad $x + y * z$, \quad $z * y \divslash z$, \quad
$(x+y)\,(x-y) = x^2 - y^2$,

\noindent $x \times y \cdot z = [x\, y\, z]$, \quad $x\circ y \bullet z$, \quad
$x\cup y \cap z$, \quad $x\sqcup y \sqcap z$, \quad

\noindent $x \vee y \wedge z$, \quad $x\pm y\mp z$, \quad
$x=y \divslash z$, \;\; $x \coloneqq y$, \;\; $x\le y \ne z$, \;\; $x \sim y \simeq z$
$x \equiv y \nequiv z$, \;\; $x\subset y \subseteq z$

\noindent $\sin2\theta=2\sin\theta\cos\theta$, \quad
$\hbox{O}(n\log n\log n)$, \quad
$\Pr(X>x)=\exp(-x \divslash \mu)$,

\noindent $\bigl(x\in A(n)\bigm|x\in B(n)\bigr)$, \quad
$\bigcup_n X_n\bigm\|\bigcap_n Y_n$

% page 178

\noindent In-text matrices $\binom{1\ 1}{0\ 1}$ and $\bigl(\genfrac{}{}{0pt}{}{a}{1}\genfrac{}{}{0pt}{}{b}{m}\genfrac{}{}{0pt}{}{c}{n}\bigr)$.

% page 142

$$a_0+\frac1{\displaystyle a_1 +
	{\strut \frac1{\displaystyle a_2 +
			{\strut \frac1{\displaystyle a_3 +
					{\strut \frac1{\displaystyle a_4}}}}}}}$$

% page 143

$$\binom{p}{2}x^2y^{p-2} - \frac1{1 - x}\frac{1}{1 - x^2}
=
\frac{a+1}{b}\bigg/\frac{c+1}{d}.$$

%% page 145

$$\sqrt{1+\sqrt{1+\sqrt{1+\sqrt{1+\sqrt{1+x}}}}}$$

$$\sqrt[n]{1+\sqrt[k]{1+\sqrt[5]{1+\sqrt[4]{1+\sqrt[3]{1+x}}}}}$$

%% page 147

$$\left(\frac{\partial^2}{\partial x^2} + \frac{\partial^2}{\partial y^2}\right)
\bigl|\varphi(x+\mathup{i}y)\bigr|^2=0$$

%% page 149

% $$\pi(n)=\sum_{m=2}^n\left\lfloor\biggl(\sum_{k=1}^{m-1}\bigl
% \lfloor(m/k)\big/\lceil m/k\rceil\bigr\rfloor\biggr)^{-1}\right\rfloor.$$

$$\pi(n)=\sum_{m=2}^n\left\lfloor\Biggl(\sum_{k=1}^{m-1}\bigl
\lfloor(m \divslash k) \big/ \lceil m \divslash k\rceil\bigr\rfloor\Biggr)^{-1}\right\rfloor.$$

% page 168

$$\int_0^\infty \frac{t - \mathup{i} b}{t^2 + b^2}e^{\mathup{i}at}\,\mathup{d}t=e^{ab}E_1(ab), \quad
a,b > 0.$$

% page 176

$$\mathbf{A} \coloneq \begin{pmatrix}x-\lambda&1&0\\
0&x-\lambda&1\\
0&0&x-\lambda\end{pmatrix}.$$

$$\left\lgroup\begin{matrix}a&b&c\\ d&e&f\\\end{matrix}\right\rgroup
\left\lgroup\begin{matrix}u&x\cr v&y\cr w&z\end{matrix}\right\rgroup$$

% page 177

$$\mathbf{A} = \begin{pmatrix}a_{11}&a_{12}&\ldots&a_{1n}\\
a_{21}&a_{22}&\ldots&a_{2n}\\
\vdots&\vdots&\ddots&\vdots\\
a_{m1}&a_{m2}&\ldots&a_{mn}\end{pmatrix}$$

$$\mathbf{M}=\bordermatrix{&C&I&C'\cr
	C&1&0&0\cr I&b&1-b&0\cr C'&0&a&1-a}$$

%% page 186

$$\sum_{n=0}^\infty a_nz^n\quad\hbox{converges if}\quad
|z|<\Bigl(\limsup_{n\to\infty}\root n\of{|a_n|}\,\Bigr)^{-1}.$$

$$\frac{f(x+\mathup{\Delta} x)-f(x)}{\mathup{\Delta} x}\to f'(x)
\qquad \hbox{as $\mathup{\Delta} x\to0$.}$$

$$\|u_i\|=1,\qquad u_i\cdot u_j=0\quad\hbox{if $i\ne j$.}$$

%% page 191

$$\hbox{The confluent image of}\quad
\begin{Bmatrix}\hbox{an arc}\hfill\\%
\hbox{a circle}\hfill\\%
\hbox{a fan}\hfill\end{Bmatrix}
\quad\hbox{is}\quad
\begin{Bmatrix}\hbox{an arc}\hfill\\%
\hbox{an arc or a circle}\hfill\\%
\hbox{a fan or an arc}\hfill\end{Bmatrix}.$$

%% page 191

\begin{align*}
T(n)\le T(2^{\lceil\lg n\rceil})
&\le c(3^{\lceil\lg n\rceil}-2^{\lceil\lg n\rceil})\\
&<3c\cdot3^{\lg n}\\
&=3c\,n^{\lg3}.
\end{align*}

%\begin{align*}
%\left\{%
%\begin{gathered}\alpha&=f(z)\\ \beta&=f(z^2)\\ \gamma&=f(z^3)
%\end{gathered}
%\right\}
%\qquad
%\left\{%
%\begin{gathered}
%x&=\alpha^2-\beta\\ y&=2\gamma
%\end{gathered}
%\right\}%
%\end{align*}

%$$\left\{
%\begin{align}
%\alpha&=f(z)\cr \beta&=f(z^2)\cr \gamma&=f(z^3)\\
%%\end{align}
%\right\}
%\qquad
%\left\{
%%\begin{align}
%x&=\alpha^2-\beta\cr y&=2\gamma\\
%\end{align}
%\right\}.$$
%%% page 192

\begin{align*}
\begin{aligned}
(x+y)(x-y)&=x^2-xy+yx-y^2\\
&=x^2-y^2\\
(x+y)^2&=x^2+2xy+y^2.
\end{aligned}
\end{align*}

%% page 192

\begin{align*}
\begin{aligned}
\left( \int\limits_{-\infty}^\infty \mathup{e}^{-x^2}\,\mathup{d}x \right)^2
&=\int_{-\infty}^\infty\int_{-\infty}^\infty \mathup{e}^{-(x^2+y^2)}\,\mathup{d}x\,\mathup{d}y\\
&=\int_0^{2\piup}\int_0^\infty \mathup{e}^{-r^2}\,\mathup{d}r\,\mathup{d}\theta\\
&=\int_0^{2\piup}\biggl(\mathup{e}^{-\frac{r^2}{2}}\biggl|_{r=0}^{r=\infty}\,\biggr)\,\mathup{d}\theta\\
&=\piup.
\end{aligned}
\end{align*}


%% page 197

$$\prod_{k\ge0}\frac{1}{(1-q^kz)}=
\sum_{n\ge0}z^n \bigg/ \!\!\prod_{1\le k\le n}(1-q^k).$$

$$\sum_{\substack{\scriptstyle 0< i\le m\\\scriptstyle0<j\le n}}p(i,j) \,\ne
%
% $$\sum_{i=1}^p \sum_{j=1}^q \sum_{k=1}^r a_{ij} b_{jk} c_{ki}$$
%
\sum_{i=1}^p \sum_{j=1}^q \sum_{k=1}^r a_{ij} b_{jk} c_{ki} \,\ne
%
\sum_{\substack{\scriptstyle 1\le i\le p \\ \scriptstyle 1\le j\le q\\
		\scriptstyle 1\le k\le r}} a_{ij} b_{jk} c_{ki}$$

$$\max_{1\le n\le m}\log_2P_n \quad \hbox{and} \quad
\lim_{x\to0}\frac{\sin x}{x}=1$$
Inline math:
$\max_{1\le n\le m}\log_2P_n \quad \hbox{and} \quad
\lim_{x\to0}\frac{\sin x}{x}=1$
$$p_1(n)=\lim_{m\to\infty}\sum_{\nu=0}^\infty\bigl(1-\cos^{2m}(\nu!^n\piup \divslash n)\bigr)$$
Inline math:
$p_1(n)=\lim_{m\to\infty}\sum_{\nu=0}^\infty\bigl(1-\cos^{2m}(\nu!^n\piup \divslash n)\bigr)$

\clearpage

\begingroup
\bfseries\boldmath

\section{Math \texorpdfstring{``}{“}Torture Test\texorpdfstring{''}{”} \texttt{\textbackslash boldmath}}

Most of the following examples are taken from \textit{The \TeX book} \citep[][see \url{https://ctan.org/pkg/texbook}]{knuth:ct:a} and were adapted for \LaTeX\ from Karl Berry's torture test for plain \TeX\ math fonts.

\noindent $x + y - z$, \quad $x + y * z$, \quad $z * y \divslash z$, \quad
$(x+y)(x-y) = x^2 - y^2$,

\noindent $x \times y \cdot z = [x\, y\, z]$, \quad $x\circ y \bullet z$, \quad
$x\cup y \cap z$, \quad $x\sqcup y \sqcap z$, \quad

\noindent $x \vee y \wedge z$, \quad $x\pm y\mp z$, \quad
$x=y \divslash z$, \;\; $x:=y$, \;\; $x\le y \ne z$, \;\; $x \sim y \simeq z$
$x \equiv y \nequiv z$, \;\; $x\subset y \subseteq z$

\noindent $\sin2\theta=2\sin\theta\cos\theta$, \quad
$\hbox{O}(n\log n\log n)$, \quad
$\Pr(X>x)=\exp(-x \divslash \mu)$,

\noindent $\bigl(x\in A(n)\bigm|x\in B(n)\bigr)$, \quad
$\bigcup_n X_n\bigm\|\bigcap_n Y_n$

% page 178

\noindent In-text matrices $\binom{1\,1}{0\,1}$ and $\bigl(\genfrac{}{}{0pt}{}{a}{1}\genfrac{}{}{0pt}{}{b}{m}\genfrac{}{}{0pt}{}{c}{n}\bigr)$.

% page 142

$$a_0+\frac1{\displaystyle a_1 +
	{\strut \frac1{\displaystyle a_2 +
			{\strut \frac1{\displaystyle a_3 +
					{\strut \frac1{\displaystyle a_4}}}}}}}$$

% page 143

$$\binom{p}{2}x^2y^{p-2} - \frac1{1 - x}\frac{1}{1 - x^2}
=
\frac{a+1}{b}\bigg/\frac{c+1}{d}.$$

%% page 145

$$\sqrt{1+\sqrt{1+\sqrt{1+\sqrt{1+\sqrt{1+x}}}}}$$

$$\sqrt[n]{1+\sqrt[k]{1+\sqrt[5]{1+\sqrt[4]{1+\sqrt[3]{1+x}}}}}$$

%% page 147

$$\left(\frac{\partial^2}{\partial x^2} + \frac{\partial^2}{\partial y^2}\right)
\bigl|\varphi(x+\mathup{i}y)\bigr|^2=0$$

%% page 149

% $$\pi(n)=\sum_{m=2}^n\left\lfloor\biggl(\sum_{k=1}^{m-1}\bigl
% \lfloor(m/k)\big/\lceil m/k\rceil\bigr\rfloor\biggr)^{-1}\right\rfloor.$$

$$\pi(n)=\sum_{m=2}^n\left\lfloor\Biggl(\sum_{k=1}^{m-1}\bigl
\lfloor(m \divslash k) \big/ \lceil m \divslash k\rceil\bigr\rfloor\Biggr)^{-1}\right\rfloor.$$

% page 168

$$\int_0^\infty \frac{t - \mathup{i} b}{t^2 + b^2}e^{\mathup{i}at}\,\mathup{d}t=e^{ab}E_1(ab), \quad
a,b > 0.$$

% page 176

$$\mathbf{A} \coloneq \begin{pmatrix}x-\lambda&1&0\\
0&x-\lambda&1\\
0&0&x-\lambda\end{pmatrix}.$$

$$\left\lgroup\begin{matrix}a&b&c\\ d&e&f\\\end{matrix}\right\rgroup
\left\lgroup\begin{matrix}u&x\cr v&y\cr w&z\end{matrix}\right\rgroup$$

% page 177

$$\mathbf{A} = \begin{pmatrix}a_{11}&a_{12}&\ldots&a_{1n}\\
a_{21}&a_{22}&\ldots&a_{2n}\\
\vdots&\vdots&\ddots&\vdots\\
a_{m1}&a_{m2}&\ldots&a_{mn}\end{pmatrix}$$

$$\mathbf{M}=\bordermatrix{&C&I&C'\cr
	C&1&0&0\cr I&b&1-b&0\cr C'&0&a&1-a}$$

%% page 186

$$\sum_{n=0}^\infty a_nz^n\quad\hbox{converges if}\quad
|z|<\Bigl(\limsup_{n\to\infty}\root n\of{|a_n|}\,\Bigr)^{-1}.$$

$$\frac{f(x+\mathup{\Delta} x)-f(x)}{\mathup{\Delta} x}\to f'(x)
\qquad \hbox{as $\mathup{\Delta} x\to0$.}$$

$$\|u_i\|=1,\qquad u_i\cdot u_j=0\quad\hbox{if $i\ne j$.}$$

%% page 191

$$\hbox{The confluent image of}\quad
\begin{Bmatrix}\hbox{an arc}\hfill\\\hbox{a circle}\hfill\\
\hbox{a fan}\hfill\\\end{Bmatrix}
\quad\hbox{is}\quad
\begin{Bmatrix}\hbox{an arc}\hfill\\
\hbox{an arc or a circle}\hfill\\
\hbox{a fan or an arc}\hfill\end{Bmatrix}.$$

%% page 191

\begin{align*}
T(n)\le T(2^{\lceil\lg n\rceil})
&\le c(3^{\lceil\lg n\rceil}-2^{\lceil\lg n\rceil})\\
&<3c\cdot3^{\lg n}\\
&=3c\,n^{\lg3}.
\end{align*}

%\begin{align*}
%\left\{%
%\begin{gathered}\alpha&=f(z)\\ \beta&=f(z^2)\\ \gamma&=f(z^3)
%\end{gathered}
%\right\}
%\qquad
%\left\{%
%\begin{gathered}
%x&=\alpha^2-\beta\\ y&=2\gamma
%\end{gathered}
%\right\}%
%\end{align*}

%$$\left\{
%\begin{align}
%\alpha&=f(z)\cr \beta&=f(z^2)\cr \gamma&=f(z^3)\\
%%\end{align}
%\right\}
%\qquad
%\left\{
%%\begin{align}
%x&=\alpha^2-\beta\cr y&=2\gamma\\
%\end{align}
%\right\}.$$
%%% page 192

\begin{align*}
\begin{aligned}
(x+y)(x-y)&=x^2-xy+yx-y^2\\
&=x^2-y^2\\
(x+y)^2&=x^2+2xy+y^2.
\end{aligned}
\end{align*}

%% page 192

\begin{align*}
\begin{aligned}
\left( \int\limits_{-\infty}^\infty \mathup{e}^{-x^2}\,\mathup{d}x \right)^2
&=\int_{-\infty}^\infty\int_{-\infty}^\infty \mathup{e}^{-(x^2+y^2)}\,\mathup{d}x\,\mathup{d}y\\
&=\int_0^{2\piup}\int_0^\infty \mathup{e}^{-r^2}\,\mathup{d}r\,\mathup{d}\theta\\
&=\int_0^{2\piup}\biggl(\mathup{e}^{-\frac{r^2}{2}}\biggl|_{r=0}^{r=\infty}\,\biggr)\,\mathup{d}\theta\\
&=\piup.
\end{aligned}
\end{align*}


%% page 197

$$\prod_{k\ge0}\frac{1}{(1-q^kz)}=
\sum_{n\ge0}z^n \bigg/ \!\!\prod_{1\le k\le n}(1-q^k).$$

$$\sum_{\substack{\scriptstyle 0< i\le m\\\scriptstyle0<j\le n}}p(i,j) \,\ne
%
% $$\sum_{i=1}^p \sum_{j=1}^q \sum_{k=1}^r a_{ij} b_{jk} c_{ki}$$
%
\sum_{i=1}^p \sum_{j=1}^q \sum_{k=1}^r a_{ij} b_{jk} c_{ki} \,\ne
%
\sum_{\substack{\scriptstyle 1\le i\le p \\ \scriptstyle 1\le j\le q\\
		\scriptstyle 1\le k\le r}} a_{ij} b_{jk} c_{ki}$$

\medskip
$$\max_{1\le n\le m}\log_2P_n \quad \hbox{and} \quad
\lim_{x\to0}\frac{\sin x}{x}=1$$

\medskip
Inline math:
$\max_{1\le n\le m}\log_2P_n \quad \hbox{and} \quad
\lim_{x\to0}\frac{\sin x}{x}=1$
$$p_1(n)=\lim_{m\to\infty}\sum_{\nu=0}^\infty\bigl(1-\cos^{2m}(\nu!^n\piup \divslash n)\bigr)$$

\medskip
Inline math:
$p_1(n)=\lim_{m\to\infty}\sum_{\nu=0}^\infty\bigl(1-\cos^{2m}(\nu!^n\piup \divslash n)\bigr)$

\endgroup

\clearpage

\section{Math Test \showfamily}

\subsection{Spacing}

$$\frac{a \divslash b + \frac{a \divslash b + c}{x}}{x} \qquad \sin x \divslash \cos x \qquad n \divslash \log n$$

{\boldmath $$\frac{a \divslash b + \frac{a \divslash b + c}{x}}{x} \qquad \sin x \divslash \cos x \qquad n \divslash \log n$$}

\begin{theorem}[simplest form of the Central Limit Theorem]
	\label{theorem:SFCLT}
	Let $X_1, X_2, \ldots, X_n$ be a~sequence of i.i.d. random variables with mean $0$
	and variance $1$ on a~probability space $(\Omega, \mathcal{F}, \mathbb{P})$. Then
	\[
		\mathbb{P}{\left(\frac{X_1+\cdots+X_n}{\sqrt{n}}\le y\right)} \to
		\mathfrak{N}(y) \coloneq
		\int_{-\infty}^y \frac{\mathup{e}^{-v^2 \divslash 2}}{\sqrt{2\mathup{\pi}}}\,\mathup{d}v
		\quad\text{as} \quad n\to\infty,
	\]
	or, equivalently, letting $S_n \coloneq \sum_1^n X_k$,
	\[
		\mathbb{E} f \big( S_n \divslash \sqrt{n} \big) \to
		\int_{-\infty}^\infty f(v) \frac{\mathup{e}^{-v^2 \divslash 2}}{\sqrt{2\piup}}\,\mathup{d}v
		\quad \mbox{as $n\to\infty$, for every $f \in \mathup{b}\mathcal{C}(\mathbb{R})$.}
	\]
\end{theorem}

\subsection{Overview \showfamily}

{\parindent 0pt
Default: $a \alpha \alphaup b \beta G \Gamma \upGamma \epsilon \varepsilon \theta \vartheta P \Pi \Sigma \sigma$; $\sigma_\epsilon, c^\alpha$

mathnormal: $\mathnormal{a \alpha \alphaup b \beta G \Gamma \upGamma \epsilon \varepsilon \theta \vartheta P \Pi \Sigma \sigma}$

mathrm: $\mathrm{a \alpha \alphaup b \beta G \Gamma \upGamma \epsilon \varepsilon \theta \vartheta P \Pi \Sigma \sigma}$

mathup: $\mathup{a \alpha \alphaup b \beta G \Gamma \upGamma \epsilon \varepsilon \theta \vartheta P \Pi \Sigma \sigma}$

mathit: $\mathit{a \alpha \alphaup b \beta G \Gamma \upGamma \epsilon \varepsilon \theta \vartheta P \Pi \Sigma \sigma}$

mathbf: $\mathbf{a \alphaup b \beta G \Gamma \upGamma \epsilon \varepsilon \theta \vartheta P \Pi \Sigma \sigma}$

mathbfit: $\mathbfit{a \alpha b \beta G \Gamma \upGamma \epsilon \varepsilon \theta \vartheta P \Pi \Sigma \sigma}$

mathbfup: $\mathbfup{a \alpha b \beta G \Gamma \upGamma \epsilon \varepsilon \theta \vartheta P \Pi \Sigma \sigma}$

\bigskip

{\bfseries
Default: $a \alpha \alphaup b \beta G \Gamma \upGamma \epsilon \varepsilon \theta \vartheta P \Pi \Sigma \sigma$; $\sigma_\epsilon, c^\alpha$

mathnormal: $\mathnormal{a \alpha \alphaup b \beta G \Gamma \upGamma \epsilon \varepsilon \theta \vartheta P \Pi \Sigma \sigma}$

mathrm: $\mathrm{a \alpha \alphaup b \beta G \Gamma \upGamma \epsilon \varepsilon \theta \vartheta P \Pi \Sigma \sigma}$

mathup: $\mathup{a \alpha \alphaup b \beta G \Gamma \upGamma \epsilon \varepsilon \theta \vartheta P \Pi \Sigma \sigma}$

mathit: $\mathit{a \alpha \alphaup b \beta G \Gamma \upGamma \epsilon \varepsilon \theta \vartheta P \Pi \Sigma \sigma}$

mathbf: $\mathbf{a \alpha \alphaup b \beta G \Gamma \upGamma \epsilon \varepsilon \theta \vartheta P \Pi \Sigma \sigma}$

mathbfit: $\mathbfit{a \alpha \alphaup b \beta G \Gamma \upGamma \epsilon \varepsilon \theta \vartheta P \Pi \Sigma \sigma}$

mathbfup: $\mathbfup{a \alpha \alphaup b \beta G \Gamma \upGamma \epsilon \varepsilon \theta \vartheta P \Pi \Sigma \sigma}$
}

\bigskip

{\sffamily\mdseries
Default: $a \alpha \alphaup b \beta G \Gamma \upGamma \epsilon \varepsilon \theta \vartheta P \Pi \Sigma \sigma$; $\sigma_\epsilon, c^\alpha$

mathnormal: $\mathnormal{a \alpha \alphaup b \beta G \Gamma \upGamma \epsilon \varepsilon \theta \vartheta P \Pi \Sigma \sigma}$

mathrm: $\mathrm{a \alpha \alphaup b \beta G \Gamma \upGamma \epsilon \varepsilon \theta \vartheta P \Pi \Sigma \sigma}$

mathup: $\mathup{a \alpha \alphaup b \beta G \Gamma \upGamma \epsilon \varepsilon \theta \vartheta P \Pi \Sigma \sigma}$

mathit: $\mathit{a \alpha \alphaup b \beta G \Gamma \upGamma \epsilon \varepsilon \theta \vartheta P \Pi \Sigma \sigma}$

mathbf: $\mathbf{a \alpha \alphaup b \beta G \Gamma \upGamma \epsilon \varepsilon \theta \vartheta P \Pi \Sigma \sigma}$

mathbfit: $\mathbfit{a \alpha \alphaup b \beta G \Gamma \upGamma \epsilon \varepsilon \theta \vartheta P \Pi \Sigma \sigma}$

mathbfup: $\mathbfup{a \alpha \alphaup b \beta G \Gamma \upGamma \epsilon \varepsilon \theta \vartheta P \Pi \Sigma \sigma}$
}

\bigskip

{\sffamily\bfseries

Default: $a \alpha \alphaup b \beta G \Gamma \upGamma \epsilon \varepsilon \theta \vartheta P \Pi \Sigma \sigma$; $\sigma_\epsilon, c^\alpha$

mathnormal: $\mathnormal{a \alpha \alphaup b \beta G \Gamma \upGamma \epsilon \varepsilon \theta \vartheta P \Pi \Sigma \sigma}$

mathrm: $\mathrm{a \alpha \alphaup b \beta G \Gamma \upGamma \epsilon \varepsilon \theta \vartheta P \Pi \Sigma \sigma}$

mathup: $\mathup{a \alpha \alphaup b \beta G \Gamma \upGamma \epsilon \varepsilon \theta \vartheta P \Pi \Sigma \sigma}$

mathit: $\mathit{a \alpha \alphaup b \beta G \Gamma \upGamma \epsilon \varepsilon \theta \vartheta P \Pi \Sigma \sigma}$

mathbf: $\mathbf{a \alpha \alphaup b \beta G \Gamma \upGamma \epsilon \varepsilon \theta \vartheta P \Pi \Sigma \sigma}$

mathbfit: $\mathbfit{a \alpha \alphaup b \beta G \Gamma \upGamma \epsilon \varepsilon \theta \vartheta P \Pi \Sigma \sigma}$

mathbfup: $\mathbfup{a \alpha \alphaup b \beta G \Gamma \upGamma \epsilon \varepsilon \theta \vartheta P \Pi \Sigma \sigma}$
}
}

\subsection{Formulas \showfamily}

\setlength{\parindent}{0pt}

\noindent%
\checkgreekletters

\noindent%
{\boldmath\checkgreekletters}

\noindent%
{\sffamily\selectfont \checkgreekletters}

\noindent%
{\sffamily\bfseries\selectfont \checkgreekletters}

\noindent%
{\sffamily $\alpha a > 0, \beta b + (3 \times 27), \Gamma G = 7 < 8, \lambda$}

\noindent%
$\alpha a > 0, \beta b + (3 \times 27), \Gamma G = 7 < 8, \lambda$

$\lim_{\nu \to \infty} v(\nu) = \max_{s \in S} \{s \pm 3 \gamma + y - 1\} = 4 \times 7$

$\hat{\beta} = (X'X)^{-1}X'y$

$$\lim_{N \to \infty} \sum_{i=0}^{N} x^i = \min_{x \in \mathbb{R}} S(x)$$

$$\int_{-\infty}^{\infty} x\,f(x)\,\mathup{d}x = \left( \frac{27}{2} \right)$$

Disambiguation: $0$~O~$O$, $1$~l~I~$|$~$l$~$I$~$/$, $i$~$j$, $rn$~$m$, $\theta$~$\Theta$, $\phi$~$\psi$, --~$-$

Latin vs. Greek: $a$~$\alpha$, $d$~$\delta$, $e$~$\epsilon$, $i$~$\iota$, $k$~$\kappa$, $n$~$\eta$, $o$~$\sigma$, $p$~$\rho$, \textit{\ss} $\beta$, $u$~$\upsilon$, $v$~$\nu$, $w$~$\omega$, $x$~$\chi$, $y$~$\gamma$, $A$~$\Delta$~$\Lambda$, $O$~$\Theta$~$\Omega$, $T$~$\Gamma$, $Y$~$\Upsilon$.

\noindent%
{\bfseries%
$\alpha a > 0, \beta b + (3 \times 27), \Gamma G = 7 < 8, \lambda$

$\lim_{\nu \to \infty} v(\nu) = \max_{s \in S} \{s \pm 3 \gamma + y - 1\} = 4 \times 7$

$\hat{\beta} = (X'X)^{-1}X'y$

$$\lim_{N \to \infty} \sum_{i=0}^{N} x^i = \min_{x \in \mathbb{R}} S(x)$$

$$\int_{-\infty}^{\infty} x\,f(x)\,\mathup{d}x = \left( \frac{27}{2} \right)$$

Disambiguation: $0$~O~$O$, $1$~l~I~$|$~$l$~$I$~$/$, $i$~$j$, $rn$~$m$, $\theta$~$\Theta$, $\phi$~$\psi$, --~$-$

Latin vs. Greek: $a$~$\alpha$, $d$~$\delta$, $e$~$\epsilon$, $i$~$\iota$, $k$~$\kappa$, $n$~$\eta$, $o$~$\sigma$, $p$~$\rho$, \textit{\ss} $\beta$, $u$~$\upsilon$, $v$~$\nu$, $w$~$\omega$, $x$~$\chi$, $y$~$\gamma$, $A$~$\Delta$~$\Lambda$, $O$~$\Theta$~$\Omega$, $T$~$\Gamma$, $Y$~$\Upsilon$.
}

\noindent%
{\sffamily%
$\alpha a > 0, \beta b + (3 \times 27), \Gamma G = 7 < 8, \lambda$

$\lim_{\nu \to \infty} v(\nu) = \max_{s \in S} \{s \pm 3 \gamma + y - 1\} = 4 \times 7$

$\hat{\beta} = (X'X)^{-1}X'y$

$$\lim_{N \to \infty} \sum_{i=0}^{N} x^i = \min_{x \in \mathbb{R}} S(x)$$

$$\int_{-\infty}^{\infty} x\,f(x)\,\mathup{d}x = \left( \frac{27}{2} \right)$$

Disambiguation: $0$~O~$O$, $1$~l~I~$|$~$l$~$I$~$/$, $i$~$j$, $rn$~$m$, $\theta$~$\Theta$, $\phi$~$\psi$, --~$-$

Latin vs. Greek: $a$~$\alpha$, $d$~$\delta$, $e$~$\epsilon$, $i$~$\iota$, $k$~$\kappa$, $n$~$\eta$, $o$~$\sigma$, $p$~$\rho$, \textit{\ss} $\beta$, $u$~$\upsilon$, $v$~$\nu$, $w$~$\omega$, $x$~$\chi$, $y$~$\gamma$, $A$~$\Delta$~$\Lambda$, $O$~$\Theta$~$\Omega$, $T$~$\Gamma$, $Y$~$\Upsilon$.
}

\noindent%
{\sffamily\bfseries%
$\alpha a > 0, \beta b + (3 \times 27), \Gamma G = 7 < 8, \lambda$

$\lim_{\nu \to \infty} v(\nu) = \max_{s \in S} \{s \pm 3 \gamma + y - 1\} = 4 \times 7$

$\hat{\beta} = (X'X)^{-1}X'y$

$$\lim_{N \to \infty} \sum_{i=0}^{N} x^i = \min_{x \in \mathbb{R}} S(x)$$

$$\int_{-\infty}^{\infty} x\,f(x)\,\mathup{d}x = \left( \frac{27}{2} \right)$$

Disambiguation: $0$~O~$O$, $1$~l~I~$|$~$l$~$I$~$/$, $i$~$j$, $rn$~$m$, $\theta$~$\Theta$, $\phi$~$\psi$, --~$-$

Latin vs. Greek: $a$~$\alpha$, $d$~$\delta$, $e$~$\epsilon$, $i$~$\iota$, $k$~$\kappa$, $n$~$\eta$, $o$~$\sigma$, $p$~$\rho$, \textit{\ss} $\beta$, $u$~$\upsilon$, $v$~$\nu$, $w$~$\omega$, $x$~$\chi$, $y$~$\gamma$, $A$~$\Delta$~$\Lambda$, $O$~$\Theta$~$\Omega$, $T$~$\Gamma$, $Y$~$\Upsilon$.
}


\subsection{Math Alphabets \showfamily}
\label{app:math-test:alphabets}

%\sffamily\selectfont

Default
\def\test#1{#1,}
\begin{eqnarray*}
  && {\testnums}\\
  && {\testupper}\\
  && {\testlower}\\
  && {\testupgreek}\\
  && {\testlowgreek}
\end{eqnarray*}%

Math Normal (\texttt{\string\mathnormal})
\def\test#1{\mathnormal{#1},}
\begin{eqnarray*}
  && {\testnums}\\
  && {\testupper}\\
  && {\testlower}\\
  && {\testupgreek}\\
  && {\testlowgreek}
\end{eqnarray*}%

Math Italic (\texttt{\string\mathit})
\def\test#1{\mathit{#1},}
\begin{eqnarray*}
  && {\testnums}\\
  && {\testupper}\\
  && {\testlower}\\
  && {\testupgreek}\\
  && {\testlowgreek}
\end{eqnarray*}%

Math Roman (\texttt{\string\mathrm})
\def\test#1{\mathrm{#1},}
\begin{eqnarray*}
  && {\testnums}\\
  && {\testupper}\\
  && {\testlower}\\
  && {\testupgreek}\\
  && {\testlowgreek}
\end{eqnarray*}%

%Math Italic Bold (\texttt{\string\mathbm})
%\def\test#1{\mathbm{#1},}
%\begin{eqnarray*}
%  && {\testnums}\\
%  && {\testupper}\\
%  && {\testlower}\\
%  && {\testupgreek}\\
%  && {\testlowgreek}
%\end{eqnarray*}%

Math Bold (\texttt{\string\mathbf})
\def\test#1{\mathbf{#1},}
\begin{eqnarray*}
  && {\testnums}\\
  && {\testupper}\\
  && {\testlower}\\
  && {\testupgreek}\\
  && {\testlowgreek}
\end{eqnarray*}%

Caligraphic (\texttt{\string\mathcal})
\def\test#1{\mathcal{#1},}
\begin{eqnarray*}
  && {\testupper}
\end{eqnarray*}%

Script (\texttt{\string\mathscr})
\def\test#1{\mathscr{#1},}
\begin{eqnarray*}
  && {\testupper}
\end{eqnarray*}%

Fraktur (\texttt{\string\mathfrak})
\def\test#1{\mathfrak{#1},}
\begin{eqnarray*}
  && {\testupper}\\
  && {\testlower}
\end{eqnarray*}%

Blackboard Bold (\texttt{\string\mathbb})
\def\test#1{\mathbb{#1},}
\begin{eqnarray*}
  && {\testupper}
\end{eqnarray*}%

\subsection{Character Sidebearings \showfamily}

Default
\def\test#1{|#1|+{}}
\begin{eqnarray*}
  && {\testupperi}\\
  && {\testupperii}\\
  && {\testloweri}\\
  && {\testlowerii}\\
  && {\testupgreeki}\\
  && {\testupgreekii}\\
  && {\testlowgreeki}\\
  && {\testlowgreekii}\\
  && {\testlowgreekiii}
\end{eqnarray*}%

Math Roman (\texttt{\string\mathrm})
\def\test#1{|\mathrm{#1}|+{}}%
\begin{eqnarray*}
  && {\testupperi}\\
  && {\testupperii}\\
  && {\testloweri}\\
  && {\testlowerii}\\
  && {\testupgreeki}\\
  && {\testupgreekii}
\end{eqnarray*}%

%Math Italic Bold (\texttt{\string\mathbm})
%\def\test#1{|\mathbm{#1}|+{}}%
%\begin{eqnarray*}
%  && {\testupperi}\\
%  && {\testupperii}\\
%  && {\testloweri}\\
%  && {\testlowerii}\\
%  && {\testupgreeki}\\
%  && {\testupgreekii}\\
%  && {\testlowgreeki}\\
%  && {\testlowgreekii}\\
%  && {\testlowgreekiii}
%\end{eqnarray*}%

Math Bold (\texttt{\string\mathbf})
\def\test#1{|\mathbf{#1}|+{}}%
\begin{eqnarray*}
  && {\testupperi}\\
  && {\testupperii}\\
  && {\testloweri}\\
  && {\testlowerii}\\
  && {\testupgreeki}\\
  && {\testupgreekii}
\end{eqnarray*}%

Math Calligraphic (\texttt{\string\mathcal})
\def\test#1{|\mathcal{#1}|+{}}%
\begin{eqnarray*}
  && {\testupperi}\\
  && {\testupperii}
\end{eqnarray*}%


\subsection{Superscript Positioning \showfamily}

Default
\def\test#1{#1^{2}+{}}%
\begin{eqnarray*}
  && {\testupperi}\\
  && {\testupperii}\\
  && {\testloweri}\\
  && {\testlowerii}\\
  && {\testupgreeki}\\
  && {\testupgreekii}\\
  && {\testlowgreeki}\\
  && {\testlowgreekii}\\
  && {\testlowgreekiii}
\end{eqnarray*}%

Math Roman (\texttt{\string\mathrm})
\def\test#1{\mathrm{#1}^{2}+{}}%
\begin{eqnarray*}
  && {\testupperi}\\
  && {\testupperii}\\
  && {\testloweri}\\
  && {\testlowerii}\\
  && {\testupgreeki}\\
  && {\testupgreekii}
\end{eqnarray*}%

%Math Italic Bold (\texttt{\string\mathbm})
%\def\test#1{\mathbm{#1}^{2}+{}}%
%\begin{eqnarray*}
%  && {\testupperi}\\
%  && {\testupperii}\\
%  && {\testloweri}\\
%  && {\testlowerii}\\
%  && {\testupgreeki}\\
%  && {\testupgreekii}\\
%  && {\testlowgreeki}\\
%  && {\testlowgreekii}\\
%  && {\testlowgreekiii}
%\end{eqnarray*}%

Math Bold (\texttt{\string\mathbf})
\def\test#1{\mathbf{#1}^{2}+{}}%
\begin{eqnarray*}
  && {\testupperi}\\
  && {\testupperii}\\
  && {\testloweri}\\
  && {\testlowerii}\\
  && {\testupgreeki}\\
  && {\testupgreekii}
\end{eqnarray*}

Math Calligraphic (\texttt{\string\mathcal})
\def\test#1{\mathcal{#1}^{2}+{}}%
\begin{eqnarray*}
  && {\testupperi}\\
  && {\testupperii}
\end{eqnarray*}%


\subsection{Subscript Positioning \showfamily}

Default
\def\test#1{\mathnormal{#1}_{i}+{}}%
\begin{eqnarray*}
  && {\testupperi}\\
  && {\testupperii}\\
  && {\testloweri}\\
  && {\testlowerii}\\
  && {\testupgreeki}\\
  && {\testupgreekii}\\
  && {\testlowgreeki}\\
  && {\testlowgreekii}\\
  && {\testlowgreekiii}
\end{eqnarray*}%

Math Roman (\texttt{\string\mathrm})
\def\test#1{\mathrm{#1}_{i}+{}}%
\begin{eqnarray*}
  && {\testupperi}\\
  && {\testupperii}\\
  && {\testloweri}\\
  && {\testlowerii}\\
  && {\testupgreeki}\\
  && {\testupgreekii}
\end{eqnarray*}%

%Math Bold Italic (\texttt{\string\mathbm})
%\def\test#1{\mathbm{#1}_{i}+{}}%
%\begin{eqnarray*}
%  && {\testupperi}\\
%  && {\testupperii}\\
%  && {\testloweri}\\
%  && {\testlowerii}\\
%  && {\testupgreeki}\\
%  && {\testupgreekii}\\
%  && {\testlowgreeki}\\
%  && {\testlowgreekii}\\
%  && {\testlowgreekiii}
%\end{eqnarray*}

Math Bold (\texttt{\string\mathbf})
\def\test#1{\mathbf{#1}_{i}+{}}%
\begin{eqnarray*}
  && {\testupperi}\\
  && {\testupperii}\\
  && {\testloweri}\\
  && {\testlowerii}\\
  && {\testupgreeki}\\
  && {\testupgreekii}
\end{eqnarray*}%

Math Calligraphic (\texttt{\string\mathcal})
\def\test#1{\mathcal{#1}_{i}+{}}%
\begin{eqnarray*}
  && {\testupperi}\\
  && {\testupperii}
\end{eqnarray*}%


\subsection{Accent Positioning \showfamily}

Default
\def\test#1{\hat{#1}+{}}%
\begin{eqnarray*}
  && {\testnums}\\
  && {\testupperi}\\
  && {\testupperii}\\
  && {\testloweri}\\
  && {\testlowerii}\\
  && {\testupgreeki}\\
  && {\testupgreekii}\\
  && {\testlowgreeki}\\
  && {\testlowgreekii}\\
  && {\testlowgreekiii}
\end{eqnarray*}%

Math Italic (\texttt{\string\mathit})
\def\test#1{\hat{\mathit{#1}}+{}}%
\begin{eqnarray*}
  && {\testnums}\\
  && {\testupperi}\\
  && {\testupperii}\\
  && {\testloweri} \test\ell \test\wp \test\imath \test\jmath \tilde{i} \\
  && {\testlowerii}\\
  && {\testupgreeki}\\
  && {\testupgreekii}\\
  && {\testlowgreeki}\\
  && {\testlowgreekii}\\
  && {\testlowgreekiii}
\end{eqnarray*}%

Math Roman (\texttt{\string\mathrm})
\def\test#1{\hat{\mathrm{#1}}+{}}%
\begin{eqnarray*}
  && {\testnums}\\
  && {\testupperi}\\
  && {\testupperii}\\
  && {\testloweri}\\
  && {\testlowerii}\\
  && {\testupgreeki}\\
  && {\testupgreekii}
\end{eqnarray*}%

%Math Italic Bold (\texttt{\string\mathbm})
%\def\test#1{\hat{\mathbm{#1}}+{}}%
%\begin{eqnarray*}
%  && {\testnums}\\
%  && {\testupperi}\\
%  && {\testupperii}\\
%  && {\testloweri}\\
%  && {\testlowerii}\\
%  && {\testupgreeki}\\
%  && {\testupgreekii}\\
%  && {\testlowgreeki}\\
%  && {\testlowgreekii}\\
%  && {\testlowgreekiii}
%\end{eqnarray*}%

Math Bold (\texttt{\string\mathbf})
\def\test#1{\hat{\mathbf{#1}}+{}}%
\begin{eqnarray*}
  && {\testnums}\\
  && {\testupperi}\\
  && {\testupperii}\\
  && {\testloweri}\\
  && {\testlowerii}\\
  && {\testupgreeki}\\
  && {\testupgreekii}
\end{eqnarray*}

Math Calligraphic (\texttt{\string\mathcal})
\def\test#1{\hat{\mathcal{#1}}+{}}%
\begin{eqnarray*}
  && {\testupperi}\\
  && {\testupperii}
\end{eqnarray*}%


\subsection{Differentials \showfamily}

\begin{eqnarray*}
\gdef\test#1{\dit #1+{}}%
  && {\testupperi}\\
  && {\testupperii}\\
  && {\testloweri}\\
  && {\testlowerii}\\
  && {\testupgreeki}\\
  && {\testupgreekii}\\
  && {\testlowgreeki}\\
  && {\testlowgreekii}\\
  && {\testlowgreekiii}\\
\gdef\test#1{\dit \mathrm{#1}+{}}%
  && {\testupgreeki}\\
  && {\testupgreekii}
\end{eqnarray*}%

\begin{eqnarray*}
\gdef\test#1{\dup #1+{}}%
  && {\testupperi}\\
  && {\testupperii}\\
  && {\testloweri}\\
  && {\testlowerii}\\
  && {\testupgreeki}\\
  && {\testupgreekii}\\
  && {\testlowgreeki}\\
  && {\testlowgreekii}\\
  && {\testlowgreekiii}\\
\gdef\test#1{\dup \mathrm{#1}+{}}%
  && {\testupgreeki}\\
  && {\testupgreekii}
\end{eqnarray*}%

\begin{eqnarray*}
\gdef\test#1{\partial #1+{}}%
  && {\testupperi}\\
  && {\testupperii}\\
  && {\testloweri}\\
  && {\testlowerii}\\
  && {\testupgreeki}\\
  && {\testupgreekii}\\
  && {\testlowgreeki}\\
  && {\testlowgreekii}\\
  && {\testlowgreekiii}\\
\gdef\test#1{\partial \mathrm{#1}+{}}%
  && {\testupgreeki}\\
  && {\testupgreekii}
\end{eqnarray*}%


\subsection{Slash Kerning \showfamily}

\def\test#1{1 \divslash #1+{}}
\begin{eqnarray*}
  && {\testupperi}\\
  && {\testupperii}\\
  && {\testloweri}\\
  && {\testlowerii}\\
  && {\testupgreeki}\\
  && {\testupgreekii}\\
  && {\testlowgreeki}\\
  && {\testlowgreekii}\\
  && {\testlowgreekiii}
\end{eqnarray*}

\def\test#1{#1 \divslash 2+{}}
\begin{eqnarray*}
  && {\testupperi}\\
  && {\testupperii}\\
  && {\testloweri}\\
  && {\testlowerii}\\
  && {\testupgreeki}\\
  && {\testupgreekii}\\
  && {\testlowgreeki}\\
  && {\testlowgreekii}\\
  && {\testlowgreekiii}
\end{eqnarray*}


\subsection{(Big) Operators \showfamily}

\def\testop#1{#1_{i=1}^{n} x^{n} \quad}
$
	\testop\sum
	\testop\prod
	\testop\coprod
	\testop\int
	\testop\oint
$

\noindent%
$
	\testop\bigotimes
	\testop\bigoplus
	\testop\bigodot
	\testop\bigwedge
	\testop\bigvee
	\testop\biguplus
	\testop\bigcup
	\testop\bigcap
	\testop\bigsqcup
	% \testop\bigsqcap
$

\begin{displaymath}
  \testop\sum
  \testop\prod
  \testop\coprod
  \testop\int
  \testop\oint
\end{displaymath}
\begin{displaymath}
  \testop\bigotimes
  \testop\bigoplus
  \testop\bigodot
  \testop\bigwedge
  \testop\bigvee
  \testop\biguplus
  \testop\bigcup
  \testop\bigcap
  \testop\bigsqcup
% \testop\bigsqcap
\end{displaymath}


\subsection{Radicals \showfamily}

\begin{displaymath}
  \sqrt{x+y} \qquad \sqrt{x^{2}+y^{2}} \qquad
  \sqrt{x_{i}^{2}+y_{j}^{2}} \qquad
  \sqrt{\left(\frac{\cos x}{2}\right)} \qquad
  \sqrt{\left(\frac{\sin x}{2}\right)}
\end{displaymath}

\begingroup
\delimitershortfall-1pt
\begin{displaymath}
  \sqrt{\sqrt{\sqrt{\sqrt{\sqrt{\sqrt{\sqrt{x+y}}}}}}}
\end{displaymath}
\endgroup % \delimitershortfall


\subsection{Over- and Underbraces \showfamily}

\begin{displaymath}
  \overbrace{x} \quad
  \overbrace{x+y} \quad
  \overbrace{x^{2}+y^{2}} \quad
  \overbrace{x_{i}^{2}+y_{j}^{2}} \quad
  \underbrace{x} \quad
  \underbrace{x+y} \quad
  \underbrace{x_{i}+y_{j}} \quad
  \underbrace{x_{i}^{2}+y_{j}^{2}} \quad
\end{displaymath}


\subsection{Normal and Wide Accents \showfamily}

\begin{displaymath}
  \dot{x} \quad
  \ddot{x} \quad
  \vec{x} \quad
  \bar{x} \quad
  \overline{x} \quad
  \overline{xx} \quad
  \tilde{x} \quad
  \widetilde{x} \quad
  \widetilde{xx} \quad
  \widetilde{xxx} \quad
  \hat{x} \quad
  \widehat{x} \quad
  \widehat{xx} \quad
  \widehat{xxx} \quad
\end{displaymath}

\begin{displaymath}
  \hat{x} \quad
  \check{x} \quad
  \tilde{x} \quad
  \acute{x} \quad
  \grave{x} \quad
  \dot{x} \quad
  \ddot{x} \quad
  \breve{x} \quad
  \bar{x} \quad
  \vec{x} \quad
\end{displaymath}


\subsection{Long Arrows \showfamily}

\begin{displaymath}
  \leftarrow \mathrel{-} \rightarrow \quad
  \leftrightarrow \quad
  \longleftarrow  \quad
  \longrightarrow \quad
  \longleftrightarrow \quad
  \Leftarrow = \Rightarrow \quad
  \Leftrightarrow \quad
  \Longleftarrow  \quad
  \Longrightarrow \quad
  \Longleftrightarrow \quad
\end{displaymath}


\subsection{Left and Right Delimiters \showfamily}

\def\testdelim#1#2{ - #1 f #2 - }
\begin{displaymath}
  \testdelim()
  \testdelim[]
  \testdelim\lfloor\rfloor
  \testdelim\lceil\rceil
  \testdelim\langle\rangle
  \testdelim\{\}
\end{displaymath}

Using {\tt\string\left} and {\tt\string\right}.
\def\testdelim#1#2{ - \left#1 f \right#2 - }
\begin{displaymath}
  \testdelim()
  \testdelim[]
  \testdelim\lfloor\rfloor
  \testdelim\lceil\rceil
  \testdelim\langle\rangle
  \testdelim\{\}
% \testdelim\lgroup\rgroup
% \testdelim\lmoustache\rmoustache
\end{displaymath}
\begin{displaymath}
  \testdelim)(
  \testdelim][
  \testdelim//
  \testdelim\backslash\backslash
  \testdelim/\backslash
  \testdelim\backslash/
\end{displaymath}


\subsection{Big-g-g Delimiters \showfamily}

\def\testdelim#1#2{%
  - \left#1\left#1\left#1\left#1\left#1\left#1\left#1\left#1 -
  \right#2\right#2\right#2\right#2\right#2\right#2\right#2\right#2 -}

\begingroup
\delimitershortfall-1pt
\begin{displaymath}
  \testdelim\lfloor\rfloor
  \qquad
  \testdelim()
\end{displaymath}
\begin{displaymath}
  \testdelim\lceil\rceil
  \qquad
  \testdelim\{\}
\end{displaymath}
\begin{displaymath}
  \testdelim[]
  \qquad
  \testdelim\lgroup\rgroup
\end{displaymath}
\begin{displaymath}
  \testdelim\langle\rangle
  \qquad
  \testdelim\lmoustache\rmoustache
\end{displaymath}
\begin{displaymath}
  \testdelim\uparrow\downarrow \quad
  \testdelim\Uparrow\Downarrow \quad
\end{displaymath}
\endgroup % \delimitershortfall

\def\X#1{$x #1 y$ &\tt\string#1}
\def\Y#1{$\big#1$ &\tt\string#1}
\def\Z#1{$x #1 y$}
\def\W#1#2{$#1{#2}$ &\tt\string#1\string{#2\string}}


\subsection{Binary Operators \showfamily}

\begin{tabular}{*8l}
\X\pm           &\X\cap         &\X\diamond             &\X\oplus     \\
\X\mp           &\X\cup         &\X\bigtriangleup       &\X\ominus    \\
\X\times        &\X\uplus       &\X\bigtriangledown     &\X\otimes    \\
\X\div          &\X\sqcap       &\X\triangleleft        &\X\oslash    \\
\X\ast          &\X\sqcup       &\X\triangleright       &\X\odot      \\
\X\star         &\X\vee         &\X\lhd                 &\X\bigcirc   \\
\X\circ         &\X\wedge       &\X\rhd                 &\X\dagger    \\
\X\bullet       &\X\setminus    &\X\unlhd               &\X\ddagger   \\
\X\cdot         &\X\wr          &\X\unrhd               &\X\S         \\
\X+             &\X-            &\X\amalg               &\X\P
\end{tabular}


\subsection{Relations \showfamily}

\begin{tabular}{*8l}
\X\leq          &\X\geq         &\X\equiv       &\X\models      \\
\X\prec         &\X\succ        &\X\sim         &\X\perp        \\
\X\preceq       &\X\succeq      &\X\simeq       &\X\mid         \\
\X\ll           &\X\gg          &\X\asymp       &\X\parallel    \\
\X\subset       &\X\supset      &\X\approx      &\X\bowtie      \\
\X\subseteq     &\X\supseteq    &\X\cong        &\X\Join        \\
\X\sqsubset     &\X\sqsupset    &\X\neq         &\X\smile       \\
\X\sqsubseteq   &\X\sqsupseteq  &\X\doteq       &\X\frown       \\
\X\in           &\X\ni          &\X\propto      &\X=            \\
\X\vdash        &\X\dashv       &\X<            &\X>            \\
\X:
\end{tabular}


\subsection{Punctuation \showfamily}

\begin{tabular}{*{5}{lp{3.2em}}}
\X,     &\X;    &\X\colon       &\X\ldotp       &\X\cdotp
\end{tabular}


\subsection{Arrows \showfamily}

\begin{tabular}{*6l}
\X\leftarrow            &\X\longleftarrow       &\X\uparrow     \\
\X\Leftarrow            &\X\Longleftarrow       &\X\Uparrow     \\
\X\rightarrow           &\X\longrightarrow      &\X\downarrow   \\
\X\Rightarrow           &\X\Longrightarrow      &\X\Downarrow   \\
\X\leftrightarrow       &\X\longleftrightarrow  &\X\updownarrow \\
\X\Leftrightarrow       &\X\Longleftrightarrow  &\X\Updownarrow \\
\X\mapsto               &\X\longmapsto          &\X\nearrow     \\
\X\hookleftarrow        &\X\hookrightarrow      &\X\searrow     \\
\X\leftharpoonup        &\X\rightharpoonup      &\X\swarrow     \\
\X\leftharpoondown      &\X\rightharpoondown    &\X\nwarrow     \\
\X\rightleftharpoons    &\X\leadsto
\end{tabular}


\subsection{Miscellaneous Symbols \showfamily}

\begin{tabular}{*8l}
\X\ldots        &\X\cdots       &\X\vdots       &\X\ddots       \\
\X\aleph        &\X\prime       &\X\forall      &\X\infty       \\
\X\hbar         &\X\emptyset    &\X\exists      &\X\Box         \\
\X\imath        &\X\nabla       &\X\neg         &\X\Diamond     \\
\X\jmath        &\X\surd        &\X\flat        &\X\triangle    \\
\X\ell          &\X\top         &\X\natural     &\X\clubsuit    \\
\X\wp           &\X\bot         &\X\sharp       &\X\diamondsuit \\
\X\Re           &\X\|           &\X\backslash   &\X\heartsuit   \\
\X\Im           &\X\angle       &\X\partial     &\X\spadesuit   \\
\X\mho          &\X.            &\X|            &\X!
\end{tabular}


\subsection{Variable-Sized Operators \showfamily}

\begin{tabular}{*6l}
\X\sum          &\X\bigcap      &\X\bigodot     \\
\X\prod         &\X\bigcup      &\X\bigotimes   \\
\X\coprod       &\X\bigsqcup    &\X\bigoplus    \\
\X\int          &\X\bigvee      &\X\biguplus    \\
\X\oint         &\X\bigwedge
\end{tabular}


\subsection{Log-Like Operators \showfamily}

\begin{tabular}{*8l}
\Z\arccos &\Z\cos  &\Z\csc &\Z\exp &
           \Z\ker    &\Z\limsup &\Z\min &\Z\sinh \\
\Z\arcsin &\Z\cosh &\Z\deg &\Z\gcd &
           \Z\lg     &\Z\ln     &\Z\Pr  &\Z\sup  \\
\Z\arctan &\Z\cot  &\Z\det &\Z\hom &
           \Z\lim    &\Z\log    &\Z\sec &\Z\tan  \\
\Z\arg    &\Z\coth &\Z\dim &\Z\inf &
           \Z\liminf &\Z\max    &\Z\sin &\Z\tanh
\end{tabular}


\subsection{Delimiters \showfamily}

\begin{tabular}{*8l}
\X(             &\X)            &\X\uparrow     &\X\Uparrow     \\
\X[             &\X]            &\X\downarrow   &\X\Downarrow   \\
\X\{            &\X\}           &\X\updownarrow &\X\Updownarrow \\
\X\lfloor       &\X\rfloor      &\X\lceil       &\X\rceil       \\
\X\langle       &\X\rangle      &\X/            &\X\backslash   \\
\X|             &\X\|
\end{tabular}


\subsection{Large Delimiters \showfamily}

\begin{tabular}{*8l}
\Y\rmoustache&  \Y\lmoustache&  \Y\rgroup&      \Y\lgroup\\[5pt]
\Y\arrowvert&   \Y\Arrowvert&   \Y\bracevert
\end{tabular}


\subsection{Math Mode Accents \showfamily}

\begin{tabular}{*{10}l}
\W\hat{a}     &\W\acute{a}  &\W\bar{a}    &\W\dot{a}    &\W\breve{a}\\
\W\check{a}   &\W\grave{a}  &\W\vec{a}    &\W\ddot{a}   &\W\tilde{a}\\
\end{tabular}


\subsection{Miscellaneous Constructions \showfamily}

\begin{tabular}{*4l}
\W\widetilde{abc}       &\W\widehat{abc}                        \\
\W\overleftarrow{abc}   &\W\overrightarrow{abc}                 \\
\W\overline{abc}        &\W\underline{abc}                      \\
\W\overbrace{abc}       &\W\underbrace{abc}                     \\[5pt]
\W\sqrt{abc}            &$\sqrt[n]{abc}$&\verb|\sqrt[n]{abc}|   \\
$f'$&\verb|f'|          &$\frac{abc}{xyz}$&\verb|\frac{abc}{xyz}|
\end{tabular}


\subsection{{\textls[40]{AMS}} Delimiters \showfamily}

\begin{tabular}{*8l}
\X\ulcorner&\X\urcorner&\X\llcorner&\X\lrcorner
\end{tabular}


\subsection{{\textls[40]{AMS}} Arrows \showfamily}

\begin{tabular}{*8l}
\X\dashrightarrow       &\X\dashleftarrow
        \\ \X\leftleftarrows      &\X\leftrightarrows     \\
\X\Lleftarrow           &\X\twoheadleftarrow
        \\ \X\leftarrowtail       &\X\looparrowleft       \\
\X\leftrightharpoons    &\X\curvearrowleft
        \\ \X\circlearrowleft     &\X\Lsh                 \\
\X\upuparrows           &\X\upharpoonleft
        \\ \X\downharpoonleft     &\X\multimap            \\
\X\leftrightsquigarrow  &\X\rightrightarrows
        \\ \X\rightleftarrows     &\X\rightrightarrows    \\
\X\rightleftarrows      &\X\twoheadrightarrow
        \\ \X\rightarrowtail      &\X\looparrowright      \\
\X\rightleftharpoons    &\X\curvearrowright
        \\ \X\circlearrowright    &\X\Rsh                 \\
\X\downdownarrows       &\X\upharpoonright
        \\ \X\downharpoonright    &\X\rightsquigarrow
\end{tabular}


\subsection{{\textls[40]{AMS}} Negated Arrows \showfamily}

\begin{tabular}{*8l}
\X\nleftarrow   &\X\nrightarrow \\ \X\nLeftarrow  &\X\nRightarrow \\
\X\nleftrightarrow&\X\nLeftrightarrow
\end{tabular}


\subsection{{\textls[40]{AMS}} Greek \showfamily}

\begin{tabular}{*4l}
\X\digamma      &\X\varkappa
\end{tabular}


\subsection{{\textls[40]{AMS}} Hebrew \showfamily}

\begin{tabular}{*6l}
\X\beth &\X\daleth      &\X\gimel
\end{tabular}


\subsection{{\textls[40]{AMS}} Miscellaneous \showfamily}

\begin{tabular}{*8l}
\X\hbar         &\X\hslash      \\ \X\vartriangle &\X\triangledown      \\
\X\square       &\X\lozenge     \\ \X\circledS    &\X\angle             \\
\X\measuredangle&\X\nexists     \\ \X\mho         &\X\Finv$^u$          \\
\X\Game$^u$     &\X\Bbbk$^u$    \\ \X\backprime   &\X\varnothing        \\
\X\blacktriangle&\X\blacktriangledown \\ \X\blacksquare&\X\blacklozenge  \\
\X\bigstar      &\X\sphericalangle     \\ \X\complement  &\X\eth       \\
\X\diagup$^u$   &\X\diagdown$^u$
\end{tabular}

$^u$ Not defined in {\tt amssymb.sty}, define using the
\verb|\newsymbol|  command.


\subsection{{\textls[40]{AMS}} Binary Operators \showfamily}

\begin{tabular}{*8l}
\X\dotplus      &\X\smallsetminus \\ \X\Cap        &\X\Cup               \\
\X\barwedge     &\X\veebar      \\ \X\doublebarwedge&\X\boxminus        \\
\X\boxtimes     &\X\boxdot      \\ \X\boxplus     &\X\divideontimes     \\
\X\ltimes       &\X\rtimes      \\ \X\leftthreetimes&\X\rightthreetimes \\
\X\curlywedge   &\X\curlyvee    \\ \X\circleddash &\X\circledast        \\
\X\circledcirc  &\X\centerdot   \\ \X\intercal
\end{tabular}


\subsection{{\textls[40]{AMS}} Relations \showfamily}

\begin{tabular}{*2l}
\X\leqslant    \\\X\lesssim    \\
\X\approxeq    \\\X\lll        \\
\X\lesseqgtr   \\\X\doteqdot   \\
\X\fallingdotseq\\\X\backsimeq  \\
\X\Subset      \\\X\preccurlyeq\\
\X\precsim     \\\X\vartriangleleft\\
\X\vDash      \\\X\smallsmile \\
\X\bumpeq      \\\X\geqq       \\
\X\eqslantgtr  \\\X\gtrapprox  \\
\X\ggg         \\\X\gtreqless  \\
\X\eqcirc      \\\X\triangleq  \\
\X\thickapprox \\\X\Supset     \\
\X\succcurlyeq \\\X\succsim    \\
\X\vartriangleright\\\X\Vdash      \\
\X\shortparallel\\\X\pitchfork  \\
\X\blacktriangleleft \\\X\backepsilon\\
\X\because
\end{tabular}


\subsection{{\textls[40]{AMS}} Negated Relations \showfamily}

\begin{tabular}{*8l}
\X\nless        &\X\nleq        \\ \X\nleqslant   &\X\nleqq       \\
\X\lneq         &\X\lneqq       \\ \X\lvertneqq   &\X\lnsim       \\
\X\lnapprox     &\X\nprec       \\ \X\npreceq     &\X\precnsim    \\
\X\precnapprox  &\X\nsim        \\ \X\nshortmid   &\X\nmid        \\
\X\nvdash       &\X\nvDash      \\ \X\ntriangleleft&\X\ntrianglelefteq\\
\X\nsubseteq    &\X\subsetneq   \\ \X\varsubsetneq&\X\subsetneqq  \\
\X\varsubsetneqq&\X\ngtr        \\ \X\ngeq        &\X\ngeqslant   \\
\X\ngeqq        &\X\gneq        \\ \X\gneqq       &\X\gvertneqq   \\
\X\gnsim        &\X\gnapprox    \\ \X\nsucc       &\X\nsucceq     \\
\X\nsucceqq     &\X\succnsim    \\ \X\succnapprox &\X\ncong       \\
\X\nshortparallel&\X\nparallel  \\ \X\nvDash      &\X\nVDash      \\
\X\ntriangleright&\X\ntrianglerighteq \\ \X\nsupseteq&\X\nsupseteqq\\
\X\supsetneq    &\X\varsupsetneq \\ \X\supsetneqq  &\X\varsupsetneqq
\end{tabular}

\clearpage

\end{appendix}


\renewcommand{\thepage}{}


\clearpage

\ifthenelse{\equal{\manuscripttype}{BA} \OR \equal{\manuscripttype}{MA}}{%
	\ifthenelse{%
		\equal{\manuscriptlanguage}{DE}%
	}{% German
		\section*{Selbstständigkeitserklärung}
		\pdfbookmark[1]{Selbstständigkeitserklärung}{declaration}
		Ich versichere hiermit, dass ich die vorstehende \ifthenelse{\equal{\manuscripttype}{BA}}{Bachelorarbeit}{Masterarbeit} selbstständig verfasst und keine anderen als die angegebenen Quellen und Hilfsmittel benutzt habe, dass die vorgelegte Arbeit noch an keiner anderen Hochschule zur Prüfung vorgelegt wurde und dass sie weder ganz noch in Teilen bereits veröffentlicht wurde. Wörtliche Zitate und Stellen, die anderen Werken dem Sinn nach entnommen sind, habe ich in jedem einzelnen Fall kenntlich gemacht.%
	}{% English
		\section*{Declaration}
		\pdfbookmark[1]{Declaration}{declaration}
		I~hereby declare that I~have written the above \ifthenelse{\equal{\manuscripttype}{BA}}{bachelor}{master}'s thesis independently and have used no other sources and aids than those indicated, that the submitted work has not been previously presented for examination at another university, and that it has not been published either in whole or in part. I~have identified all direct quotes and paraphrased ideas taken from other works in each individual case.%
	}
	\par
	\vspace{\baselineskip}
	\noindent\printdate{\manuscriptdate} \par
	\vspace{3\baselineskip} \par
	\noindent\studentfullname%
}{}

%\clearpage


\end{document}