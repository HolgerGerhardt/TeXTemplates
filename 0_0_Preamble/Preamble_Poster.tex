% !TeX program = pdflatex
% !TeX TXS-program:compile = txs:///pdflatex/
% !TeX TS-program = pdflatex
% !BIB program = biber
% !TeX TXS-program:bibliography = txs:///biber




%%%%%%%%%%%%%%%%%%%%%%%%%%%%
%%  FUNDAMENTAL PACKAGES  %%
%%%%%%%%%%%%%%%%%%%%%%%%%%%%


\usepackage[utf8]{inputenc}

\usepackage[LGR, T1]{fontenc}  % LGR needed for sansserif math

\usepackage[ngerman, american, USenglish]{babel}  % German and US English hyphenation and quotation marks
\selectlanguage{USenglish}

\usepackage{calc}  % Enables calculations for lengths
\usepackage[nomessages]{fp}	 % Enables calculations in LaTeX

\usepackage{etoolbox}  % Enables manipulating LaTeX commmands via \preto, \appto, \patchcmd, etc.
\usepackage{xpatch}  % Enables manipulating LaTeX commmands via \xpatchcmd etc.

\PassOptionsToPackage{cmyk}{xcolor}
\definecolor{UBonnBlue}  {cmyk}{1.00, 0.70, 0.00, 0.00}
\definecolor{UBonnYellow}{cmyk}{0.00, 0.20, 1.00, 0.00}
\definecolor{UBonnGray}  {cmyk}{0.00, 0.00, 0.15, 0.55}
\colorlet{SpotColor}{UBonnBlue}
\definecolor{CMYKBlack}{cmyk}{0.00, 0.00, 0.00, 1.00}
\colorlet{CMYKGray}{CMYKBlack!50!}
\setbeamercolor{normal text}{fg=CMYKBlack}
\setbeamercolor{alerted text}{fg=SpotColor}
\setbeamerfont{alerted text}{family=\sffamily, series=\bfseries}

\setbeamercolor{block alerted title}{bg=SpotColor, fg=white}
\setbeamercolor{block body}{fg=CMYKBlack}

\setbeamercolor{title in headline}{fg=SpotColor}
\setbeamercolor{institute in headline}{fg=CMYKBlack}

\hypersetup{
	colorlinks = false,
	urlcolor   = SpotColor,
	linkcolor  = SpotColor,
	citecolor  = SpotColor
}
\urlstyle{same}
\let \urlorig \url
\renewcommand{\url}[1]{\textcolor{SpotColor}{\urlorig{#1}}}
\let \hreforig \href
\renewcommand{\href}[2]{\textcolor{SpotColor}{\hreforig{#1}{#2}}}
\newcommand{\email}[1]{\href{mailto:#1}{\nolinkurl{#1}}}

\usepackage[absolute,overlay]{textpos}
\setlength{\TPboxrulesize}{1pt}
\textblockrulecolor{SpotColor}

\usepackage{ragged2e}

% microtype enables ``hanging punctuation'' and other neat microtypographic features
\ifxetex
	\usepackage[protrusion=true, expansion=false]{microtype}
\else
	\usepackage[protrusion=true, expansion=false, kerning=true]{microtype}
\fi

% Use the bm (= boldmath) package for better support of setting math in bold ==>
% Prevent the "Too many math fonts used" error:
\newcommand{\bmmax}{0}
\newcommand{\hmmax}{0}
\usepackage{bm}
% <==




%%%%%%%%%%%%%%%%%%%%%%%
%%  BEAMER SETTINGS  %%
%%%%%%%%%%%%%%%%%%%%%%%


% Use the beamerposter package for laying out the poster:
% \usepackage[size=a0, scale=1.1]{beamerposter}
	% A0: 840 mm by 1188 mm in beamerposter
\usepackage[size=custom, width=84.1, height=118.9, scale=1.2]{beamerposter}
	% 841 mm by 1189 mm: A0
\beamertemplatenavigationsymbolsempty

\usefonttheme{professionalfonts}

\newcommand{\highlight}[1]{\alert{\textbf{#1}}}

% Make itemize justified instead of flushleft
% ==>
\makeatletter
\renewcommand{\itemize}[1][]{%
	\beamer@ifempty{#1}{}{\def\beamer@defaultospec{#1}}%
	\ifnum \@itemdepth >2\relax\@toodeep\else
	\advance\@itemdepth\@ne
	\beamer@computepref\@itemdepth% sets \beameritemnestingprefix
	\usebeamerfont{itemize/enumerate \beameritemnestingprefix body}%
	\usebeamercolor[fg]{itemize/enumerate \beameritemnestingprefix body}%
	\usebeamertemplate{itemize/enumerate \beameritemnestingprefix body begin}%
	\list
	{\usebeamertemplate{itemize \beameritemnestingprefix item}}
	{\def\makelabel##1{%
			{%
				\hss\llap{{%
						\usebeamerfont*{itemize \beameritemnestingprefix item}%
						\usebeamercolor[fg]{itemize \beameritemnestingprefix item}##1}}%
			}%
		}%
	}
	\fi%
	\beamer@cramped%
	\justifying% NEW
	%\raggedright% ORIGINAL
	\beamer@firstlineitemizeunskip%
}
\makeatother
% <==

% Add white space before lists and between items inside lists
% ==>
\makeatletter
\def\@listi{\leftmargin\leftmarginii
	\topsep 0.333333\baselineskip % Spacing before lists
	\parsep 0\p@ \@plus\p@
	\itemsep 0.333333\baselineskip} % Spacing between items
\makeatother
% <==

\setbeamertemplate{itemize item}{\raisebox{-1.5pt}{\textbf{\textbullet}}}
\setbeamercolor{item}{fg=SpotColor}

\setbeamerfont{structure}{family=\sffamily, shape=\upshape, series=\bfseries}

\setbeamerfont{title}{series=\bfseries}
\setbeamercolor{title}{fg=SpotColor}

% Poster title
\setbeamertemplate{headline}{
	\leavevmode
	\begin{columns}
		\begin{column}{\linewidth}
			\vskip-0.6\topmargin
			% \centering
			\Huge\textbf{\textsf{\textcolor{SpotColor}{\inserttitle}}}\\[-57pt]
			{\color{SpotColor}\rule{\linewidth}{1pt}\\[-31pt]}
			\usebeamercolor{author in headline}{\Large\insertauthor}\\[-42pt]
			\usebeamercolor{institute in headline}{\large\textsf{\insertinstitute}}\\[1ex]
			\vskip1cm
		\end{column}
		\vspace{1cm}
	\end{columns}
	\vspace{0.5in}
	% \hspace{0.5in}\begin{beamercolorbox}[wd=47in,colsep=0.15cm]{cboxb}\end{beamercolorbox}
	% \hspace{\leftmargin}\rule{\textwidth-\leftmargin-\rightmargin}{0.4pt}
	\vspace{0.1in}
}

% Regular block definition
\setbeamerfont{block title}{family=\sffamily, series=\bfseries}
\setbeamercolor{block title}{fg=SpotColor}
\setbeamertemplate{block begin}{
	\par\vskip\medskipamount
	\begin{beamercolorbox}[colsep*=0ex, dp=30pt, left]{block title}
		\vskip-0.25cm
		\usebeamerfont{block title}%
		\large\insertblocktitle\\[-22pt]
		\rule{\linewidth}{1pt}\\[22pt]
	\end{beamercolorbox}
	{\parskip0pt\par}
	\ifbeamercolorempty[bg]{block title}
	{}
	{\ifbeamercolorempty[bg]{block body}{}{\nointerlineskip\vskip-0.5pt}}
	\usebeamerfont{block body}
	\vskip-0.75cm
	\begin{beamercolorbox}[colsep*=0ex,vmode]{block body}
		\setlength{\baselineskip}{36pt}%
		\sloppypar\justifying%
}
\setbeamertemplate{block end}{
	\end{beamercolorbox}
	\vskip\smallskipamount
}

% Alert block definition (with frame)
\setbeamerfont{block alerted title}{parent=structure, series=\fontseries{\savesfbfseries}\selectfont}
\newlength{\inboxwd}
\newlength{\iinboxwd}
\newlength{\inboxrule}
\setbeamertemplate{block alerted begin}{
  \par
  \begin{beamercolorbox}[sep=25pt, rounded=false, dp=2pt, leftskip=12pt, rightskip=12pt]{block alerted title}
    \usebeamerfont{block alerted title}%
    {\sloppy\large\insertblocktitle}
  \end{beamercolorbox}
  {\parskip0pt\par}
  \usebeamerfont{block body}
  \vskip-0.25in
  \begin{beamercolorbox}[sep=0.5cm, rounded=false, center]{block body}
  \setlength{\inboxwd}{\linewidth}
  \addtolength{\inboxwd}{-1cm}
  \begin{beamercolorbox}[rounded=false, wd={\inboxwd}, center]{block body}
  \setlength{\iinboxwd}{\inboxwd}
  \setlength{\inboxrule}{\inboxwd}
  \addtolength{\iinboxwd}{-0.5cm}
  \addtolength{\inboxrule}{0.5cm}
  \begin{center}
  \begin{minipage}{\iinboxwd}
  \centering%
  \sffamily
}
\setbeamertemplate{block alerted end}{
  \end{minipage}
  \end{center}
  \end{beamercolorbox}
  \end{beamercolorbox}
  \vskip\smallskipamount
}




%%%%%%%%%%%%%%%%%%%%%%%%%%%%%
%%  BIBLIOGRAPHY SETTINGS  %%
%%%%%%%%%%%%%%%%%%%%%%%%%%%%%


%\usepackage{natbib}
\newlength{\bibindent}
\newlength{\bibitemsep}
\newlength{\bibsep}

\setbeamertemplate{bibliography item}{}
\setbeamercolor{bibliography entry author}  {fg=SpotColor}
\setbeamercolor{bibliography entry title}   {fg=SpotColor}
\setbeamercolor{bibliography entry location}{fg=SpotColor}
\setbeamercolor{bibliography entry note}    {fg=SpotColor}




%%%%%%%%%%%%%%%%%%%%%
%%  MISCELLANEOUS  %%
%%%%%%%%%%%%%%%%%%%%%


\newcommand{\affilindicator}[1]{\textsuperscript{\kern.75pt \textit{#1}}}
\usepackage[math]{blindtext}  % For debugging only
\makeatletter
\def\blindtext@american{}
\makeatother

% Allow for fine-grained scaling of font sizes
% ==>
\usepackage{relsize}
\renewcommand\RSpercentTolerance{1}
% Enabling slightly reduced font for CAPS:
\newcommand{\caps}[1]{\textscale{0.96}{\textls[35]{\MakeUppercase{#1}}}}
% <==




%%%%%%%%%%%%%%%%%%%%%%%
%%  DOCUMENT LAYOUT  %%
%%%%%%%%%%%%%%%%%%%%%%%


\setlength{\leftmargin}{4cm}
\setlength{\rightmargin}{4cm}

\newenvironment{parblock}[1]%
	{\begin{block}{#1}%
	 \setlength{\parindent}{\baselinestretch\baselineskip}%
	 \noindent\ignorespaces%
	}%
	{\end{block}}

\newcommand{\numcols}{3}
\FPeval{\numcolsminusone}{\numcols-1}

\newlength{\colsep}
\setlength{\colsep}{1.5cm}
\newlength{\totalcolsep}
\setlength{\totalcolsep}{\numcolsminusone\colsep}

\newlength{\colwidth}
\setlength{\colwidth}{(\paperwidth-\leftmargin-\rightmargin-\totalcolsep)/\numcols}

\setbeamersize{text margin left=\leftmargin, text margin right=\rightmargin}

\setlength{\fboxsep}{0pt}%
\setlength{\fboxrule}{1pt}%

\frenchspacing