% !TeX program = pdflatex
% !TeX TXS-program:compile = txs:///pdflatex/
% !TeX TS-program = pdflatex
% !BIB program = biber
% !TeX TXS-program:bibliography = txs:///biber




%%%%%%%%%%%%%%%%%%%%%%%%%%%%%%%%%%%%%%%%%%%%%%%%%
%%  SANS-SERIF MATH IN SANS-SERIF ENVIRONMENT  %%
%%%%%%%%%%%%%%%%%%%%%%%%%%%%%%%%%%%%%%%%%%%%%%%%%


% See https://tex.stackexchange.com/questions/41497/how-to-typeset-some-text-including-math-content-in-sans-serif
% See https://tex.stackexchange.com/questions/33165/make-mathfont-respect-the-surrounding-family

% Necessary for use of kpfonts
% ==>
\makeatletter
\newif\ifkp@upRm
\newif\ifkp@osm
\newif\ifkp@vosm
\makeatother
% <==

\DeclareMathVersion{normalup}
\DeclareMathVersion{boldup}
\DeclareMathVersion{sans}

%\SetSymbolFont{operators}{sans}{OT1}{jkpss}{m}{n}
%	% From http://mirrors.ctan.org/fonts/kpfonts/latex/kpfonts.sty
\SetSymbolFont{operators}   {sans}{OT1}{mdbch}{m}{n}
\SetSymbolFont{letters}     {sans}{OML}{jkpss}{m}{it}
	% From http://mirrors.ctan.org/fonts/kpfonts/latex/kpfonts.sty
%\SetSymbolFont{letters}     {sans}{OML}{cmbrm}{m}{it}
%\SetSymbolFont{symbols}     {sans}{OMS}{cmbrs}{m}{n}
\SetSymbolFont{symbols}     {sans}{OMS}{jkp}  {m}{n}
	% From http://mirrors.ctan.org/fonts/kpfonts/latex/kpfonts.sty
\DeclareSymbolFont{extrasymbols}  {OMS}{cmbrs}{m}{n}
\SetSymbolFont{extrasymbols}{sans}{OMS}{cmbrs}{m}{n}
	% Some symbols (e.g., \prime) look weird in kpfonts.
	% This provides the option to replace them by symbols from mathdesign-charter.

\SetMathAlphabet{\mathit} {sans}{T1}{\savesffamily}{\savesfmdseries}{it}
\SetMathAlphabet{\mathbf} {sans}{T1}{\savesffamily}{\savesfbfseries}{n}
\SetMathAlphabet{\mathtt} {sans}{OT1}{cmtl}{m}{n}
\SetMathAlphabet{\mathcal}{sans}{OMS}{ntxsy}{m}{n}
	% See https://tex.stackexchange.com/questions/231583/import-mathcal-symbols-from-txfonts
%\SetSymbolFont{largesymbols}{sans}{OMX}{jkpss}{m}{n}
%	% From http://mirrors.ctan.org/fonts/kpfonts/latex/kpfonts.sty
\SetSymbolFont{largesymbols} {sans}{OMX}{mdbch}{m}{n}
	% Using symbols like \int, \left(, etc. from mathdesign-charter because they look better than the ones included in kpfonts

\DeclareMathVersion{sansup}
\SetSymbolFont{letters}  {sansup}{OML}{jkpss}{m}{it}
\SetSymbolFont{symbols}  {sansup}{OMS}{jkp}  {m}{n}

\DeclareMathVersion{boldsans}
%\SetSymbolFont{operators}{boldsans}{OT1}{jkpss}{b}{n}
%	% From http://mirrors.ctan.org/fonts/kpfonts/latex/kpfonts.sty
\SetSymbolFont{operators}{boldsans}{OT1}{mdbch}{bx}{n}
\SetSymbolFont{letters}  {boldsans}{OML}{jkpss}{bx}{it}
	% From http://mirrors.ctan.org/fonts/kpfonts/latex/kpfonts.sty
%\SetSymbolFont{letters}  {boldsans}{OML}{mdbch}{bx}{it}
%\SetSymbolFont{letters}{boldsans}{OML}{cmbrm}{b}{it}
\SetSymbolFont{symbols}  {boldsans}{OMS}{jkp}  {bx}{n}
	% From http://mirrors.ctan.org/fonts/kpfonts/latex/kpfonts.sty
%\SetMathAlphabet{\mathrm}{boldsans}{OT1}{\savesffamily}{\savesfbfseries}{n}
\SetMathAlphabet{\mathit} {boldsans}{T1}{\savesffamily}{\savesfbfseries}{it}
\SetMathAlphabet{\mathtt} {boldsans}{T1}{cmtl}{b}{n}
\SetMathAlphabet{\mathcal}{boldsans}{OMS}{ntxsy}{b}{n}
%\SetSymbolFont{largesymbols}{boldsans}{OMX}{jkpss}{bx}{n}
%	% From http://mirrors.ctan.org/fonts/kpfonts/latex/kpfonts.sty
\SetSymbolFont{largesymbols}{boldsans}{OMX}{mdbch}{bx}{n}
	% Using symbols like \int, \left(, etc. from mathdesign-charter because they look better than the ones included in kpfonts

\DeclareMathVersion{boldsansup}
\SetSymbolFont{letters}{boldsansup}{OML}{jkpss}{bx}{it}
\SetSymbolFont{symbols}{boldsansup}{OMS}{jkp}  {bx}{n}

% Using glyphs for math mode from the custom sansserif font
\DeclareSymbolFont{uprightglyphs}{T1}{\savermfamily}{\savermmdseries}{n}
\SetSymbolFont{uprightglyphs}{normal}    {T1}{\savermfamily}{\savermmdseries}{n}
\SetSymbolFont{uprightglyphs}{normalup}  {T1}{\savermfamily}{\savermmdseries}{n}
\SetSymbolFont{uprightglyphs}{bold}      {T1}{\savermfamily}{\savermbfseries}{n}
\SetSymbolFont{uprightglyphs}{boldup}    {T1}{\savermfamily}{\savermbfseries}{n}
\SetSymbolFont{uprightglyphs}{sans}      {T1}{\savesffamily}{\savesfmdseries}{n}
\SetSymbolFont{uprightglyphs}{sansup}    {T1}{\savesffamily}{\savesfmdseries}{n}
\SetSymbolFont{uprightglyphs}{boldsans}  {T1}{\savesffamily}{\savesfbfseries}{n}
\SetSymbolFont{uprightglyphs}{boldsansup}{T1}{\savesffamily}{\savesfbfseries}{n}
\DeclareSymbolFont{italicglyphs} {T1}{\savermfamily}{\savermmdseries}{it}
\SetSymbolFont{italicglyphs} {normal}    {T1}{\savermfamily}{\savermmdseries}{it}
\SetSymbolFont{italicglyphs} {normalup}  {T1}{\savermfamily}{\savermmdseries}{n}
\SetSymbolFont{italicglyphs} {bold}      {T1}{\savermfamily}{\savermbfseries}{it}
\SetSymbolFont{italicglyphs} {boldup}    {T1}{\savermfamily}{\savermbfseries}{n}
\SetSymbolFont{italicglyphs} {sans}      {T1}{\savesffamily}{\savesfmdseries}{it}
\SetSymbolFont{italicglyphs} {sansup}    {T1}{\savesffamily}{\savesfmdseries}{n}
\SetSymbolFont{italicglyphs} {boldsans}  {T1}{\savesffamily}{\savesfbfseries}{it}
\SetSymbolFont{italicglyphs} {boldsansup}{T1}{\savesffamily}{\savesfbfseries}{n}

% Syntax of \DeclareMathSymobl:
% \DeclareMathSymbol {<symbol>} {<type>} {<sym-font>} {<slot>}
% Type              Meaning	            Example
% 0 or \mathord     Ordinary             $\alpha$
% 1 or \mathop      Large operator       $\sum$
% 2 or \mathbin     Binary operation     $\times$
% 3 or \mathrel     Relation             $\leq$
% 4 or \mathopen    Opening              $\langle$
% 5 or \mathclose   Closing              $\rangle$
% 6 or \mathpunct   Punctuation          ;
% 7 or \mathalpha   Alphabet character   A
% Example declaration:
% \DeclareMathSymbol{b}{0}{letters}{`b}

% Digits
\DeclareMathSymbol{0}{\mathalpha}{uprightglyphs}{`0}
\DeclareMathSymbol{1}{\mathalpha}{uprightglyphs}{`1}
\DeclareMathSymbol{2}{\mathalpha}{uprightglyphs}{`2}
\DeclareMathSymbol{3}{\mathalpha}{uprightglyphs}{`3}
\DeclareMathSymbol{4}{\mathalpha}{uprightglyphs}{`4}
\DeclareMathSymbol{5}{\mathalpha}{uprightglyphs}{`5}
\DeclareMathSymbol{6}{\mathalpha}{uprightglyphs}{`6}
\DeclareMathSymbol{7}{\mathalpha}{uprightglyphs}{`7}
\DeclareMathSymbol{8}{\mathalpha}{uprightglyphs}{`8}
\DeclareMathSymbol{9}{\mathalpha}{uprightglyphs}{`9}
% Operators and punctuation
\DeclareMathSymbol{+}{\mathbin}  {operators}    {`+}
	% Not from uprightglyphs due to bad spacing
\DeclareMathSymbol{=}{\mathrel}  {operators}    {`=}
	% Not from uprightglyphs due to bad spacing
\DeclareMathSymbol{.}{\mathord}  {uprightglyphs}{`.}
\DeclareMathSymbol{,}{\mathpunct}{uprightglyphs}{`,}
\DeclareMathSymbol{;}{\mathpunct}{uprightglyphs}{`;}
\DeclareMathSymbol{/}{\mathord}  {uprightglyphs}{`/}
%\DeclareMathSymbol{/}{\mathop}   {uprightglyphs}{`/}
%	% This would icrease the spacing around the division slash slightly
%\DeclareMathSymbol{(}{\mathopen} {uprightglyphs}{`(}
%\DeclareMathSymbol{)}{\mathclose}{uprightglyphs}{`)}
%\DeclareMathSymbol{[}{\mathopen} {uprightglyphs}{`[}
%\DeclareMathSymbol{]}{\mathclose}{uprightglyphs}{`]}
\DeclareMathSymbol{\prime}{\mathord}{extrasymbols}{"30}
	% Use \prime from mathdesign-charter because it looks better than the one in kpfonts
\DeclareMathDelimiter{(}      {\mathopen} {uprightglyphs}{`(} {largesymbols}{"00}
\DeclareMathDelimiter{)}      {\mathclose}{uprightglyphs}{`)} {largesymbols}{"01}
\DeclareMathDelimiter{[}      {\mathopen} {uprightglyphs}{`[} {largesymbols}{"02}
\DeclareMathDelimiter{]}      {\mathclose}{uprightglyphs}{`]} {largesymbols}{"03}
\DeclareMathDelimiter{\lbrace}{\mathopen} {uprightglyphs}{`\{}{largesymbols}{"08}
\DeclareMathDelimiter{\rbrace}{\mathclose}{uprightglyphs}{`\}}{largesymbols}{"09}
% Uppercase Latin characters
\DeclareMathSymbol{A}{\mathalpha}{italicglyphs}{`A}
\DeclareMathSymbol{B}{\mathalpha}{italicglyphs}{`B}
\DeclareMathSymbol{C}{\mathalpha}{italicglyphs}{`C}
\DeclareMathSymbol{D}{\mathalpha}{italicglyphs}{`D}
\DeclareMathSymbol{E}{\mathalpha}{italicglyphs}{`E}
\DeclareMathSymbol{F}{\mathalpha}{italicglyphs}{`F}
\DeclareMathSymbol{G}{\mathalpha}{italicglyphs}{`G}
\DeclareMathSymbol{H}{\mathalpha}{italicglyphs}{`H}
\DeclareMathSymbol{I}{\mathalpha}{italicglyphs}{`I}
\DeclareMathSymbol{J}{\mathalpha}{italicglyphs}{`J}
\DeclareMathSymbol{K}{\mathalpha}{italicglyphs}{`K}
\DeclareMathSymbol{L}{\mathalpha}{italicglyphs}{`L}
\DeclareMathSymbol{M}{\mathalpha}{italicglyphs}{`M}
\DeclareMathSymbol{N}{\mathalpha}{italicglyphs}{`N}
\DeclareMathSymbol{O}{\mathalpha}{italicglyphs}{`O}
\DeclareMathSymbol{P}{\mathalpha}{italicglyphs}{`P}
\DeclareMathSymbol{Q}{\mathalpha}{italicglyphs}{`Q}
\DeclareMathSymbol{R}{\mathalpha}{italicglyphs}{`R}
\DeclareMathSymbol{S}{\mathalpha}{italicglyphs}{`S}
\DeclareMathSymbol{T}{\mathalpha}{italicglyphs}{`T}
\DeclareMathSymbol{U}{\mathalpha}{italicglyphs}{`U}
\DeclareMathSymbol{V}{\mathalpha}{italicglyphs}{`V}
\DeclareMathSymbol{W}{\mathalpha}{italicglyphs}{`W}
\DeclareMathSymbol{X}{\mathalpha}{italicglyphs}{`X}
\DeclareMathSymbol{Y}{\mathalpha}{italicglyphs}{`Y}
\DeclareMathSymbol{Z}{\mathalpha}{italicglyphs}{`Z}
% lowercase Latin characters
\DeclareMathSymbol{a}{\mathalpha}{italicglyphs}{`a}
\DeclareMathSymbol{b}{\mathalpha}{italicglyphs}{`b}
\DeclareMathSymbol{c}{\mathalpha}{italicglyphs}{`c}
\DeclareMathSymbol{d}{\mathalpha}{italicglyphs}{`d}
\DeclareMathSymbol{e}{\mathalpha}{italicglyphs}{`e}
\DeclareMathSymbol{f}{\mathalpha}{italicglyphs}{`f}
\DeclareMathSymbol{g}{\mathalpha}{italicglyphs}{`g}
\DeclareMathSymbol{h}{\mathalpha}{italicglyphs}{`h}
\DeclareMathSymbol{i}{\mathalpha}{italicglyphs}{`i}
\DeclareMathSymbol{\imath}{\mathalpha}{italicglyphs}{"19}
\DeclareMathSymbol{j}{\mathalpha}{italicglyphs}{`j}
\DeclareMathSymbol{\jmath}{\mathalpha}{italicglyphs}{"1A}
\DeclareMathSymbol{k}{\mathalpha}{italicglyphs}{`k}
\DeclareMathSymbol{l}{\mathalpha}{italicglyphs}{`l}
\DeclareMathSymbol{m}{\mathalpha}{italicglyphs}{`m}
\DeclareMathSymbol{n}{\mathalpha}{italicglyphs}{`n}
\DeclareMathSymbol{o}{\mathalpha}{italicglyphs}{`o}
\DeclareMathSymbol{p}{\mathalpha}{italicglyphs}{`p}
\DeclareMathSymbol{q}{\mathalpha}{italicglyphs}{`q}
\DeclareMathSymbol{r}{\mathalpha}{italicglyphs}{`r}
\DeclareMathSymbol{s}{\mathalpha}{italicglyphs}{`s}
\DeclareMathSymbol{t}{\mathalpha}{italicglyphs}{`t}
\DeclareMathSymbol{u}{\mathalpha}{italicglyphs}{`u}
\DeclareMathSymbol{v}{\mathalpha}{italicglyphs}{`v}
\DeclareMathSymbol{w}{\mathalpha}{italicglyphs}{`w}
\DeclareMathSymbol{x}{\mathalpha}{italicglyphs}{`x}
\DeclareMathSymbol{y}{\mathalpha}{italicglyphs}{`y}
\DeclareMathSymbol{z}{\mathalpha}{italicglyphs}{`z}

%% Sansserif Greek letters
%\DeclareSymbolFont{lgrgreek}{LGR}{\savesffamily}{\savesfmdseries}{it}
%\SetSymbolFont{lgrgreek}{sans}    {LGR}{\savesffamily}{\savesfmdseries}{it}
%\SetSymbolFont{lgrgreek}{boldsans}{LGR}{\savesffamily}{\savesfbfseries}{it}

% The following is taken from
% https://tex.stackexchange.com/questions/116389/automatic-upright-math-when-text-is-in-italic/116399#116399
% Filling in ``missing'' Greek glyphs for completeness
% (not really necessary, since they look identical to Latin glyphs and are thus almost never used)
% ==>
\newcommand{\omicron}{o}
\newcommand{\Digamma}{F}
\newcommand{\Alpha}  {A}
\newcommand{\Beta}   {B}
\newcommand{\Epsilon}{E}
\newcommand{\Zeta}   {Z}
\newcommand{\Eta}    {H}
\newcommand{\Iota}   {I}
\newcommand{\Kappa}  {K}
\newcommand{\Mu}     {M}
\newcommand{\Nu}     {N}
\newcommand{\Omicron}{O}
\newcommand{\Rho}    {P}
\newcommand{\Tau}    {T}
\newcommand{\Chi}    {X}
% <==

% Save original definitions of the Greek letters
% ==>
\makeatletter
\@for\@tempa:=%
	alpha,beta,gamma,delta,epsilon,zeta,eta,theta,iota,kappa,lambda,mu,nu,xi,%
	omicron,pi,rho,sigma,varsigma,tau,upsilon,phi,chi,psi,omega,digamma,%
	Alpha,Beta,Gamma,Delta,Epsilon,Zeta,Eta,Theta,Iota,Kappa,Lambda,Mu,Nu,Xi,%
	Omicron,Pi,Rho,Sigma,Tau,Upsilon,Phi,Chi,Psi,Omega,Digamma%
	\do{%
		\expandafter\let\csname\@tempa orig\expandafter\endcsname\csname\@tempa\endcsname%
		\expandafter\let\csname\@tempa uporig\expandafter\endcsname\csname\@tempa up\endcsname%
	}%
\makeatother
% <==

% LGR-encoded Greek letters
% ==>
\newcommand{\textformath}[1]{%
	\IfInBoldMode%
		\IfInUpMode\textbf{#1}\else\textit{\bfseries #1}\fi\relax%
	\else
		\IfInUpMode\textup{#1}\else\textit{#1}\fi\relax%
	\fi\relax%
}
% The double curly braces in this section are necessary to be able to use Greek letters
% in subscripts and superscripts without having to enclose theme in curly braces;
% for example, $\sigma_\epsilon$ instead of $\sigma_{\epsilon}$.
% Uppercase
\newcommand{\AlphaLGR}   {{\mathord{\textformath{\fontencoding{LGR}\selectfont A}}}}
\newcommand{\BetaLGR}    {{\mathord{\textformath{\fontencoding{LGR}\selectfont B}}}}
\newcommand{\GammaLGR}   {{\mathord{\textformath{\fontencoding{LGR}\selectfont G}}}}
\newcommand{\DeltaLGR}   {{\mathord{\textformath{\fontencoding{LGR}\selectfont D}}}}
\newcommand{\EpsilonLGR} {{\mathord{\textformath{\fontencoding{LGR}\selectfont E}}}}
\newcommand{\ZetaLGR}    {{\mathord{\textformath{\fontencoding{LGR}\selectfont Z}}}}
\newcommand{\EtaLGR}     {{\mathord{\textformath{\fontencoding{LGR}\selectfont H}}}}
\newcommand{\ThetaLGR}   {{\mathord{\textformath{\fontencoding{LGR}\selectfont J}}}}
\newcommand{\IotaLGR}    {{\mathord{\textformath{\fontencoding{LGR}\selectfont I}}}}
\newcommand{\KappaLGR}   {{\mathord{\textformath{\fontencoding{LGR}\selectfont K}}}}
\newcommand{\LambdaLGR}  {{\mathord{\textformath{\fontencoding{LGR}\selectfont L}}}}
\newcommand{\MuLGR}      {{\mathord{\textformath{\fontencoding{LGR}\selectfont M}}}}
\newcommand{\NuLGR}      {{\mathord{\textformath{\fontencoding{LGR}\selectfont N}}}}
\newcommand{\XiLGR}      {{\mathord{\textformath{\fontencoding{LGR}\selectfont X}}}}
\newcommand{\OmicronLGR} {{\mathord{\textformath{\fontencoding{LGR}\selectfont O}}}}
\newcommand{\PiLGR}      {{\mathord{\textformath{\fontencoding{LGR}\selectfont P}}}}
\newcommand{\RhoLGR}     {{\mathord{\textformath{\fontencoding{LGR}\selectfont R}}}}
\newcommand{\SigmaLGR}   {{\mathord{\textformath{\fontencoding{LGR}\selectfont S}}}}
\newcommand{\TauLGR}     {{\mathord{\textformath{\fontencoding{LGR}\selectfont T}}}}
\newcommand{\UpsilonLGR} {{\mathord{\textformath{\fontencoding{LGR}\selectfont U}}}}
\newcommand{\PhiLGR}     {{\mathord{\textformath{\fontencoding{LGR}\selectfont F}}}}
\newcommand{\ChiLGR}     {{\mathord{\textformath{\fontencoding{LGR}\selectfont Q}}}}
\newcommand{\PsiLGR}     {{\mathord{\textformath{\fontencoding{LGR}\selectfont Y}}}}
\newcommand{\OmegaLGR}   {{\mathord{\textformath{\fontencoding{LGR}\selectfont W}}}}
\newcommand{\DigammaLGR} {{\mathord{\textformath{\fontencoding{LGR}\selectfont \char195}}}}
% lowercase
\newcommand{\alphaLGR}   {{\mathord{\textformath{\fontencoding{LGR}\selectfont a}}}}
\newcommand{\betaLGR}    {{\mathord{\textformath{\fontencoding{LGR}\selectfont b}}}}
\newcommand{\gammaLGR}   {{\mathord{\textformath{\fontencoding{LGR}\selectfont g}}}}
\newcommand{\deltaLGR}   {{\mathord{\textformath{\fontencoding{LGR}\selectfont d}}}}
\newcommand{\epsilonLGR} {{\mathord{\textformath{\fontencoding{LGR}\selectfont e}}}}
\newcommand{\zetaLGR}    {{\mathord{\textformath{\fontencoding{LGR}\selectfont z}}}}
\newcommand{\etaLGR}     {{\mathord{\textformath{\fontencoding{LGR}\selectfont h}}}}
\newcommand{\thetaLGR}   {{\mathord{\textformath{\fontencoding{LGR}\selectfont j}}}}
\newcommand{\iotaLGR}    {{\mathord{\textformath{\fontencoding{LGR}\selectfont i}}}}
\newcommand{\kappaLGR}   {{\mathord{\textformath{\fontencoding{LGR}\selectfont k}}}}
\newcommand{\lambdaLGR}  {{\mathord{\textformath{\fontencoding{LGR}\selectfont l}}}}
\newcommand{\muLGR}      {{\mathord{\textformath{\fontencoding{LGR}\selectfont m}}}}
\newcommand{\nuLGR}      {{\mathord{\textformath{\fontencoding{LGR}\selectfont n}}}}
\newcommand{\xiLGR}      {{\mathord{\textformath{\fontencoding{LGR}\selectfont x}}}}
\newcommand{\omicronLGR} {{\mathord{\textformath{\fontencoding{LGR}\selectfont o}}}}
\newcommand{\piLGR}      {{\mathord{\textformath{\fontencoding{LGR}\selectfont p}}}}
\newcommand{\rhoLGR}     {{\mathord{\textformath{\fontencoding{LGR}\selectfont r}}}}
\newcommand{\sigmaLGR}   {{\mathord{\textformath{\fontencoding{LGR}\selectfont s\noboundary}}}}
	% \noboundary prevents sigma from being replaced by the word-end sigma (varsigma),
	% see http://mirrors.ctan.org/macros/latex/contrib/textgreek/textgreek.pdf
\newcommand{\varsigmaLGR}{{\mathord{\textformath{\fontencoding{LGR}\selectfont c}}}}
\newcommand{\tauLGR}     {{\mathord{\textformath{\fontencoding{LGR}\selectfont t}}}}
\newcommand{\upsilonLGR} {{\mathord{\textformath{\fontencoding{LGR}\selectfont u}}}}
\newcommand{\phiLGR}     {{\mathord{\textformath{\fontencoding{LGR}\selectfont f}}}}
\newcommand{\chiLGR}     {{\mathord{\textformath{\fontencoding{LGR}\selectfont q}}}}
\newcommand{\psiLGR}     {{\mathord{\textformath{\fontencoding{LGR}\selectfont y}}}}
\newcommand{\omegaLGR}   {{\mathord{\textformath{\fontencoding{LGR}\selectfont w}}}}
\newcommand{\digammaLGR} {{\mathord{\textformath{\fontencoding{LGR}\selectfont \char147}}}}
% Uppercase, upright
\newcommand{\AlphaupLGR}   {{\mathord{\textup{\fontencoding{LGR}\selectfont A}}}}
\newcommand{\BetaupLGR}    {{\mathord{\textup{\fontencoding{LGR}\selectfont B}}}}
\newcommand{\GammaupLGR}   {{\mathord{\textup{\fontencoding{LGR}\selectfont G}}}}
\newcommand{\DeltaupLGR}   {{\mathord{\textup{\fontencoding{LGR}\selectfont D}}}}
\newcommand{\EpsilonupLGR} {{\mathord{\textup{\fontencoding{LGR}\selectfont E}}}}
\newcommand{\ZetaupLGR}    {{\mathord{\textup{\fontencoding{LGR}\selectfont Z}}}}
\newcommand{\EtaupLGR}     {{\mathord{\textup{\fontencoding{LGR}\selectfont H}}}}
\newcommand{\ThetaupLGR}   {{\mathord{\textup{\fontencoding{LGR}\selectfont J}}}}
\newcommand{\IotaupLGR}    {{\mathord{\textup{\fontencoding{LGR}\selectfont I}}}}
\newcommand{\KappaupLGR}   {{\mathord{\textup{\fontencoding{LGR}\selectfont K}}}}
\newcommand{\LambdaupLGR}  {{\mathord{\textup{\fontencoding{LGR}\selectfont L}}}}
\newcommand{\MuupLGR}      {{\mathord{\textup{\fontencoding{LGR}\selectfont M}}}}
\newcommand{\NuupLGR}      {{\mathord{\textup{\fontencoding{LGR}\selectfont N}}}}
\newcommand{\XiupLGR}      {{\mathord{\textup{\fontencoding{LGR}\selectfont X}}}}
\newcommand{\OmicronupLGR} {{\mathord{\textup{\fontencoding{LGR}\selectfont O}}}}
\newcommand{\PiupLGR}      {{\mathord{\textup{\fontencoding{LGR}\selectfont P}}}}
\newcommand{\RhoupLGR}     {{\mathord{\textup{\fontencoding{LGR}\selectfont R}}}}
\newcommand{\SigmaupLGR}   {{\mathord{\textup{\fontencoding{LGR}\selectfont S}}}}
\newcommand{\TauupLGR}     {{\mathord{\textup{\fontencoding{LGR}\selectfont T}}}}
\newcommand{\UpsilonupLGR} {{\mathord{\textup{\fontencoding{LGR}\selectfont U}}}}
\newcommand{\PhiupLGR}     {{\mathord{\textup{\fontencoding{LGR}\selectfont F}}}}
\newcommand{\ChiupLGR}     {{\mathord{\textup{\fontencoding{LGR}\selectfont Q}}}}
\newcommand{\PsiupLGR}     {{\mathord{\textup{\fontencoding{LGR}\selectfont Y}}}}
\newcommand{\OmegaupLGR}   {{\mathord{\textup{\fontencoding{LGR}\selectfont W}}}}
\newcommand{\DigammaupLGR} {{\mathord{\textup{\fontencoding{LGR}\selectfont \char195}}}}
% lowercase, upright
\newcommand{\alphaupLGR}   {{\mathord{\textup{\fontencoding{LGR}\selectfont a}}}}
\newcommand{\betaupLGR}    {{\mathord{\textup{\fontencoding{LGR}\selectfont b}}}}
\newcommand{\gammaupLGR}   {{\mathord{\textup{\fontencoding{LGR}\selectfont g}}}}
\newcommand{\deltaupLGR}   {{\mathord{\textup{\fontencoding{LGR}\selectfont d}}}}
\newcommand{\epsilonupLGR} {{\mathord{\textup{\fontencoding{LGR}\selectfont e}}}}
\newcommand{\zetaupLGR}    {{\mathord{\textup{\fontencoding{LGR}\selectfont z}}}}
\newcommand{\etaupLGR}     {{\mathord{\textup{\fontencoding{LGR}\selectfont h}}}}
\newcommand{\thetaupLGR}   {{\mathord{\textup{\fontencoding{LGR}\selectfont j}}}}
\newcommand{\iotaupLGR}    {{\mathord{\textup{\fontencoding{LGR}\selectfont i}}}}
\newcommand{\kappaupLGR}   {{\mathord{\textup{\fontencoding{LGR}\selectfont k}}}}
\newcommand{\lambdaupLGR}  {{\mathord{\textup{\fontencoding{LGR}\selectfont l}}}}
\newcommand{\muupLGR}      {{\mathord{\textup{\fontencoding{LGR}\selectfont m}}}}
\newcommand{\nuupLGR}      {{\mathord{\textup{\fontencoding{LGR}\selectfont n}}}}
\newcommand{\xiupLGR}      {{\mathord{\textup{\fontencoding{LGR}\selectfont x}}}}
\newcommand{\omicronupLGR} {{\mathord{\textup{\fontencoding{LGR}\selectfont o}}}}
\newcommand{\piupLGR}      {{\mathord{\textup{\fontencoding{LGR}\selectfont p}}}}
\newcommand{\rhoupLGR}     {{\mathord{\textup{\fontencoding{LGR}\selectfont r}}}}
\newcommand{\sigmaupLGR}   {{\mathord{\textup{\fontencoding{LGR}\selectfont s\noboundary}}}}
	% \noboundary prevents sigma from being replaced by the word-end sigma (varsigma),
	% see http://mirrors.ctan.org/macros/latex/contrib/textgreek/textgreek.pdf
\newcommand{\varsigmaupLGR}{{\mathord{\textup{\fontencoding{LGR}\selectfont c}}}}
\newcommand{\tauupLGR}     {{\mathord{\textup{\fontencoding{LGR}\selectfont t}}}}
\newcommand{\upsilonupLGR} {{\mathord{\textup{\fontencoding{LGR}\selectfont u}}}}
\newcommand{\phiupLGR}     {{\mathord{\textup{\fontencoding{LGR}\selectfont f}}}}
\newcommand{\chiupLGR}     {{\mathord{\textup{\fontencoding{LGR}\selectfont q}}}}
\newcommand{\psiupLGR}     {{\mathord{\textup{\fontencoding{LGR}\selectfont y}}}}
\newcommand{\omegaupLGR}   {{\mathord{\textup{\fontencoding{LGR}\selectfont w}}}}
\newcommand{\digammaupLGR} {{\mathord{\textup{\fontencoding{LGR}\selectfont \char147}}}}
% <==

% Based on description of the TS1 encoding in
% http://ctan.math.illinois.edu/macros/latex/doc/encguide.pdf:
%\let \oldpm    \pm
%\let \oldtimes \times
%\let \olddiv   \div
%\makeatletter
%\newcommand{\pmsf}   {\mathbin{\text{\usefont{TS1}{\sfdefault}{\f@series}{n}\char"B1}}}
%\newcommand{\timessf}{\mathbin{\text{\usefont{TS1}{\sfdefault}{\f@series}{n}\char"D6}}}
%\newcommand{\divsf}  {\mathbin{\text{\usefont{TS1}{\sfdefault}{\f@series}{n}\char"F6}}}
%\makeatother

% Use LGR-encoded Greek letters for \mathversion{sans}
% ==>
\makeatletter

% Save original definition of \varepsilon etc.
\@for\@tempa:=%
epsilon,theta,pi,rho,phi%
\do{%
	\expandafter\let\csname var\@tempa orig\expandafter\endcsname\csname var\@tempa\endcsname%
}

\newcommand*{\sansmath}{%
	\@for\@tempa:=%
		alpha,beta,gamma,delta,epsilon,zeta,eta,theta,iota,kappa,lambda,mu,nu,xi,%
		omicron,pi,rho,sigma,varsigma,tau,upsilon,phi,chi,psi,omega,digamma,%
		Alpha,Beta,Gamma,Delta,Epsilon,Zeta,Eta,Theta,Iota,Kappa,Lambda,Mu,Nu,Xi,%
		Omicron,Pi,Rho,Sigma,Tau,Upsilon,Phi,Chi,Psi,Omega,Digamma%
	\do{%
		\expandafter\let\csname\@tempa\expandafter\endcsname\csname\@tempa LGR\endcsname%
		\expandafter\let\csname\@tempa up\expandafter\endcsname\csname\@tempa upLGR\endcsname%
		\expandafter\let\csname up\@tempa\expandafter\endcsname\csname\@tempa upLGR\endcsname%
	}%
	\@for\@tempa:=%
		epsilon,theta,pi,rho,phi%
	\do{%
		\expandafter\let\csname var\@tempa\expandafter\endcsname\csname\@tempa\endcsname%
	}%
	%\renewcommand{\pm}{\pmsf}%
	%\renewcommand{\times}{\timessf}%
	%\renewcommand{\div}{\divsf}%
}
% <==

% Switch back to the original Greek letters for \mathversion{normal}, i.e., the serif font
% ==>
\newcommand*{\unsansmath}{%
	\@for\@tempa:=%
		alpha,beta,gamma,delta,epsilon,zeta,eta,theta,iota,kappa,lambda,mu,nu,xi,%
		omicron,pi,rho,sigma,varsigma,tau,upsilon,phi,chi,psi,omega,digamma,%
		Alpha,Beta,Gamma,Delta,Epsilon,Zeta,Eta,Theta,Iota,Kappa,Lambda,Mu,Nu,Xi,%
		Omicron,Pi,Rho,Sigma,Tau,Upsilon,Phi,Chi,Psi,Omega,Digamma%
	\do{%
		\expandafter\let\csname\@tempa\expandafter\endcsname\csname\@tempa orig\endcsname%
		\expandafter\let\csname\@tempa up\expandafter\endcsname\csname\@tempa uporig\endcsname%
		\expandafter\let\csname up\@tempa\expandafter\endcsname\csname\@tempa uporig\endcsname%
	}%
	\@for\@tempa:=%
		epsilon,theta,pi,rho,phi%
	\do{%
		\expandafter\let\csname var\@tempa\expandafter\endcsname\csname var\@tempa orig\endcsname%
	}%
	%\renewcommand{\pm}{\oldpm}%
	%\renewcommand{\times}{\oldtimes}%
	%\renewcommand{\div}{\olddiv}%
}
% <==

%% If you would like to use LGR-encoded Greek letters also for the serif font
%% ==>
%\renewcommand*{\unsansmath}{%
%	\@for\@tempa:=%
%		alpha,beta,gamma,delta,epsilon,zeta,eta,theta,iota,kappa,lambda,mu,nu,xi,%
%		omicron,pi,rho,sigma,varsigma,tau,upsilon,phi,chi,psi,omega,digamma,%
%		Alpha,Beta,Gamma,Delta,Epsilon,Zeta,Eta,Theta,Iota,Kappa,Lambda,Mu,Nu,Xi,%
%		Omicron,Pi,Rho,Sigma,Tau,Upsilon,Phi,Chi,Psi,Omega,Digamma%
%		\do{%
%			\expandafter\let\csname\@tempa\expandafter\endcsname\csname\@tempa LGR\endcsname%
%			\expandafter\let\csname\@tempa up\expandafter\endcsname\csname\@tempa upLGR\endcsname%
%			\expandafter\let\csname up\@tempa\expandafter\endcsname\csname\@tempa upLGR\endcsname%
%		}%
%	%\renewcommand{\pm}{\pmsf}%
%	%\renewcommand{\times}{\timessf}%
%	%\renewcommand{\div}{\divsf}%
%}
%% <==

\newcommand*{\upgreekletters}{%
	\@for\@tempa:=%
		alpha,beta,gamma,delta,epsilon,zeta,eta,theta,iota,kappa,lambda,mu,nu,xi,%
		omicron,pi,rho,sigma,varsigma,tau,upsilon,phi,chi,psi,omega,digamma,%
		Alpha,Beta,Gamma,Delta,Epsilon,Zeta,Eta,Theta,Iota,Kappa,Lambda,Mu,Nu,Xi,%
		Omicron,Pi,Rho,Sigma,Tau,Upsilon,Phi,Chi,Psi,Omega,Digamma%
		\do{%
			\expandafter\let\csname\@tempa\expandafter\endcsname\csname\@tempa up\endcsname%
		}%
}
\newcommand*{\itgreekletters}{%
	\@for\@tempa:=%
		alpha,beta,gamma,delta,epsilon,zeta,eta,theta,iota,kappa,lambda,mu,nu,xi,%
		omicron,pi,rho,sigma,varsigma,tau,upsilon,phi,chi,psi,omega,digamma,%
		Alpha,Beta,Gamma,Delta,Epsilon,Zeta,Eta,Theta,Iota,Kappa,Lambda,Mu,Nu,Xi,%
		Omicron,Pi,Rho,Sigma,Tau,Upsilon,Phi,Chi,Psi,Omega,Digamma%
		\do{%
			\expandafter\let\csname\@tempa\expandafter\endcsname\csname\@tempa orig\endcsname%
		}%
}

\makeatother

%\makeatletter
%	\@for\@tempa:=%
%	%alpha,beta,gamma,delta,epsilon,zeta,eta,theta,iota,kappa,lambda,mu,nu,xi,%
%	%pi,rho,sigma,varsigma,tau,upsilon,phi,chi,psi,omega,digamma,%
%	Gamma,Delta,Theta,Lambda,Xi,Pi,Sigma,Upsilon,Phi,Psi,Omega%
%	\do{\expandafter\let\csname\@tempa\expandafter\endcsname\csname other\@tempa\endcsname}%
%\makeatother

% Fix the \bm command so that it also works properly in the sans mathversions
% ==>
\let \bmorig \bm
\renewcommand{\bm}[1]{%
	\IfInSansMode%
		\textbf{\mathversion{boldsans}\(#1\)}%
	\else%
		\bmorig{#1}%
	\fi\relax%
}
% <==
\renewcommand{\mathbf}[1]{\bm{#1}}
\renewcommand{\boldsymbol}[1]{\bm{#1}}
\newcommand{\mathbfit}[1]{\mathbf{\mathit{#1}}}
\renewcommand{\mathcal}[1]{\mathscr{#1}}

% Apply sansmath etc. automagically
% ==>
\newif\IfInSansMode
\newif\IfInBoldMode
\newif\IfInUpMode
\let \oldsf \sffamily
\renewcommand*{\sffamily}{%
	\oldsf\sansmath\InSansModetrue%
	\IfInBoldMode\mathversion{boldsans}\else\mathversion{sans}\fi\relax%
}
\let \oldbf \bfseries
\renewcommand*{\bfseries}{%
	\oldbf\InBoldModetrue%
	\IfInSansMode\sansmath\mathversion{boldsans}\else\mathversion{bold}\fi\relax%
}
\let \oldmd \mdseries
\renewcommand*{\mdseries}{%
	\oldmd\InBoldModefalse%
	\IfInSansMode\sansmath\mathversion{sans}\else\mathversion{normal}\fi\relax%
}
\let \oldnorm \normalfont
\renewcommand*{\normalfont}{%
	\oldnorm\InSansModefalse\InBoldModefalse\mathversion{normal}%
	\unsansmath%
}
\let \oldrm \rmfamily
\renewcommand*{\rmfamily}{%
	\oldrm\InSansModefalse%
	\IfInBoldMode\mathversion{bold}\else\mathversion{normal}\fi\relax%
	\unsansmath%
}
% <==

% Make \mathnormal obey the currently active \mathversion ==>
\let \mathnormalorig \mathnormal
\renewcommand{\mathnormal}[1]{%
	\IfInSansMode%
		\IfInBoldMode%
			\mathversion{boldsans}%
			{\textbf{\(#1\)}}%
		\else%
			\mathversion{sans}%
			{\textmd{\(#1\)}}%
		\fi\relax%	
	\else%
		\mathnormalorig{#1}%
	\fi\relax%
}
% <==

% Adjust \mathrm to the curretly active \mathversion.
% We set it up such that also in sansserif mode, \mathrm activates the serif font.
% ==>
\let \mathrmorig \mathrm
\renewcommand{\mathrm}[1]{%
	\IfInSansMode%
		{\textrm{%
			\IfInBoldMode%
				\mathversion{bold}%
				\(\mathrmorig{#1}\)%
			\else%
				\mathversion{normal}%
				\(\mathrmorig{#1}\)%
			\fi\relax%
		}}%
	\else%
		\mathrmorig{#1}%
	\fi\relax%
}
% <==

% Define \mathup to activate \upshape without switching to the serif font
% (in contrast to \mathrm)
% ==>
\newcommand{\mathup}[1]{%
	\IfInSansMode%
		{\textup{%
			\InUpModetrue%
			\IfInBoldMode%
				\mathversion{boldsansup}%
				\(#1\)%
			\else%
				\mathversion{sansup}%
				\(#1\)%
			\fi\relax%
		}}%
	\else%
		{\upgreekletters\mathrm{#1}\itgreekletters}%
	\fi\relax%
}
\newcommand{\mathbfup}[1]{%
	\IfInSansMode%
		{\mathbf{\mathup{#1}}}%
	\else%
		{\upgreekletters\mathbf{\mathrm{#1}}\itgreekletters}%
	\fi\relax%
}
% <==

%% If you would like to redefine \mathup also for the serif font:
%% ==>
%\renewcommand{\mathup}[1]{%
%	\IfInSansMode%
%		{\textup{%
%			\InUpModetrue%
%			\IfInBoldMode%
%				\mathversion{boldsansup}%
%				\(#1\)%
%			\else%
%				\mathversion{sansup}%
%				\(#1\)%
%			\fi\relax%
%		}}%
%	\else%
%		{\textup{%
%			\InUpModetrue%
%			\IfInBoldMode%
%				\mathversion{boldup}%
%				\(#1\)%
%			\else%
%				\mathversion{normalup}%
%				\(#1\)%
%			\fi\relax%
%		}}%
%	\fi\relax%
%}
%\renewcommand{\mathbfup}[1]{%
%	\IfInSansMode%
%		{\mathbf{\mathup{#1}}}%
%	\else%
%		{\upgreekletters\mathbf{\mathup{#1}}\itgreekletters}%
%	\fi\relax%
%}
%% <==

% Make the LaTeX-defined operators obey sansserif math
% ==>
\let \operatornameorig \operatorname
\renewcommand{\operatorname}[1]{%
	\operatornameorig{\mathup{#1}}%
}
\makeatletter
\@for\@tempa:=%
	arccos,arccot,arccsc,arcsec,arcsin,arctan,arg,cos,cosh,cot,coth,csc,%
	deg,det,dim,exp,gcd,hom,inf,ker,lg,lim,liminf,limsup,ln,log,max,min,%
	Pr,sec,sin,sinh,sup,tan,tanh%
	\do{%
		\expandafter\let\csname\@tempa\endcsname\relax%
	}%
\makeatother
\DeclareMathOperator {\arccos}{\mathup{arccos}}
\DeclareMathOperator {\arccot}{\mathup{arccot}}
\DeclareMathOperator {\arccsc}{\mathup{arccsc}}
\DeclareMathOperator {\arcsec}{\mathup{arcsec}}
\DeclareMathOperator {\arcsin}{\mathup{arcsin}}
\DeclareMathOperator {\arctan}{\mathup{arctan}}
\DeclareMathOperator {\arg}   {\mathup{arg}}
\DeclareMathOperator {\cos}   {\mathup{cos}}
\DeclareMathOperator {\cosh}  {\mathup{cosh}}
\DeclareMathOperator {\cot}   {\mathup{cot}}
\DeclareMathOperator {\coth}  {\mathup{coth}}
\DeclareMathOperator {\csc}   {\mathup{csc}}
\DeclareMathOperator {\deg}   {\mathup{deg}}
\DeclareMathOperator {\det}   {\mathup{det}}
\DeclareMathOperator {\dim}   {\mathup{dim}}
\DeclareMathOperator {\exp}   {\mathup{exp}}
\DeclareMathOperator {\gcd}   {\mathup{gcd}}
\DeclareMathOperator*{\hom}   {\mathup{hom}}
\DeclareMathOperator*{\inf}   {\mathup{inf}}
\DeclareMathOperator {\ker}   {\mathup{ker}}
\DeclareMathOperator {\lg}    {\mathup{lg}}
\DeclareMathOperator*{\lim}   {\mathup{lim}}
\DeclareMathOperator*{\liminf}{\mathup{lim\,inf}}
\DeclareMathOperator*{\limsup}{\mathup{lim\,sup}}
\DeclareMathOperator {\ln}    {\mathup{ln}}
\DeclareMathOperator {\log}   {\mathup{log}}
\DeclareMathOperator*{\max}   {\mathup{max}}
\DeclareMathOperator*{\min}   {\mathup{min}}
\DeclareMathOperator {\Pr}    {\mathup{Pr}}
\DeclareMathOperator {\sec}   {\mathup{sec}}
\DeclareMathOperator {\sin}   {\mathup{sin}}
\DeclareMathOperator {\sinh}  {\mathup{sinh}}
\DeclareMathOperator*{\sup}   {\mathup{sup}}
\DeclareMathOperator {\tan}   {\mathup{tan}}
\DeclareMathOperator {\tanh}  {\mathup{tanh}}
% <==