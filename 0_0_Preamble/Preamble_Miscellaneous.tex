% !TeX TXS-program:compile = txs:///pdflatex/
% !TeX TS-program = pdflatex
% !TeX TXS-program:bibliography = txs:///biber
% !BIB program = biber
% !TeX root = ../Diss-Template.tex




%%%%%%%%%%%%%%%%%%%%%%%%%%%%%%%%%%%
%%  OTHER PACKAGES AND COMMANDS  %%
%%%%%%%%%%%%%%%%%%%%%%%%%%%%%%%%%%%


% Some math-related definitions

%\newcommand*{\coloneqq}{\mathrel{%
%	\mathrel{%
%		\raisebox{ 0.18ex}{\scalebox{0.85}{:}}\hspace{-0.2pt}%
%	}%
%	=%
%}}
% Provided by the mathtools package

%\newcommand{\la}[1]{\lambda}

\newcommand{\Corr}{\operatorname{Corr}}
\newcommand{\Cov} {\operatorname{Cov}}
\newcommand{\E}   {\operatorname{E}}
\newcommand{\Var} {\operatorname{Var}}

\newcommand{\dd}  {\mathup{d}}  % Differential d
\newcommand{\e}   {\mathup{e}}  % Euler's e

\newcommand{\sigstar}{\raisebox{0.66ex}{\scalebox{0.95}{$\star$}}}

% Balanced/unbalanced sliders for text:
\newcommand{\bal}{\mbox{\caps{BAL}}\xspace}
\newcommand{\unbal}{\mbox{\caps{UNBAL}}\xspace}
\newcommand{\balA}[1][1]{\mbox{\caps{BAL}$^{\mathup{I}}_{#1:#1}$}\xspace}
\newcommand{\balB}[1][1]{\mbox{\caps{BAL}$^{\mathup{II}}_{#1:#1}$}\xspace}
\newcommand{\unbalA}[1][n]{\mbox{\caps{UNBAL}$^{\mathup{I}}_{1:#1}$}\xspace}
\newcommand{\unbalB}[1][n]{\mbox{\caps{UNBAL}$^{\mathup{II}}_{#1:1}$}\xspace}

% Balanced/unbalanced slider choice sets for math:
\newcommand{\CS}[1][C]{{\mathbf{#1}}}
\newcommand{\CbalA}[1][1]{\CS^{\mathup{BAL,\,I}}_{#1:#1}}
\newcommand{\CbalB}[1][1]{\CS^{\mathup{BAL,\,II}}_{#1:#1}}
\newcommand{\CunbalA}[1][n]{\CS^{\mathup{UNBAL,\,I}}_{1:#1}}
\newcommand{\CunbalB}[1][n]{\CS^{\mathup{UNBAL,\,II}}_{#1:1}}
\newcommand{\cse}[1][c]{{\mathbf{#1}}}
\newcommand{\cbalA}[1][1]{\cse^{\mathup{BAL,\,I}}_{#1:#1}}
\newcommand{\cbalB}[1][1]{\cse^{\mathup{BAL,\,II}}_{#1:#1}}
\newcommand{\cunbalA}[1][n]{\cse^{\mathup{UNBAL,\,I}}_{1:#1}}
\newcommand{\cunbalB}[1][n]{\cse^{\mathup{UNBAL,\,II}}_{#1:1}}

% Balanced/unbalanced choice lists for text:
\newcommand{\balCL}[1][1]{\mbox{\caps{BAL}$_{\mathup{CL}}$}\xspace}
\newcommand{\unbalCLA}[1][1]{\mbox{\caps{UNBAL}$^{\mathup{I}}_{\mathup{CL}}$}\xspace}
\newcommand{\unbalCLB}[1][1]{\mbox{\caps{UNBAL}$^{\mathup{II}}_{\mathup{CL}}$}\xspace}

% Balanced/unbalanced choice-list choice sets for math:
\newcommand{\CbalCL}{{\CS}^{\textup{BAL}}_{\textup{CL}}}
\newcommand{\CunbalCLA}{{\CS}^{\textup{UNBAL,\,I}}_{\textup{CL}}}
\newcommand{\CunbalCLB}{{\CS}^{\textup{UNBAL,\,II}}_{\textup{CL}}}
\newcommand{\cbalCL}{{\cse}^{\textup{BAL}}_{\textup{CL}}}
\newcommand{\cunbalCLA}{{\cse}^{\textup{UNBAL,\,I}}_{\textup{CL}}}
\newcommand{\cunbalCLB}{{\cse}^{\textup{UNBAL,\,II}}_{\textup{CL}}}

%% Command to suppress text: ==>
%\makeatletter
%\font\dummyft@=dummy \relax
%\def\suppress{%
%	\begingroup\par
%	\parskip\z@
%	\offinterlineskip
%	\baselineskip=\z@skip
%	\lineskip=\z@skip
%	\lineskiplimit=\maxdimen
%	\dummyft@
%	\count@\sixt@@n
%	\loop\ifnum\count@ >\z@
%	\advance\count@\m@ne
%	\textfont\count@\dummyft@
%	\scriptfont\count@\dummyft@
%	\scriptscriptfont\count@\dummyft@
%	\repeat
%	\let\selectfont\relax
%	\let\mathversion\@gobble
%	\let\getanddefine@fonts\@gobbletwo
%	\tracinglostchars\z@
%	\frenchspacing
%	\hbadness\@M}
%\def\endsuppress{\par\endgroup}
%\makeatother
%% <==