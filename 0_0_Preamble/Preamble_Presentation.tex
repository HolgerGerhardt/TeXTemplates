% !TeX TXS-program:compile = txs:///pdflatex/
% !BIB program = bibtex




%%%%%%%%%%%%%%%%%%%%%%%%%%%%
%%  FUNDAMENTAL PACKAGES  %%
%%%%%%%%%%%%%%%%%%%%%%%%%%%%


\usepackage{calc}

\usepackage[LY1, LGR, TS1, T1]{fontenc}
	% LGR (Greek) is needed to enable the ``sansserif math'' features.
\usepackage[utf8]{inputenc}

\usepackage{tabularx}

\ifxetex
	\usepackage[protrusion=true, expansion=false]{microtype}
\else
	\usepackage[protrusion=true, expansion=false, kerning=true]{microtype}
\fi
	% The microtype package enables so-called hanging punctuation. That is, when a punctuation sign like ":", ".", "-", etc. is found at the beginning or end of a line, it is protruded a little into the page margin. This results in "optical margin alignment," because the protrusion makes the margin alignment look straighter.


\definecolor{darkgray}{rgb}{0.4,0.4,0.4}

\usepackage[absolute, overlay]{textpos}
\usepackage{graphicx}
\usepackage{amsmath}
%\usepackage{amsfonts}
%\usepackage{amssymb}
\usepackage{epstopdf}
\DeclareGraphicsRule{.tif}{png}{.png}{`convert #1 `dirname #1`/`basename #1 .tif`.png}

\usepackage[ngerman, english]{babel}

\usepackage{multirow}

\usepackage{xcolor}
\usepackage{pgf}%,pgfarrows,pgfnodes,pgfshade}
\usepackage{tikz}
\usepackage{pgfplots}
\usetikzlibrary{mindmap, trees, patterns}

\usepackage[hang, center last, footnotesize, bf]{caption}

\usepackage[round]{natbib}
\renewcommand{\bibfont}{\scriptsize}

% Better-than-the-default table design
\usepackage{booktabs}
% \heavyrulewidth=.05em
% \belowbottomsep=-2.5pt
\usepackage{colortbl} % e.g., for colored rules
\arrayrulecolor{SpotColor} % Color all table rules blue
% <--

\usepackage{xspace}

\selectlanguage{english}
% \usepackage{eurosym}

%\usepackage{etoolbox}  % loaded automatically by beamer

\hypersetup{%
	pdfpagemode=FullScreen,
	colorlinks, linkcolor=, urlcolor=SpotColor, citecolor=%
}
\urlstyle{same}
\newcommand{\email}[1]{\href{mailto:#1}{\nolinkurl{#1}}}




%%%%%%%%%%%%%%%%%%%%%
%%  FONT SETTINGS  %%
%%%%%%%%%%%%%%%%%%%%%


% !TeX program = pdflatex
% !TeX TXS-program:compile = txs:///pdflatex/
% !TeX TS-program = pdflatex
% !BIB program = biber
% !TeX TXS-program:bibliography = txs:///biber




%%%%%%%%%%%%%%%%%%%%%%%%%%%%%%%%%%%%
%%  FONT SETTINGS AND TYPOGRAPHY  %%
%%%%%%%%%%%%%%%%%%%%%%%%%%%%%%%%%%%%


% Use Fira Sans as the sans-serif font.
% Load first because it otherwise doesn't work properly.
% ==>
\usepackage[lining, scale=0.88]{FiraMono}  % To match the x-height of the Charter serif font
\usepackage[book, semibold, lining, tabular, scale=0.88]{FiraSans}[2018/05/23]
\makeatletter
\@ifpackagelater{FiraSans}{2018/05/23}
	{}
	{\PackageError{FiraSans}
	     {Outdated 'FiraSans' package}
	     {Upgrade to FiraSans 2018/05/23 or newer!\MessageBreak
	      Otherwise the sans-serif font may not work properly, or the compilation might fail.\MessageBreak
	      This is a fatal error. I'm aborting now.}%
	\endinput%
   }
\makeatother
% In case you wish to use the ``light'' options of FiraSans: -->
\ifthenelse{\isundefined{\AfterPackage}}{\usepackage{afterpackage}}{}
\AfterPackage{siunitx}{%
  \providecommand*{\lseries}{\fontseries{l}\selectfont}
}
% See https://tex.stackexchange.com/questions/164791/combination-of-siunitx-and-light-weight-font-fails-to-compile.
% <--
% <==

%% This template also works with the ``IBM Plex Sans'' and ``IBM Plex Mono'' fonts,
%% https://ctan.org/pkg/plex ==>
%\usepackage[semibold, scale=0.89]{plex-mono}  % To match the x-height of the Charter serif font
%\usepackage[semibold, scale=0.89]{plex-sans}[2018/12/31]
%	% Earlier versions of the ``plex-sans'' package do not include support for LGR Greek.
%% <==

%% This template also works with the ``Alegreya'' and ``Alegreya Sans'' fonts,
%% https://ctan.org/pkg/alegreya ==>
%\usepackage[semibold, scale=0.89]{plex-mono}  % To match the x-height of the Charter serif font
%\usepackage[lining, tabular, scale=1.02]{AlegreyaSans}[2018/08/02]
%	% Earlier versions of the ``Alegreya'' package do not include support for LGR Greek.
%% <==

%% This template also works with the ``Google Noto'' fonts,
%% https://ctan.org/pkg/noto ==>
%\usepackage[scale=0.85]{noto-mono}
%\usepackage[lining, tabular, scale=0.85]{noto-sans}[2019/01/11]
%	% Earlier versions of the ``noto'' package do not include support for LGR Greek.
%% <==

% Save the font family and font series definitions for later use: ==>
\newcommand{\savesffamily}{\sfdefault}
\makeatletter
\newcommand{\savesfmdseries}{\mdseries@sf}
\newcommand{\savesfbfseries}{\bfseries@sf}
\makeatother
% <==

% Use Bitstream Charter as the math font
% ==>
\usepackage[charter, greeklowercase=italicized, greekuppercase=italicized, sfscaled=false, ttscaled=false, euro=false]{mathdesign}
%% Use Alegreya as the text font -->
%\usepackage[lining, tabular, scale=1.02]{Alegreya}[2018/08/02]
%\ifxetex \else \DisableLigatures[T]{family = {rm*}} \fi
%% <--
%% Use Noto Serif as the text font -->
%\usepackage[lining, tabular, scale=0.85]{noto-serif}[2019/01/11]
%% <--
% Save the font family and font series definitions for later use:
\let \savermfamily   \rmdefault
\let \savermmdseries \mddefault
\let \savermbfseries \bfdefault
% <==

% mathdesign provides upright Greek letters as \alphaup, \Betaup, etc. while other packages
% (e.g., unicode-math) call these \upalpha, \upBeta, etc. We make also the latter available.
% ==>
\makeatletter
\@for\@tempa:=%
	alpha,beta,gamma,delta,epsilon,varepsilon,zeta,eta,theta,vartheta,iota,kappa,lambda,mu,nu,xi,%
	omicron,pi,varpi,rho,varrho,sigma,varsigma,tau,upsilon,phi,varphi,chi,psi,omega,digamma,%
	Alpha,Beta,Gamma,Delta,Epsilon,Zeta,Eta,Theta,Iota,Kappa,Lambda,Mu,Nu,Xi,%
	Omicron,Pi,Rho,Sigma,Tau,Upsilon,Phi,Chi,Psi,Omega,Digamma%
	\do{%
		\expandafter\let\csname up\@tempa\expandafter\endcsname\csname\@tempa up\endcsname%
	}%
\makeatother
% <==

% Provide a \nequiv symbol
\ifx\nequiv\undefined
	\newcommand{\nequiv}{\not\equiv}
\fi

% Use Bitstream Charter as the text font
% ==>
\usepackage[scaled=.96, lining, sups]{XCharter}  % option ``sups'' not working properly with subimport
% Save the font family and font series definitions for later use:
\let \savermdefaultfortext \rmdefault
\let \savemddefaultfortext \mddefault
\let \savebfdefaultfortext \bfdefault
\renewcommand{\textellipsis}{\mbox{.{\kern.09em}.{\kern.09em}.}}
\makeatletter
\newcommand{\@makefnmarkorig}{%
	\hbox{\sufigures\hspace*{.04em}\@thefnmark\hspace*{.04em}}%
}  % Copied from XCharter.sty
\makeatother
% <==

\ifxetex
	% Do nothing.
\else
	\DisableLigatures[f]{family = {rm*, tt*}}
	% Disable the f* ligatures for Charter because the font does not provide sufficient support.
	% Disable the f* ligatures for the ``typewriter'' font because they make no sense for a monospaced font.
	\SetExtraKerning[unit=space]
		{encoding=*, family=*, series=*, size={*, normalsize, footnotesize}, font = */*/*/*/*}
		{\textemdash = {325, 325},
		           / = {100, 100},
		           : = { 50,   0},
		           ; = { 50,   0}}
\fi

\usepackage{xfrac}	% Provides \sfrac

% Since math mode uses a different font encoding, issuing \euro/\texteuro in math mode
% produces an incorrect sign. We fix this. ==>
\AtBeginDocument{%
	\providecommand{\euro}[1]{\relax\ifmmode\text{\texteuro}#1\else\texteuro #1\fi}%
}
% <==

% Necessary to enable semibold weight as \bfseries for Fira Sans:
%\ProvidesPackage{mweights}
%  [2017/03/30 (Bob Tennent)  Support package for multiple-weight font packages. ]


\makeatletter

\def\mweights@init{%
% Define any undefined \mdseries@rm etc. 
% Defined \mdseries@rm etc. assumed to be fully expanded!
\ifdefined\mdseries@rm\else\edef\mdseries@rm{\mddefault}\fi
\ifdefined\bfseries@rm\else\edef\bfseries@rm{\bfdefault}\fi
\ifdefined\mdseries@sf\else\edef\mdseries@sf{\mddefault}\fi
\ifdefined\bfseries@sf\else\edef\bfseries@sf{\bfdefault}\fi
\ifdefined\mdseries@tt\else\edef\mdseries@tt{\mddefault}\fi
\ifdefined\bfseries@tt\else\edef\bfseries@tt{\bfdefault}\fi
% In case any unexpanded macros present in \rmdefault, etc
\edef\rmdef@ult{\rmdefault}%
\edef\sfdef@ult{\sfdefault}%
\edef\ttdef@ult{\ttdefault}%
\edef\bfdef@ult{\bfdefault}%
\edef\mddef@ult{\mddefault}%
\edef\famdef@ult{\familydefault}%
}

\DeclareRobustCommand\bfseries{%
\mweights@init
\not@math@alphabet\bfseries\mathbf
\ifx\f@family\rmdef@ult\fontseries\bfseries@rm
\else\ifx\f@family\sfdef@ult\fontseries\bfseries@sf
\else\ifx\f@family\ttdef@ult\fontseries\bfseries@tt
\else\fontseries\bfdefault\fi\fi\fi\selectfont}%

\DeclareRobustCommand\mdseries{%
\mweights@init
\not@math@alphabet\mdseries\relax
\ifx\f@family\rmdef@ult\fontseries\mdseries@rm
\else\ifx\f@family\sfdef@ult\fontseries\mdseries@sf
\else\ifx\f@family\ttdef@ult\fontseries\mdseries@tt
\else\fontseries\mddefault\fi\fi\fi\selectfont}

\DeclareRobustCommand\rmfamily{%
\mweights@init
\not@math@alphabet\rmfamily\mathrm
% change the current series before changing the family
\ifx\f@family\sfdef@ult
    \ifx\f@series\mdseries@sf\fontseries\mdseries@rm
    \else\ifx\f@series\bfseries@sf\fontseries\bfseries@rm
    \else\ifx\f@series\mddef@ult\fontseries\mdseries@rm
    \else\ifx\f@series\bfdef@ult\fontseries\bfseries@rm
    \else\fontseries\mdseries@rm
    \fi\fi\fi\fi
\else\ifx\f@family\ttdef@ult
    \ifx\f@series\mdseries@tt\fontseries\mdseries@rm
    \else\ifx\f@series\bfseries@tt\fontseries\bfseries@rm
    \else\ifx\f@series\mddef@ult\fontseries\mdseries@rm
    \else\ifx\f@series\bfdef@ult\fontseries\bfseries@rm
    \else\fontseries\mdseries@rm
    \fi\fi\fi\fi
\else\ifx\f@family\rmdef@ult
    \ifx\f@series\mdseries@rm\fontseries\mdseries@rm
    \else\ifx\f@series\bfseries@rm\fontseries\bfseries@rm
    \else\ifx\f@series\mddef@ult\fontseries\mdseries@rm
    \else\ifx\f@series\bfdef@ult\fontseries\bfseries@rm
    \else\fontseries\mdseries@rm
    \fi\fi\fi\fi
\else\fontseries\mdseries@rm
\fi\fi\fi\fontfamily\rmdefault\selectfont}

\DeclareRobustCommand\sffamily{%
\mweights@init
\not@math@alphabet\sffamily\mathsf
% change the current series before changing the family
\ifx\f@family\rmdef@ult
    \ifx\f@series\mdseries@rm\fontseries\mdseries@sf
    \else\ifx\f@series\bfseries@rm\fontseries\bfseries@sf
    \else\ifx\f@series\mddef@ult\fontseries\mdseries@sf
    \else\ifx\f@series\bfdef@ult\fontseries\bfseries@sf
    \else\fontseries\mdseries@sf
    \fi\fi\fi\fi
\else\ifx\f@family\ttdef@ult
    \ifx\f@series\mdseries@tt\fontseries\mdseries@sf
    \else\ifx\f@series\bfseries@tt\fontseries\bfseries@sf
    \else\ifx\f@series\mddef@ult\fontseries\mdseries@sf
    \else\ifx\f@series\bfdef@ult\fontseries\bfseries@sf
    \else\fontseries\mdseries@sf
    \fi\fi\fi\fi
\else\ifx\f@family\sfdef@ult
    \ifx\f@series\mdseries@sf\fontseries\mdseries@sf
    \else\ifx\f@series\bfseries@sf\fontseries\bfseries@sf
    \else\ifx\f@series\mddef@ult\fontseries\mdseries@sf
    \else\ifx\f@series\bfdef@ult\fontseries\bfseries@sf
    \else\fontseries\mdseries@sf
    \fi\fi\fi\fi
\else\fontseries\mdseries@sf
\fi\fi\fi\fontfamily\sfdefault\selectfont}

\DeclareRobustCommand\ttfamily{%
\mweights@init
\not@math@alphabet\ttfamily\mathtt
% change the current series before changing the family
\ifx\f@family\rmdef@ult
    \ifx\f@series\mdseries@rm\fontseries\mdseries@tt
    \else\ifx\f@series\bfseries@rm\fontseries\bfseries@tt
    \else\ifx\f@series\mddef@ult\fontseries\mdseries@tt
    \else\ifx\f@series\bfdef@ult\fontseries\bfseries@tt
    \else\fontseries\mdseries@tt
    \fi\fi\fi\fi
\else\ifx\f@family\sfdef@ult
    \ifx\f@series\mdseries@sf\fontseries\mdseries@tt
    \else\ifx\f@series\bfseries@sf\fontseries\bfseries@tt
    \else\ifx\f@series\mddef@ult\fontseries\mdseries@tt
    \else\ifx\f@series\bfdef@ult\fontseries\bfseries@tt
    \else\fontseries\mdseries@tt
    \fi\fi\fi\fi
\else\ifx\f@family\ttdef@ult
    \ifx\f@series\mdseries@tt\fontseries\mdseries@tt
    \else\ifx\f@series\bfseries@tt\fontseries\bfseries@tt
    \else\ifx\f@series\mddef@ult\fontseries\mdseries@tt
    \else\ifx\f@series\bfdef@ult\fontseries\bfseries@tt
    \else\fontseries\mdseries@tt
    \fi\fi\fi\fi
\else\fontseries\mdseries@tt
\fi\fi\fi\fontfamily\ttdefault\selectfont}

% override default family with new \familydefault
\AtBeginDocument{\mweights@init
\ifx\famdef@ult\rmdef@ult\rmfamily
\else\ifx\famdef@ult\sfdef@ult\sffamily
\else\ifx\famdef@ult\ttdef@ult\ttfamily
\fi\fi\fi}

\makeatother

\endinput

\newcommand{\serifbodyfont}{1}
	% 0: Body is set in sans-serif font, ``structure'' (headings etc.) font is serif
	% 1: Body is set in serif font, ``structure'' (headings etc.) font is sans-serif

\ifnum \serifbodyfont=1
	% If serif font for body text =>
	\usefonttheme{serif}
	\setbeamerfont{normal text}{family=\rmfamily}
	\setbeamerfont{structure}{family=\sffamily, shape=\upshape, series=\bfseries}
	\setbeamerfont{frametitle}{size=\large, parent=structure}
	% <=
\else
	% If sansserif font for body text =>
	\setbeamerfont{normal text}{family=\sffamily}
	\renewcommand{\familydefault}{\savesffamily}
	\renewcommand{\mddefault}{\savesfmdseries}
	\renewcommand{\bfdefault}{\savesfbfseries}
	\setbeamerfont{structure}{family=\rmfamily, shape=\upshape, series=\bfseries}
	\setbeamerfont{frametitle}{size=\large, parent=structure}
	% <=
\fi
% Make frame titles and headlines bold
\usefonttheme{structurebold}
\usefonttheme{professionalfonts}

%% Use the bm (= boldmath) package for better support of setting math in bold ==>
%% Prevent the "Too many math fonts used" error:
\newcommand{\bmmax}{0}
\newcommand{\hmmax}{0}
\usepackage{bm}
%% <==

\usepackage{fontawesome}

\usepackage{mathtools}
%\mathtoolsset{centercolon}
	% This makes the compilation fail in combination with tikz. See
	% https://tex.stackexchange.com/questions/89467/why-does-pdftex-hang-on-this-file.
%% Inspired by https://tex.stackexchange.com/questions/251460/how-to-put-symbols-of-equal-size-on-top-of-each-other
\newcommand{\succeqq}{%
  \mathrel{%
    \vcenter{\offinterlineskip
      \ialign{##\cr$\succ$\cr\noalign{\kern 1pt}$=$\cr}%
    }%
  }%
}
\newcommand{\nsucceqq}{\mathrel{\not\succeqq}}
\newcommand*{\coloneqq}{\mathrel{%
	\mathrel{%
		\raisebox{0.15ex}{\scalebox{0.85}{\ensuremath{:}}}\hspace{-0.2pt}%
	}%
	=%
}}

% !TeX program = pdflatex
% !TeX TXS-program:compile = txs:///pdflatex/
% !TeX TS-program = pdflatex
% !BIB program = biber
% !TeX TXS-program:bibliography = txs:///biber




%%%%%%%%%%%%%%%%%%%%%%%%%%%%%%%%%%%%%%%%%%%%%%%%%
%%  SANS-SERIF MATH IN SANS-SERIF ENVIRONMENT  %%
%%%%%%%%%%%%%%%%%%%%%%%%%%%%%%%%%%%%%%%%%%%%%%%%%


% See https://tex.stackexchange.com/questions/41497/how-to-typeset-some-text-including-math-content-in-sans-serif
% See https://tex.stackexchange.com/questions/33165/make-mathfont-respect-the-surrounding-family

% Necessary for use of kpfonts
% ==>
\makeatletter
\newif\ifkp@upRm
\newif\ifkp@osm
\newif\ifkp@vosm
\makeatother
% <==

\DeclareMathVersion{normalup}
\DeclareMathVersion{boldup}
\DeclareMathVersion{sans}

%\SetSymbolFont{operators}{sans}{OT1}{jkpss}{m}{n}
%	% From http://mirrors.ctan.org/fonts/kpfonts/latex/kpfonts.sty
\SetSymbolFont{operators}   {sans}{OT1}{mdbch}{m}{n}
\SetSymbolFont{letters}     {sans}{OML}{jkpss}{m}{it}
	% From http://mirrors.ctan.org/fonts/kpfonts/latex/kpfonts.sty
%\SetSymbolFont{letters}     {sans}{OML}{cmbrm}{m}{it}
%\SetSymbolFont{symbols}     {sans}{OMS}{cmbrs}{m}{n}
\SetSymbolFont{symbols}     {sans}{OMS}{jkp}  {m}{n}
	% From http://mirrors.ctan.org/fonts/kpfonts/latex/kpfonts.sty
\DeclareSymbolFont{extrasymbols}  {OMS}{cmbrs}{m}{n}
\SetSymbolFont{extrasymbols}{sans}{OMS}{cmbrs}{m}{n}
	% Some symbols (e.g., \prime) look weird in kpfonts.
	% This provides the option to replace them by symbols from mathdesign-charter.

\SetMathAlphabet{\mathit} {sans}{OT1}{\savesffamily}{\savesfmdseries}{it}
\SetMathAlphabet{\mathbf} {sans}{OT1}{\savesffamily}{\savesfbfseries}{n}
\SetMathAlphabet{\mathtt} {sans}{OT1}{cmtl}{m}{n}
\SetMathAlphabet{\mathcal}{sans}{OMS}{ntxsy}{m}{n}
	% See https://tex.stackexchange.com/questions/231583/import-mathcal-symbols-from-txfonts
%\SetSymbolFont{largesymbols}{sans}{OMX}{jkpss}{m}{n}
%	% From http://mirrors.ctan.org/fonts/kpfonts/latex/kpfonts.sty
\SetSymbolFont{largesymbols} {sans}{OMX}{mdbch}{m}{n}
	% Using symbols like \int, \left(, etc. from mathdesign-charter because they look better than the ones included in kpfonts

\DeclareMathVersion{sansup}
\SetSymbolFont{letters}  {sansup}{OML}{jkpss}{m}{it}
\SetSymbolFont{symbols}  {sansup}{OMS}{jkp}  {m}{n}

\DeclareMathVersion{boldsans}
%\SetSymbolFont{operators}{boldsans}{OT1}{jkpss}{b}{n}
%	% From http://mirrors.ctan.org/fonts/kpfonts/latex/kpfonts.sty
\SetSymbolFont{operators}{boldsans}{OT1}{mdbch}{bx}{n}
\SetSymbolFont{letters}  {boldsans}{OML}{jkpss}{bx}{it}
	% From http://mirrors.ctan.org/fonts/kpfonts/latex/kpfonts.sty
%\SetSymbolFont{letters}  {boldsans}{OML}{mdbch}{bx}{it}
%\SetSymbolFont{letters}{boldsans}{OML}{cmbrm}{b}{it}
\SetSymbolFont{symbols}  {boldsans}{OMS}{jkp}  {bx}{n}
	% From http://mirrors.ctan.org/fonts/kpfonts/latex/kpfonts.sty
%\SetMathAlphabet{\mathrm}{boldsans}{OT1}{\savesffamily}{\savesfbfseries}{n}
\SetMathAlphabet{\mathit} {boldsans}{OT1}{\savesffamily}{\savesfbfseries}{it}
\SetMathAlphabet{\mathtt} {boldsans}{OT1}{cmtl}{b}{n}
\SetMathAlphabet{\mathcal}{boldsans}{OMS}{ntxsy}{b}{n}
%\SetSymbolFont{largesymbols}{boldsans}{OMX}{jkpss}{bx}{n}
%	% From http://mirrors.ctan.org/fonts/kpfonts/latex/kpfonts.sty
\SetSymbolFont{largesymbols}{boldsans}{OMX}{mdbch}{bx}{n}
	% Using symbols like \int, \left(, etc. from mathdesign-charter because they look better than the ones included in kpfonts

\DeclareMathVersion{boldsansup}
\SetSymbolFont{letters}{boldsansup}{OML}{jkpss}{bx}{it}
\SetSymbolFont{symbols}{boldsansup}{OMS}{jkp}  {bx}{n}

% Using glyphs for math mode from the custom sansserif font
\DeclareSymbolFont{uprightglyphs}{T1}{\savermfamily}{\savermmdseries}{n}
\SetSymbolFont{uprightglyphs}{normal}    {T1}{\savermfamily}{\savermmdseries}{n}
\SetSymbolFont{uprightglyphs}{normalup}  {T1}{\savermfamily}{\savermmdseries}{n}
\SetSymbolFont{uprightglyphs}{bold}      {T1}{\savermfamily}{\savermbfseries}{n}
\SetSymbolFont{uprightglyphs}{boldup}    {T1}{\savermfamily}{\savermbfseries}{n}
\SetSymbolFont{uprightglyphs}{sans}      {T1}{\savesffamily}{\savesfmdseries}{n}
\SetSymbolFont{uprightglyphs}{sansup}    {T1}{\savesffamily}{\savesfmdseries}{n}
\SetSymbolFont{uprightglyphs}{boldsans}  {T1}{\savesffamily}{\savesfbfseries}{n}
\SetSymbolFont{uprightglyphs}{boldsansup}{T1}{\savesffamily}{\savesfbfseries}{n}
\DeclareSymbolFont{italicglyphs} {T1}{\savermfamily}{\savermmdseries}{it}
\SetSymbolFont{italicglyphs} {normal}    {T1}{\savermfamily}{\savermmdseries}{it}
\SetSymbolFont{italicglyphs} {normalup}  {T1}{\savermfamily}{\savermmdseries}{n}
\SetSymbolFont{italicglyphs} {bold}      {T1}{\savermfamily}{\savermbfseries}{it}
\SetSymbolFont{italicglyphs} {boldup}    {T1}{\savermfamily}{\savermbfseries}{n}
\SetSymbolFont{italicglyphs} {sans}      {T1}{\savesffamily}{\savesfmdseries}{it}
\SetSymbolFont{italicglyphs} {sansup}    {T1}{\savesffamily}{\savesfmdseries}{n}
\SetSymbolFont{italicglyphs} {boldsans}  {T1}{\savesffamily}{\savesfbfseries}{it}
\SetSymbolFont{italicglyphs} {boldsansup}{T1}{\savesffamily}{\savesfbfseries}{n}

% Syntax of \DeclareMathSymobl:
% \DeclareMathSymbol {<symbol>} {<type>} {<sym-font>} {<slot>}
% Type              Meaning	            Example
% 0 or \mathord     Ordinary             $\alpha$
% 1 or \mathop      Large operator       $\sum$
% 2 or \mathbin     Binary operation     $\times$
% 3 or \mathrel     Relation             $\leq$
% 4 or \mathopen    Opening              $\langle$
% 5 or \mathclose   Closing              $\rangle$
% 6 or \mathpunct   Punctuation          ;
% 7 or \mathalpha   Alphabet character   A
% Example declaration:
% \DeclareMathSymbol{b}{0}{letters}{`b}

% Digits
\DeclareMathSymbol{0}{\mathalpha}{uprightglyphs}{`0}
\DeclareMathSymbol{1}{\mathalpha}{uprightglyphs}{`1}
\DeclareMathSymbol{2}{\mathalpha}{uprightglyphs}{`2}
\DeclareMathSymbol{3}{\mathalpha}{uprightglyphs}{`3}
\DeclareMathSymbol{4}{\mathalpha}{uprightglyphs}{`4}
\DeclareMathSymbol{5}{\mathalpha}{uprightglyphs}{`5}
\DeclareMathSymbol{6}{\mathalpha}{uprightglyphs}{`6}
\DeclareMathSymbol{7}{\mathalpha}{uprightglyphs}{`7}
\DeclareMathSymbol{8}{\mathalpha}{uprightglyphs}{`8}
\DeclareMathSymbol{9}{\mathalpha}{uprightglyphs}{`9}
% Operators and punctuation
\DeclareMathSymbol{+}{\mathbin}  {operators}    {`+}
	% Not from uprightglyphs due to bad spacing
\DeclareMathSymbol{=}{\mathrel}  {operators}    {`=}
	% Not from uprightglyphs due to bad spacing
\DeclareMathSymbol{.}{\mathord}  {uprightglyphs}{`.}
\DeclareMathSymbol{,}{\mathpunct}{uprightglyphs}{`,}
\DeclareMathSymbol{;}{\mathpunct}{uprightglyphs}{`;}
\DeclareMathSymbol{/}{\mathord}  {uprightglyphs}{`/}
%\DeclareMathSymbol{/}{\mathop}   {uprightglyphs}{`/}
%	% This would icrease the spacing around the division slash slightly
%\DeclareMathSymbol{(}{\mathopen} {uprightglyphs}{`(}
%\DeclareMathSymbol{)}{\mathclose}{uprightglyphs}{`)}
%\DeclareMathSymbol{[}{\mathopen} {uprightglyphs}{`[}
%\DeclareMathSymbol{]}{\mathclose}{uprightglyphs}{`]}
\DeclareMathSymbol{\prime}{\mathord}{extrasymbols}{"30}
	% Use \prime from mathdesign-charter because it looks better than the one in kpfonts
\DeclareMathDelimiter{(}      {\mathopen} {uprightglyphs}{`(} {largesymbols}{"00}
\DeclareMathDelimiter{)}      {\mathclose}{uprightglyphs}{`)} {largesymbols}{"01}
\DeclareMathDelimiter{[}      {\mathopen} {uprightglyphs}{`[} {largesymbols}{"02}
\DeclareMathDelimiter{]}      {\mathclose}{uprightglyphs}{`]} {largesymbols}{"03}
\DeclareMathDelimiter{\lbrace}{\mathopen} {uprightglyphs}{`\{}{largesymbols}{"08}
\DeclareMathDelimiter{\rbrace}{\mathclose}{uprightglyphs}{`\}}{largesymbols}{"09}
% Uppercase Latin characters
\DeclareMathSymbol{A}{\mathalpha}{italicglyphs}{`A}
\DeclareMathSymbol{B}{\mathalpha}{italicglyphs}{`B}
\DeclareMathSymbol{C}{\mathalpha}{italicglyphs}{`C}
\DeclareMathSymbol{D}{\mathalpha}{italicglyphs}{`D}
\DeclareMathSymbol{E}{\mathalpha}{italicglyphs}{`E}
\DeclareMathSymbol{F}{\mathalpha}{italicglyphs}{`F}
\DeclareMathSymbol{G}{\mathalpha}{italicglyphs}{`G}
\DeclareMathSymbol{H}{\mathalpha}{italicglyphs}{`H}
\DeclareMathSymbol{I}{\mathalpha}{italicglyphs}{`I}
\DeclareMathSymbol{J}{\mathalpha}{italicglyphs}{`J}
\DeclareMathSymbol{K}{\mathalpha}{italicglyphs}{`K}
\DeclareMathSymbol{L}{\mathalpha}{italicglyphs}{`L}
\DeclareMathSymbol{M}{\mathalpha}{italicglyphs}{`M}
\DeclareMathSymbol{N}{\mathalpha}{italicglyphs}{`N}
\DeclareMathSymbol{O}{\mathalpha}{italicglyphs}{`O}
\DeclareMathSymbol{P}{\mathalpha}{italicglyphs}{`P}
\DeclareMathSymbol{Q}{\mathalpha}{italicglyphs}{`Q}
\DeclareMathSymbol{R}{\mathalpha}{italicglyphs}{`R}
\DeclareMathSymbol{S}{\mathalpha}{italicglyphs}{`S}
\DeclareMathSymbol{T}{\mathalpha}{italicglyphs}{`T}
\DeclareMathSymbol{U}{\mathalpha}{italicglyphs}{`U}
\DeclareMathSymbol{V}{\mathalpha}{italicglyphs}{`V}
\DeclareMathSymbol{W}{\mathalpha}{italicglyphs}{`W}
\DeclareMathSymbol{X}{\mathalpha}{italicglyphs}{`X}
\DeclareMathSymbol{Y}{\mathalpha}{italicglyphs}{`Y}
\DeclareMathSymbol{Z}{\mathalpha}{italicglyphs}{`Z}
% lowercase Latin characters
\DeclareMathSymbol{a}{\mathalpha}{italicglyphs}{`a}
\DeclareMathSymbol{b}{\mathalpha}{italicglyphs}{`b}
\DeclareMathSymbol{c}{\mathalpha}{italicglyphs}{`c}
\DeclareMathSymbol{d}{\mathalpha}{italicglyphs}{`d}
\DeclareMathSymbol{e}{\mathalpha}{italicglyphs}{`e}
\DeclareMathSymbol{f}{\mathalpha}{italicglyphs}{`f}
\DeclareMathSymbol{g}{\mathalpha}{italicglyphs}{`g}
\DeclareMathSymbol{h}{\mathalpha}{italicglyphs}{`h}
\DeclareMathSymbol{i}{\mathalpha}{italicglyphs}{`i}
\DeclareMathSymbol{j}{\mathalpha}{italicglyphs}{`j}
\DeclareMathSymbol{k}{\mathalpha}{italicglyphs}{`k}
\DeclareMathSymbol{l}{\mathalpha}{italicglyphs}{`l}
\DeclareMathSymbol{m}{\mathalpha}{italicglyphs}{`m}
\DeclareMathSymbol{n}{\mathalpha}{italicglyphs}{`n}
\DeclareMathSymbol{o}{\mathalpha}{italicglyphs}{`o}
\DeclareMathSymbol{p}{\mathalpha}{italicglyphs}{`p}
\DeclareMathSymbol{q}{\mathalpha}{italicglyphs}{`q}
\DeclareMathSymbol{r}{\mathalpha}{italicglyphs}{`r}
\DeclareMathSymbol{s}{\mathalpha}{italicglyphs}{`s}
\DeclareMathSymbol{t}{\mathalpha}{italicglyphs}{`t}
\DeclareMathSymbol{u}{\mathalpha}{italicglyphs}{`u}
\DeclareMathSymbol{v}{\mathalpha}{italicglyphs}{`v}
\DeclareMathSymbol{w}{\mathalpha}{italicglyphs}{`w}
\DeclareMathSymbol{x}{\mathalpha}{italicglyphs}{`x}
\DeclareMathSymbol{y}{\mathalpha}{italicglyphs}{`y}
\DeclareMathSymbol{z}{\mathalpha}{italicglyphs}{`z}

%% Sansserif Greek letters
%\DeclareSymbolFont{lgrgreek}{LGR}{\savesffamily}{\savesfmdseries}{it}
%\SetSymbolFont{lgrgreek}{sans}    {LGR}{\savesffamily}{\savesfmdseries}{it}
%\SetSymbolFont{lgrgreek}{boldsans}{LGR}{\savesffamily}{\savesfbfseries}{it}

% The following is taken from
% https://tex.stackexchange.com/questions/116389/automatic-upright-math-when-text-is-in-italic/116399#116399
% Filling in ``missing'' Greek glyphs for completeness
% (not really necessary, since they look identical to Latin glyphs and are thus almost never used)
% ==>
\newcommand{\omicron}{o}
\newcommand{\Digamma}{F}
\newcommand{\Alpha}  {A}
\newcommand{\Beta}   {B}
\newcommand{\Epsilon}{E}
\newcommand{\Zeta}   {Z}
\newcommand{\Eta}    {H}
\newcommand{\Iota}   {I}
\newcommand{\Kappa}  {K}
\newcommand{\Mu}     {M}
\newcommand{\Nu}     {N}
\newcommand{\Omicron}{O}
\newcommand{\Rho}    {P}
\newcommand{\Tau}    {T}
\newcommand{\Chi}    {X}
% <==

% Save original definitions of the Greek letters
% ==>
\makeatletter
\@for\@tempa:=%
	alpha,beta,gamma,delta,epsilon,zeta,eta,theta,iota,kappa,lambda,mu,nu,xi,%
	omicron,pi,rho,sigma,varsigma,tau,upsilon,phi,chi,psi,omega,digamma,%
	Alpha,Beta,Gamma,Delta,Epsilon,Zeta,Eta,Theta,Iota,Kappa,Lambda,Mu,Nu,Xi,%
	Omicron,Pi,Rho,Sigma,Tau,Upsilon,Phi,Chi,Psi,Omega,Digamma%
	\do{%
		\expandafter\let\csname\@tempa orig\expandafter\endcsname\csname\@tempa\endcsname%
		\expandafter\let\csname\@tempa uporig\expandafter\endcsname\csname\@tempa up\endcsname%
	}%
\makeatother
% <==

% LGR-encoded Greek letters
% ==>
\newcommand{\textformath}[1]{%
	\IfInBoldMode%
		\IfInUpMode\textbf{#1}\else\textit{\bfseries #1}\fi\relax%
	\else
		\IfInUpMode\textup{#1}\else\textit{#1}\fi\relax%
	\fi\relax%
}
% The double curly braces in this section are necessary to be able to use Greek letters
% in subscripts and superscripts without having to enclose theme in curly braces;
% for example, $\sigma_\epsilon$ instead of $\sigma_{\epsilon}$.
% Uppercase
\newcommand{\AlphaLGR}   {{\mathord{\textformath{\fontencoding{LGR}\selectfont A}}}}
\newcommand{\BetaLGR}    {{\mathord{\textformath{\fontencoding{LGR}\selectfont B}}}}
\newcommand{\GammaLGR}   {{\mathord{\textformath{\fontencoding{LGR}\selectfont G}}}}
\newcommand{\DeltaLGR}   {{\mathord{\textformath{\fontencoding{LGR}\selectfont D}}}}
\newcommand{\EpsilonLGR} {{\mathord{\textformath{\fontencoding{LGR}\selectfont E}}}}
\newcommand{\ZetaLGR}    {{\mathord{\textformath{\fontencoding{LGR}\selectfont Z}}}}
\newcommand{\EtaLGR}     {{\mathord{\textformath{\fontencoding{LGR}\selectfont H}}}}
\newcommand{\ThetaLGR}   {{\mathord{\textformath{\fontencoding{LGR}\selectfont J}}}}
\newcommand{\IotaLGR}    {{\mathord{\textformath{\fontencoding{LGR}\selectfont I}}}}
\newcommand{\KappaLGR}   {{\mathord{\textformath{\fontencoding{LGR}\selectfont K}}}}
\newcommand{\LambdaLGR}  {{\mathord{\textformath{\fontencoding{LGR}\selectfont L}}}}
\newcommand{\MuLGR}      {{\mathord{\textformath{\fontencoding{LGR}\selectfont M}}}}
\newcommand{\NuLGR}      {{\mathord{\textformath{\fontencoding{LGR}\selectfont N}}}}
\newcommand{\XiLGR}      {{\mathord{\textformath{\fontencoding{LGR}\selectfont X}}}}
\newcommand{\OmicronLGR} {{\mathord{\textformath{\fontencoding{LGR}\selectfont O}}}}
\newcommand{\PiLGR}      {{\mathord{\textformath{\fontencoding{LGR}\selectfont P}}}}
\newcommand{\RhoLGR}     {{\mathord{\textformath{\fontencoding{LGR}\selectfont R}}}}
\newcommand{\SigmaLGR}   {{\mathord{\textformath{\fontencoding{LGR}\selectfont S}}}}
\newcommand{\TauLGR}     {{\mathord{\textformath{\fontencoding{LGR}\selectfont T}}}}
\newcommand{\UpsilonLGR} {{\mathord{\textformath{\fontencoding{LGR}\selectfont U}}}}
\newcommand{\PhiLGR}     {{\mathord{\textformath{\fontencoding{LGR}\selectfont F}}}}
\newcommand{\ChiLGR}     {{\mathord{\textformath{\fontencoding{LGR}\selectfont Q}}}}
\newcommand{\PsiLGR}     {{\mathord{\textformath{\fontencoding{LGR}\selectfont Y}}}}
\newcommand{\OmegaLGR}   {{\mathord{\textformath{\fontencoding{LGR}\selectfont W}}}}
\newcommand{\DigammaLGR} {{\mathord{\textformath{\fontencoding{LGR}\selectfont \char195}}}}
% lowercase
\newcommand{\alphaLGR}   {{\mathord{\textformath{\fontencoding{LGR}\selectfont a}}}}
\newcommand{\betaLGR}    {{\mathord{\textformath{\fontencoding{LGR}\selectfont b}}}}
\newcommand{\gammaLGR}   {{\mathord{\textformath{\fontencoding{LGR}\selectfont g}}}}
\newcommand{\deltaLGR}   {{\mathord{\textformath{\fontencoding{LGR}\selectfont d}}}}
\newcommand{\epsilonLGR} {{\mathord{\textformath{\fontencoding{LGR}\selectfont e}}}}
\newcommand{\zetaLGR}    {{\mathord{\textformath{\fontencoding{LGR}\selectfont z}}}}
\newcommand{\etaLGR}     {{\mathord{\textformath{\fontencoding{LGR}\selectfont h}}}}
\newcommand{\thetaLGR}   {{\mathord{\textformath{\fontencoding{LGR}\selectfont j}}}}
\newcommand{\iotaLGR}    {{\mathord{\textformath{\fontencoding{LGR}\selectfont i}}}}
\newcommand{\kappaLGR}   {{\mathord{\textformath{\fontencoding{LGR}\selectfont k}}}}
\newcommand{\lambdaLGR}  {{\mathord{\textformath{\fontencoding{LGR}\selectfont l}}}}
\newcommand{\muLGR}      {{\mathord{\textformath{\fontencoding{LGR}\selectfont m}}}}
\newcommand{\nuLGR}      {{\mathord{\textformath{\fontencoding{LGR}\selectfont n}}}}
\newcommand{\xiLGR}      {{\mathord{\textformath{\fontencoding{LGR}\selectfont x}}}}
\newcommand{\omicronLGR} {{\mathord{\textformath{\fontencoding{LGR}\selectfont o}}}}
\newcommand{\piLGR}      {{\mathord{\textformath{\fontencoding{LGR}\selectfont p}}}}
\newcommand{\rhoLGR}     {{\mathord{\textformath{\fontencoding{LGR}\selectfont r}}}}
\newcommand{\sigmaLGR}   {{\mathord{\textformath{\fontencoding{LGR}\selectfont s\noboundary}}}}
	% \noboundary prevents sigma from being replaced by the word-end sigma (varsigma),
	% see http://mirrors.ctan.org/macros/latex/contrib/textgreek/textgreek.pdf
\newcommand{\varsigmaLGR}{{\mathord{\textformath{\fontencoding{LGR}\selectfont c}}}}
\newcommand{\tauLGR}     {{\mathord{\textformath{\fontencoding{LGR}\selectfont t}}}}
\newcommand{\upsilonLGR} {{\mathord{\textformath{\fontencoding{LGR}\selectfont u}}}}
\newcommand{\phiLGR}     {{\mathord{\textformath{\fontencoding{LGR}\selectfont f}}}}
\newcommand{\chiLGR}     {{\mathord{\textformath{\fontencoding{LGR}\selectfont q}}}}
\newcommand{\psiLGR}     {{\mathord{\textformath{\fontencoding{LGR}\selectfont y}}}}
\newcommand{\omegaLGR}   {{\mathord{\textformath{\fontencoding{LGR}\selectfont w}}}}
\newcommand{\digammaLGR} {{\mathord{\textformath{\fontencoding{LGR}\selectfont \char147}}}}
% Uppercase, upright
\newcommand{\AlphaupLGR}   {{\mathord{\textup{\fontencoding{LGR}\selectfont A}}}}
\newcommand{\BetaupLGR}    {{\mathord{\textup{\fontencoding{LGR}\selectfont B}}}}
\newcommand{\GammaupLGR}   {{\mathord{\textup{\fontencoding{LGR}\selectfont G}}}}
\newcommand{\DeltaupLGR}   {{\mathord{\textup{\fontencoding{LGR}\selectfont D}}}}
\newcommand{\EpsilonupLGR} {{\mathord{\textup{\fontencoding{LGR}\selectfont E}}}}
\newcommand{\ZetaupLGR}    {{\mathord{\textup{\fontencoding{LGR}\selectfont Z}}}}
\newcommand{\EtaupLGR}     {{\mathord{\textup{\fontencoding{LGR}\selectfont H}}}}
\newcommand{\ThetaupLGR}   {{\mathord{\textup{\fontencoding{LGR}\selectfont J}}}}
\newcommand{\IotaupLGR}    {{\mathord{\textup{\fontencoding{LGR}\selectfont I}}}}
\newcommand{\KappaupLGR}   {{\mathord{\textup{\fontencoding{LGR}\selectfont K}}}}
\newcommand{\LambdaupLGR}  {{\mathord{\textup{\fontencoding{LGR}\selectfont L}}}}
\newcommand{\MuupLGR}      {{\mathord{\textup{\fontencoding{LGR}\selectfont M}}}}
\newcommand{\NuupLGR}      {{\mathord{\textup{\fontencoding{LGR}\selectfont N}}}}
\newcommand{\XiupLGR}      {{\mathord{\textup{\fontencoding{LGR}\selectfont X}}}}
\newcommand{\OmicronupLGR} {{\mathord{\textup{\fontencoding{LGR}\selectfont O}}}}
\newcommand{\PiupLGR}      {{\mathord{\textup{\fontencoding{LGR}\selectfont P}}}}
\newcommand{\RhoupLGR}     {{\mathord{\textup{\fontencoding{LGR}\selectfont R}}}}
\newcommand{\SigmaupLGR}   {{\mathord{\textup{\fontencoding{LGR}\selectfont S}}}}
\newcommand{\TauupLGR}     {{\mathord{\textup{\fontencoding{LGR}\selectfont T}}}}
\newcommand{\UpsilonupLGR} {{\mathord{\textup{\fontencoding{LGR}\selectfont U}}}}
\newcommand{\PhiupLGR}     {{\mathord{\textup{\fontencoding{LGR}\selectfont F}}}}
\newcommand{\ChiupLGR}     {{\mathord{\textup{\fontencoding{LGR}\selectfont Q}}}}
\newcommand{\PsiupLGR}     {{\mathord{\textup{\fontencoding{LGR}\selectfont Y}}}}
\newcommand{\OmegaupLGR}   {{\mathord{\textup{\fontencoding{LGR}\selectfont W}}}}
\newcommand{\DigammaupLGR} {{\mathord{\textup{\fontencoding{LGR}\selectfont \char195}}}}
% lowercase, upright
\newcommand{\alphaupLGR}   {{\mathord{\textup{\fontencoding{LGR}\selectfont a}}}}
\newcommand{\betaupLGR}    {{\mathord{\textup{\fontencoding{LGR}\selectfont b}}}}
\newcommand{\gammaupLGR}   {{\mathord{\textup{\fontencoding{LGR}\selectfont g}}}}
\newcommand{\deltaupLGR}   {{\mathord{\textup{\fontencoding{LGR}\selectfont d}}}}
\newcommand{\epsilonupLGR} {{\mathord{\textup{\fontencoding{LGR}\selectfont e}}}}
\newcommand{\zetaupLGR}    {{\mathord{\textup{\fontencoding{LGR}\selectfont z}}}}
\newcommand{\etaupLGR}     {{\mathord{\textup{\fontencoding{LGR}\selectfont h}}}}
\newcommand{\thetaupLGR}   {{\mathord{\textup{\fontencoding{LGR}\selectfont j}}}}
\newcommand{\iotaupLGR}    {{\mathord{\textup{\fontencoding{LGR}\selectfont i}}}}
\newcommand{\kappaupLGR}   {{\mathord{\textup{\fontencoding{LGR}\selectfont k}}}}
\newcommand{\lambdaupLGR}  {{\mathord{\textup{\fontencoding{LGR}\selectfont l}}}}
\newcommand{\muupLGR}      {{\mathord{\textup{\fontencoding{LGR}\selectfont m}}}}
\newcommand{\nuupLGR}      {{\mathord{\textup{\fontencoding{LGR}\selectfont n}}}}
\newcommand{\xiupLGR}      {{\mathord{\textup{\fontencoding{LGR}\selectfont x}}}}
\newcommand{\omicronupLGR} {{\mathord{\textup{\fontencoding{LGR}\selectfont o}}}}
\newcommand{\piupLGR}      {{\mathord{\textup{\fontencoding{LGR}\selectfont p}}}}
\newcommand{\rhoupLGR}     {{\mathord{\textup{\fontencoding{LGR}\selectfont r}}}}
\newcommand{\sigmaupLGR}   {{\mathord{\textup{\fontencoding{LGR}\selectfont s\noboundary}}}}
	% \noboundary prevents sigma from being replaced by the word-end sigma (varsigma),
	% see http://mirrors.ctan.org/macros/latex/contrib/textgreek/textgreek.pdf
\newcommand{\varsigmaupLGR}{{\mathord{\textup{\fontencoding{LGR}\selectfont c}}}}
\newcommand{\tauupLGR}     {{\mathord{\textup{\fontencoding{LGR}\selectfont t}}}}
\newcommand{\upsilonupLGR} {{\mathord{\textup{\fontencoding{LGR}\selectfont u}}}}
\newcommand{\phiupLGR}     {{\mathord{\textup{\fontencoding{LGR}\selectfont f}}}}
\newcommand{\chiupLGR}     {{\mathord{\textup{\fontencoding{LGR}\selectfont q}}}}
\newcommand{\psiupLGR}     {{\mathord{\textup{\fontencoding{LGR}\selectfont y}}}}
\newcommand{\omegaupLGR}   {{\mathord{\textup{\fontencoding{LGR}\selectfont w}}}}
\newcommand{\digammaupLGR} {{\mathord{\textup{\fontencoding{LGR}\selectfont \char147}}}}
% <==

% Based on description of the TS1 encoding in
% http://ctan.math.illinois.edu/macros/latex/doc/encguide.pdf:
%\let \oldpm    \pm
%\let \oldtimes \times
%\let \olddiv   \div
%\makeatletter
%\newcommand{\pmsf}   {\mathbin{\text{\usefont{TS1}{\sfdefault}{\f@series}{n}\char"B1}}}
%\newcommand{\timessf}{\mathbin{\text{\usefont{TS1}{\sfdefault}{\f@series}{n}\char"D6}}}
%\newcommand{\divsf}  {\mathbin{\text{\usefont{TS1}{\sfdefault}{\f@series}{n}\char"F6}}}
%\makeatother

% Use LGR-encoded Greek letters for \mathversion{sans}
% ==>
\makeatletter

\newcommand*{\sansmath}{%
	\@for\@tempa:=%
		alpha,beta,gamma,delta,epsilon,zeta,eta,theta,iota,kappa,lambda,mu,nu,xi,%
		omicron,pi,rho,sigma,varsigma,tau,upsilon,phi,chi,psi,omega,digamma,%
		Alpha,Beta,Gamma,Delta,Epsilon,Zeta,Eta,Theta,Iota,Kappa,Lambda,Mu,Nu,Xi,%
		Omicron,Pi,Rho,Sigma,Tau,Upsilon,Phi,Chi,Psi,Omega,Digamma%
		\do{%
			\expandafter\let\csname\@tempa\expandafter\endcsname\csname\@tempa LGR\endcsname%
			\expandafter\let\csname\@tempa up\expandafter\endcsname\csname\@tempa upLGR\endcsname%
			\expandafter\let\csname up\@tempa\expandafter\endcsname\csname\@tempa upLGR\endcsname%
		}%
	%\renewcommand{\pm}{\pmsf}%
	%\renewcommand{\times}{\timessf}%
	%\renewcommand{\div}{\divsf}%
}
% <==

% Switch back to the original Greek letters for \mathversion{normal}, i.e., the serif font
% ==>
\newcommand*{\unsansmath}{%
	\@for\@tempa:=%
		alpha,beta,gamma,delta,epsilon,zeta,eta,theta,iota,kappa,lambda,mu,nu,xi,%
		omicron,pi,rho,sigma,varsigma,tau,upsilon,phi,chi,psi,omega,digamma,%
		Alpha,Beta,Gamma,Delta,Epsilon,Zeta,Eta,Theta,Iota,Kappa,Lambda,Mu,Nu,Xi,%
		Omicron,Pi,Rho,Sigma,Tau,Upsilon,Phi,Chi,Psi,Omega,Digamma%
		\do{%
			\expandafter\let\csname\@tempa\expandafter\endcsname\csname\@tempa orig\endcsname%
			\expandafter\let\csname\@tempa up\expandafter\endcsname\csname\@tempa uporig\endcsname%
			\expandafter\let\csname up\@tempa\expandafter\endcsname\csname\@tempa uporig\endcsname%
		}%
	%\renewcommand{\pm}{\oldpm}%
	%\renewcommand{\times}{\oldtimes}%
	%\renewcommand{\div}{\olddiv}%
}
% <==

%% If you would like to use LGR-encoded Greek letters also for the serif font
%% ==>
%\renewcommand*{\unsansmath}{%
%	\@for\@tempa:=%
%		alpha,beta,gamma,delta,epsilon,zeta,eta,theta,iota,kappa,lambda,mu,nu,xi,%
%		omicron,pi,rho,sigma,varsigma,tau,upsilon,phi,chi,psi,omega,digamma,%
%		Alpha,Beta,Gamma,Delta,Epsilon,Zeta,Eta,Theta,Iota,Kappa,Lambda,Mu,Nu,Xi,%
%		Omicron,Pi,Rho,Sigma,Tau,Upsilon,Phi,Chi,Psi,Omega,Digamma%
%		\do{%
%			\expandafter\let\csname\@tempa\expandafter\endcsname\csname\@tempa LGR\endcsname%
%			\expandafter\let\csname\@tempa up\expandafter\endcsname\csname\@tempa upLGR\endcsname%
%			\expandafter\let\csname up\@tempa\expandafter\endcsname\csname\@tempa upLGR\endcsname%
%		}%
%	%\renewcommand{\pm}{\pmsf}%
%	%\renewcommand{\times}{\timessf}%
%	%\renewcommand{\div}{\divsf}%
%}
%% <==

\newcommand*{\upgreekletters}{%
	\@for\@tempa:=%
		alpha,beta,gamma,delta,epsilon,zeta,eta,theta,iota,kappa,lambda,mu,nu,xi,%
		omicron,pi,rho,sigma,varsigma,tau,upsilon,phi,chi,psi,omega,digamma,%
		Alpha,Beta,Gamma,Delta,Epsilon,Zeta,Eta,Theta,Iota,Kappa,Lambda,Mu,Nu,Xi,%
		Omicron,Pi,Rho,Sigma,Tau,Upsilon,Phi,Chi,Psi,Omega,Digamma%
		\do{%
			\expandafter\let\csname\@tempa\expandafter\endcsname\csname\@tempa up\endcsname%
		}%
}
\newcommand*{\itgreekletters}{%
	\@for\@tempa:=%
		alpha,beta,gamma,delta,epsilon,zeta,eta,theta,iota,kappa,lambda,mu,nu,xi,%
		omicron,pi,rho,sigma,varsigma,tau,upsilon,phi,chi,psi,omega,digamma,%
		Alpha,Beta,Gamma,Delta,Epsilon,Zeta,Eta,Theta,Iota,Kappa,Lambda,Mu,Nu,Xi,%
		Omicron,Pi,Rho,Sigma,Tau,Upsilon,Phi,Chi,Psi,Omega,Digamma%
		\do{%
			\expandafter\let\csname\@tempa\expandafter\endcsname\csname\@tempa orig\endcsname%
		}%
}

\makeatother

%\makeatletter
%	\@for\@tempa:=%
%	%alpha,beta,gamma,delta,epsilon,zeta,eta,theta,iota,kappa,lambda,mu,nu,xi,%
%	%pi,rho,sigma,varsigma,tau,upsilon,phi,chi,psi,omega,digamma,%
%	Gamma,Delta,Theta,Lambda,Xi,Pi,Sigma,Upsilon,Phi,Psi,Omega%
%	\do{\expandafter\let\csname\@tempa\expandafter\endcsname\csname other\@tempa\endcsname}%
%\makeatother

% Fix the \bm command so that it also works properly in the sans mathversions
% ==>
\let \bmorig \bm
\renewcommand{\bm}[1]{%
	\IfInSansMode%
		\textbf{\mathversion{boldsans}\(#1\)}%
	\else%
		\bmorig{#1}%
	\fi\relax%
}
% <==
\renewcommand{\mathbf}[1]{\bm{#1}}
\renewcommand{\boldsymbol}[1]{\bm{#1}}
\newcommand{\mathbfit}[1]{\mathbf{\mathit{#1}}}
\renewcommand{\mathcal}[1]{\mathscr{#1}}

% Apply sansmath etc. automagically
% ==>
\newif\IfInSansMode
\newif\IfInBoldMode
\newif\IfInUpMode
\let \oldsf \sffamily
\renewcommand*{\sffamily}{%
	\oldsf\sansmath\InSansModetrue%
	\IfInBoldMode\mathversion{boldsans}\else\mathversion{sans}\fi\relax%
}
\let \oldbf \bfseries
\renewcommand*{\bfseries}{%
	\oldbf\InBoldModetrue%
	\IfInSansMode\sansmath\mathversion{boldsans}\else\mathversion{bold}\fi\relax%
}
\let \oldmd \mdseries
\renewcommand*{\mdseries}{%
	\oldmd\InBoldModefalse%
	\IfInSansMode\sansmath\mathversion{sans}\else\mathversion{normal}\fi\relax%
}
\let \oldnorm \normalfont
\renewcommand*{\normalfont}{%
	\oldnorm\InSansModefalse\InBoldModefalse\mathversion{normal}%
	\unsansmath%
}
\let \oldrm \rmfamily
\renewcommand*{\rmfamily}{%
	\oldrm\InSansModefalse%
	\IfInBoldMode\mathversion{bold}\else\mathversion{normal}\fi\relax%
	\unsansmath%
}
% <==

% Make \mathnormal obey the currently active \mathversion ==>
\let \mathnormalorig \mathnormal
\renewcommand{\mathnormal}[1]{%
	\IfInSansMode%
		\IfInBoldMode%
			\mathversion{boldsans}%
			{\textbf{\(#1\)}}%
		\else%
			\mathversion{sans}%
			{\textmd{\(#1\)}}%
		\fi\relax%	
	\else%
		\mathnormalorig{#1}%
	\fi\relax%
}
% <==

% Adjust \mathrm to the curretly active \mathversion.
% We set it up such that also in sansserif mode, \mathrm activates the serif font.
% ==>
\let \mathrmorig \mathrm
\renewcommand{\mathrm}[1]{%
	\IfInSansMode%
		{\textrm{%
			\IfInBoldMode%
				\mathversion{bold}%
				\(\mathrmorig{#1}\)%
			\else%
				\mathversion{normal}%
				\(\mathrmorig{#1}\)%
			\fi\relax%
		}}%
	\else%
		\mathrmorig{#1}%
	\fi\relax%
}
% <==

% Define \mathup to activate \upshape without switching to the serif font
% (in contrast to \mathrm)
% ==>
\newcommand{\mathup}[1]{%
	\IfInSansMode%
		{\textup{%
			\InUpModetrue%
			\IfInBoldMode%
				\mathversion{boldsansup}%
				\(#1\)%
			\else%
				\mathversion{sansup}%
				\(#1\)%
			\fi\relax%
		}}%
	\else%
		{\upgreekletters\mathrm{#1}\itgreekletters}%
	\fi\relax%
}
\newcommand{\mathbfup}[1]{%
	\IfInSansMode%
		{\mathbf{\mathup{#1}}}%
	\else%
		{\upgreekletters\mathbf{\mathrm{#1}}\itgreekletters}%
	\fi\relax%
}
% <==

%% If you would like to redefine \mathup also for the serif font:
%% ==>
%\renewcommand{\mathup}[1]{%
%	\IfInSansMode%
%		{\textup{%
%			\InUpModetrue%
%			\IfInBoldMode%
%				\mathversion{boldsansup}%
%				\(#1\)%
%			\else%
%				\mathversion{sansup}%
%				\(#1\)%
%			\fi\relax%
%		}}%
%	\else%
%		{\textup{%
%			\InUpModetrue%
%			\IfInBoldMode%
%				\mathversion{boldup}%
%				\(#1\)%
%			\else%
%				\mathversion{normalup}%
%				\(#1\)%
%			\fi\relax%
%		}}%
%	\fi\relax%
%}
%\renewcommand{\mathbfup}[1]{%
%	\IfInSansMode%
%		{\mathbf{\mathup{#1}}}%
%	\else%
%		{\upgreekletters\mathbf{\mathup{#1}}\itgreekletters}%
%	\fi\relax%
%}
%% <==

% Make the LaTeX-defined operators obey sansserif math
% ==>
\let \operatornameorig \operatorname
\renewcommand{\operatorname}[1]{%
	\operatornameorig{\mathup{#1}}%
}
\makeatletter
\@for\@tempa:=%
	arccos,arccot,arccsc,arcsec,arcsin,arctan,arg,cos,cosh,cot,coth,csc,%
	deg,det,dim,exp,gcd,hom,inf,ker,lg,lim,liminf,limsup,ln,log,max,min,%
	Pr,sec,sin,sinh,sup,tan,tanh%
	\do{%
		\expandafter\let\csname\@tempa\endcsname\relax%
	}%
\makeatother
\DeclareMathOperator {\arccos}{\mathup{arccos}}
\DeclareMathOperator {\arccot}{\mathup{arccot}}
\DeclareMathOperator {\arccsc}{\mathup{arccsc}}
\DeclareMathOperator {\arcsec}{\mathup{arcsec}}
\DeclareMathOperator {\arcsin}{\mathup{arcsin}}
\DeclareMathOperator {\arctan}{\mathup{arctan}}
\DeclareMathOperator {\arg}   {\mathup{arg}}
\DeclareMathOperator {\cos}   {\mathup{cos}}
\DeclareMathOperator {\cosh}  {\mathup{cosh}}
\DeclareMathOperator {\cot}   {\mathup{cot}}
\DeclareMathOperator {\coth}  {\mathup{coth}}
\DeclareMathOperator {\csc}   {\mathup{csc}}
\DeclareMathOperator {\deg}   {\mathup{deg}}
\DeclareMathOperator {\det}   {\mathup{det}}
\DeclareMathOperator {\dim}   {\mathup{dim}}
\DeclareMathOperator {\exp}   {\mathup{exp}}
\DeclareMathOperator {\gcd}   {\mathup{gcd}}
\DeclareMathOperator*{\hom}   {\mathup{hom}}
\DeclareMathOperator*{\inf}   {\mathup{inf}}
\DeclareMathOperator {\ker}   {\mathup{ker}}
\DeclareMathOperator {\lg}    {\mathup{lg}}
\DeclareMathOperator*{\lim}   {\mathup{lim}}
\DeclareMathOperator*{\liminf}{\mathup{lim\,inf}}
\DeclareMathOperator*{\limsup}{\mathup{lim\,sup}}
\DeclareMathOperator {\ln}    {\mathup{ln}}
\DeclareMathOperator {\log}   {\mathup{log}}
\DeclareMathOperator*{\max}   {\mathup{max}}
\DeclareMathOperator*{\min}   {\mathup{min}}
\DeclareMathOperator {\Pr}    {\mathup{Pr}}
\DeclareMathOperator {\sec}   {\mathup{sec}}
\DeclareMathOperator {\sin}   {\mathup{sin}}
\DeclareMathOperator {\sinh}  {\mathup{sinh}}
\DeclareMathOperator*{\sup}   {\mathup{sup}}
\DeclareMathOperator {\tan}   {\mathup{tan}}
\DeclareMathOperator {\tanh}  {\mathup{tanh}}
% <==

\AtBeginDocument{%
	\renewcommand{\euro}{\texteuro}%
}

% Allow for fine-grained scaling of font sizes
% ==>
\usepackage{relsize}
\renewcommand\RSpercentTolerance{1}
% Enabling slightly reduced font for CAPS:
\newcommand{\caps}[1]{\textscale{0.96}{\textls[35]{\MakeUppercase{#1}}}}
% <==




%%%%%%%%%%%%%%%%%%%%%%%%%%%%%%%%%%%%%%%%%%%%%%%%
%%  ADJUSTING THE DESIGN OF THE PRESENTATION  %%
%%%%%%%%%%%%%%%%%%%%%%%%%%%%%%%%%%%%%%%%%%%%%%%%


\setbeamerfont{title}{size=\LARGE, parent=structure}

% For convenience, let the ``paperheight'' of the slides be identical,
% no matter what the aspect ratio of the slides ==>
\makeatletter
% 4:3 aspect ratio:
\@ifclasswith{beamer}{aspectratio=43}{%
	\beamer@paperwidth 12.00cm%
	\beamer@paperheight 9.00cm%
	\setbeamerfont{frametitle}{size*={11}{14}, parent=structure}
	\setbeamerfont{title}{size*={14}{20}, parent=structure}
}{}
%% 14:9 aspect ratio: Nothing needs to be changed
%\@ifclasswith{beamer}{aspectratio=149}{%
%	\beamer@paperwidth 14.00cm%
%	\beamer@paperheight 9.00cm%
%}{}
%% 16:9 aspect ratio: Nothing needs to be changed
%\@ifclasswith{beamer}{aspectratio=169}{%
%	\beamer@paperwidth 16.00cm%
%	\beamer@paperheight 9.00cm%
%}{}
% 16:10 aspect ratio:
\@ifclasswith{beamer}{aspectratio=1610}{%
	\beamer@paperwidth 14.40cm%
	\beamer@paperheight 9.00cm%
}{}
\makeatother
% <==

% Margins ==>
\newcommand{\margintop}{12.5pt}
\newcommand{\marginleft}{27pt}
\newcommand{\margincenter}{24pt}
\newcommand{\marginright}{18pt}
\setbeamersize{text margin left=\marginleft}
\setbeamersize{text margin right=\marginright}
% <==

\setlength{\parskip}{\medskipamount}
	% Inserts some space between paragraphs

% Enable hyphenation on Beamer slides:
\usepackage{ragged2e}
\let \raggedright \RaggedRight
\sloppy
\hyphenpenalty=500

% Make frame titles and headlines bold
\usefonttheme{structurebold}
%\setbeamercolor{frametitle}{fg=black}
\setbeamerfont{subtitle}{size=\normalsize, series=\bfseries, parent=structure}
\setbeamerfont{author}{size=\large, series=\mdseries}
\setbeamerfont{institute}{size=\small, series=\mdseries, shape=\itshape}
%\setbeamercolor{institute}{fg=darkgray}
\setbeamerfont{date}{size=\normalsize, series=\bfseries, parent=structure}
%\setbeamercolor{date}{fg=darkgray}
\setbeamerfont{alerted text}{series=\bfseries}

%% Lorenz Götte's color scheme:
%\usecolortheme{whale}
%\definecolor{beamer@blendedblue}{rgb}{0.137,0.466,0.741}
%\setbeamercolor{structure}{fg=beamer@blendedblue}
%\definecolor{SpotColor}{rgb}{0.1,0.4,0.7} % Lorenz' Blue
\definecolor{SpotColor}{rgb}{0.00, 0.25, 0.55}
\definecolor{darkgray} {rgb}{0.40, 0.40, 0.40}
\definecolor{darkred}  {RGB}{204, 0,    0}
\definecolor{nicegreen}{RGB}{ 51, 204,  0}
\setbeamercolor{structure}{fg=SpotColor}
\setbeamercolor{alerted text}{fg=SpotColor}
\setbeamercolor{author in head/foot}{fg=white, bg=SpotColor}
\setbeamercolor{button}{bg=SpotColor,fg=white}

\newcommand{\highlight}[1]{\textcolor{SpotColor}{#1}}
\newcommand{\heading}[1]{%
	\par%
	\textbf{\usebeamerfont{structure}\usebeamercolor[fg]{frametitle}#1}%
	\par%
}
\newcommand{\sigstar}{\highlight{*}}

\renewcommand{\alert}[1]{\highlight{\textbf{#1}}}

\newlength{\rulelength}
\setlength{\rulelength}{\paperwidth - \marginleft - \marginright}
\newcommand{\sliderule}{%
	\textcolor{SpotColor}{\rule{\rulelength}{.35pt}}%
}

% Remove navigation symbols from the slide footer
\beamertemplatenavigationsymbolsempty

% Adjust layout of frametitle
% ==>
\setbeamertemplate{frametitle}{%
	\vspace{\margintop}\vspace{-5pt}%
	{\large\usebeamercolor{frametitle}\usebeamerfont{frametitle}\insertframetitle\\[-1.2ex]}%
	\sliderule%
}
% <==

% Add a slide number (and [short]title) to the footer of each slide
% ==>
\BeforeBeginEnvironment{frame}{%
	\setbeamertemplate{footline}{\vskip-15pt
		\textcolor{SpotColor}{%
			%\fontseries{rm}\selectfont%
			\hspace{\marginleft}%
			\sliderule \vspace{1.1ex} \linebreak
			\mbox{\hspace{\marginleft}}%
			\textmd{\insertshortauthor: ``\insertshorttitle''}
			\hfill%
			\insertframenumber/\inserttotalframenumber%
			\hspace{\marginright}%
		}%
		\vspace{12pt}%
	}
}
% <==

% See https://tex.stackexchange.com/questions/427257/how-to-remove-footer-for-specific-type-of-slides-frames
% ==>
\makeatletter
\define@key{beamerframe}{standout}[true]{%
	\setbeamertemplate{footline}{\vskip-15pt
		\textcolor{SpotColor}{%
			%\fontseries{rm}\selectfont%
			\hspace{\marginleft}%
			\sliderule \vspace{1.1ex} \linebreak
			\mbox{\hspace{\marginleft}}%
			\textmd{\phantom{\insertshortauthor: ``\insertshorttitle''}}
			\hfill%
			\phantom{\insertframenumber/\inserttotalframenumber}%
			\hspace{\marginright}%
		}%
		\vspace{12pt}%
	}%
}
\makeatother
% <==

\frenchspacing  % Prevent increased whitespace after periods and colons

\setbeamertemplate{button}{%
	\tikz
		\node[
			inner xsep=3pt,
			inner ysep=2pt,
			draw=structure!100,
			fill=structure!100,
			rounded corners=1.5pt
		]{\raisebox{1.5pt}{\usebeamerfont{structure}\insertbuttontext}};%
}

%\renewcommand{\texteuro}{\fontencoding{TS1}\selectfont\char"BF\fontencoding{T1}\selectfont}
%\newcommand{\euro}{\texteuro}

% Adding framenumbers automatically to the frametitles
\newcommand{\pageinsection}{\number\numexpr\insertpagenumber-\insertsectionstartpage+1}
% From https://tex.stackexchange.com/questions/308343/how-to-create-mini-sections-mini-subsections-and-mini-frames-in-beamer-presenta
% ==>
%\usepackage{etoolbox}  % Automatically loaded by beamer
\makeatletter
\newcount\beamer@sectionstartframe
\beamer@sectionstartframe=1
\apptocmd{\beamer@section}{%
	\addtocontents{nav}{\protect\headcommand{%
		\protect\beamer@sectionframes{\the\beamer@sectionstartframe}{\the\c@framenumber}}}%
}{}{}
\apptocmd{\beamer@section}{%
	\beamer@sectionstartframe=\c@framenumber\advance\beamer@sectionstartframe by1\relax%
}{}{}
\AtEndDocument{%
	\immediate\write\@auxout{\string\@writefile{nav}%
		{\noexpand\headcommand{\noexpand\beamer@sectionframes{\the\beamer@sectionstartframe}{\the\c@framenumber}}}}%
}{}{}
\def\beamer@startframeofsection{1}
\def\beamer@endframeofsection{1}
\def\beamer@sectionframes#1#2{%
	\ifnum\c@framenumber<#1%
	\else%
		\ifnum\c@framenumber>#2%
		\else%
			\gdef\beamer@startframeofsection{#1}%
			\gdef\beamer@endframeofsection{#2}%
		\fi%
	\fi%
}
\newcommand\insertsectionstartframe{\beamer@startframeofsection}
\newcommand\insertsectionendframe{\beamer@endframeofsection}
\makeatother
% <==
\newcommand{\frameinsection}{\number\numexpr\insertframenumber-\insertsectionstartframe+1}
% https://tex.stackexchange.com/questions/228684/two-counters-for-beamer-presentations
%\newcommand{\titleprefix}{\insertsection~{\pageinsection}}
\newcommand{\titleprefix}{\insertsection~{\frameinsection}}
% <==

% Continuation counter for frames with the ``allowframebreaks'' option.
% ==>
%% Inspired by https://tex.stackexchange.com/questions/275044/how-do-i-insert-the-total-continuation-count-in-the-allowframbreaks-frame-title:
%\newcounter{totalcontinuationcount}
%\makeatletter
%\setbeamertemplate{frametitle continuation}{%
%	\setcounter{totalcontinuationcount}{\beamer@endpageofframe}%
%	\addtocounter{totalcontinuationcount}{1}%
%	\addtocounter{totalcontinuationcount}{-\beamer@startpageofframe}%
%	\ifnum \value{totalcontinuationcount} > 1
%		\textmd{(\insertcontinuationcount/\arabic{totalcontinuationcount})}%
%	\fi
%}
%\makeatother
% More elegant version based on https://github.com/josephwright/beamer/issues/423#issuecomment-456494500:
\makeatletter
\defbeamertemplate*{frametitle continuation}{only if multiple}{%
	\ifnum \numexpr\beamer@endpageofframe+1-\beamer@startpageofframe\relax > 1
		\textmd{(%
			\insertcontinuationcount/%
			\the\numexpr\beamer@endpageofframe+1-\beamer@startpageofframe%
		)}%
	\fi%
}
\makeatother
% <==



%%%%%%%%%%%%%%%%%%%%%%%%%%%%%%%%%
%%  LAYOUT OF THE TITLE SLIDE  %%
%%%%%%%%%%%%%%%%%%%%%%%%%%%%%%%%%


% Make the title page left-aligned
\setbeamertemplate{title page}[default][left]

\newcommand\Wider[2][0pt]{%
	\makebox[\linewidth][c]{%
		\hspace{-\marginleft}\hspace{\marginright}%
		\begin{minipage}{\dimexpr\textwidth+\marginleft-\marginright\relax}
			\raggedright%
			\smallskip%
			#2
		\end{minipage}%
	}%
}

\makeatletter
\renewcommand{\beamer@insttitle}[1]{\highlight{\textsuperscript{\kern.75pt \textit{#1}}}}
\renewcommand{\beamer@instinst}[1]{\beamer@insttitle{#1}\ignorespaces}
\renewcommand{\beamer@andinst}{\\[0.33\baselineskip]}
\makeatother
\makeatother




%%%%%%%%%%%%%%%%%%%%%%%%%%%%%%%%%%%%%%%
%%  LAYOUT OF THE TABLE OF CONTENTS  %%
%%%%%%%%%%%%%%%%%%%%%%%%%%%%%%%%%%%%%%%


\useinnertheme{circles}
\setbeamertemplate{section in toc}{%
	\leavevmode\leftskip=0ex%
	\llap{%
		\usebeamerfont*{section number projected}%
		\usebeamercolor{section number projected}%
		\begin{pgfpicture}{-1ex}{0ex}{1ex}{2ex}
			\color{bg}
			\pgfpathcircle{\pgfpoint{0pt}{.75ex}}{1.2ex}
			\pgfusepath{fill}
			\pgftext[base]{\color{fg}\raisebox{0.07ex}{%
					% \addfontfeatures{Numbers={Lining, Monospaced}}%
					\usebeamerfont{structure}%
					\small\inserttocsectionnumber}%
			}
		\end{pgfpicture}\kern1.5ex%
	}%
	\usebeamerfont{normal text}%
	\inserttocsection\par%
}
\setbeamertemplate{subsection in toc}{%
	\leavevmode\leftskip=2em$\bullet$\hskip1em\inserttocsubsection\par%
}

\makeatletter
\patchcmd{\beamer@sectionintoc}
	{\vfill}
	{\medskip}
	{}
	{}
\makeatother




%%%%%%%%%%%%%
%%  LISTS  %%
%%%%%%%%%%%%%


\AtBeginEnvironment{itemize}{\vspace{-\parskip}\setlength{\labelsep}{0.55em}}
\AtBeginEnvironment{enumerate}{\vspace{-\parskip}\setlength{\labelsep}{0.5em}}

% Adjust spacing of lists ==>
% Unfortunately, the ``enumitem'' package is incompatible with the ``beamer'' class.
% Hence, we need to adjust left margins etc. manually:
\setlength{\leftmargini}{0pt}
\setlength{\leftmarginii}{15pt}
\setlength{\leftmarginiii}{18pt}
%\setlength{\rightmargin}{0in}
%\setlength{\itemindent}{0in}
%\usepackage{enumitem}
%\setlist[enumerate, 1]{
%	leftmargin=-\parindent, listparindent=\parindent, labelsep=0.42\parindent, itemsep=\smallskipamount, parsep=0pt
%}
%
%\setlist[itemize, 1]{
%	leftmargin=0pt, listparindent=\parindent, labelsep=0.75em, itemsep=\smallskipamount, parsep=0pt, topsep=1.2ex, label=\usebeamerfont*{itemize item}	\usebeamercolor[fg]{itemize item} \usebeamertemplate{itemize item}, before={\RaggedRight \hyphenpenalty=1000}
%}
%\setlist[itemize, 2]{
%	leftmargin=1.2em, listparindent=\parindent, labelsep=0.6em, itemsep=\smallskipamount, parsep=0pt, topsep=0.6ex, label=\usebeamercolor[fg]{itemize item} --, before={\RaggedRight \hyphenpenalty=1000}
%}
% <==

% ``Medium'' as the bold series:
%\makeatletter
%\def\bfseries@sf{mb}
%\makeatother

%\useinnertheme{circles}

\setbeamertemplate{enumerate items}     [default]
\setbeamertemplate{enumerate subitem}   {\alph{enumii}.}
\setbeamertemplate{enumerate subsubitem}{\roman{enumiii}.}

\setbeamertemplate{itemize item}        {\raisebox{-0.5pt}{\textrm{\textbf{\textbullet}}}}
\setbeamertemplate{itemize subitem}     {\usebeamerfont{normal text}--}
\setbeamertemplate{itemize subsubitem}  {\raisebox{1.5pt}{\tiny$\blacktriangleright$}}

\setbeamerfont*{itemize/enumerate body}{size=\normalsize}
\setbeamerfont*{itemize/enumerate subbody}{parent=itemize/enumerate body}
\setbeamerfont*{itemize/enumerate subsubbody}{parent=itemize/enumerate body}

\renewenvironment{quote}{%
	\list{}{\leftmargin\leftmarginii \rightmargin\leftmarginii}%
	\RaggedLeft
	\itshape%
	\item\relax%
}
{\endlist}




%%%%%%%%%%%%%%%%%%%%%%%%%%%%%%%%%
%%  APPENDIX-RELATED SETTINGS  %%
%%%%%%%%%%%%%%%%%%%%%%%%%%%%%%%%%


\usepackage{appendixnumberbeamer}
\AtBeginEnvironment{appendix}{%
	\let \insertframenumberorig \insertframenumber
	\renewcommand{\insertframenumber}{Appendix Slide \insertframenumberorig}
	%\let \inserttotalframenumberorig \inserttotalframenumber
	%\renewcommand{\inserttotalframenumber}{A-\inserttotalframenumberorig}
}




%%%%%%%%%%%%%%%%%%%%%%%%%%
%%  FOR DEBUGGING ONLY  %%
%%%%%%%%%%%%%%%%%%%%%%%%%%


\usepackage{blindtext}
\blindmathtrue

% An auxiliary command to display the current font settings ==>
\makeatletter
\newcommand{\showfont}{%
	\textit{encoding:} \f@encoding{},
	\textit{family:} \f@family{},
	\textit{series:} \f@series{},
	\textit{shape:} \f@shape{},
	\textit{size:} \f@size{}
}
\makeatother
% <==