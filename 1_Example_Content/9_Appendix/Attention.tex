% !TeX program = pdflatex
% !BIB program = biber



\section{Put More Complicated Derivations and Proofs Here}
\label{sec:app:derivations}

\subsection{Appendix Subsection}
\label{sec:app:derivations:one}

\blindmathtrue
\Blindtext[4]

\subsection{Salience}
\label{sec:app:salience}

Salience theory \citep{Bordalo2012, Bordalo2013} represents a~behavioral model according to which the most distinctive features of the available alternatives receive a~particularly large share of attention and are therefore over-weighted. More precisely, a~particular attribute out of all attributes of an~alternative becomes the more salient, the more it differs from that attribute's average level over all available alternatives.

Formally, alternatives are assumed to be uniquely characterized by the values they take in ${T \geq 1}$ attributes (or, ``dimensions''). Utility is assumed to be additively separable in attributes, and salience attaches a~decision weight to each attribute of each good which indicates how salient the respective attribute is for that good. Suppose an agent chooses one alternative from some finite choice set $\CS$. Let $t$ index the $T$~different attributes, and let $k$ index the $K$~available alternatives. Let $u_t(\cdot)$ denote the function which assigns utility to values in dimension~$t$.
Denote by $a^k_{t}$ the level of attribute $t$ of good $k$ and define ${u^k_t \coloneqq u_t(a^k_t)}$ as the utility that dimension $t$ of good~$k$ yields. Let $\overline{u}_t$ be the average utility level, across all $K$~goods, of dimension~$t$. The salience of each dimension of good~$k$ is determined by a~symmetric and continuous salience function ${\sigma(\cdot, \cdot)}$ that satisfies the following two properties:
\begin{enumerate}
\item \emph{Ordering.} Let ${\mu \coloneqq \mathrm{sgn}(u^k_t - \overline{u}_t)}$. Then for any ${\epsilon,\epsilon' \geq 0}$ with ${\epsilon + \epsilon' > 0}$, it holds that~
\begin{equation}
	\sigma(u^k_t + \mu\,\epsilon, \overline{u}_t - \mu\,\epsilon') > \sigma (u^k_t,\overline{u}_t).
\end{equation}
\item \emph{Diminishing sensitivity.} For any ${u^k_t, \overline{u}_t \geq 0}$ and all ${\epsilon > 0}$, it holds that
\begin{equation}
	\sigma(u^k_t + \epsilon, \overline{u}_t + \epsilon) < \sigma(u^k_t,\overline{u}_t).
\end{equation}
\end{enumerate}

Following the smooth salience characterization proposed in \cite[p.\,1255]{Bordalo2012}, each dimension $t$ of good $k$ receives weight $\Delta^{-\sigma(u^k_t, \overline{u}_t)}$, where ${\Delta \in (0, 1]}$ is a~constant that captures an~agent's susceptibility to salience. ${\Delta = 1}$ gives rise to a~rational decision maker, and the smaller $\Delta$, the stronger is the salience bias. We call an~agent with ${\Delta < 1}$ a~salient thinker.