% !TeX program = pdflatex
% !TeX TS-program = pdflatex
% !TeX TXS-program:compile = txs:///pdflatex/

% !BIB program = biber
% !BIB TS-program = biber
% !TeX TXS-program:bibliography = txs:///biber




%%%%%%%%%%%%%%%%
%%  PREAMBLE  %%
%%%%%%%%%%%%%%%%


\newcommand{\thesistitle}{%
	A\unskip\nobreak\ Template for Bachelor's\unskip\nobreak\ Theses and\unskip\nobreak\ Master's\unskip\nobreak\ Theses%
}
\newcommand{\thesisauthor} {Lou~E. V{\"u}{\'i}-T{\o}n}
\newcommand{\studentID}    {7654321}
\newcommand{\thesistype}   {BA}  % BA for bachelor's, MA for master's thesis
\newcommand{\advisorname}  {Prof.~Dr. Vae-Ree Smart}
\newcommand{\advisorgender}{m}  % f for female, m for male

% From https://www.vwlpamt.uni-bonn.de/pruefungsamt/pdfs/formulare/bachelorarbeitsmerkblatt-formalia and
% https://www.vwlpamt.uni-bonn.de/pruefungsamt/pdfs/formulare/masterarbeitsmerkblatt-formalia:
% ``Die Bachelorarbeit darf höchstens einen Textumfang, inklusive Grafiken, Bilder, Formeln etc.,
% von 20 [Master: 40] einseitig bedruckten DIN-A4-Seiten haben, wobei Inhaltsverzeichnis, Anhang und Literaturverzeichnis
% nicht mitzählen.''
% ``Es ist Papier im DIN-A4-Format zu benutzen. Verwenden Sie den Schrifttyp Times New Roman
% oder eine Schriftart der gleichen Kategorie (Serifen), die Schriftgröße 12, anderthalbfachen Zeilenabstand
% und Blocksatz.''
% This makes an incredibly ugly layout, :-( but we follow the guidelines nevertheless.
% =>>
\documentclass[12pt, a4paper, oneside, full]{article}
\usepackage[left=3cm, top=2cm, right=2cm, bottom=2cm]{geometry}
	% ``Die Seitenränder sollen links 3 cm sowie rechts, oben und unten 2 cm betragen.''
\newcommand{\linesperpagedesired}{35}
	% In Microsoft Word, A4 paper with margins of 2 cm at the top and bottom and the font set to
	% 12-pt Times New Roman, a line spacing of 1.5 yields 35 lines of text per page.
% <==

% !TeX TXS-program:compile = txs:///pdflatex/
% !TeX TS-program = pdflatex
% !TeX TXS-program:bibliography = txs:///biber
% !BIB program = biber




%%%%%%%%%%%%%%%%%%%%%%%%%%%%%%%%%%%%%%%%%
%%  FUNDAMENTAL PACKAGES AND COMMANDS  %%
%%%%%%%%%%%%%%%%%%%%%%%%%%%%%%%%%%%%%%%%%


\usepackage{ifxetex}  % Detect if engine is XeTeX/XeLaTeX

\usepackage{ifthen}

\usepackage[utf8]{inputenc}  % so that umlauts can be input without having to use TeX code

\ifxetex
	\usepackage[euenc]{fontspec}
\else
	\usepackage[LGR, T1]{fontenc}  % LGR needed for sansserif math
\fi

\usepackage[ngerman, american, USenglish]{babel}  % German and US English hyphenation and quotation marks
\selectlanguage{USenglish}
\usepackage[ngerman, USenglish]{isodate}

\usepackage[babel, german=quotes]{csquotes}  % Needed for correct German quotes via BibLaTeX's \mkbibquote{...}
%\usepackage[babel, german=guillemets]{csquotes}  % Needed for correct German quotes via BibLaTeX's \mkbibquote{...}

\usepackage{calc}  % Enables calculations for lengths; provides, e.g., \widthof{text}
\usepackage{fp}  % Enables calculations in LaTeX
% \usepackage[nomessages]{fp}

\usepackage{etoolbox}  % Enables manipulating LaTeX commmands via \preto, \appto, \patchcmd, etc.
\usepackage{xpatch}  % Enables manipulating LaTeX commmands via \xpatchcmd etc.
\usepackage{letltxmacro}
\usepackage{xparse}

%\usepackage{geometry}  % See geometry.pdf to learn the layout options. There are lots.
%\usepackage[mathscr]{eucal}
%\geometry{a4paper, top=20mm, bottom=20mm, right=20mm, left=30mm}
%\geometry{landscape}  % Activate for for rotated page geometry

\usepackage{ragged2e}
% Provides, among others, the \RaggedRight environment, which is \raggedright but with hyphenation enabled.
%\renewcommand{\raggedright}{\RaggedRight}

\usepackage{import}	 % To allow for relative paths in nested \input's (\import's)

% \usepackage{float} % damit die figure da ist wo sie sein soll
\usepackage{placeins}  % Improve placing of floats (figures, tables), provides \FloatBarrier

% DO NOT USE \usepackage{amssymb}!
% The AMS symbols are included in the mathdesign package or the fourier package.
\usepackage{amsmath}
\MakeRobust{\eqref}
	% See https://tex.stackexchange.com/questions/61764/eqref-in-captions-with-mathtools
\renewcommand{\eqref}[1]{(\ref{#1})}  % Necessary to make the font switch from serif to sansserif where required.
\usepackage{amsthm}	% provides \newtheoremstyle
\usepackage{mathtools}
%\mathtoolsset{centercolon}
	% This makes the compilation fail in combination with tikz. See
	% https://tex.stackexchange.com/questions/89467/why-does-pdftex-hang-on-this-file.
% Inspired by https://tex.stackexchange.com/questions/251460/how-to-put-symbols-of-equal-size-on-top-of-each-other
\newcommand{\succeqq}{%
  \mathrel{%
    \vcenter{\offinterlineskip
      \ialign{##\cr$\succ$\cr\noalign{\kern 1pt}$=$\cr}%
    }%
  }%
}
\newcommand{\nsucceqq}{\mathrel{\not\succeqq}}

\usepackage[
	colorlinks=true,
	linkcolor=UBonnBlue,
	citecolor=UBonnBlue,
	filecolor=black,
	urlcolor=UBonnBlue,
	bookmarks=true,
	bookmarksnumbered=true,
	bookmarksopenlevel=2,
	pdfstartview=Fit,
	pdfpagelayout=SinglePage,
	plainpages=false,
	pdfpagelabels=true
]{hyperref}
\urlstyle{same}   % Sets URLs in the text font instead of the typewriter font
\Urlmuskip = 0mu\relax  % Prevent additional whitespace before/after breakable characters in URLs
%\setlength{\Urlmuskip}{0mu}
\newcommand{\email}[1]{\href{mailto:#1}{\nolinkurl{#1}}}
% Change ``([sub]sub)section'' to ``Section'' in \autoref
% and add na
% ==>
\addto\extrasUSenglish{%
	\renewcommand{\chapterautorefname}{Chapter}%  instead of ``chapter''
	\renewcommand{\sectionautorefname}{Section}%  instead of ``section''
	\renewcommand{\subsectionautorefname}{Section}%  instead of ``subsection''
	\renewcommand{\subsubsectionautorefname}{Section}%  instead of ``subsubsection''
}
\addto\extrasamerican{%
	\renewcommand{\chapterautorefname}{Chapter}%  instead of ``chapter''
	\renewcommand{\sectionautorefname}{Section}%  instead of ``section''
	\renewcommand{\subsectionautorefname}{Section}%  instead of ``subsection''
	\renewcommand{\subsubsectionautorefname}{Section}%  instead of ``subsubsection''
}
\addto\extrasngerman{%
	\renewcommand{\subsectionautorefname}{Abschnitt}%  instead of ``Unterabschnitt''
	\renewcommand{\subsubsectionautorefname}{Abschnitt}%  instead of ``Unterunterabschnitt''
}
\newcommand*{\Appendixautorefname}{Appendix}
	% See https://tex.stackexchange.com/questions/207744/no-autoref-name-for-appendix
\newcommand*{\hypothesisautorefname}{Hypothesis}
\newcommand*{\resultautorefname}{Result}
% <==
% Enclose the back references in the bibliography to the pages on which a reference is cited in square brackets (Econometrica style):
% (only applicable if using BibTeX instead of BibLaTeX)
%\let \backrefold \backref
%\renewcommand*{\backref}[1]{[\backrefold{#1}]}

% \usepackage{cleveref}	% Provides \cref{...} etc. for flexible referencing.

\usepackage{verbatim}

%% Command to suppress text:
%% See https://tex.stackexchange.com/questions/97347/selectively-suppress-generation-of-typeset-output
%% ==>
%\makeatletter
%\font\dummyft@=dummy \relax
%\def\suppress{%
%	\begingroup\par
%	\parskip\z@
%	\offinterlineskip
%	\baselineskip=\z@skip
%	\lineskip=\z@skip
%	\lineskiplimit=\maxdimen
%	\dummyft@
%	\count@\sixt@@n
%	\loop\ifnum\count@ >\z@
%	\advance\count@\m@ne
%	\textfont\count@\dummyft@
%	\scriptfont\count@\dummyft@
%	\scriptscriptfont\count@\dummyft@
%	\repeat
%	\let\selectfont\relax
%	\let\mathversion\@gobble
%	\let\getanddefine@fonts\@gobbletwo
%	\tracinglostchars\z@
%	\frenchspacing
%	\hbadness\@M}
%\def\endsuppress{\par\endgroup}
%\makeatother
%% <==




%%%%%%%%%%%%%%%%%%%%%%%%%%%%%%%%%
%%  GRAPHICS-RELATED PACKAGES  %%
%%%%%%%%%%%%%%%%%%%%%%%%%%%%%%%%%


\usepackage{graphicx}
\DeclareGraphicsRule{.tif}{png}{.png}{`convert #1 `dirname #1`/`basename #1 .tif`.png}

\usepackage{epstopdf}  % Has to be loaded after graphic{s,x}

\usepackage[table]{xcolor}
\definecolor{UBonnBlue}   {RGB}{0007,0082,0154}
\definecolor{darkblue}    {rgb}{0.00,0.20,0.40}
\definecolor{darkred}     {rgb}{0.80,0.00,0.00}
\colorlet   {darkred25}   {darkred!25!white}
\definecolor{customgreen} {rgb}{0.15,0.55,0.00}
\definecolor{custompurple}{rgb}{0.15,0.00,0.75}

\usepackage{pgf, pgfarrows, pgfnodes, pgfshade}
\usepackage{pgfplots}

\usepackage{tikz}
\usetikzlibrary{mindmap, trees, patterns}

\usepackage{pdflscape}
% To set single pages in landscape ortientation

\usepackage{afterpage}
% To wrap text around landscape-oriented pages



%%%%%%%%%%%%%%%%%%%%%%%%%%%%%%%%
%%  ADVANCED TEXT FORMATTING  %%
%%%%%%%%%%%%%%%%%%%%%%%%%%%%%%%%


\usepackage[normalem]{ulem}  % Do NOT change \emph to \uline!!!

\usepackage[full]{textcomp}  % ``full'' option tequired some packages (e.g., ``newtxtext'')
\usepackage{xfrac}	% Provides \sfrac; loads textcomp (without ``full'' option)

\usepackage{soul}  % Provides a~highlighting command: \hl{...}

% Allow for fine-grained scaling of font sizes
% ==>
\usepackage{relsize}
\renewcommand\RSpercentTolerance{1}
% Enabling slightly reduced font for CAPS:
\ifxetex
	\newcommand{\caps}[1]{\textscale{0.96}{\addfontfeature{LetterSpace=5}\MakeUppercase{#1}}}
\else
	\newcommand{\caps}[1]{\textscale{0.96}{\textls[35]{\MakeUppercase{#1}}}}
\fi
% <==

\frenchspacing	% Prevent excessively large whitespace after periods
\sloppy




%%%%%%%%%%%%%%%%%%%%%%%%%%%%%%%%%%%%
%%  COMMANDS FOR TROUBLESHOOTING  %%
%%%%%%%%%%%%%%%%%%%%%%%%%%%%%%%%%%%%


\usepackage{printlen}  % Enables outputting the current values of lengths

\usepackage[math]{blindtext}
\makeatletter
\def\blindtext@american{}
\renewcommand{\blindmathpaper}{%
	\blindtext
	\blindtext@formula\par
	\blindtext
	\blindtext@formula
	\blindtext
	\blindtext@formula\par
	\blindtext
	\blindtext@formula
	\blindtext
	\blindtext@formula\par
	\blindtext\relax%
}
\makeatother
\setcounter{blindtext}{1}
\setcounter{Blindtext}{1}
% The ``blindtext'' package does not recognize ``USenglish'' as identical to ``american''.
% Fix this -->
\LetLtxMacro{\blindtextblindtext}{\blindtext}
\LetLtxMacro{\blindtextBlindlist}{\Blindlist}
\LetLtxMacro{\blindtextBlindtext}{\Blindtext}
\RenewDocumentCommand{\blindtext}{O{\value{blindtext}}}{%
	\begingroup%
	\iflanguage{USenglish}{\selectlanguage{american}}{}%
	\blindtextblindtext[#1]%
	\endgroup%
}
\RenewDocumentCommand{\Blindtext}{O{\value{blindtext}} O{\value{Blindtext}}}{%
	\begingroup%
	\iflanguage{USenglish}{\selectlanguage{american}}{}%
	\blindtextBlindtext[#1][#2]%
	\endgroup%
}
\RenewDocumentCommand{\Blindlist}{m O{\value{blindlist}}}{%
	\begingroup%
	\iflanguage{USenglish}{\selectlanguage{american}}{}%
	\blindtextBlindlist{#1}[#2]%
	\endgroup%
}
% Based on https://tex.stackexchange.com/questions/299954/styling-blindtext-and-blindtext-aka-renewcommand-with-optional-arguments
% <==

%\usepackage{showframe}
%\renewcommand*\ShowFrameColor{\color{magenta}}
%\usepackage[pagewise]{lineno}
%\addtolength{\linenumbersep}{7.5pt}
%\renewcommand{\linenumberfont}{\sffamily\tiny\color{gray}}
%\linenumbers
% An auxiliary command to display the current font settings -->
\makeatletter
\newcommand{\showfont}{{%
	\color{magenta}
	\textit{Encoding:} \f@encoding{},
	\textit{family:}   \f@family{},
	\textit{series:}   \f@series{},
	\textit{shape:}    \f@shape{},
	\textit{size:}     \f@size{}.
}}
\makeatother
\newcommand{\showfamily}{\f@family{}}
% <--

\makeatletter
\newcommand*{\checkgreekletters}{%
	\@for\@tempa:=%
	alpha,beta,gamma,delta,epsilon,zeta,eta,theta,iota,kappa,lambda,mu,nu,xi,%
	omicron,pi,rho,sigma,varsigma,tau,upsilon,phi,chi,psi,omega,digamma,%
	Alpha,Beta,Gamma,Delta,Epsilon,Zeta,Eta,Theta,Iota,Kappa,Lambda,Mu,Nu,Xi,%
	Omicron,Pi,Rho,Sigma,Tau,Upsilon,Phi,Chi,Psi,Omega,Digamma%
	\do{$\csname\@tempa\endcsname,$ }%
}
\makeatother

\usepackage{fonttable}
% Color slot numbers in \xfonttable gray instead of black -->
\makeatletter
\renewcommand*\f@placedecimal[2]{#1\ {\color{gray}\tiny #2}}
\renewcommand*{\f@toct}[1]{\hbox{\color{gray}\rmfamily\'{}\kern-.2em\itshape#1\/\kern.05em}} % octal constant
\renewcommand*{\f@thex}[1]{\hbox{\color{gray}\rmfamily\H{}\ttfamily#1}} % hexadecimal constant
\makeatother
% <--
% !TeX program = pdflatex
% !TeX TXS-program:compile = txs:///pdflatex/
% !TeX TS-program = pdflatex
% !BIB program = biber
% !TeX TXS-program:bibliography = txs:///biber




%%%%%%%%%%%%%%%%%%%%%%%%%%%%%%%%%%%%%%%%%%%%%%%%%
%%  GRID-BASED TYPESETTING AS FAR AS POSSIBLE  %%
%%%%%%%%%%%%%%%%%%%%%%%%%%%%%%%%%%%%%%%%%%%%%%%%%


\flushbottom

\newlength{\origbaselineskip}
\setlength{\origbaselineskip}{\baselineskip}

\ifdef{\linesperpagedesired}
	{}% If already defined, do nothing.
	{\newcommand{\linesperpagedesired}{42}}
	% Number of text lines per page, see https://en.wikipedia.org/wiki/Gutenberg_Bible

\newcommand{\linesperpagecurrent}{\numexpr (\textheight - \topskip) / \baselineskip + 1 \relax}  % Integer division!
\makeatletter
%\newcommand{\newbaselinestretch}{\strip@pt\dimexpr (\linesperpagecurrent pt - 1pt) / (\linesperpagedesired - 1)}
\newcommand{\newbaselinestretch}{1.1}
\makeatother
\linespread{\newbaselinestretch}
\newlength{\newbaselineskip}
\setlength{\newbaselineskip}{
	\dimexpr (\textheight - \topskip) / (\linesperpagedesired - 1)
}
\setlength{\textheight}{\dimexpr \numexpr \linesperpagedesired - 1 \relax \newbaselineskip + \topskip}  % to prevent too small \textheigtht due to rounding errors
\newlength{\newparindent}
\setlength{\newparindent}{1.15\newbaselineskip}%
\AtBeginDocument{%
	\setlength{\baselineskip}{\newbaselineskip}%
	\setlength{\parindent}{\newparindent}
	\setlength{\lineskiplimit}{0pt}
		% Prevents increased line spacing in response to spacious inline formulas.
}
\newlength{\abovedisplayauxskip}
\newlength{\belowdisplayauxskip}
\setlength{\abovedisplayauxskip}{0pt plus 0.5\baselineskip}
\setlength{\belowdisplayauxskip}{0pt plus 0.5\baselineskip}
\newcommand{\predisplaycmd}{%
	\ifvmode\else\unskip\fi%
	\nopagebreak[2]%
	\vspace{\abovedisplayauxskip}%
}
\makeatletter
\def\@itemize@name{itemize}
\def\@enumerate@name{enumerate}
\def\@description@name{description}
\newcommand{\postdisplaycmd}{%
	\ifx\@currenvir\@itemize@name%
	\else%
		\ifx\@currenvir\@enumerate@name%
		\else%
			\ifx\@currenvir\@description@name%
			\else%
				\vskip\belowdisplayauxskip%
			\fi%
		\fi%
	\fi%
	\noindent%
}
\makeatother
\AtBeginEnvironment {align}      {\predisplaycmd}
\AfterEndEnvironment{align}      {\postdisplaycmd}
\AtBeginEnvironment {align*}     {\predisplaycmd}
\AfterEndEnvironment{align*}     {\postdisplaycmd}
\AtBeginEnvironment {alignat}    {\predisplaycmd}
\AfterEndEnvironment{alignat}    {\postdisplaycmd}
\AtBeginEnvironment {alignat*}   {\predisplaycmd}
\AfterEndEnvironment{alignat*}   {\postdisplaycmd}
\AtBeginEnvironment {displaymath}{\predisplaycmd}
\AfterEndEnvironment{displaymath}{\postdisplaycmd}
\AtBeginEnvironment {eqnarray}   {\predisplaycmd}
\AfterEndEnvironment{eqnarray}   {\postdisplaycmd}
\AtBeginEnvironment {eqnarray*}  {\predisplaycmd}
\AfterEndEnvironment{eqnarray*}  {\postdisplaycmd}
\AtBeginEnvironment {equation}   {\predisplaycmd}
\AfterEndEnvironment{equation}   {\postdisplaycmd}
\AtBeginEnvironment {equation*}  {\predisplaycmd}
\AfterEndEnvironment{equation*}  {\postdisplaycmd}
\AtBeginEnvironment {flalign}    {\predisplaycmd}
\AfterEndEnvironment{flalign}    {\postdisplaycmd}
\AtBeginEnvironment {flalign*}   {\predisplaycmd}
\AfterEndEnvironment{flalign*}   {\postdisplaycmd}
\AtBeginEnvironment {gather}     {\predisplaycmd}
\AfterEndEnvironment{gather}     {\postdisplaycmd}
\AtBeginEnvironment {gather*}    {\predisplaycmd}
\AfterEndEnvironment{gather*}    {\postdisplaycmd}
\AtBeginEnvironment {multiline}  {\predisplaycmd}
\AfterEndEnvironment{multiline}  {\postdisplaycmd}
\AtBeginEnvironment {multiline*} {\predisplaycmd}
\AfterEndEnvironment{multiline*} {\postdisplaycmd}
% \setlength{\topskip}{\newbaselineskip pt}

\setlength{\parskip}{0pt}  % Prevents whitespace from being added between paragraphs
%\setlength{\parskip}{0pt plus 0.0001pt}
% Add whitepsace between paragraphs only in emergency cases




%%%%%%%%%%%%%%%%%%%%%%%%%%%%%%%%%
%%  FORMATTING OF MATHEMATICS  %%
%%%%%%%%%%%%%%%%%%%%%%%%%%%%%%%%%


%\setlength{\mathindent}{\newparindent}  % Only for option ``fleqn''

% Prevent stretching of spaces around operators in inline formulas; see ==>
%  http://tex.stackexchange.com/questions/83746/keeping-the-distance-between-mathematical-symbols-consistent
% \thinmuskip=3.0mu (already without glue)
\medmuskip=1\medmuskip	% Formerly 4.0mu plus 2.0mu minus 4.0mu -> 4.0mu
\thickmuskip=1\thickmuskip	% Formerly 5.0mu plus 5.0mu -> 5.0mu
% <==

% Increase spacing around relational symbols (<, =, >, \le, \ge, \equiv; +, -, \times, etc.) in display formulae ==>
\newcommand{\regularmu}{\thickmuskip= 5mu \medmuskip=4mu}
\newcommand{\thickmu}  {\thickmuskip=10mu \medmuskip=5mu}
\AtBeginEnvironment{align}      {\thickmu}
\AtBeginEnvironment{align*}     {\thickmu}
\AtBeginEnvironment{alignat}    {\thickmu}
\AtBeginEnvironment{alignat*}   {\thickmu}
\AtBeginEnvironment{displaymath}{\thickmu}
\AtBeginEnvironment{eqnarray}   {\thickmu}
\AtBeginEnvironment{eqnarray*}  {\thickmu}
\AtBeginEnvironment{equation}   {\thickmu}
\AtBeginEnvironment{equation*}  {\thickmu}
\AtBeginEnvironment{flalign}    {\thickmu}
\AtBeginEnvironment{flalign*}   {\thickmu}
\AtBeginEnvironment{gather}     {\thickmu}
\AtBeginEnvironment{gather*}    {\thickmu}
\AtBeginEnvironment{multiline}  {\thickmu}
\AtBeginEnvironment{multiline*} {\thickmu}
% <==

% Use the bm (= boldmath) package for better support of setting math in bold ==>
% Prevent the "Too many math fonts used" error:
\newcommand{\bmmax}{0}
\newcommand{\hmmax}{0}
\usepackage{bm}
% <==

% Allow paragraph breaks after a display equation
\makeatletter
\predisplaypenalty=\@medpenalty
\postdisplaypenalty=0
\makeatother




%%%%%%%%%%%%%%%%%%%%%%%%%%%%%
%%  LAYOUT AND SECTIONING  %%
%%%%%%%%%%%%%%%%%%%%%%%%%%%%%

% Disable single lines that start a~paragraph at the end of a~page (widows/Schusterjungen)
% and disable single lines at the end of a~paragraph that start a~new page (orphans/Hurenkinder):
\usepackage[all]{nowidow}

\AtBeginDocument{%
	% Reduce the amount of white space after a period (and enforce this throughout the entire document, because babel's \selectlanguage{...} resets \frenchspacing):
	\let\nonfrenchspacing=\frenchspacing
	\frenchspacing
	\sloppy%  % Prevent overfull hboxes (at the expense of more uneven whitespace)
}

\ifxetex
	\usepackage[protrusion=true, expansion=false]{microtype}
\else
	\usepackage[protrusion=true, expansion=false, kerning=true]{microtype}
\fi

% Reduce by how much an interword space can shrink; see
% http://tex.stackexchange.com/questions/19236/how-to-change-the-interword-spacing and http://tex.stackexchange.com/questions/88991/what-do-different-fontdimennum-mean:
\spaceskip=\fontdimen2\font plus \fontdimen3\font minus 0.75\fontdimen4\font

% Let even relatively big floats (long tables, spacious figures) be placed on text pages ==>
\renewcommand{\textfraction}{0.05}
\renewcommand{\topfraction}{0.95}
\renewcommand{\bottomfraction}{0.95}
% <== See http://tex.stackexchange.com/questions/39017/how-to-influence-the-position-of-float-environments-like-figure-and-table-in-lat/39020#39020

% Set margins of block quotes
% ==>
\renewenvironment{quote}%
  {\list{}{\leftmargin=\parindent \rightmargin=\parindent}%
   \linespread{\newbaselinestretch}\item[]\itshape\small}%
  {\endlist}
\renewenvironment{quotation}%
  {\list{}{\leftmargin=\parindent \rightmargin=\parindent
           \listparindent=\parindent \parsep=0pt}%
   \item[]}%
  {\endlist}
% <==

\usepackage{csquotes}

\usepackage{nameref}
\makeatletter
\newcommand*{\currentname}{\@currentlabelname}
\makeatother




%%%%%%%%%%%%%%%%%%%%%%%%%%%%%%%
%%  FORMATTING OF FOOTNOTES  %%
%%%%%%%%%%%%%%%%%%%%%%%%%%%%%%%


\usepackage[multiple, bottom, norule, splitrule, marginal]{footmisc}
% \renewcommand{\multfootsep}{,\,}
% AER-like style: no regular footnote rule, half-page splitrule
% ==>
\setlength{\footnotemargin}{0.85\newparindent}%
\renewcommand{\footnotelayout}{\hspace{0.15\newparindent}}
\renewcommand{\hangfootparindent}{\newparindent}
\preto{\footnote}{\setlength{\parindent}{\newparindent}}
% Add some space around the footnoterule:
\let\oldfootnoterule\footnoterule
\addtolength{\skip\footins}{\bigskipamount}
\AtBeginDocument{%
	\renewcommand{\splitfootnoterule}{{\hrule width 0.5\textwidth}}%
	\renewcommand{\footnoterule}{\oldfootnoterule\medskip}%
}
% <==
%\setlength{\footnotesep}{0.8\baselineskip}

% Chicago Manual of Style (16th edition):
% ``Note reference numbers in text are set as superior (superscript) numbers.
% In the notes themselves, they are normally full size, not raised, and followed by a period.''
% (also REStud and JEEA style)
% ==>
\usepackage{xstring}
\newlength{\textparindent}
\setlength{\textparindent}{\parindent}
\newlength{\templength}
\makeatletter
\let \@makefntextorig \@makefntext
    % Saving the original definition so we can reuse it if necessary.
\newcommand{\@makefntextcustom}[1]{%
	\parindent 2\textparindent%
	\hspace{-\textparindent}%
	\settowidth{\templength}{0}%
	\ifnum\value{footnote}<10 \hspace{\templength}\else\fi%
	\thefootnote.\enskip #1%
}
\renewcommand{\@makefntext}[1]{\@makefntextcustom{#1}}
\makeatother
%\usepackage{scrextend}
%\newlength{\footnoteflmargin}
%\setlength{\footnoteflmargin}{\parindent}
%% \deffootnote[\footnoteflmargin]{0pt}{0pt}{\thefootnotemark.\:\,}
%\deffootnote[\footnoteflmargin]{0pt}{\footnoteflmargin}{}
%\renewcommand{\footnote}[1]{%
%	\footnoteorig{%
%		\ifnum\value{footnote}<10 \phantom{0}\else\fi%
%		\thefootnotemark.\enskip%
%		#1%
%	}%
%}
% <==
\let \thefootnoteorig \thefootnote
\DefineFNsymbols*{star}{%
	{$\mathrm{\star}$}{$\mathrm{\star\star}$}{$\mathrm{\ddagger}$}%
	{$\mathrm{\ddagger\ddagger}$}{\S}{\S\S}{\P}{\P\P}{$\mathrm{\|}$}{$\mathrm{\|\|}$}%
}
\setfnsymbol{star}




%%%%%%%%%%%%%%%%%%%%%%%%%%
%%  FIGURES AND TABLES  %%
%%%%%%%%%%%%%%%%%%%%%%%%%%


\usepackage[singlelinecheck=on]{caption}
\DeclareCaptionLabelSeparator{periodlargespace}{.\:\:}
\captionsetup{
	singlelinecheck=on,
	figureposition=below,
	tableposition=above,
	format = plain,
	labelsep = periodlargespace,
	margin = 0pt,
	font = {sf, small},
	labelfont = {sf, bf, small},
	justification = justified
}

% Referencing ``subfigures'' (i.e., individual panels of which a figure is comprised):
%\usepackage{subfigure}
%\usepackage{subfig}
% Justus says this is the newer and preferable package.
% However, see https://tex.stackexchange.com/questions/13625/subcaption-vs-subfig-best-package-for-referencing-a-subfigure:
\usepackage{subcaption}

% Packages for creating better-looking tables
\usepackage{booktabs}
\setlength{\cmidrulewidth}{.035em}
\setlength{\lightrulewidth}{.035em}
\setlength{\heavyrulewidth}{.09em}
\setlength{\abovetopsep}{-5pt}
\addtolength{\aboverulesep}{1.5pt}	% Make tables a little more spacious
\addtolength{\belowrulesep}{1.5pt}
% \setlength{\belowbottomsep}{-2pt}

\usepackage{tabularx}	% Provides environment tabularx to adjust width of tables
% \usepackage{tabulary}	% For some reason, tabulary doesn't obey the width argument ...
% Emulate the "tabulary" column types:
\newcolumntype{C}{>{\centering\arraybackslash}X}
\newcolumntype{J}{>{\arraybackslash}X}
\newcolumntype{L}{>{\RaggedRight\arraybackslash}X}
\newcolumntype{R}{>{\RaggedLeft\arraybackslash}X}
% \usepackage[flushleft]{threeparttable}	% Provides the tablenotes environment

% Remove superfluous whitespace at beginning and end of table rows -->
\LetLtxMacro{\oldtabular}{\tabular}
\LetLtxMacro{\endoldtabular}{\endtabular}
\RenewDocumentEnvironment{tabular}{O{c} m}{%
	\oldtabular[#1]{@{}#2@{}}%
}{%
	\endoldtabular%
}
\LetLtxMacro{\oldtabularx}{\tabularx}
\LetLtxMacro{\endoldtabularx}{\endtabularx}
\RenewDocumentEnvironment{tabularx}{m O{c} m}{%
	\oldtabularx{#1}[#2]{@{}#3@{}}%
}{%
	\endoldtabularx%
}

\usepackage{siunitx}[=v2]
	% Allows, among others, for alignment of decimal numbers in tables at the decimal point.
\sisetup{
	detect-all,
	round-integer-to-decimal = true,
	group-digits             = true,
	group-minimum-digits     = 5,
	group-separator          = {\kern 1pt},
	table-align-text-pre     = false,
	table-align-text-post    = false,
	input-signs              = + -,
	input-symbols            = {*} {**} {***} \sigstar,
	input-open-uncertainty 	 = ,
	input-close-uncertainty  = ,
	retain-explicit-plus
}
\newcolumntype{T}[1]{@{}S[table-format = #1, table-space-text-pre = {***}, table-space-text-post = {***}]}
% Fix incompatibility of siunitx (v2018-05-17) with FiraSans (v2019-06-06),
% based on https://tex.stackexchange.com/questions/213605/siunitx-does-not-detect-semi-bold-font
% ==>
\ExplSyntaxOn\makeatletter
\newcommand{\thisseries}{\f@series}
\cs_set_protected:Npn \__siunitx_detect_font_weight_text: {%
	\let\origmdseries\mdseries@sf%
	\let\origbfseries\bfseries@sf%
	\let\currentseries\f@series%
	\edef\XcurrentseriesX{/\f@series/}%
		% Store the current fontseries but enclose it in some kind of delimiter,
		% because otherwise one-letter fontseries may erroneously triger \boldmath
		% (for instance, ``m'' is contained in ``semibold'')
	% Use \boldmath for any weight above semibold:
	\tl_if_in:noTF
		{ /sb/ /b/ /bx/ /eb/ /ub/ /bold/ /extrabold/ /ultrabold/ /heavy/ /black/ /demibold/ /semibold/ }
		{ \XcurrentseriesX }
		{% if included in the above list: switch to \mathversion{bold}
			\cs_set:Nn \__siunitx_font_weight: {%
				\boldmath%
				\let\bfseries@sf\currentseries%
					% necessary because siunitx in some way accesses
					% \bfseries@sf from the mweights package
				\fontseries{\bfseries@sf}\selectfont%
			}%
			\let\bfseries@sf\origbfseries%  % restore \bfseries@sf
		}
		{% if not: use \mathversion{normal}
			\cs_set:Nn \__siunitx_font_weight: {%
				\unboldmath
				\let\mdseries@sf\currentseries%
					% necessary because siunitx in some way accesses
					% \bfseries@sf from the mweights package
				\fontseries{\mdseries@sf}\selectfont%
			}%
			\let\mdseries@sf\origmdseries%  % restore \mdseries@sf
		}%
}
\makeatother\ExplSyntaxOff
% <==

% Ability to add footnotes to tables:
% ==>
\usepackage[restart, breakwithin, indentafter]{parnotes}
	% BEWARE: For some reason, \parnotes removes the vertical space before the following section/the indent of the following paragraph if not included in the table itself!
\renewcommand{\parnotevskip}{0pt}
% \renewcommand{\parnoteintercmd}{\\}
\renewcommand{\theparnotemark}{\alph{parnotemark}}
\renewcommand{\parnotefmt}[1]{\footnotesize\noindent\justify #1\par}
%\renewcommand{\parnotefmt}[1]{\footnotesize\rmfamily%
%	\noindent\rule{\linewidth}{1pt}\\%
%	\noindent#1\par%
%	\noindent\rule{\linewidth}{1pt}\\%
%}
%\renewcommand{\parnoteintercmd}{\;$\bullet$\;}
% <==

\usepackage{makecell}

\usepackage{longtable}

\usepackage{multirow}

% Redefine figure environment so that all figures are centered
% ==>
\makeatletter
\let\oldfigure\figure
\def\figure{\@ifnextchar[\figure@i \figure@ii}
\def\figure@i[#1]{\oldfigure[#1]\centering}
\def\figure@ii{\oldfigure\centering}
\makeatother
% <==

\usepackage[font={sf, small}]{floatrow}	% Set the font in tables to sansserif small
\floatsetup[table]{style=Plaintop}
\renewcommand{\floatfootskip}{\smallskipamount}
\newcommand{\tablenotes}[2][Notes:]{%
	\floatfoot*{%
		\setlength{\baselineskip}{11pt}%
		\textit{#1} #2%
	}%
}
\newcommand{\figurenotes}[2][Notes:]{%
	\floatfoot{%
		\setlength{\baselineskip}{11pt}%
		\vspace{-\floatfootskip}%
		\vspace{\medskipamount}%
		\\[-\baselineskip]
		\textit{#1} #2%
	}%
}
% \usepackage{float}	% Allows the inclusion of figures inside a minipage	% Do not use in combination with floatrow.
\usepackage{placeins} % improve placing of floats (figures, tables), provides \FloatBarrier

% Adjust the minimum distance between a float (figure, table) and the body text:
\setlength{\textfloatsep}{1.5\newbaselineskip plus 0.5\newbaselineskip minus 0.0pt}
% originally, \textfloatsep: 20.0pt plus 2.0pt minus 4.0pt

\usepackage{verbatim}

\usepackage{longtable}

%\usepackage{setspace}

%\DeclareCaptionLabelFormat{REStudTable}{\MakeUppercase{\tablename}~#2}
%\DeclareCaptionTextFormat{REStudTableT}{\textit{#1}}
%\captionsetup[table]{labelformat=REStudTable, textformat=REStudTableT}




%%%%%%%%%%%%%%%%%%%%%%%%%%%
%%  FORMATTING OF LISTS  %%
%%%%%%%%%%%%%%%%%%%%%%%%%%%


\usepackage[inline]{enumitem}
% General settings:
\setlist{leftmargin=\parindent, listparindent=\parindent, itemsep=\smallskipamount, parsep=0pt}
%\setlist[1]{topsep=\medskipamount, partopsep=0pt}
% Type-specific settings
%\setlist[enumerate]{labelwidth=\parindent, labelindent=0pt, labelsep=!, align=left}
\setlist[enumerate]{leftmargin=\parindent, labelsep=*}
	% Fine as long as the list does not include more than 9 items.
\setlist[enumerate, 1]{label=(\arabic*), labelindent=-0.5pt}
\setlist[enumerate, 2]{label=\alph*., align=right}
	% Taken from the Chicago Manual of Style (16th ed., Section 6.126)
\setlist[enumerate, 3]{label=\roman*., align=right, widest*=3, labelsep=0.3\parindent}
\setlist[itemize]{labelsep=0.435\parindent}
\setlist[description]{font=\rmfamily\normalsize}




%%%%%%%%%%%%%%%%%%%%%%%%%%%%%%
%%  FORMATTING OF THEOREMS  %%
%%%%%%%%%%%%%%%%%%%%%%%%%%%%%%


\newtheoremstyle{Standard}% name
	{\topsep}    % Space above: Use \topsep to make the space identical to the one around lists
	{\topsep}    % Space below
	{\itshape}   % Body font
	{}           % Indent amount (empty = no indent, \parindent = paragraph indent)
	{\bfseries}  % Theorem head font
	{.}          % Punctuation after theorem head
	{.5em}       % Space after theorem head: " " = normal interword space; \newline = linebreak
	{\thmname{#1}\thmnumber{\:#2}\thmnote{\bfseries\upshape\ (#3)}}
		% Theorem head spec (changed such that also the ``theorem note'' is printed in boldface)

\theoremstyle{Standard}
\newtheorem{theorem}{Theorem}
\newtheorem{corollary}[theorem]{Corollary}
\newtheorem{lemma}[theorem]{Lemma}
\newtheorem{proposition}[theorem]{Proposition}
\newtheorem{hypothesis}{Hypothesis}
\newtheorem{result}{Result}

\theoremstyle{definition}
\newtheorem{definition}[theorem]{Definition}
\newtheorem{example}[theorem]{Example}
\newtheorem{conjecture}[theorem]{Conjecture}

% Make numbering chapter-specific if we are compiling the dissertation template:
\makeatletter
\@ifclassloaded{book}{%
	\numberwithin{theorem}{chapter}%
	%\numberwithin{corollary}{chapter}%
	%\numberwithin{lemma}{chapter}%
	%\numberwithin{proposition}{chapter}%
	\numberwithin{hypothesis}{chapter}%
	\numberwithin{result}{chapter}%
	%\numberwithin{definition}{chapter}%
	%\numberwithin{example}{chapter}%
	%\numberwithin{conjecture}{chapter}%
}
\makeatother



%%%%%%%%%%%%%%%%%%%%%%%%%%%%%%%%%%%%%%%%%%%%%%%%%%%%%%%%%
%%  AUTOMATIC CAPITALIZATION OF HEADINGS AND CAPTIONS  %%
%%%%%%%%%%%%%%%%%%%%%%%%%%%%%%%%%%%%%%%%%%%%%%%%%%%%%%%%%


\usepackage{mfirstuc-english}
% \gMFUnocap{xxx} adds ``xxx'' to the words not to be capitalized
% Do not capitalize prepositions:
\gMFUnocap{about}
\gMFUnocap{at}
\gMFUnocap{against}
\gMFUnocap{around}
\gMFUnocap{between}
\gMFUnocap{by}
\gMFUnocap{from}
\gMFUnocap{on}
\gMFUnocap{over}
\gMFUnocap{per}
\gMFUnocap{to}
\gMFUnocap{versus}
\gMFUnocap{vs.}
\gMFUnocap{vis-\`a-vis}
\gMFUnocap{within}
\gMFUnocap{without}

\ifcase 0
	% 0 for disabling auto-capitalization, 1 for enabling auto-capitalization of headings, 2 for headings + figure/table captions
	% Case 0: Make capitalization commands ineffective
	\renewcommand{\capitalisewords}[1]{#1}
	\renewcommand{\ecapitalisewords}[1]{#1}
	\renewcommand{\xcapitalisewords}[1]{#1}
\or
	% Case 1:
	% Add auto-capitalization to table of contents -->
	\let\SavedContentsline\contentsline
	\renewcommand{\contentsline}[4]{%
		\SavedContentsline{#1}{\capitalisewords{#2}}{#3}{#4}%
	}
	% <--
\else
	% Case 2: Also auto-capitalize figure and table captions:
	% Add auto-capitalization to table of contents -->
	\let\SavedContentsline\contentsline
	\renewcommand{\contentsline}[4]{%
		\SavedContentsline{#1}{\capitalisewords{#2}}{#3}{#4}%
	}
	% <--
	\LetLtxMacro{\SavedCaption}{\caption}
	\RenewDocumentCommand{\caption}{ O{\shortcaption} m }{%
		\def\shortcaption{%
			\xcapitalisewords{%
				% \TestForPunct{%
				#2%
				% }%
			}%
		}%
		\SavedCaption[#1]{%
			\xcapitalisewords{%
				% \TestForPunct{%
				#2%
				% }%
			}%
		}%
	}
\fi




%%%%%%%%%%%%%%%%%%%%%%
%%  APPENDIX STYLE  %%
%%%%%%%%%%%%%%%%%%%%%%


\usepackage[title, titletoc]{appendix}  % allows, e.g., for appendices within chapters
\usepackage{chngcntr}  % to innclude section numbers/letters in the figure/table/equation counters

\AtBeginEnvironment{appendices}{%
	\counterwithin{figure}{section}%
	\counterwithin{table}{section}%
	\counterwithin{equation}{section}%
}
\AtBeginEnvironment{subappendices}{%
	\counterwithin{figure}{section}%
	\counterwithin{table}{section}%
	\counterwithin{equation}{section}%
}

% Revoke the changes at the end of the (sub)appendices environment
% if we are compiling the dissertation template:
\makeatletter
\@ifclassloaded{book}{%
	\AfterEndEnvironment{appendices}{%
		\counterwithout{figure}{section}%
		\counterwithin{figure}{chapter}%
		\counterwithout{table}{section}%
		\counterwithin{table}{chapter}%
		\counterwithout{equation}{section}%
		\counterwithin{equation}{chapter}%
	}%
	\AfterEndEnvironment{subappendices}{%
		\counterwithout{figure}{section}%
		\counterwithin{figure}{chapter}%
		\counterwithout{table}{section}%
		\counterwithin{table}{chapter}%
		\counterwithout{equation}{section}%
		\counterwithin{equation}{chapter}%
	}%
}{}
\makeatother
% !TeX program = pdflatex
% !TeX TXS-program:compile = txs:///pdflatex/
% !TeX TS-program = pdflatex

% !TeX TXS-program:bibliography = txs:///biber
% !BIB program = biber




%%%%%%%%%%%%%%%%%%%%%%%%%%%%%%%%%%
%%  ADDITIONAL LAYOUT SETTINGS  %%
%%%%%%%%%%%%%%%%%%%%%%%%%%%%%%%%%%


\usepackage[raggedright]{titlesec}	% Fine-tuning of headings
\titlespacing{\paragraph}
	{0pt}{0.5\baselineskip plus 0.5\baselineskip}{3\wordsep}[]
\titlespacing{\subparagraph}
	{\parindent}{0pt}{3\wordsep}[]

\usepackage{epigraph}
\makeatletter
\newlength\epitextskip
\pretocmd{\@epitext}{\em}{}{}
\apptocmd{\@epitext}{\em}{}{}
\patchcmd{\epigraph}{\@epitext{#1}\\}{\@epitext{#1}\\[\epitextskip]}{}{}
\makeatother
\setlength\epigraphrule{0pt}
\setlength\epitextskip{0pt}
\setlength\epigraphwidth{.8\textwidth}

\graphicspath{{Thesis/images/}}

\usepackage[titles]{tocloft}
\renewcommand{\cftdotsep}{\cftnodots}  % Remove the leader (dots) from the table of contents
\setlength{\cftfigindent}{0pt}  % Remove indentation from list of figures
\setlength{\cfttabindent}{0pt}  % Remove indentation from list of tables

% \marginpar settings for the commenting functions
\setlength{\marginparwidth}{2.3cm}
\setlength{\marginparsep}{0.3cm}
\setlength{\marginparpush}{0.3cm}
\AtBeginDocument{\reversemarginpar}




%%%%%%%%%%%%%%%%%%%%%%%%%%%%%%
%%  TYPOGRAPHICAL SETTINGS  %%
%%%%%%%%%%%%%%%%%%%%%%%%%%%%%%


% Adjust the design of captions
\DeclareCaptionLabelSeparator{periodlargespace}{.\:\:}
\captionsetup{
	singlelinecheck=on,
	figureposition=below,
	tableposition=above,
	format = plain,
	labelsep = periodlargespace,  % period,
	margin = 0pt,
	font = {small},  % {sf, small},
	labelfont = {bf, small},  % {sf, bf, small},
	justification = justified
}

% Apply \mathversion{bold} automagically
% ==>
\let\oldbf\bfseries
\renewcommand*{\bfseries}{\oldbf\mathversion{bold}}
\let\oldmd\mdseries
\renewcommand*{\mdseries}{\oldmd\mathversion{normal}}
\let\oldnorm\normalfont
\renewcommand*{\normalfont}{\oldnorm\mathversion{normal}}
% <==




%%%%%%%%%%%%%%%%%%%%%%%%%%%%%%
%% TABLE AND FIGURE LAYOUT  %%
%%%%%%%%%%%%%%%%%%%%%%%%%%%%%%


\floatsetup{font={rm, small}}  % Set the font in tables (back) to serif small
\floatsetup[table]{style=plaintop}




%%%%%%%%%%%%%%%%%%%%%%%%%%%%%%%%%
%%  FOOTNOTE-RELATED SETTINGS  %%
%%%%%%%%%%%%%%%%%%%%%%%%%%%%%%%%%


\usepackage[multiple, bottom, splitrule, marginal]{footmisc}
% \renewcommand{\multfootsep}{,\,}
\setlength{\footnotemargin}{0.85\newparindent}%
\renewcommand{\footnotelayout}{\hspace{0.15\newparindent}}
\renewcommand{\hangfootparindent}{\newparindent}
\preto{\footnote}{\setlength{\parindent}{\newparindent}}
% Add some space around the footnoterule:
\let\oldfootnoterule\footnoterule
\addtolength{\skip\footins}{\bigskipamount}
\AtBeginDocument{%
	\renewcommand{\footnoterule}{\oldfootnoterule\smallskip}%
}
% !TeX TXS-program:compile = txs:///pdflatex/
% !TeX TXS-program:bibliography = txs:///biber
% !TeX program = pdflatex
% !BIB program = biber




%%%%%%%%%%%%%%%%%%%%%%%%%%%%%%%%%%%%%%%%%%%%%
%  CITATION COMMANDS AND BIBLIOGRAPHY STYLE %
%%%%%%%%%%%%%%%%%%%%%%%%%%%%%%%%%%%%%%%%%%%%%


% AER/JEL/JEP style

\usepackage[backend=biber, natbib=true, bibencoding=inputenc, bibstyle=authoryear, citestyle=authoryear-comp, mincitenames=1, maxcitenames=3, minbibnames=99, maxbibnames=99, uniquename=false, uniquelist=true, backref=true, backrefstyle=three, doi=true, isbn=false, dashed=false, sorting=ynt, sortcites=true, mergedate=true, dateabbrev=false, abbreviate=false, citetracker=true]{biblatex}
% sortcites sorts the in-text citations by year of publication
\DeclareBibliographyAlias{newspaper}{article}

% Full author list on first citation, ``et al.'' only from second citation onwards:
% https://tex.stackexchange.com/questions/48846/biblatex-et-al-beginning-from-second-citation
% ==>
\AtEveryCitekey{\ifciteseen{}{\clearfield{namehash}}}
\xpatchbibmacro{cite}
	{\printnames{labelname}}
	{\ifciteseen
		{\printnames{labelname}}
		{\printnames[][1-5]{labelname}}%
	}
	{}
	{}
\xpatchbibmacro{textcite}
	{\printnames{labelname}}
	{\ifciteseen
		{\printnames{labelname}}
		{\printnames[][1-5]{labelname}}%
	}
	{}
	{}
% <==

\let\citeorig\cite
\renewcommand{\cite}{\citet}
\renewcommand{\citealp}{\citeorig}

\defbibheading{subbibliography}[\refname]{%
	\section*{#1}%
	\sectionmark{#1}%
	\addcontentsline{toc}{section}{#1}%
}

\renewcommand{\bibfont}{\sffamily\small}
	% Reduce font size for the bibliography, make the font sans-serif.
\setlength{\bibindent}{\parindent}
\setlength{\bibitemsep}{0pt}

\DefineBibliographyStrings{english}{%
	andothers = {et~al\adddot},
	volume = {vol\adddot}
}

% Remove parentheses from the date:
\xpatchbibmacro{date+extradate}{%
	\printtext[parens]%
}{%
	\setunit*{\mkbibbold{\addperiod\space}}%
	\mdseries\selectfont\printtext%
}{}{}
% Make the author/editor name(s) bold ==>
% See https://tex.stackexchange.com/questions/41468/biblatex-textcite-author-formatting-vs-bibliography-author-formatting
\xpatchbibmacro{author}{\printnames{author}}{\mkbibbold{\printnames{author}\addperiod}}{}{}
%\xpretobibmacro{author}{\mkbibbold\bgroup}{}{}
%\xapptobibmacro{author}{\egroup}{}{}
\xpretobibmacro{editor+others}{\mkbibbold\bgroup}{}{}%
\xapptobibmacro{editor+others}{\egroup\clearname{editor}}{}{}%
\xpretobibmacro{translator+others}{\mkbibbold\bgroup}{}{}%
\xapptobibmacro{translator+others}{\egroup\clearname{translator}}{}{}
% Set up page range compression (e.g., 1034-1067 => 1034-67)
\setcounter{mincomprange}{100}
\setcounter{maxcomprange}{100000}
\setcounter{mincompwidth}{10}
\DeclareFieldFormat{pages}{\mkcomprange{#1}}
% Removes ``pp.'' from pages and compresses page ranges:
\DeclareFieldFormat[article, inbook, incollection, inproceedings, patent, thesis, unpublished, newspaper]{pages}{{\nopp\mkcomprange{#1}}} 
% Include comma also after first name in the case of only two authors:
\AtEveryBibitem{
	\renewcommand*{\finalnamedelim}{\finalandcomma\addspace\bibstring{and}\space}
	\renewcommand*{\finallistdelim}{\finalandcomma\addspace\bibstring{and}\space}
}
\xpretobibmacro{byeditor+others}  % Undo this for the listing of editors
	{%
		\renewcommand*{\finalnamedelim}{%
			\ifnumgreater{\value{liststop}}{2}{\finalandcomma}{}%
			\addspace\bibstring{and}\space%
		}%
	}
	{}
	{\DoNotContinue}

% \renewcommand*{\newunitpunct}{\addcomma\space}
\renewcommand*{\bibnamedash}{---{\kern -2.25pt}---{\kern -2.25pt}---\addcomma\space}

%\DeclareCiteCommand{\citejournal}
%  {\usebibmacro{prenote}}
%  {\usebibmacro{citeindex}%
%   \usebibmacro{journal}}
%  {\multicitedelim}
%  {\usebibmacro{postnote}}

% Remove ``in'' for journal and newspaper articles:
\renewcommand*{\intitlepunct}{\space}
\renewbibmacro{in:}{%
	\ifentrytype{article}%
		{}%
		{\ifentrytype{newspaper}%
			{}%
			{\printtext{\bibstring{in}\intitlepunct}}%
		}%
}

% How to handle DOIs and URLs
% ==>
\renewbibmacro*{doi+eprint+url}
{%
	\iffieldundef{doi}
	{}
	{\printfield{doi}}
	\newunit\newblock
	\iftoggle{bbx:eprint}
	{\usebibmacro{eprint}}
	{}%
	\newunit\newblock
	\iffieldundef{doi}  % If no DOI provided, use URL
	{\usebibmacro{url+urldate}}
	{}
}
\DeclareFieldFormat{url}{%
	% \setlength{\Urlmuskip}{0mu}%
	\Urlmuskip = 0mu\relax%
	\caps{URL}\addcolon\space\url{#1}%
}
% Link DOIs automatically to https://dx.doi.org/DOI:
\DeclareFieldFormat{doi}{%
	% \setlength{\Urlmuskip}{0pt}%
	\Urlmuskip = 0mu\relax
	\caps{DOI}\addcolon\space\href{https://dx.doi.org/#1}{\nolinkurl{#1}}%
}
% <==

% Make journal title italic:
\DeclareFieldFormat[article]{journaltitle}{%
	\mkbibemph{#1}%
	\iffieldundef{volume}{\addcomma}{}%
}

% Handling of newspaper articles
\DeclareFieldFormat[newspaper]{journaltitle}{%
	\mkbibemph{#1} \mkbibparens{\thefield{edition}}\addcomma\addspace%
	\iflanguage{ngerman}{%
		\addnbspace\thefield{day}\addperiod\addnbspace\mkbibmonth{\thefield{month}}\addspace\thefield{year}%
	}{% else: default to the USenglish version
		\mkbibmonth{\thefield{month}}\addnbspace\thefield{day}\addcomma\addspace\thefield{year}%
	}%
	\iffieldundef{volume}{\addcolon}{\addcomma}%
}
\DeclareFieldFormat[newspaper]{date}{%
	\thefield{year}%
}

% Format the ``month'' field:
\DeclareFieldFormat{month}{\mkbibmonth{#1}}

% Put the title of articles in quotation marks:
\DeclareFieldFormat[thesis, unpublished, report, misc, newspaper]{title}{\mkbibquote{#1}}
%% Make title of books italic:
%\DeclareFieldFormat[book]{title}{\mkbibemph{#1}\nopunct}

% Makes first character of document type uppercase:
\DeclareFieldFormat[thesis, unpublished, report, misc]{type}{\MakeCapital{#1}}

%% Remove punctuation after ``series'' field:
%\DeclareFieldFormat[incollection]{series}{#1\nopunct}

% Add ``vol.'' for books etc.:
\DeclareFieldFormat[book, inbook, incollection, inproceedings]{volume}{\bibstring{volume}\addnbspace#1\addcomma}
% Formatting VV\,(NN):
\DeclareFieldFormat[article]{volume}{%
	\iffieldundef{volume}{}{%
		\addspace#1%
		\iffieldundef{number}{\addcolon}{\nopunct}%
	}
}
\DeclareFieldFormat[article]{number}{\addnbthinspace\mkbibparens{#1}\addcolon}
\DeclareFieldFormat[article]{date}{#1}

% \AtEveryCitekey{\clearfield{month}}
% Replace number by month if number is undefined:
% ==>
\DeclareSourcemap{
    \maps[datatype=bibtex]{
        \map{
            \pertype{article}
			\step[notfield=number, final]
			\step[fieldsource=month, fieldtarget=number]
            \step[fieldset=month, null]
            \step[fieldset=month_numeric, null]
        }
    }
}
% <==

% Adjust formatting of back-references
% ==>
\DefineBibliographyStrings{english}{%
	backrefpage  = {},	% originally ``cit. on p.''
	backrefpages = {}	% originally ``cit. on pp.''
}
\DefineBibliographyStrings{ngerman}{%
	backrefpage  = {},	% originally ``Siehe Seite''
	backrefpages = {}	% originally ``Siehe Seiten''
}
\renewcommand*{\finentrypunct}{}
\renewbibmacro*{pageref}{%
	\addperiod
	\iflistundef{pageref}
	{}
	{\printtext[brackets]{%
			\ifnumgreater{\value{pageref}}{1}
				{\bibstring{backrefpages}}
				{\bibstring{backrefpage}}%
			\printlist[pageref][-\value{listtotal}]{pageref}%
		}%
	}%
}
% <==

% Move notes to the end of an entry by copying the ``note'' information into ``addendum'': -->
\DeclareSourcemap{
	\maps[datatype=bibtex]{
		% Copy values of the Mendeley-created ``annote'' field to the ``note'' field:
		\map[overwrite]{
		 	\step[fieldsource=annote]
		 	\step[fieldset=note, origfieldval, append]
			\step[fieldset=annote, null]
		}
		\map{
			\step[fieldsource=note, final]
			\step[fieldset=addendum, origfieldval, final]
			\step[fieldset=note, null]
		}
	}
}
% \DeclareFieldFormat{addendum}{(#1)} % Enclose addendum/note in parentheses.
% <--

\AtEveryBibitem{%
	\ifentrytype{newspaper}%
		{\clearfield{addendum}}% then
		{\clearfield{month}}% else
}

% Add ``working paper'' as the default value for type of ``techreports'' ==>
% see https://tex.stackexchange.com/questions/212362/how-to-use-declaresourcemap-to-add-default-value-to-a-field
\DeclareSourcemap{
	\maps[datatype=bibtex]{
		\map{% Will overwrite fields without the ``overwrite'' option
			\pertype{techreport}
			\step[fieldset=type, fieldvalue={Working paper}]
		}
	}
}
% <==

% Add “doctoral dissertation” as the default value for type of “phdthesis” ==>
% see https://tex.stackexchange.com/questions/212362/how-to-use-declaresourcemap-to-add-default-value-to-a-field
\DeclareSourcemap{
	\maps[datatype=bibtex]{
		\map{% Will overwrite fields without the ``overwrite'' option
			\pertype{phdthesis}
			\step[fieldset=type, fieldvalue={Doctoral dissertation}]
		}
	}
}
% <==

% Remove superfluous ``The'' from journal names ==>
\DeclareSourcemap{ 
	\maps[datatype=bibtex]{
		\map{
			\step[fieldsource=journal, match={\regexp{^The\s}}, replace={}]
		}
	}      
}
% <==
%%%%%%%%%%%%%%%%%%%%%%%%%%%%
%%  FONT SETTINGS THESIS  %%
%%%%%%%%%%%%%%%%%%%%%%%%%%%%


\usepackage{amsfonts, amssymb}  % Additional symbols from AMS

\usepackage{extarrows}

% Use Times Roman for text and math
\usepackage{newtxtext}
\usepackage[smallerops, varg, upint, slantedGreek]{newtxmath}

%\usepackage[scale=1.07]{tgcursor}
%	% Use Courier as the typewriter font
%\usepackage[scale=0.93]{sourcecodepro}
%	% Use Source Code Pro as the typewriter font
\usepackage[scaled=0.835]{DejaVuSansMono}  % Scaled to match the x-height of the newtxtext Times Roman
	% Use DejaVu Sans Mono as the typewriter font, also supports Greek via the LGR encoding

\usepackage{xfrac}

\usepackage[protrusion=true, expansion=false, kerning=true]{microtype}
% This package enables so-called hanging punctuation. That is, when a punctuation sign like ":", ".", "-", etc. is found at the beginning or end of a line, it is protruded a little into the page margin. This results in "optical margin alignment," because the protrusion makes the margin alignment look straighter.
%\DisableLigatures[f]{family = {sf*, rm*}}
%	% Disable the f* ligatures for both Fira Sans and Charter because both fonts provide insufficient support
%\DisableLigatures[f]{family = {rm*}}
	% Disable the f* ligatures for Charter because it provides insufficient support
\SetExtraKerning[unit=space]
{encoding=*, family=*, series=*, size={*, normalsize, footnotesize}, font = */*/*/*/*}
{\textemdash = {325, 325},
	/ = {100, 100},
	: = { 50,   0},
	; = { 50,   0}}
\renewcommand{\textellipsis}{\mbox{.{\kern.09em}.{\kern.09em}.}}

\renewcommand{\bibfont}{\small} % Reduce font size for the bibliography, make the font sans-serif

\captionsetup{footfont={rm, footnotesize}}  % Set font in \floatfoot to serif and footnotesize




%%%%%%%%%%%%%%%%%%%%%%%%%%%%%%%%
%%  MATH-RELATED ADJUSTMENTS  %%
%%%%%%%%%%%%%%%%%%%%%%%%%%%%%%%%


\usepackage{mathrsfs}  % Provides the \mathscr math alphabet

% The following is taken from
% https://tex.stackexchange.com/questions/116389/automatic-upright-math-when-text-is-in-italic/116399#116399
% Filling in ``missing'' Greek glyphs for completeness
% (not really necessary, since they look identical to Latin glyphs and are thus almost never used)
% ==>
\newcommand{\omicron}{o}
\newcommand{\Digamma}{F}
\newcommand{\Alpha}  {A}
\newcommand{\Beta}   {B}
\newcommand{\Epsilon}{E}
\newcommand{\Zeta}   {Z}
\newcommand{\Eta}    {H}
\newcommand{\Iota}   {I}
\newcommand{\Kappa}  {K}
\newcommand{\Mu}     {M}
\newcommand{\Nu}     {N}
\newcommand{\Omicron}{O}
\newcommand{\Rho}    {P}
\newcommand{\Tau}    {T}
\newcommand{\Chi}    {X}
% <==

% Upright Greek letters for the \mathup command
% ==>
\makeatletter
% Save original definitions of the Greek letters
\@for\@tempa:=%
	alpha,beta,gamma,delta,epsilon,zeta,eta,theta,iota,kappa,lambda,mu,nu,xi,%
	omicron,pi,rho,sigma,varsigma,tau,upsilon,phi,chi,psi,omega,digamma,%
	Alpha,Beta,Gamma,Delta,Epsilon,Zeta,Eta,Theta,Iota,Kappa,Lambda,Mu,Nu,Xi,%
	Omicron,Pi,Rho,Sigma,Tau,Upsilon,Phi,Chi,Psi,Omega,Digamma%
	\do{%
		\expandafter\let\csname\@tempa orig\expandafter\endcsname\csname\@tempa\endcsname%
		\expandafter\let\csname\@tempa uporig\expandafter\endcsname\csname\@tempa up\endcsname%
	}%
\newcommand*{\upgreekletters}{%
	\@for\@tempa:=%
		alpha,beta,gamma,delta,epsilon,zeta,eta,theta,iota,kappa,lambda,mu,nu,xi,%
		omicron,pi,rho,sigma,varsigma,tau,upsilon,phi,chi,psi,omega,digamma,%
		Alpha,Beta,Gamma,Delta,Epsilon,Zeta,Eta,Theta,Iota,Kappa,Lambda,Mu,Nu,Xi,%
		Omicron,Pi,Rho,Sigma,Tau,Upsilon,Phi,Chi,Psi,Omega,Digamma%
		\do{%
			\expandafter\let\csname\@tempa\expandafter\endcsname\csname\@tempa up\endcsname%
		}%
}
\newcommand*{\itgreekletters}{%
	\@for\@tempa:=%
		alpha,beta,gamma,delta,epsilon,zeta,eta,theta,iota,kappa,lambda,mu,nu,xi,%
		omicron,pi,rho,sigma,varsigma,tau,upsilon,phi,chi,psi,omega,digamma,%
		Alpha,Beta,Gamma,Delta,Epsilon,Zeta,Eta,Theta,Iota,Kappa,Lambda,Mu,Nu,Xi,%
		Omicron,Pi,Rho,Sigma,Tau,Upsilon,Phi,Chi,Psi,Omega,Digamma%
		\do{%
			\expandafter\let\csname\@tempa\expandafter\endcsname\csname\@tempa orig\endcsname%
		}%
}
\makeatother
% <==

\newcommand{\mathup}[1]{\upgreekletters\mathrm{#1}\itgreekletters}

\renewcommand{\mathbf}[1]{\bm{#1}}
\newcommand{\mathbfit}[1]{\mathbf{\mathit{#1}}}
\newcommand{\mathbfup}[1]{\upgreekletters\mathbf{\mathrm{#1}}\itgreekletters}

% Since math mode uses a different font encoding, issuing \euro/\texteuro in math mode
% produces an incorrect sign. We fix this. ==>
\AtBeginDocument{%
	\newcommand*{\euro}[1]{%
		\relax\ifmmode\text{\texteuro}#1\else\texteuro #1\fi%
	}%
}
% <==
% !TeX TXS-program:compile = txs:///pdflatex/
% !TeX TXS-program:bibliography = txs:///biber
% !TeX program = pdflatex
% !BIB program = biber




%%%%%%%%%%%%%%%%%%%%%%%%%%%
%%  COMMENTING COMMANDS  %%
%%%%%%%%%%%%%%%%%%%%%%%%%%%


\usepackage{tcolorbox}

\definecolor{blazeorange}{rgb}{1,0.40,0.00}
\newcommand{\highlight}[1]{\textcolor{blazeorange}{#1}}

\newlength{\boxheight}
\newlength{\boxwidth}
\newcommand{\trackchanges}[2][]{%
	\highlight{#2}%
	\settoheight{\boxheight}{\parbox{\textwidth}{#2}}%
	\marginpar{\vspace{-2\boxheight}%
		\begin{tcolorbox}[width=1.25in, height=2\boxheight, left=1.5mm, right=1.5mm, top=1.5mm, bottom=1.5mm, colback=blazeorange, boxrule=0pt, arc=0pt, opacityfill=0.5]
			\tiny\RaggedRight%
			\textcolor{white}{\textsf{%
					\ifthenelse{\equal{#1}{}}{\textbf{Insertion}}{\textbf{Replaces:} ``#1''}%
			}}
		\end{tcolorbox}%
	}%
}
% To ``accept'' all changes in the document:
% \renewcommand{\trackchanges}[2][]{#2}	% Sets changes in plain text.
\newcommand{\XXX}[2][]{\trackchanges[#1]{#2}}

\usepackage{marginnote}  % Customizable marginal paragraphs
\setlength{\marginparwidth}{3.5cm}
\setlength{\marginparsep}{0.6cm}
\setlength{\marginparpush}{0.6cm}

% \usepackage{soul} % Already loaded earlier
% Provides a~highlighting command: \hl{...}
% Change the highlighting color:
\sethlcolor{darkred25}


\newcommand{\remark}[4][0pt]{
	\marginnote{\RaggedRight\sloppypar\sffamily%
		%\fontseries{l}\selectfont
		\vspace{-10pt}%
		\scriptsize\setlength{\baselineskip}{2.35ex}\textcolor{darkred}{\textbf{%
				%\fontseries{sb}\selectfont
				#3}\\ #4}}[#1]%
	\hl{#2}%
}

\newcommand{\commentgeneric}[4][0pt]{%
	\setbox99=\hbox{\textbf{\v{ }}}%
	\makebox[0pt][l]{\hspace{-1.75pt}\textcolor{#4}{\textbf{\raisebox{1.75pt}{\v{ }}\hspace{-0.985\wd99}|}}}%
	\marginnote{\RaggedRight\sloppypar\sffamily%
		%\fontseries{l}\selectfont
		\vspace{-10pt}%
		\scriptsize\setlength{\baselineskip}{2.35ex}\textcolor{#4}{%
			\textbf{#2} \\
			#3%
		}%
	}[#1]%
}
\newcommand{\commentfreddie}[2][0pt]{%
	\commentgeneric[#1]{Freddie says:}{#2}{darkblue}%
}
\newcommand{\commentholger}[2][0pt]{%
	\commentgeneric[#1]{Holger says:}{#2}{customgreen}%
}
\newcommand{\commentlouis}[2][0pt]{%
	\commentgeneric[#1]{Louis says:}{#2}{custompurple}%
}
\newcommand{\commentmarkus}[2][0pt]{%
	\commentgeneric[#1]{Markus says:}{#2}{darkred}%
}
\newcommand{\commentgerhard}[2][0pt]{%
	\commentgeneric[#1]{Gerhard says:}{#2}{blazeorange}%
}
% To suppress all comments in the document:
% \renewcommand{\commentgeneric}[4][0pt]{}{}{}

\usepackage{pdfcomment}
\newenvironment{holgeradded}{%
	\color{customgreen}%
	\begin{pdfsidelinecomment}[color=customgreen,caption=inline,linebegin={/None},lineend={/None},linewidth=2bp,linesep=1cm]{Holger}\ignorespaces}%
	{\end{pdfsidelinecomment}%
}
% To ``accept'' all insertions in the document:
% \renewenvironment{holgeradded}{}{}
% !TeX TXS-program:compile = txs:///pdflatex/
% !TeX TXS-program:bibliography = txs:///biber
% !TeX program = pdflatex
% !BIB program = biber




%%%%%%%%%%%%%%%%%%%%%%%%%%%%%%%%%%%%%%%%%%%%%%%%%%%%%%%%    
%%  OVERWRITE COMMENTING COMMANDS TO SWITCH THEM OFF  %%
%%%%%%%%%%%%%%%%%%%%%%%%%%%%%%%%%%%%%%%%%%%%%%%%%%%%%%%%


% Suppress all comments:
\renewcommand{\commentgeneric}[4][0pt]{}{}{}

% Suppress all highlighting:
\renewcommand{\highlight}[1]{#1}

% ``Accept'' all changes in the document:
\renewcommand{\trackchanges}[2][]{#2}	% Sets changes in plain text.
\renewcommand{\XXX}[2][]{\trackchanges[#1]{#2}}

% ``Accept'' all insertions in the document:
\renewenvironment{holgeradded}{}{}
% !TeX program = pdflatex
% !TeX TXS-program:compile = txs:///pdflatex/
% !TeX TS-program = pdflatex
% !BIB program = biber
% !TeX TXS-program:bibliography = txs:///biber




%%%%%%%%%%%%%%%%%%%%%%%%%%%%%%%%%%%
%%  OTHER PACKAGES AND COMMANDS  %%
%%%%%%%%%%%%%%%%%%%%%%%%%%%%%%%%%%%


% Some math-related definitions

%\newcommand*{\coloneqq}{\mathrel{%
%	\mathrel{%
%		\raisebox{0.18ex}{\scalebox{0.85}{\ensuremath{:}}\hspace{-0.2pt}%
%	}%
%	=%
%}}
% Provided by the mathtools package

\newcommand{\Corr}{\operatorname{Corr}}
\newcommand{\Cov} {\operatorname{Cov}}
\newcommand{\E}   {\operatorname{E}}
\newcommand{\Var} {\operatorname{Var}}

\newcommand{\dd}  {\mathup{d}}  % Differential d
\newcommand{\e}   {\mathup{e}}  % Euler's e

\newcommand{\sigstar}{\raisebox{0.66ex}{\scalebox{0.95}{$\star$}}}

% Balanced/unbalanced sliders for text:
\newcommand{\bal}{\mbox{\caps{BAL}}\xspace}
\newcommand{\unbal}{\mbox{\caps{UNBAL}}\xspace}
\newcommand{\balA}[1][1]{\mbox{\caps{BAL}$^{\mathup{I}}_{#1:#1}$}\xspace}
\newcommand{\balB}[1][1]{\mbox{\caps{BAL}$^{\mathup{II}}_{#1:#1}$}\xspace}
\newcommand{\unbalA}[1][n]{\mbox{\caps{UNBAL}$^{\mathup{I}}_{1:#1}$}\xspace}
\newcommand{\unbalB}[1][n]{\mbox{\caps{UNBAL}$^{\mathup{II}}_{#1:1}$}\xspace}

% Balanced/unbalanced slider choice sets for math:
\newcommand{\CS}[1][C]{{\mathbf{#1}}}
\newcommand{\CbalA}[1][1]{\CS^{\mathup{BAL,\,I}}_{#1:#1}}
\newcommand{\CbalB}[1][1]{\CS^{\mathup{BAL,\,II}}_{#1:#1}}
\newcommand{\CunbalA}[1][n]{\CS^{\mathup{UNBAL,\,I}}_{1:#1}}
\newcommand{\CunbalB}[1][n]{\CS^{\mathup{UNBAL,\,II}}_{#1:1}}
\newcommand{\cse}[1][c]{{\mathbf{#1}}}
\newcommand{\cbalA}[1][1]{\cse^{\mathup{BAL,\,I}}_{#1:#1}}
\newcommand{\cbalB}[1][1]{\cse^{\mathup{BAL,\,II}}_{#1:#1}}
\newcommand{\cunbalA}[1][n]{\cse^{\mathup{UNBAL,\,I}}_{1:#1}}
\newcommand{\cunbalB}[1][n]{\cse^{\mathup{UNBAL,\,II}}_{#1:1}}

% Balanced/unbalanced choice lists for text:
\newcommand{\balCL}[1][1]{\mbox{\caps{BAL}$_{\mathup{CL}}$}\xspace}
\newcommand{\unbalCLA}[1][1]{\mbox{\caps{UNBAL}$^{\mathup{I}}_{\mathup{CL}}$}\xspace}
\newcommand{\unbalCLB}[1][1]{\mbox{\caps{UNBAL}$^{\mathup{II}}_{\mathup{CL}}$}\xspace}

% Balanced/unbalanced choice-list choice sets for math:
\newcommand{\CbalCL}{{\CS}^{\textup{BAL}}_{\textup{CL}}}
\newcommand{\CunbalCLA}{{\CS}^{\textup{UNBAL,\,I}}_{\textup{CL}}}
\newcommand{\CunbalCLB}{{\CS}^{\textup{UNBAL,\,II}}_{\textup{CL}}}
\newcommand{\cbalCL}{{\cse}^{\textup{BAL}}_{\textup{CL}}}
\newcommand{\cunbalCLA}{{\cse}^{\textup{UNBAL,\,I}}_{\textup{CL}}}
\newcommand{\cunbalCLB}{{\cse}^{\textup{UNBAL,\,II}}_{\textup{CL}}}




%%%%%%%%%%%%%%%%%%%%%%%%%%%%%%%%%%%%
%%  ILLUSTRATE THE BASELINE GRID  %%
%%%%%%%%%%%%%%%%%%%%%%%%%%%%%%%%%%%%


\usetikzlibrary{calc}
\newcommand{\displaybaselinegrid}{%
	\begin{tikzpicture}[remember picture, overlay, x = 1mm, y = \baselineskip]
		\draw[orange] ($(current page.north west) + (0mm, -1in-\topmargin-\headheight-\headsep)$) -- ($(current page.north east) + (0, -1in-\topmargin-\headheight-\headsep)$);
		\foreach \i in {-5,...,-1,0,1,2,...,\linesperpagedesired}{
			\draw[magenta!50] ($(current page.north west) + (0mm, -1in-\voffset-\topmargin-\headheight-\headsep-\topskip) + (0, -\i-2)$) -- ($(current page.north east) + (0mm, -1in-\voffset-\topmargin-\headheight-\headsep-\topskip) + (0, -\i-2)$);
		}
		\draw[orange] ($(current page.north west) + (0mm, -1in-\topmargin-\headheight-\headsep-\textheight)$) -- ($(current page.north east) + (0mm, -1in-\topmargin-\headheight-\headsep-\textheight)$);
	\end{tikzpicture}%
}

\bibliography{Library.bib}

\hypersetup{
	pdftitle  = {\thesistitle},
	pdfauthor = {\thesisauthor}
}




%%%%%%%%%%%%
%%  BODY  %%
%%%%%%%%%%%%


\begin{document}


\title{\thesistitle}
\author{\thesisauthor}
\date{\today}

% !TeX program = pdflatex
% !TeX TXS-program:compile = txs:///pdflatex/
% !TeX TS-program = pdflatex
% !TeX TXS-program:bibliography = txs:///biber
% !BIB program = biber




\selectlanguage{ngerman}

\pagenumbering{roman}

\makeatletter
\renewcommand*{\maketitle}{%
	%\begin{titlepage}
	\thispagestyle{bullshitheader}
	\begin{center}\selectlanguage{ngerman}\large\mbox{}
		
		\vfill
		
		{\LARGE\bfseries\@title\par}
		
		\vfill%\vspace{3\baselineskip}
		
		{\Large\ifthenelse{\equal{\thesistype}{BA}}{Bachelorarbeit}{Masterarbeit}}
		
		\vspace{2\baselineskip}
		
		zur Erlangung des Grades
		\ifthenelse{\equal{\thesistype}{BA}}{Bachelor of Science (B.\,Sc.)}{Master of Science (M.\,Sc.)}
		
		im Studiengang Volkswirtschaftslehre
		
		an der Rheinischen Friedrich-Wilhelms-Universität Bonn
		
		\vspace{2\baselineskip}
		
		Themensteller\ifthenelse{\equal{\advisorgender}{f}}{in}{}:
		
		\advisorname
		
		\vfill%\vspace{2\baselineskip}
		
		Vorgelegt im \datengerman\printdayoff\today\ von
		
		\vspace{\baselineskip}
		
		{\Large \@author}
		
		\vspace{1\baselineskip}
		
		Matrikelnummer: \studentID
		
		\vfill
		
		\selectlanguage{USenglish}
	\end{center}
	%\end{titlepage}
	\clearpage
}
\makeatother

\maketitle

{
	\linespread{1}
	\tableofcontents
}

\clearpage

\pagenumbering{arabic}

% \subimport enables execution of \input commands in subdocuments without having to adjust paths
\subimport*{1_Example_Content/1_Introduction/}{Introduction_Thesis}
% Avoid that the line spacing in tikzpictures becomes too large
% as a consequence of the 1.5 line spacing of the body text
% ==>
\tikzset{every picture/.append style={
		execute at begin picture={%
			\edef\baselinestretch{0.667}%
		}
	}
}
% <==
\subimport*{1_Example_Content/2_Methods/}{Design}
\subimport*{1_Example_Content/2_Methods/}{Predictions}
\subimport*{1_Example_Content/3_Results/}{Results}
\subimport*{1_Example_Content/3_Results/}{Structural_Estimation}
\subimport*{1_Example_Content/4_Discussion/}{Discussion}
\subimport*{1_Example_Content/5_Conclusion/}{Conclusion}

\clearpage

\begin{appendices}
	\label{sec:appendix}
	\FloatBarrier
	\subimport*{1_Example_Content/9_Appendix/}{Attention}
	\FloatBarrier
	\clearpage
	\subimport*{1_Example_Content/9_Appendix/}{Additional_Figures}
	\FloatBarrier
	\subimport*{1_Example_Content/9_Appendix/}{siunitx_examples}
	\clearpage
	%%%%%%%%%%%%%%%%%%%%%%%%%%
%%  TESTING MATH FONTS  %%
%%%%%%%%%%%%%%%%%%%%%%%%%%

\newcommand{\dit}{\mathit{d}}
\newcommand{\dup}{\mathup{d}}

\def\test#1{#1}

\def\testnums{%
  \test 0 \test 1 \test 2 \test 3 \test 4 \test 5 \test 6 \test 7
  \test 8 \test 9 }
\def\testupperi{%
  \test A \test B \test C \test D \test E \test F \test G \test H
  \test I \test J \test K \test L \test M }
\def\testupperii{%
  \test N \test O \test P \test Q \test R \test S \test T \test U
  \test V \test W \test X \test Y \test Z }
\def\testupper{%
  \testupperi\testupperii}

\def\testloweri{%
  \test a \test b \test c \test d \test e \test f \test g \test h
  \test i \test j \test k \test l \test m }
\def\testlowerii{%
  \test n \test o \test p \test q \test r \test s \test t \test u
  \test v \test w \test x \test y \test z }
\def\testlower{%
  \testloweri\testlowerii}

\def\testupgreeki{%
  \test A \test B \test\Gamma \test\Delta \test E \test Z \test H
  \test\Theta \test I \test K \test\Lambda \test M }
\def\testupgreekii{%
  \test N \test\Xi \test O \test\Pi \test P \test\Sigma \test T
  \test\Upsilon \test\Phi \test X \test\Psi \test\Omega }
\def\testupgreek{%
  \testupgreeki\testupgreekii}

\def\testlowgreeki{%
  \test\alpha \test\beta \test\gamma \test\delta \test\epsilon
  \test\zeta \test\eta \test\theta \test\iota \test\kappa \test\lambda
  \test\mu }
\def\testlowgreekii{%
  \test\nu \test\xi \test o \test\pi \test\rho \test\sigma \test\tau
  \test\upsilon \test\phi \test\chi \test\psi \test\omega }
\def\testlowgreekiii{%
  \test\varepsilon \test\vartheta \test\varpi \test\varrho
  \test\varsigma \test\varphi}
\def\testlowgreek{%
  \testlowgreeki\testlowgreekii\testlowgreekiii}
	\renewcommand{\showfamily}{{\color{magenta}%
		Serif%
	}}
	{\rmfamily\mdseries\section{Math Test \showfamily}

\subsection{Overview \showfamily}

{\parindent 0pt
Default: $a \alpha \alphaup b \beta G \Gamma \upGamma \epsilon \varepsilon \theta \vartheta P \Pi \Sigma \sigma$; $\sigma_\epsilon, c^\alpha$

mathnormal: $\mathnormal{a \alpha \alphaup b \beta G \Gamma \upGamma \epsilon \varepsilon \theta \vartheta P \Pi \Sigma \sigma}$

mathrm: $\mathrm{a \alpha \alphaup b \beta G \Gamma \upGamma \epsilon \varepsilon \theta \vartheta P \Pi \Sigma \sigma}$

mathup: $\mathup{a \alpha \alphaup b \beta G \Gamma \upGamma \epsilon \varepsilon \theta \vartheta P \Pi \Sigma \sigma}$

mathit: $\mathit{a \alpha \alphaup b \beta G \Gamma \upGamma \epsilon \varepsilon \theta \vartheta P \Pi \Sigma \sigma}$

mathbf: $\mathbf{a \alphaup b \beta G \Gamma \upGamma \epsilon \varepsilon \theta \vartheta P \Pi \Sigma \sigma}$

mathbfit: $\mathbfit{a \alpha b \beta G \Gamma \upGamma \epsilon \varepsilon \theta \vartheta P \Pi \Sigma \sigma}$

mathbfup: $\mathbfup{a \alpha b \beta G \Gamma \upGamma \epsilon \varepsilon \theta \vartheta P \Pi \Sigma \sigma}$

\bigskip

{\bfseries
Default: $a \alpha \alphaup b \beta G \Gamma \upGamma \epsilon \varepsilon \theta \vartheta P \Pi \Sigma \sigma$; $\sigma_\epsilon, c^\alpha$

mathnormal: $\mathnormal{a \alpha \alphaup b \beta G \Gamma \upGamma \epsilon \varepsilon \theta \vartheta P \Pi \Sigma \sigma}$

mathrm: $\mathrm{a \alpha \alphaup b \beta G \Gamma \upGamma \epsilon \varepsilon \theta \vartheta P \Pi \Sigma \sigma}$

mathup: $\mathup{a \alpha \alphaup b \beta G \Gamma \upGamma \epsilon \varepsilon \theta \vartheta P \Pi \Sigma \sigma}$

mathit: $\mathit{a \alpha \alphaup b \beta G \Gamma \upGamma \epsilon \varepsilon \theta \vartheta P \Pi \Sigma \sigma}$

mathbf: $\mathbf{a \alpha \alphaup b \beta G \Gamma \upGamma \epsilon \varepsilon \theta \vartheta P \Pi \Sigma \sigma}$

mathbfit: $\mathbfit{a \alpha \alphaup b \beta G \Gamma \upGamma \epsilon \varepsilon \theta \vartheta P \Pi \Sigma \sigma}$

mathbfup: $\mathbfup{a \alpha \alphaup b \beta G \Gamma \upGamma \epsilon \varepsilon \theta \vartheta P \Pi \Sigma \sigma}$
}

\bigskip

{\sffamily\mdseries
Default: $a \alpha \alphaup b \beta G \Gamma \upGamma \epsilon \varepsilon \theta \vartheta P \Pi \Sigma \sigma$; $\sigma_\epsilon, c^\alpha$

mathnormal: $\mathnormal{a \alpha \alphaup b \beta G \Gamma \upGamma \epsilon \varepsilon \theta \vartheta P \Pi \Sigma \sigma}$

mathrm: $\mathrm{a \alpha \alphaup b \beta G \Gamma \upGamma \epsilon \varepsilon \theta \vartheta P \Pi \Sigma \sigma}$

mathup: $\mathup{a \alpha \alphaup b \beta G \Gamma \upGamma \epsilon \varepsilon \theta \vartheta P \Pi \Sigma \sigma}$

mathit: $\mathit{a \alpha \alphaup b \beta G \Gamma \upGamma \epsilon \varepsilon \theta \vartheta P \Pi \Sigma \sigma}$

mathbf: $\mathbf{a \alpha \alphaup b \beta G \Gamma \upGamma \epsilon \varepsilon \theta \vartheta P \Pi \Sigma \sigma}$

mathbfit: $\mathbfit{a \alpha \alphaup b \beta G \Gamma \upGamma \epsilon \varepsilon \theta \vartheta P \Pi \Sigma \sigma}$

mathbfup: $\mathbfup{a \alpha \alphaup b \beta G \Gamma \upGamma \epsilon \varepsilon \theta \vartheta P \Pi \Sigma \sigma}$
}

\bigskip

{\sffamily\bfseries

Default: $a \alpha \alphaup b \beta G \Gamma \upGamma \epsilon \varepsilon \theta \vartheta P \Pi \Sigma \sigma$; $\sigma_\epsilon, c^\alpha$

mathnormal: $\mathnormal{a \alpha \alphaup b \beta G \Gamma \upGamma \epsilon \varepsilon \theta \vartheta P \Pi \Sigma \sigma}$

mathrm: $\mathrm{a \alpha \alphaup b \beta G \Gamma \upGamma \epsilon \varepsilon \theta \vartheta P \Pi \Sigma \sigma}$

mathup: $\mathup{a \alpha \alphaup b \beta G \Gamma \upGamma \epsilon \varepsilon \theta \vartheta P \Pi \Sigma \sigma}$

mathit: $\mathit{a \alpha \alphaup b \beta G \Gamma \upGamma \epsilon \varepsilon \theta \vartheta P \Pi \Sigma \sigma}$

mathbf: $\mathbf{a \alpha \alphaup b \beta G \Gamma \upGamma \epsilon \varepsilon \theta \vartheta P \Pi \Sigma \sigma}$

mathbfit: $\mathbfit{a \alpha \alphaup b \beta G \Gamma \upGamma \epsilon \varepsilon \theta \vartheta P \Pi \Sigma \sigma}$

mathbfup: $\mathbfup{a \alpha \alphaup b \beta G \Gamma \upGamma \epsilon \varepsilon \theta \vartheta P \Pi \Sigma \sigma}$
}
}


\subsection{Formulas \showfamily}

\noindent%
\checkgreekletters

\noindent%
{\boldmath\checkgreekletters}

\noindent%
{\sffamily\selectfont \checkgreekletters}

\noindent%
{\sffamily\bfseries\selectfont \checkgreekletters}

\noindent%
{\sffamily $\alpha a > 0, \beta b + (3 \times 27), \Gamma G = 7 < 8, \lambda$}

\noindent%
$\alpha a > 0, \beta b + (3 \times 27), \Gamma G = 7 < 8, \lambda$

$\lim_{\nu \to \infty} v(\nu) = \max_{s \in S} \{s \pm 3 \gamma + y - 1\} = 4 \times 7$

$\hat{\beta} = (X'X)^{-1}X'y$

$$\lim_{N \to \infty} \sum_{i=0}^{N} x^i = \min_{x \in \mathbb{R}} S(x)$$

$$\int_{-\infty}^{\infty} x\,f(x)\,\mathup{d}x = \left( \frac{27}{2} \right)$$

\noindent%
{\bfseries%
$\alpha a > 0, \beta b + (3 \times 27), \Gamma G = 7 < 8, \lambda$

$\lim_{\nu \to \infty} v(\nu) = \max_{s \in S} \{s \pm 3 \gamma + y - 1\} = 4 \times 7$

$\hat{\beta} = (X'X)^{-1}X'y$

$$\lim_{N \to \infty} \sum_{i=0}^{N} x^i = \min_{x \in \mathbb{R}} S(x)$$

$$\int_{-\infty}^{\infty} x\,f(x)\,\mathup{d}x = \left( \frac{27}{2} \right)$$
}

\noindent%
{\sffamily%
$\alpha a > 0, \beta b + (3 \times 27), \Gamma G = 7 < 8, \lambda$

$\lim_{\nu \to \infty} v(\nu) = \max_{s \in S} \{s \pm 3 \gamma + y - 1\} = 4 \times 7$

$\hat{\beta} = (X'X)^{-1}X'y$

$$\lim_{N \to \infty} \sum_{i=0}^{N} x^i = \min_{x \in \mathbb{R}} S(x)$$

$$\int_{-\infty}^{\infty} x\,f(x)\,\mathup{d}x = \left( \frac{27}{2} \right)$$
}

\noindent%
{\sffamily\bfseries%
$\alpha a > 0, \beta b + (3 \times 27), \Gamma G = 7 < 8, \lambda$

$\lim_{\nu \to \infty} v(\nu) = \max_{s \in S} \{s \pm 3 \gamma + y - 1\} = 4 \times 7$

$\hat{\beta} = (X'X)^{-1}X'y$

$$\lim_{N \to \infty} \sum_{i=0}^{N} x^i = \min_{x \in \mathbb{R}} S(x)$$

$$\int_{-\infty}^{\infty} x\,f(x)\,\mathup{d}x = \left( \frac{27}{2} \right)$$
}


\subsection{Math Alphabets \showfamily}

%\sffamily\selectfont

Default
\def\test#1{#1,}
\begin{eqnarray*}
  && {\testnums}\\
  && {\testupper}\\
  && {\testlower}\\
  && {\testupgreek}\\
  && {\testlowgreek}
\end{eqnarray*}%

Math Normal (\texttt{\string\mathnormal})
\def\test#1{\mathnormal{#1},}
\begin{eqnarray*}
  && {\testnums}\\
  && {\testupper}\\
  && {\testlower}\\
  && {\testupgreek}\\
  && {\testlowgreek}
\end{eqnarray*}%

Math Italic (\texttt{\string\mathit})
\def\test#1{\mathit{#1},}
\begin{eqnarray*}
  && {\testnums}\\
  && {\testupper}\\
  && {\testlower}\\
  && {\testupgreek}\\
  && {\testlowgreek}
\end{eqnarray*}%

Math Roman (\texttt{\string\mathrm})
\def\test#1{\mathrm{#1},}
\begin{eqnarray*}
  && {\testnums}\\
  && {\testupper}\\
  && {\testlower}\\
  && {\testupgreek}\\
  && {\testlowgreek}
\end{eqnarray*}%

%Math Italic Bold (\texttt{\string\mathbm})
%\def\test#1{\mathbm{#1},}
%\begin{eqnarray*}
%  && {\testnums}\\
%  && {\testupper}\\
%  && {\testlower}\\
%  && {\testupgreek}\\
%  && {\testlowgreek}
%\end{eqnarray*}%

Math Bold (\texttt{\string\mathbf})
\def\test#1{\mathbf{#1},}
\begin{eqnarray*}
  && {\testnums}\\
  && {\testupper}\\
  && {\testlower}\\
  && {\testupgreek}\\
  && {\testlowgreek}
\end{eqnarray*}%

Caligraphic (\texttt{\string\mathcal})
\def\test#1{\mathcal{#1},}
\begin{eqnarray*}
  && {\testupper}
\end{eqnarray*}%

Script (\texttt{\string\mathscr})
\def\test#1{\mathscr{#1},}
\begin{eqnarray*}
  && {\testupper}
\end{eqnarray*}%

Fraktur (\texttt{\string\mathfrak})
\def\test#1{\mathfrak{#1},}
\begin{eqnarray*}
  && {\testupper}\\
  && {\testlower}
\end{eqnarray*}%

Blackboard Bold (\texttt{\string\mathbb})
\def\test#1{\mathbb{#1},}
\begin{eqnarray*}
  && {\testupper}
\end{eqnarray*}%

\subsection{Character Sidebearings \showfamily}

Default
\def\test#1{|#1|+{}}
\begin{eqnarray*}
  && {\testupperi}\\
  && {\testupperii}\\
  && {\testloweri}\\
  && {\testlowerii}\\
  && {\testupgreeki}\\
  && {\testupgreekii}\\
  && {\testlowgreeki}\\
  && {\testlowgreekii}\\
  && {\testlowgreekiii}
\end{eqnarray*}%

Math Roman (\texttt{\string\mathrm})
\def\test#1{|\mathrm{#1}|+{}}%
\begin{eqnarray*}
  && {\testupperi}\\
  && {\testupperii}\\
  && {\testloweri}\\
  && {\testlowerii}\\
  && {\testupgreeki}\\
  && {\testupgreekii}
\end{eqnarray*}%

%Math Italic Bold (\texttt{\string\mathbm})
%\def\test#1{|\mathbm{#1}|+{}}%
%\begin{eqnarray*}
%  && {\testupperi}\\
%  && {\testupperii}\\
%  && {\testloweri}\\
%  && {\testlowerii}\\
%  && {\testupgreeki}\\
%  && {\testupgreekii}\\
%  && {\testlowgreeki}\\
%  && {\testlowgreekii}\\
%  && {\testlowgreekiii}
%\end{eqnarray*}%

Math Bold (\texttt{\string\mathbf})
\def\test#1{|\mathbf{#1}|+{}}%
\begin{eqnarray*}
  && {\testupperi}\\
  && {\testupperii}\\
  && {\testloweri}\\
  && {\testlowerii}\\
  && {\testupgreeki}\\
  && {\testupgreekii}
\end{eqnarray*}%

Math Calligraphic (\texttt{\string\mathcal})
\def\test#1{|\mathcal{#1}|+{}}%
\begin{eqnarray*}
  && {\testupperi}\\
  && {\testupperii}
\end{eqnarray*}%


\subsection{Superscript Positioning \showfamily}

Default
\def\test#1{#1^{2}+{}}%
\begin{eqnarray*}
  && {\testupperi}\\
  && {\testupperii}\\
  && {\testloweri}\\
  && {\testlowerii}\\
  && {\testupgreeki}\\
  && {\testupgreekii}\\
  && {\testlowgreeki}\\
  && {\testlowgreekii}\\
  && {\testlowgreekiii}
\end{eqnarray*}%

Math Roman (\texttt{\string\mathrm})
\def\test#1{\mathrm{#1}^{2}+{}}%
\begin{eqnarray*}
  && {\testupperi}\\
  && {\testupperii}\\
  && {\testloweri}\\
  && {\testlowerii}\\
  && {\testupgreeki}\\
  && {\testupgreekii}
\end{eqnarray*}%

%Math Italic Bold (\texttt{\string\mathbm})
%\def\test#1{\mathbm{#1}^{2}+{}}%
%\begin{eqnarray*}
%  && {\testupperi}\\
%  && {\testupperii}\\
%  && {\testloweri}\\
%  && {\testlowerii}\\
%  && {\testupgreeki}\\
%  && {\testupgreekii}\\
%  && {\testlowgreeki}\\
%  && {\testlowgreekii}\\
%  && {\testlowgreekiii}
%\end{eqnarray*}%

Math Bold (\texttt{\string\mathbf})
\def\test#1{\mathbf{#1}^{2}+{}}%
\begin{eqnarray*}
  && {\testupperi}\\
  && {\testupperii}\\
  && {\testloweri}\\
  && {\testlowerii}\\
  && {\testupgreeki}\\
  && {\testupgreekii}
\end{eqnarray*}

Math Calligraphic (\texttt{\string\mathcal})
\def\test#1{\mathcal{#1}^{2}+{}}%
\begin{eqnarray*}
  && {\testupperi}\\
  && {\testupperii}
\end{eqnarray*}%


\subsection{Subscript Positioning \showfamily}

Default
\def\test#1{\mathnormal{#1}_{i}+{}}%
\begin{eqnarray*}
  && {\testupperi}\\
  && {\testupperii}\\
  && {\testloweri}\\
  && {\testlowerii}\\
  && {\testupgreeki}\\
  && {\testupgreekii}\\
  && {\testlowgreeki}\\
  && {\testlowgreekii}\\
  && {\testlowgreekiii}
\end{eqnarray*}%

Math Roman (\texttt{\string\mathrm})
\def\test#1{\mathrm{#1}_{i}+{}}%
\begin{eqnarray*}
  && {\testupperi}\\
  && {\testupperii}\\
  && {\testloweri}\\
  && {\testlowerii}\\
  && {\testupgreeki}\\
  && {\testupgreekii}
\end{eqnarray*}%

%Math Bold Italic (\texttt{\string\mathbm})
%\def\test#1{\mathbm{#1}_{i}+{}}%
%\begin{eqnarray*}
%  && {\testupperi}\\
%  && {\testupperii}\\
%  && {\testloweri}\\
%  && {\testlowerii}\\
%  && {\testupgreeki}\\
%  && {\testupgreekii}\\
%  && {\testlowgreeki}\\
%  && {\testlowgreekii}\\
%  && {\testlowgreekiii}
%\end{eqnarray*}

Math Bold (\texttt{\string\mathbf})
\def\test#1{\mathbf{#1}_{i}+{}}%
\begin{eqnarray*}
  && {\testupperi}\\
  && {\testupperii}\\
  && {\testloweri}\\
  && {\testlowerii}\\
  && {\testupgreeki}\\
  && {\testupgreekii}
\end{eqnarray*}%

Math Calligraphic (\texttt{\string\mathcal})
\def\test#1{\mathcal{#1}_{i}+{}}%
\begin{eqnarray*}
  && {\testupperi}\\
  && {\testupperii}
\end{eqnarray*}%


\subsection{Accent Positioning \showfamily}

Default
\def\test#1{\hat{#1}+{}}%
\begin{eqnarray*}
  && {\testnums}\\
  && {\testupperi}\\
  && {\testupperii}\\
  && {\testloweri}\\
  && {\testlowerii}\\
  && {\testupgreeki}\\
  && {\testupgreekii}\\
  && {\testlowgreeki}\\
  && {\testlowgreekii}\\
  && {\testlowgreekiii}
\end{eqnarray*}%

Math Italic (\texttt{\string\mathit})
\def\test#1{\hat{\mathit{#1}}+{}}%
\begin{eqnarray*}
  && {\testnums}\\
  && {\testupperi}\\
  && {\testupperii}\\
  && {\testloweri} \test\ell \test\wp \test\imath \test\jmath \tilde{i} \\
  && {\testlowerii}\\
  && {\testupgreeki}\\
  && {\testupgreekii}\\
  && {\testlowgreeki}\\
  && {\testlowgreekii}\\
  && {\testlowgreekiii}
\end{eqnarray*}%

Math Roman (\texttt{\string\mathrm})
\def\test#1{\hat{\mathrm{#1}}+{}}%
\begin{eqnarray*}
  && {\testnums}\\
  && {\testupperi}\\
  && {\testupperii}\\
  && {\testloweri}\\
  && {\testlowerii}\\
  && {\testupgreeki}\\
  && {\testupgreekii}
\end{eqnarray*}%

%Math Italic Bold (\texttt{\string\mathbm})
%\def\test#1{\hat{\mathbm{#1}}+{}}%
%\begin{eqnarray*}
%  && {\testnums}\\
%  && {\testupperi}\\
%  && {\testupperii}\\
%  && {\testloweri}\\
%  && {\testlowerii}\\
%  && {\testupgreeki}\\
%  && {\testupgreekii}\\
%  && {\testlowgreeki}\\
%  && {\testlowgreekii}\\
%  && {\testlowgreekiii}
%\end{eqnarray*}%

Math Bold (\texttt{\string\mathbf})
\def\test#1{\hat{\mathbf{#1}}+{}}%
\begin{eqnarray*}
  && {\testnums}\\
  && {\testupperi}\\
  && {\testupperii}\\
  && {\testloweri}\\
  && {\testlowerii}\\
  && {\testupgreeki}\\
  && {\testupgreekii}
\end{eqnarray*}

Math Calligraphic (\texttt{\string\mathcal})
\def\test#1{\hat{\mathcal{#1}}+{}}%
\begin{eqnarray*}
  && {\testupperi}\\
  && {\testupperii}
\end{eqnarray*}%


\subsection{Differentials \showfamily}

\begin{eqnarray*}
\gdef\test#1{\dit #1+{}}%
  && {\testupperi}\\
  && {\testupperii}\\
  && {\testloweri}\\
  && {\testlowerii}\\
  && {\testupgreeki}\\
  && {\testupgreekii}\\
  && {\testlowgreeki}\\
  && {\testlowgreekii}\\
  && {\testlowgreekiii}\\
\gdef\test#1{\dit \mathrm{#1}+{}}%
  && {\testupgreeki}\\
  && {\testupgreekii}
\end{eqnarray*}%

\begin{eqnarray*}
\gdef\test#1{\dup #1+{}}%
  && {\testupperi}\\
  && {\testupperii}\\
  && {\testloweri}\\
  && {\testlowerii}\\
  && {\testupgreeki}\\
  && {\testupgreekii}\\
  && {\testlowgreeki}\\
  && {\testlowgreekii}\\
  && {\testlowgreekiii}\\
\gdef\test#1{\dup \mathrm{#1}+{}}%
  && {\testupgreeki}\\
  && {\testupgreekii}
\end{eqnarray*}%

\begin{eqnarray*}
\gdef\test#1{\partial #1+{}}%
  && {\testupperi}\\
  && {\testupperii}\\
  && {\testloweri}\\
  && {\testlowerii}\\
  && {\testupgreeki}\\
  && {\testupgreekii}\\
  && {\testlowgreeki}\\
  && {\testlowgreekii}\\
  && {\testlowgreekiii}\\
\gdef\test#1{\partial \mathrm{#1}+{}}%
  && {\testupgreeki}\\
  && {\testupgreekii}
\end{eqnarray*}%


\subsection{Slash Kerning \showfamily}

\def\test#1{1/#1+{}}
\begin{eqnarray*}
  && {\testupperi}\\
  && {\testupperii}\\
  && {\testloweri}\\
  && {\testlowerii}\\
  && {\testupgreeki}\\
  && {\testupgreekii}\\
  && {\testlowgreeki}\\
  && {\testlowgreekii}\\
  && {\testlowgreekiii}
\end{eqnarray*}

\def\test#1{#1/2+{}}
\begin{eqnarray*}
  && {\testupperi}\\
  && {\testupperii}\\
  && {\testloweri}\\
  && {\testlowerii}\\
  && {\testupgreeki}\\
  && {\testupgreekii}\\
  && {\testlowgreeki}\\
  && {\testlowgreekii}\\
  && {\testlowgreekiii}
\end{eqnarray*}


\subsection{(Big) Operators \showfamily}

\def\testop#1{#1_{i=1}^{n} x^{n} \quad}
$
	\testop\sum
	\testop\prod
	\testop\coprod
	\testop\int
	\testop\oint
$

\noindent%
$
	\testop\bigotimes
	\testop\bigoplus
	\testop\bigodot
	\testop\bigwedge
	\testop\bigvee
	\testop\biguplus
	\testop\bigcup
	\testop\bigcap
	\testop\bigsqcup
	% \testop\bigsqcap
$

\begin{displaymath}
  \testop\sum
  \testop\prod
  \testop\coprod
  \testop\int
  \testop\oint
\end{displaymath}
\begin{displaymath}
  \testop\bigotimes
  \testop\bigoplus
  \testop\bigodot
  \testop\bigwedge
  \testop\bigvee
  \testop\biguplus
  \testop\bigcup
  \testop\bigcap
  \testop\bigsqcup
% \testop\bigsqcap
\end{displaymath}


\subsection{Radicals \showfamily}

\begin{displaymath}
  \sqrt{x+y} \qquad \sqrt{x^{2}+y^{2}} \qquad
  \sqrt{x_{i}^{2}+y_{j}^{2}} \qquad
  \sqrt{\left(\frac{\cos x}{2}\right)} \qquad
  \sqrt{\left(\frac{\sin x}{2}\right)}
\end{displaymath}

\begingroup
\delimitershortfall-1pt
\begin{displaymath}
  \sqrt{\sqrt{\sqrt{\sqrt{\sqrt{\sqrt{\sqrt{x+y}}}}}}}
\end{displaymath}
\endgroup % \delimitershortfall


\subsection{Over- and Underbraces \showfamily}

\begin{displaymath}
  \overbrace{x} \quad
  \overbrace{x+y} \quad
  \overbrace{x^{2}+y^{2}} \quad
  \overbrace{x_{i}^{2}+y_{j}^{2}} \quad
  \underbrace{x} \quad
  \underbrace{x+y} \quad
  \underbrace{x_{i}+y_{j}} \quad
  \underbrace{x_{i}^{2}+y_{j}^{2}} \quad
\end{displaymath}


\subsection{Normal and Wide Accents \showfamily}

\begin{displaymath}
  \dot{x} \quad
  \ddot{x} \quad
  \vec{x} \quad
  \bar{x} \quad
  \overline{x} \quad
  \overline{xx} \quad
  \tilde{x} \quad
  \widetilde{x} \quad
  \widetilde{xx} \quad
  \widetilde{xxx} \quad
  \hat{x} \quad
  \widehat{x} \quad
  \widehat{xx} \quad
  \widehat{xxx} \quad
\end{displaymath}

\begin{displaymath}
  \hat{x} \quad
  \check{x} \quad
  \tilde{x} \quad
  \acute{x} \quad
  \grave{x} \quad
  \dot{x} \quad
  \ddot{x} \quad
  \breve{x} \quad
  \bar{x} \quad
  \vec{x} \quad
\end{displaymath}


\subsection{Long Arrows \showfamily}

\begin{displaymath}
  \leftarrow \mathrel{-} \rightarrow \quad
  \leftrightarrow \quad
  \longleftarrow  \quad
  \longrightarrow \quad
  \longleftrightarrow \quad
  \Leftarrow = \Rightarrow \quad
  \Leftrightarrow \quad
  \Longleftarrow  \quad
  \Longrightarrow \quad
  \Longleftrightarrow \quad
\end{displaymath}


\subsection{Left and Right Delimiters \showfamily}

\def\testdelim#1#2{ - #1 f #2 - }
\begin{displaymath}
  \testdelim()
  \testdelim[]
  \testdelim\lfloor\rfloor
  \testdelim\lceil\rceil
  \testdelim\langle\rangle
  \testdelim\{\}
\end{displaymath}

Using {\tt\string\left} and {\tt\string\right}.
\def\testdelim#1#2{ - \left#1 f \right#2 - }
\begin{displaymath}
  \testdelim()
  \testdelim[]
  \testdelim\lfloor\rfloor
  \testdelim\lceil\rceil
  \testdelim\langle\rangle
  \testdelim\{\}
% \testdelim\lgroup\rgroup
% \testdelim\lmoustache\rmoustache
\end{displaymath}
\begin{displaymath}
  \testdelim)(
  \testdelim][
  \testdelim//
  \testdelim\backslash\backslash
  \testdelim/\backslash
  \testdelim\backslash/
\end{displaymath}


\subsection{Big-g-g Delimiters \showfamily}

\def\testdelim#1#2{%
  - \left#1\left#1\left#1\left#1\left#1\left#1\left#1\left#1 -
  \right#2\right#2\right#2\right#2\right#2\right#2\right#2\right#2 -}

\begingroup
\delimitershortfall-1pt
\begin{displaymath}
  \testdelim\lfloor\rfloor
  \qquad
  \testdelim()
\end{displaymath}
\begin{displaymath}
  \testdelim\lceil\rceil
  \qquad
  \testdelim\{\}
\end{displaymath}
\begin{displaymath}
  \testdelim[]
  \qquad
  \testdelim\lgroup\rgroup
\end{displaymath}
\begin{displaymath}
  \testdelim\langle\rangle
  \qquad
  \testdelim\lmoustache\rmoustache
\end{displaymath}
\begin{displaymath}
  \testdelim\uparrow\downarrow \quad
  \testdelim\Uparrow\Downarrow \quad
\end{displaymath}
\endgroup % \delimitershortfall

\def\X#1{$x #1 y$ &\tt\string#1}
\def\Y#1{$\big#1$ &\tt\string#1}
\def\Z#1{$x #1 y$}
\def\W#1#2{$#1{#2}$ &\tt\string#1\string{#2\string}}


\subsection{Binary Operators \showfamily}

\begin{tabular}{*8l}
\X\pm           &\X\cap         &\X\diamond             &\X\oplus     \\
\X\mp           &\X\cup         &\X\bigtriangleup       &\X\ominus    \\
\X\times        &\X\uplus       &\X\bigtriangledown     &\X\otimes    \\
\X\div          &\X\sqcap       &\X\triangleleft        &\X\oslash    \\
\X\ast          &\X\sqcup       &\X\triangleright       &\X\odot      \\
\X\star         &\X\vee         &\X\lhd                 &\X\bigcirc   \\
\X\circ         &\X\wedge       &\X\rhd                 &\X\dagger    \\
\X\bullet       &\X\setminus    &\X\unlhd               &\X\ddagger   \\
\X\cdot         &\X\wr          &\X\unrhd               &\X\S         \\
\X+             &\X-            &\X\amalg               &\X\P
\end{tabular}


\subsection{Relations \showfamily}

\begin{tabular}{*8l}
\X\leq          &\X\geq         &\X\equiv       &\X\models      \\
\X\prec         &\X\succ        &\X\sim         &\X\perp        \\
\X\preceq       &\X\succeq      &\X\simeq       &\X\mid         \\
\X\ll           &\X\gg          &\X\asymp       &\X\parallel    \\
\X\subset       &\X\supset      &\X\approx      &\X\bowtie      \\
\X\subseteq     &\X\supseteq    &\X\cong        &\X\Join        \\
\X\sqsubset     &\X\sqsupset    &\X\neq         &\X\smile       \\
\X\sqsubseteq   &\X\sqsupseteq  &\X\doteq       &\X\frown       \\
\X\in           &\X\ni          &\X\propto      &\X=            \\
\X\vdash        &\X\dashv       &\X<            &\X>            \\
\X:
\end{tabular}


\subsection{Punctuation \showfamily}

\begin{tabular}{*{5}{lp{3.2em}}}
\X,     &\X;    &\X\colon       &\X\ldotp       &\X\cdotp
\end{tabular}


\subsection{Arrows \showfamily}

\begin{tabular}{*6l}
\X\leftarrow            &\X\longleftarrow       &\X\uparrow     \\
\X\Leftarrow            &\X\Longleftarrow       &\X\Uparrow     \\
\X\rightarrow           &\X\longrightarrow      &\X\downarrow   \\
\X\Rightarrow           &\X\Longrightarrow      &\X\Downarrow   \\
\X\leftrightarrow       &\X\longleftrightarrow  &\X\updownarrow \\
\X\Leftrightarrow       &\X\Longleftrightarrow  &\X\Updownarrow \\
\X\mapsto               &\X\longmapsto          &\X\nearrow     \\
\X\hookleftarrow        &\X\hookrightarrow      &\X\searrow     \\
\X\leftharpoonup        &\X\rightharpoonup      &\X\swarrow     \\
\X\leftharpoondown      &\X\rightharpoondown    &\X\nwarrow     \\
\X\rightleftharpoons    &\X\leadsto
\end{tabular}


\subsection{Miscellaneous Symbols \showfamily}

\begin{tabular}{*8l}
\X\ldots        &\X\cdots       &\X\vdots       &\X\ddots       \\
\X\aleph        &\X\prime       &\X\forall      &\X\infty       \\
\X\hbar         &\X\emptyset    &\X\exists      &\X\Box         \\
\X\imath        &\X\nabla       &\X\neg         &\X\Diamond     \\
\X\jmath        &\X\surd        &\X\flat        &\X\triangle    \\
\X\ell          &\X\top         &\X\natural     &\X\clubsuit    \\
\X\wp           &\X\bot         &\X\sharp       &\X\diamondsuit \\
\X\Re           &\X\|           &\X\backslash   &\X\heartsuit   \\
\X\Im           &\X\angle       &\X\partial     &\X\spadesuit   \\
\X\mho          &\X.            &\X|            &\X!
\end{tabular}


\subsection{Variable-Sized Operators \showfamily}

\begin{tabular}{*6l}
\X\sum          &\X\bigcap      &\X\bigodot     \\
\X\prod         &\X\bigcup      &\X\bigotimes   \\
\X\coprod       &\X\bigsqcup    &\X\bigoplus    \\
\X\int          &\X\bigvee      &\X\biguplus    \\
\X\oint         &\X\bigwedge
\end{tabular}


\subsection{Log-Like Operators \showfamily}

\begin{tabular}{*8l}
\Z\arccos &\Z\cos  &\Z\csc &\Z\exp &
           \Z\ker    &\Z\limsup &\Z\min &\Z\sinh \\
\Z\arcsin &\Z\cosh &\Z\deg &\Z\gcd &
           \Z\lg     &\Z\ln     &\Z\Pr  &\Z\sup  \\
\Z\arctan &\Z\cot  &\Z\det &\Z\hom &
           \Z\lim    &\Z\log    &\Z\sec &\Z\tan  \\
\Z\arg    &\Z\coth &\Z\dim &\Z\inf &
           \Z\liminf &\Z\max    &\Z\sin &\Z\tanh
\end{tabular}


\subsection{Delimiters \showfamily}

\begin{tabular}{*8l}
\X(             &\X)            &\X\uparrow     &\X\Uparrow     \\
\X[             &\X]            &\X\downarrow   &\X\Downarrow   \\
\X\{            &\X\}           &\X\updownarrow &\X\Updownarrow \\
\X\lfloor       &\X\rfloor      &\X\lceil       &\X\rceil       \\
\X\langle       &\X\rangle      &\X/            &\X\backslash   \\
\X|             &\X\|
\end{tabular}


\subsection{Large Delimiters \showfamily}

\begin{tabular}{*8l}
\Y\rmoustache&  \Y\lmoustache&  \Y\rgroup&      \Y\lgroup\\[5pt]
\Y\arrowvert&   \Y\Arrowvert&   \Y\bracevert
\end{tabular}


\subsection{Math Mode Accents \showfamily}

\begin{tabular}{*{10}l}
\W\hat{a}     &\W\acute{a}  &\W\bar{a}    &\W\dot{a}    &\W\breve{a}\\
\W\check{a}   &\W\grave{a}  &\W\vec{a}    &\W\ddot{a}   &\W\tilde{a}\\
\end{tabular}


\subsection{Miscellaneous Constructions \showfamily}

\begin{tabular}{*4l}
\W\widetilde{abc}       &\W\widehat{abc}                        \\
\W\overleftarrow{abc}   &\W\overrightarrow{abc}                 \\
\W\overline{abc}        &\W\underline{abc}                      \\
\W\overbrace{abc}       &\W\underbrace{abc}                     \\[5pt]
\W\sqrt{abc}            &$\sqrt[n]{abc}$&\verb|\sqrt[n]{abc}|   \\
$f'$&\verb|f'|          &$\frac{abc}{xyz}$&\verb|\frac{abc}{xyz}|
\end{tabular}


\subsection{AMS Delimiters \showfamily}

\begin{tabular}{*8l}
\X\ulcorner&\X\urcorner&\X\llcorner&\X\lrcorner
\end{tabular}


\subsection{AMS Arrows \showfamily}

\begin{tabular}{*8l}
\X\dashrightarrow       &\X\dashleftarrow
        \\ \X\leftleftarrows      &\X\leftrightarrows     \\
\X\Lleftarrow           &\X\twoheadleftarrow
        \\ \X\leftarrowtail       &\X\looparrowleft       \\
\X\leftrightharpoons    &\X\curvearrowleft
        \\ \X\circlearrowleft     &\X\Lsh                 \\
\X\upuparrows           &\X\upharpoonleft
        \\ \X\downharpoonleft     &\X\multimap            \\
\X\leftrightsquigarrow  &\X\rightrightarrows
        \\ \X\rightleftarrows     &\X\rightrightarrows    \\
\X\rightleftarrows      &\X\twoheadrightarrow
        \\ \X\rightarrowtail      &\X\looparrowright      \\
\X\rightleftharpoons    &\X\curvearrowright
        \\ \X\circlearrowright    &\X\Rsh                 \\
\X\downdownarrows       &\X\upharpoonright
        \\ \X\downharpoonright    &\X\rightsquigarrow
\end{tabular}


\subsection{AMS Negated Arrows \showfamily}

\begin{tabular}{*8l}
\X\nleftarrow   &\X\nrightarrow \\ \X\nLeftarrow  &\X\nRightarrow \\
\X\nleftrightarrow&\X\nLeftrightarrow
\end{tabular}


\subsection{AMS Greek \showfamily}

\begin{tabular}{*4l}
\X\digamma      &\X\varkappa
\end{tabular}


\subsection{AMS Hebrew \showfamily}

\begin{tabular}{*6l}
\X\beth &\X\daleth      &\X\gimel
\end{tabular}


\subsection{AMS Miscellaneous \showfamily}

\begin{tabular}{*8l}
\X\hbar         &\X\hslash      \\ \X\vartriangle &\X\triangledown      \\
\X\square       &\X\lozenge     \\ \X\circledS    &\X\angle             \\
\X\measuredangle&\X\nexists     \\ \X\mho         &\X\Finv$^u$          \\
\X\Game$^u$     &\X\Bbbk$^u$    \\ \X\backprime   &\X\varnothing        \\
\X\blacktriangle&\X\blacktriangledown \\ \X\blacksquare&\X\blacklozenge  \\
\X\bigstar      &\X\sphericalangle     \\ \X\complement  &\X\eth       \\
\X\diagup$^u$   &\X\diagdown$^u$
\end{tabular}

$^u$ Not defined in {\tt amssymb.sty}, define using the
\verb|\newsymbol|  command.


\subsection{AMS Binary Operators \showfamily}

\begin{tabular}{*8l}
\X\dotplus      &\X\smallsetminus \\ \X\Cap        &\X\Cup               \\
\X\barwedge     &\X\veebar      \\ \X\doublebarwedge&\X\boxminus        \\
\X\boxtimes     &\X\boxdot      \\ \X\boxplus     &\X\divideontimes     \\
\X\ltimes       &\X\rtimes      \\ \X\leftthreetimes&\X\rightthreetimes \\
\X\curlywedge   &\X\curlyvee    \\ \X\circleddash &\X\circledast        \\
\X\circledcirc  &\X\centerdot   \\ \X\intercal
\end{tabular}


\subsection{AMS Relations \showfamily}

\begin{tabular}{*2l}
\X\leqslant    \\\X\lesssim    \\
\X\approxeq    \\\X\lll        \\
\X\lesseqgtr   \\\X\doteqdot   \\
\X\fallingdotseq\\\X\backsimeq  \\
\X\Subset      \\\X\preccurlyeq\\
\X\precsim     \\\X\vartriangleleft\\
\X\vDash      \\\X\smallsmile \\
\X\bumpeq      \\\X\geqq       \\
\X\eqslantgtr  \\\X\gtrapprox  \\
\X\ggg         \\\X\gtreqless  \\
\X\eqcirc      \\\X\triangleq  \\
\X\thickapprox \\\X\Supset     \\
\X\succcurlyeq \\\X\succsim    \\
\X\vartriangleright\\\X\Vdash      \\
\X\shortparallel\\\X\pitchfork  \\
\X\blacktriangleleft \\\X\backepsilon\\
\X\because
\end{tabular}


\subsection{AMS Negated Relations \showfamily}

\begin{tabular}{*8l}
\X\nless        &\X\nleq        \\ \X\nleqslant   &\X\nleqq       \\
\X\lneq         &\X\lneqq       \\ \X\lvertneqq   &\X\lnsim       \\
\X\lnapprox     &\X\nprec       \\ \X\npreceq     &\X\precnsim    \\
\X\precnapprox  &\X\nsim        \\ \X\nshortmid   &\X\nmid        \\
\X\nvdash       &\X\nvDash      \\ \X\ntriangleleft&\X\ntrianglelefteq\\
\X\nsubseteq    &\X\subsetneq   \\ \X\varsubsetneq&\X\subsetneqq  \\
\X\varsubsetneqq&\X\ngtr        \\ \X\ngeq        &\X\ngeqslant   \\
\X\ngeqq        &\X\gneq        \\ \X\gneqq       &\X\gvertneqq   \\
\X\gnsim        &\X\gnapprox    \\ \X\nsucc       &\X\nsucceq     \\
\X\nsucceqq     &\X\succnsim    \\ \X\succnapprox &\X\ncong       \\
\X\nshortparallel&\X\nparallel  \\ \X\nvDash      &\X\nVDash      \\
\X\ntriangleright&\X\ntrianglerighteq \\ \X\nsupseteq&\X\nsupseteqq\\
\X\supsetneq    &\X\varsupsetneq \\ \X\supsetneqq  &\X\varsupsetneqq
\end{tabular}}
\end{appendices}

\phantomsection % Needed for the bibliography bookmark to point to the correct page

\clearpage % Print References at a new page

\begin{refcontext}[sorting=nyt]  % Sort BIBLIOGRAPHY by alphabet (while CITATIONS are sorted by year)
\sloppy
\printbibliography[heading=bibintoc]
\end{refcontext}

\selectlanguage{ngerman}

\section*{Selbstständigkeitserklärung}

Ich versichere hiermit, dass ich die vorstehende Bachelorarbeit selbstständig verfasst und keine anderen als die angegebenen Quellen und Hilfsmittel benutzt habe, dass die vorgelegte Arbeit noch an keiner anderen Hochschule zur Prüfung vorgelegt wurde und dass sie weder ganz noch in Teilen bereits veröffentlicht wurde. Wörtliche Zitate und Stellen, die anderen Werken dem Sinn nach entnommen sind, habe ich in jedem einzelnen Fall kenntlich gemacht.

\vspace{\baselineskip}

\noindent%
%23.~Juli~2018
\number\DTMfetchday{currentdate}.~\DTMgermanmonthname{\DTMfetchmonth{currentdate}}~\DTMfetchyear{currentdate}
% \number strips a potentially present leading zero.

\vspace{3\baselineskip}

\noindent \thesisauthor

\thispagestyle{empty}


\end{document}